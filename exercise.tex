\documentclass[lang=cn,12pt,scheme=chinese,founder,mode=simple,black]{elegantbook}

\usepackage{array}
\newcommand{\ccr}[1]{\makecell{{\color{#1}\rule{1cm}{1cm}}}}

%# -*- coding: utf-8 -*-

% preamble_command_mathematics_v1_0.tex

% ---------- macro definition ----------
% ----- equation setting -----
% \everymath{\displaystyle}

% ----- command -----
\newcommand{\les}{\leqslant}
\newcommand{\ges}{\geqslant}
\newcommand{\degree}{^{\circ}}
\newcommand{\circA}{\overset{\circ}{\mathrm{A}}}
\newcommand{\dageq}{\stackrel{\dag}{ = }}
\newcommand{\ddageq}{\stackrel{\ddag}{ = }}
\newcommand{\dbdageq}{\stackrel{\dbdag}{ = }}
\newcommand{\stareq}{\stackrel{*}{ = }}
\newcommand{\dagiff}{\stackrel{\dag}{\iff}}
\newcommand{\ddagiff}{\stackrel{\ddag}{\iff}}
\newcommand{\dagimp}{\stackrel{\dag}{\implies}}
\newcommand{\ddagimp}{\stackrel{\ddag}{\implies}}
\newcommand{\tensor}[1]{\overleftrightarrow{#1}}
% \renewcommand\tensor[1]{\overset{\scriptscriptstyle\leftrightarrow}{#1}}
\renewcommand{\tensor}[1]{\overset{\scriptstyle\leftrightarrow}{#1}}
\newcommand{\by}{\leftarrow}
\newcommand{\quot}[2]{#1\mathbin{/}\!\!#2}

\newcommand{\join}{\Join}

% \renewcommand\pi{\piup}
\newcommand{\PI}{\uppi}
% \renewcommand\pi{\uppi}
% \renewcommand\pi{\text{\textpi}}
% \renewcommand\pi{\textrm{\greektext p}}

\newcommand{\nimplies}{\centernot\implies}
\newcommand{\nimpliedby}{\centernot\impliedby}

\renewcommand{\Re}{\mathrm{Re}\,}
\renewcommand{\Im}{\mathrm{Im}\,}
\newcommand{\const}{\mathrm{Const.}}
\newcommand{\sym}{\mathrm{sym}}
\newcommand{\cyc}{\mathrm{cyc}}
\newcommand{\ii}{\,\mathrm{i}}
\newcommand{\jj}{\,\mathrm{j}}

\DeclareMathOperator{\sgn}{sgn}
\DeclareMathOperator{\dif}{d\!}
\DeclareMathOperator{\diff}{d}
\DeclareMathOperator{\trans}{T}
\DeclareMathOperator{\tr}{tr}
\DeclareMathOperator{\grad}{\mathbf{grad}}
\DeclareMathOperator{\rank}{rank}
\DeclareMathOperator{\diag}{diag}
\DeclareMathOperator{\sinc}{sinc}
\DeclareMathOperator{\ad}{ad}
\DeclareMathOperator{\ex}{E}
\DeclareMathOperator{\var}{Var}
\DeclareMathOperator{\argmin}{arg\,min}
\DeclareMathOperator{\argmax}{arg\,max}
\DeclareMathOperator{\softmax}{softmax}
\DeclareMathOperator{\arccot}{arccot}
\DeclareMathOperator{\st}{s.t.}
\DeclareMathOperator{\iid}{i.i.d.}
\DeclareMathOperator{\Exp}{Exp}
\DeclareMathOperator{\Poi}{Poi}

\renewcommand{\parallel}{/\!\!/}

\newcommand{\bR}{\bm{\mathcal{R}}}

\newcommand{\Alpha}{\mathrm{A}}
\newcommand{\Beta}{\mathrm{B}}
\newcommand{\Epsilon}{\mathrm{E}}

\newcommand{\rA}{\mathrm{A}}
\newcommand{\rB}{\mathrm{B}}
\newcommand{\rC}{\mathrm{C}}
\newcommand{\rD}{\mathrm{D}}
\newcommand{\rE}{\mathrm{E}}
\newcommand{\rF}{\mathrm{F}}
\newcommand{\rG}{\mathrm{G}}
\newcommand{\rH}{\mathrm{H}}
\newcommand{\rI}{\mathrm{I}}
\newcommand{\rJ}{\mathrm{J}}
\newcommand{\rK}{\mathrm{K}}
\newcommand{\rL}{\mathrm{L}}
\newcommand{\rM}{\mathrm{M}}
\newcommand{\rN}{\mathrm{N}}
\newcommand{\rO}{\mathrm{O}}
\newcommand{\rP}{\mathrm{P}}
\newcommand{\rQ}{\mathrm{Q}}
\newcommand{\rR}{\mathrm{R}}
\newcommand{\rS}{\mathrm{S}}
\newcommand{\rT}{\mathrm{T}}
\newcommand{\rU}{\mathrm{U}}
\newcommand{\rV}{\mathrm{V}}
\newcommand{\rW}{\mathrm{W}}
\newcommand{\rX}{\mathrm{X}}
\newcommand{\rY}{\mathrm{Y}}
\newcommand{\rZ}{\mathrm{Z}}
\newcommand{\ra}{\mathrm{a}}
\newcommand{\rb}{\mathrm{b}}
\newcommand{\rc}{\mathrm{c}}
\newcommand{\rd}{\mathrm{d}}
\newcommand{\re}{\mathrm{e}}
\newcommand{\rf}{\mathrm{f}}
\newcommand{\rg}{\mathrm{g}}
\newcommand{\rh}{\mathrm{h}}
\newcommand{\ri}{\mathrm{i}}
\newcommand{\rj}{\mathrm{j}}
\newcommand{\rk}{\mathrm{k}}
\newcommand{\rl}{\mathrm{l}}
\renewcommand{\rm}{\mathrm{m}}
\newcommand{\rn}{\mathrm{n}}
\newcommand{\ro}{\mathrm{o}}
\newcommand{\rp}{\mathrm{p}}
\renewcommand{\rq}{\mathrm{q}}
\newcommand{\rr}{\mathrm{r}}
\newcommand{\rs}{\mathrm{s}}
\newcommand{\rt}{\mathrm{t}}
\newcommand{\ru}{\mathrm{u}}
\newcommand{\rv}{\mathrm{v}}
\newcommand{\rw}{\mathrm{w}}
\newcommand{\rx}{\mathrm{x}}
\newcommand{\ry}{\mathrm{y}}
\newcommand{\rz}{\mathrm{z}}

\newcommand{\0}{\bm{0}}
\newcommand{\1}{\bm{1}}
\newcommand{\A}{\bm{A}}
\newcommand{\B}{\bm{B}}
\newcommand{\C}{\mathbb{C}}
\newcommand{\D}{\bm{D}}
\newcommand{\E}{\bm{E}}
\newcommand{\F}{\bm{F}}
\newcommand{\G}{\bm{G}}
\renewcommand{\H}{\bm{H}}
\newcommand{\I}{\bm{I}}
\newcommand{\J}{\bm{J}}
\newcommand{\K}{\mathbb{K}}
\renewcommand{\L}{\bm{L}}
\newcommand{\M}{\bm{M}}
\newcommand{\N}{\mathbb{N}}
\renewcommand{\O}{\bm{O}}
\renewcommand{\P}{\bm{P}}
\newcommand{\Q}{\mathbb{Q}}
\newcommand{\R}{\mathbb{R}}
\renewcommand{\S}{\bm{S}}
\newcommand{\T}{\bm{T}}
\newcommand{\U}{\bm{U}}
\newcommand{\V}{\bm{V}}
\newcommand{\W}{\bm{W}}
\newcommand{\X}{\bm{X}}
\newcommand{\Y}{\bm{Y}}
\newcommand{\Z}{\mathbb{Z}}
\renewcommand{\a}{\bm{a}}
\renewcommand{\b}{\bm{b}}
\renewcommand{\c}{\bm{c}}
\renewcommand{\d}{\bm{d}}
\newcommand{\e}{\mathrm{e}}
\newcommand{\f}{\bm{f}}
\newcommand{\g}{\bm{g}}
\newcommand{\h}{\bm{h}}
\renewcommand{\i}{\mathrm{i}}
\renewcommand{\j}{\bm{j}}
\renewcommand{\k}{\bm{k}}
\renewcommand{\l}{\bm{l}}
\newcommand{\m}{\bm{m}}
\newcommand{\n}{\bm{n}}
\renewcommand{\o}{\bm{o}}
\newcommand{\p}{\bm{p}}
\newcommand{\q}{\bm{q}}
\renewcommand\r{\bm{r}}
\newcommand{\br}{\bm{r}}
\newcommand{\s}{\bm{s}}
\renewcommand{\t}{\bm{t}}
\renewcommand{\u}{\bm{u}}
\renewcommand{\v}{\bm{v}}
\newcommand{\w}{\bm{w}}
\newcommand{\x}{\bm{x}}
\newcommand{\y}{\bm{y}}
\newcommand{\z}{\bm{z}}

\newcommand{\balpha}{\bm{\alpha}}
\newcommand{\bbeta}{\bm{\beta}}
\newcommand{\bchi}{\bm{\chi}}
\newcommand{\bdelta}{\bm{\delta}}
\newcommand{\bvarepsilon}{\bm{\varepsilon}}
\newcommand{\bphi}{\bm{\phi}}
\newcommand{\bvarphi}{\bm{\varphi}}
\newcommand{\bgamma}{\bm{\gamma}}
\newcommand{\bmeta}{\bm{\eta}}
\newcommand{\biota}{\bm{\iota}}
\newcommand{\bkappa}{\bm{\kappa}}
\newcommand{\blambda}{\bm{\lambda}}
\newcommand{\bvarLambda}{\bm{\varLambda}}
\newcommand{\bmu}{\bm{\mu}}
\newcommand{\bnu}{\bm{\nu}}
\newcommand{\bpi}{\bm{\pi}}
\newcommand{\bvarpi}{\bm{\varpi}}
\newcommand{\btheta}{\bm{\theta}}
\newcommand{\bvartheta}{\bm{\vartheta}}
\newcommand{\brho}{\bm{\rho}}
\newcommand{\bsigma}{\bm{\sigma}}
\newcommand{\bvarsigma}{\bm{\varsigma}}
\newcommand{\btau}{\bm{\tau}}
\newcommand{\bupsilon}{\bm{\upsilon}}
\newcommand{\bomega}{\bm{\omega}}
\newcommand{\bxi}{\bm{\xi}}
\newcommand{\bpsi}{\bm{\psi}}
\newcommand{\bzeta}{\bm{\zeta}}
\newcommand{\bAlpha}{\bm{\Alpha}}
\newcommand{\bBeta}{\bm{\Beta}}
\newcommand{\bChi}{\bm{\Chi}}
\newcommand{\bDelta}{\bm{\Delta}}
\newcommand{\bEpsilon}{\bm{\Epsilon}}
\newcommand{\bPhi}{\bm{\Phi}}
\newcommand{\bGamma}{\bm{\Gamma}}
\newcommand{\bEta}{\bm{\Eta}}
\newcommand{\bIota}{\bm{\Iota}}
\newcommand{\bKappa}{\bm{\Kappa}}
\newcommand{\bLambda}{\bm{\Lambda}}
\newcommand{\bMu}{\bm{\Mu}}
\newcommand{\bNu}{\bm{\Nu}}
\newcommand{\bPi}{\bm{\Pi}}
\newcommand{\bTheta}{\bm{\Theta}}
\newcommand{\bRho}{\bm{\Rho}}
\newcommand{\bSigma}{\bm{\Sigma}}
\newcommand{\bvarSigma}{\bm{\varSigma}}
\newcommand{\bTau}{\bm{\Tau}}
\newcommand{\bUpsilon}{\bm{\Upsilon}}
\newcommand{\bOmega}{\bm{\Omega}}
\newcommand{\bXi}{\bm{\Xi}}
\newcommand{\bPsi}{\bm{\Psi}}
\newcommand{\bZeta}{\bm{\Zeta}}
\newcommand{\ve}{\varepsilon}
\newcommand{\laplace}{\Delta}
\newcommand{\Ln}{\mathop{\mathrm{Ln}}\nolimits}
\newcommand{\Arg}{\mathop{\mathrm{Arg}}\nolimits}
\newcommand{\supp}{\mathop{\mathrm{supp}}\nolimits}
\newcommand{\Aut}{\mathop{\mathrm{Aut}}\nolimits}
\ExplSyntaxOn

% 定义 \parfrac 宏,支持星号、阶数、括号组和逗号变量
\NewDocumentCommand{\parfrac}{ s O{1} m m } { \group_begin: \IfBooleanT {#1} { \cs_set:Npn
    \__parfrac_fractype: { \flatfrac } } \frac { \__parfrac_process_numerator:nn
    {#2}{#3} } { \__parfrac_process_denominator:nn {#2}{#4} } \group_end: }   % 分母处理(传入阶数)% 分子处理% 星号用 \flatfrac

% 处理分子(带阶数)
\cs_new_protected:Nn \__parfrac_process_numerator:nn
{
  \int_compare:nNnTF {#1} = {1}
    { \partial #2 } { \partial^{#1} #2 } }

% 处理分母(自动识别括号组/逗号变量,并应用阶数)
\cs_new_protected:Nn \__parfrac_process_denominator:nn
{
\tl_set:Nn \l_tmpa_tl {#2}
% 判断是否为括号组(如 (x,y))
\tl_if_in:NnTF \l_tmpa_tl {(}
{
\partial \l_tmpa_tl \int_compare:nNnT {#1} > {1} { ^{#1} } } { \seq_set_split:Nnn
\l_tmpa_seq { , } {#2} \int_compare:nNnTF {#1} = {1} { \seq_map_inline:Nn
\l_tmpa_seq {   % 添加阶数% 保留整体括号
  \partial ##1 } } { \seq_map_inline:Nn \l_tmpa_seq {  % 单阶数
  \partial ##1 ^{#1} } } } } % 多变量各自加阶数

\ExplSyntaxOff

\DeclareDocumentCommand\pdv{ s o m g g d() }
{
  \IfBooleanTF{#1}
  {\let\fractype\flatfrac}
  {\let\fractype\frac}
  \IfNoValueTF{#4}
  {
    \IfNoValueTF{#6}
    {\fractype{\partial \IfNoValueTF{#2}{}{^{#2}}}{\partial #3\IfNoValueTF{#2}{}{^{#2}}}} {\fractype{\partial \IfNoValueTF{#2}{}{^{#2}}}{\partial #3\IfNoValueTF{#2}{}{^{#2}}} \argopen(#6\argclose)} } { \IfNoValueTF{#5}
    {\fractype{\partial \IfNoValueTF{#2}{}{^{#2}} #3}{\partial #4\IfNoValueTF{#2}{}{^{#2}}}} {\fractype{\partial^2 #3}{\partial #4 \partial #5}} } }

\usepackage{esint}

% 重定义\eqref{}
\renewcommand{\eqref}[1]{\textup{式(\ref{#1}})}  % 输出类似 "Eq. (1)" 的格式
\newcommand{\thmref}[1]{\textup{定理~\ref{#1}}~}  % 输出类似 "Eq. (1)" 的格式
\renewcommand{\theoremautorefname}{定理}
\renewcommand{\figureautorefname}{图}

\usepackage{tkz-euclide}
\usepackage{tasks}

% 圆弧命令\wideparen
\DeclareSymbolFont{ugmL}{OMX}{mdugm}{m}{n}
\DeclareMathAccent{\wideparen}{\mathord}{ugmL}{"F3}

\let\oldexists\exists
\renewcommand{\exists}{\oldexists\,}
\newcommand{\llfloor}{\left\lfloor}
\newcommand{\rrfloor}{\right\rfloor}
\newcommand{\llceil}{\left\lceil}
\newcommand{\rrceil}{\right\rceil}

\usepackage{enumitem}
\newenvironment{enum}{\begin{enumerate}[label=\textbf{\arabic*.},ref=\arabic*,left=0cm]}
    {
  \end{enumerate}}

\newenvironment{eenum}{\begin{enumerate}[label=(\arabic*),left=0.85cm]}
    {
  \end{enumerate}}
\newenvironment{enuma}{\begin{enumerate}[label=(\arabic*),leftmargin=0.71cm]}
    {
  \end{enumerate}}
\newcommand\quan[1]{
  \tikz[baseline=(a.base)]\node(a)[inner sep=0.5pt,draw,circle]{$#1$};
}
\newcommand\closure[1]{%
  {}\mkern1mu\overline{\mkern-1mu#1}
}
\renewcommand\bar{\closure}

\newenvironment{xiti}{\par%
  \centerline{\textbf{习\quad 题\quad \thesection}}
  \begin{enum}
    }{\end{enum}}


\usepackage{tasks}

\linespread{1.5}

\addbibresource[location=local]{reference.bib}

\settasks{
    label=(\roman*),
    label-width = 3ex
}

\everymath{\displaystyle}

\newcommand\blfootnote[1]{%
    \begingroup
    \renewcommand\thefootnote{}%
    \footnote{#1}%
    \addtocounter{footnote}{-1}%
    \endgroup
}

\setcounter{secnumdepth}{1} % 只编号到 chapter;section 及以下不编号

\begin{document}

\begin{titlepage}
    \vspace*{\fill} % 垂直居中
    \begin{center}
        {\Huge \textbf{数学分析讲义(第一册)}} \\[0.5cm]

        {\Huge \textbf{习题解答}} \\[1cm]

        \date{\today}

    \end{center}
    \vspace*{\fill} % 垂直居中

    \blfootnote{如有疏漏,欢迎指正. 联系邮箱: \mailto{liuyue22@mail.ustc.edu.cn}}
\end{titlepage}

\tableofcontents

\mainmatter

\chapter*{第1章 \ 极限}
\addcontentsline{toc}{chapter}{第1章 \ 极限}

\section*{习题 1.1}
\addcontentsline{toc}{section}{习题 1.1}

\begin{exercise}[1.1.1]
    设 $a$ 是有理数, $b$ 是无理数. 求证: $a+b$ 和 $a-b$ 都是无理数; 当 $a \ne 0$ 时, $ab$ 和 $\frac{b}{a}$ 也都是无理数.
\end{exercise}

\begin{solution}
    设$a$是有理数,$b$是无理数.
    \begin{enumerate}[(1)]
        \item 若$a+b$是有理数,则$b = (a+b)-a$是有理数,矛盾.同理可证$a-b$是无理数.
        \item 若$ab$是有理数,则$b = \frac{ab}{a}$是有理数,矛盾.同理可证$\frac{b}{a}$是无理数.
    \end{enumerate}
\end{solution}

\begin{exercise}[1.1.2]
    求证: 两个不同的有理数之间有无理数.
\end{exercise}

\begin{solution}
    设$a,b$是两个不同的有理数,不妨设$a < b$.则存在正整数$k$, $N$使得
    $$\left( \sqrt{2} \right)^{2k-1} a < N < \left( \sqrt{2} \right)^{2k-1} b.$$
    具体而言,取$k > \llceil \log_2 (b-a) \rrceil$,则$k > \log_2 \frac{2 \sqrt{2}}{b-a} \Rightarrow 2^k(b-a) > 2 \sqrt{2} \Rightarrow \left( \sqrt{2} \right)^{2k-1} b - \left( \sqrt{2} \right)^{2k-1} a > 2$.因此,存在整数$N = \llfloor \left( \sqrt{2} \right)^{2k-1} b \rrfloor$,使得$\left( \sqrt{2} \right)^{2k-1} a < N < \left( \sqrt{2} \right)^{2k-1} b$.于是
    $$a < \frac{N}{\left( \sqrt{2} \right)^{2k-1}} < b.$$
    而$\frac{N}{\left( \sqrt{2} \right)^{2k-1}} = \frac{N \sqrt{2}}{2^k}$是无理数.
\end{solution}

\begin{exercise}[1.1.3]
    求证: $\sqrt{2}, \sqrt{3}$ 以及 $\sqrt{2}+\sqrt{3}$ 都是无理数.
\end{exercise}

\begin{solution}
    \begin{enumerate}[(1)]
        \item 设$\sqrt{2}$是有理数,则$\sqrt{2} = \frac{p}{q}$,其中$p,q$互素.因此$2q^2 = p^2$,由素数分解的唯一性可知$p$是偶数,设$p=2k$,则$2q^2 = 4k^2 \Rightarrow q^2 = 2k^2$,同理可知$q$也是偶数,与$p,q$互素矛盾.因此$\sqrt{2}$是无理数.
        \item 设$\sqrt{3}$是有理数,则$\sqrt{3} = \frac{p}{q}$,其中$p,q$互素.因此$3q^2 = p^2$,由素数分解的唯一性可知$p$是3的倍数,设$p=3k$,则$3q^2 = 9k^2 \Rightarrow q^2 = 3k^2$,同理可知$q$也是3的倍数,与$p,q$互素矛盾.因此$\sqrt{3}$是无理数.
        \item 设$\sqrt{2} + \sqrt{3}$是有理数,则$\sqrt{2} + \sqrt{3} = \frac{p}{q}$,其中$p,q$互素.因此$2 + 3 + 2\sqrt{6} = \frac{p^2}{q^2} \Rightarrow \sqrt{6} = \frac{p^2 - 5q^2}{2q^2}$,与$\sqrt{6}$是无理数矛盾.因此$\sqrt{2} + \sqrt{3}$是无理数.
    \end{enumerate}
\end{solution}

\begin{exercise}[1.1.4]
    把下列循环小数表示为分数:
    \begin{tasks}[label=(\arabic*)](3)
        \task $0.24999\dots$
        \task $0.\dot{3}7\dot{5}$
        \task $4.\dot{5}1\dot{8}$
    \end{tasks}
\end{exercise}

\begin{solution}
    \begin{enumerate}[(1)]
        \item 设$x = 0.24999\dots$,则$10x = 2.4999\dots$,因此$9x = 2.25 \Rightarrow x = \frac{9}{40}$.
        \item 设$x = 0.\dot{3}7\dot{5}$,则$1000x = 375.375375\dots$,因此$999x = 375 \Rightarrow x = \frac{375}{999}=\frac{125}{333}$.
        \item 设$x = 4.\dot{5}1\dot{8}$,则$1000x = 4518.518518\dots$,因此$999x = 4514 \Rightarrow x = \frac{4514}{999}=\frac{122}{27}$.
    \end{enumerate}
\end{solution}

\begin{exercise}[1.1.5]
    设 $r,s,t$ 都是有理数. 求证:
    \begin{enumerate}
        \item 若 $r+s\sqrt{2}=0$, 则 $r=s=0$;
        \item 若 $r+s\sqrt{2}+t\sqrt{3}=0$, 则 $r=s=t=0$.
    \end{enumerate}
\end{exercise}

\begin{solution}
    \begin{enumerate}[(1)]
        \item 假设$s \ne 0$,则$\sqrt{2} = -\frac{r}{s}$是有理数,与$\sqrt{2}$是无理数矛盾.因此$s=0$,从而$r=0$.
        \item $r + s\sqrt{2} + t\sqrt{3} = 0 \Rightarrow r^2 = 2s^2 + 3t^2 + 2st\sqrt{6} \Rightarrow \left( r^2 - 2s^2 - 3t^2 \right) + \left( -2st \right) \sqrt{6} = 0$. :

              与(1)类似,若$st \ne 0$,则$\sqrt{6} = \frac{r^2 - 2s^2 - 3t^2}{2st}$是有理数,与$\sqrt{6}$是无理数矛盾.故$st = 0$,
              \begin{enumerate}
                  \item 若$t = 0$,则$r + s\sqrt{2} = 0$,由(1)可知$r=s=0$;
                  \item 若$s = 0$,则$r + t\sqrt{3} = 0$,同理可知$r=t=0$.
              \end{enumerate}

    \end{enumerate}
\end{solution}

\begin{exercise}[1.1.6]
    设 $a_1, a_2, \dots, a_n$ 有相同的符号, 且都大于 $-1$. 证明:
    $$ (1+a_1)(1+a_2)\cdots(1+a_n) \ges 1+a_1+a_2+\cdots+a_n. $$
\end{exercise}

\begin{solution}
    利用数学归纳法:

    当$n=1$时,等式为$$1+a_1 \ges 1+a_1,$$显然成立.

    假设当$n=k$时,等式成立,即$$(1+a_1)(1+a_2)\cdots(1+a_k) \ges 1+a_1+a_2+\cdots+a_k.$$以此作为条件,当$n=k+1$时,由$a_{k+1} > -1$,可知$1+a_{k+1} > 0$,因此
    \begin{align*}
        (1+a_1)(1+a_2)\cdots(1+a_k)(1+a_{k+1}) & \ges (1+a_1+a_2+\cdots+a_k)(1+a_{k+1})                                       \\
                                               & = 1 + a_1 + a_2 + \cdots + a_k + a_{k+1} + a_{k+1}(a_1 + a_2 + \cdots + a_k) \\
                                               & \ges 1 + a_1 + a_2 + \cdots + a_k + a_{k+1}.
    \end{align*}
    其中$a_{k+1}(a_1 + a_2 + \cdots + a_k) = a_1 a_{k+1} + a_2 a_{k+1} + \cdots + a_k a_{k+1} \ges 0$,因为$a_i$与$a_{k+1}$符号相同.
\end{solution}

\begin{exercise}[1.1.7]
    设 $a, b$ 是实数, 且 $|a|<1, |b|<1$. 证明:
    $$ \left|\frac{a+b}{1+ab}\right| < 1. $$
\end{exercise}

\begin{solution}
    由$|a|<1, |b|<1$,可知$ab \ne -1$.因此
    $$\left|\frac{a+b}{1+ab}\right| < 1 \Leftrightarrow |a+b| < |1+ab| \Leftrightarrow (a+b)^2 < (1+ab)^2.$$
    即$$a^2 + b^2 + 2ab < 1 + a^2 b^2 + 2ab \Leftrightarrow a^2 + b^2 < 1 + a^2 b^2 \Leftrightarrow (1 - a^2)(1 - b^2) > 0.$$
    显然成立.
\end{solution}


\newpage
\section*{习题 1.2}
\addcontentsline{toc}{section}{习题 1.2}

\begin{exercise}[1.2.1]
    用定义证明下面的结论:
    \begin{tasks}[label=(\arabic*)](2)
        \task $\lim_{n\to\infty} \frac{n}{5+3n} = \frac{1}{3}$;
        \task $\lim_{n\to\infty} \frac{\sin n}{n} = 0$;
        \task $\lim_{n\to\infty} (-1)^n \frac{1}{\sqrt{n+1}} = 0$;
        \task $\lim_{n\to\infty} \frac{n!}{n^n} = 0$.
    \end{tasks}
\end{exercise}

\begin{solution}
    \begin{enumerate}[(1)]
        \item $\forall \varepsilon > 0$, 取$N = \llceil \frac{5}{9\varepsilon} \rrceil$, 则当 $n > N$ 时, 有
              $$
                  \left| \frac{n}{5 + 3n} - \frac{1}{3} \right| = \left| \frac{3n - (5 + 3n)}{3(5 + 3n)} \right| = \frac{5}{3(5 + 3n)} < \frac{5}{9n} < \varepsilon.
              $$
        \item $\forall \varepsilon > 0$, 取$N = \llceil \frac{1}{\varepsilon} \rrceil$, 则当 $n > N$ 时, 有
              $$
                  \left| \frac{\sin n}{n} - 0 \right| = \frac{|\sin n|}{n} \les \frac{1}{n} < \varepsilon.
              $$
        \item $\forall \varepsilon > 0$, 取$N = \llceil \frac{1}{\varepsilon^2} - 1 \rrceil$, 则当 $n > N$ 时, 有
              $$
                  \left| (-1)^n \frac{1}{\sqrt{n + 1}} - 0 \right| = \frac{1}{\sqrt{n + 1}} < \varepsilon.
              $$
        \item $\forall \varepsilon > 0$, 取$N = \llceil \frac{1}{\varepsilon} \rrceil$, 则当 $n > N$ 时, 有
              $$
                  \left| \frac{n!}{n^n} - 0 \right| = \frac{n!}{n^n} = \frac{1}{n} \cdot \frac{2}{n} \cdots \frac{n - 1}{n} \cdot \frac{n}{n} < \frac{1}{n} < \varepsilon.
              $$
    \end{enumerate}
\end{solution}

\begin{exercise}[1.2.2]
    若数列 $\{a_n\}$ ($n \ges 1$) 满足条件: 任给正数 $\varepsilon$, 存在正整数 $N$, 使得当 $n > N$ 时, 有 $|a_n - a| < M\varepsilon$ (其中 $M$ 为常数), 则 $\{a_n\}$ 必以 $a$ 为极限.
\end{exercise}

$M$为常数指的是$M$不依赖于$\ve$和$n$. 例如$M=2, M=1000$等都是常数.也就是说, 上述$(2)$ 其实等价于
$\forall M > 0, \forall \ve >0, \exists N\in \mathbb{N}^*, \forall n>N $都有 $|a_n-a|<M \ve$ 成立.

\begin{exercise}[1.2.3]
    证明: 当且仅当 $\lim_{n\to\infty} (a_n - a) = 0$ 时, 有 $\lim_{n\to\infty} a_n = a$. (数列极限的许多证明问题, 都可用同样的方法处理.)
\end{exercise}

\begin{proof}
    充分性: 由$\lim_{n \to \infty} (a_n - a) = 0$,则$\forall \ve > 0, \exists N \in \mathbb{N}^*, \forall n > N$都有$|a_n - a| < \ve$成立.因此$\lim_{n \to \infty} a_n = a$.

    必要性: 由$\lim_{n \to \infty} a_n = a$,则$\forall \ve > 0, \exists N \in \mathbb{N}^*, \forall n > N$都有$|a_n - a| < \ve$成立.因此$\lim_{n \to \infty} (a_n - a) = 0$.
\end{proof}

\begin{exercise}[1.2.4]
    证明: 若 $\lim_{n\to\infty} a_n = a$, 则 $\lim_{n\to\infty} |a_n| = |a|$; 反之不一定成立 (试举例说明). 但若 $\lim_{n\to\infty} |a_n| = 0$, 则有 $\lim_{n\to\infty} a_n = 0$.
\end{exercise}

\begin{proof}
    由 $\lim_{n\to\infty} a_n = a \Rightarrow \forall \ve > 0, \exists N \in \mathbb{N}^*$, 当 $n > N$ 时, 有 $|a_n - a| < \ve$. 则
    $$
        ||a_n| - |a|| \les |a_n - a| < \ve.
    $$
    即 $\lim_{n\to\infty} |a_n| = |a|$.

    反之不一定成立,如数列 $a_n = (-1)^n$, 则 $\lim_{n\to\infty} |a_n| = 1$, 但 $\{a_n\}$ 发散.

    若 $\lim_{n\to\infty} |a_n| = 0$, 则 $\forall \ve > 0, \exists N \in \mathbb{N}^*$, 当 $n > N$ 时, 有 $||a_n| - 0| < \ve$. 则
    $$
        |a_n - 0| = |a_n| < \ve.
    $$
    即 $\lim_{n\to\infty} a_n = 0$.
\end{proof}

\begin{exercise}[1.2.5]
    证明: 若 $\lim_{n\to\infty} a_n = 0$, 又 $|b_n| \les M$, ($n=1,2,\cdots$), 则 $\lim_{n\to\infty} a_n b_n = 0$.
\end{exercise}

\begin{proof}
    由 $\lim_{n\to\infty} a_n = 0 \Rightarrow \forall \ve > 0, \exists N \in \mathbb{N}^*$, 当 $n > N$ 时, 有 $|a_n - 0| < \frac{\ve}{M}$. 则
    $$
        |a_n b_n - 0| = |a_n| |b_n| < \frac{\ve}{M} \cdot M = \ve.
    $$
    即 $\lim_{n\to\infty} a_n b_n = 0$.
\end{proof}

\begin{exercise}[1.2.6]
    证明: 若数列 $\{a_n\}$ 满足 $\lim_{k\to\infty} a_{2k+1} = a$, 及 $\lim_{k\to\infty} a_{2k} = a$, 则 $\lim_{n\to\infty} a_n = a$.
\end{exercise}

\begin{solution}
    按已知条件 \(\forall \varepsilon > 0, \exists N_1 > 0\),当 \(n > N_1\) 时 \(|x_{2n} - a| < \varepsilon\).
    又 \(\exists N_2 > 0\),当 \(n > N_2\) 时 \(|x_{2n+1} - a| < \varepsilon\).于是令\( N = \max\{2N_1, 2N_2 + 1\}, \)
    则 \(n > N\) 时恒有 \(|x_n - a| < \varepsilon\).故 \(\lim_{n \to \infty} x_n = a\).

\end{solution}

\begin{exercise}[1.2.7]
    证明下列数列不收敛:
    \begin{tasks}[label=(\arabic*)](2)
        \task $a_n = (-1)^n \frac{n}{n+1}$;
        \task $a_n = 5 \left(1 - \frac{2}{n}\right) + (-1)^n$.
    \end{tasks}
\end{exercise}

\begin{solution}
    \begin{enumerate}[(1)]
        \item 取$a_{2n} = \frac{2n}{2n+1}, a_{2n+1} = -\frac{2n+1}{2n+2}$,则$\lim_{n \to \infty} a_{2n} = 1, \lim_{n \to \infty} a_{2n+1} = -1$,而如果$\{a_n\}$收敛,则$\lim_{n \to \infty} a_{2n} = \lim_{n \to \infty} a_{2n+1}$,矛盾.
        \item 取$a_{2n} = 5 \left(1 - \frac{1}{n}\right)+1, a_{2n+1} = 5 \left(1 - \frac{2}{2n+1}\right)-1$,则$\lim_{n \to \infty} a_{2n} = 6, \lim_{n \to \infty} a_{2n+1} = 4$,而如果$\{a_n\}$收敛,则$\lim_{n \to \infty} a_{2n} = \lim_{n \to \infty} a_{2n+1}$,矛盾.
    \end{enumerate}
\end{solution}

\begin{exercise}[1.2.8]
    求下列极限:
    \begin{enumerate}[(1)]
        \item $a_n = \frac{4n^2 + 5n + 2}{3n^2 + 2n + 1}$;
        \item $a_n = \frac{1}{1 \cdot 2} + \frac{1}{2 \cdot 3} + \cdots + \frac{1}{(n-1)n}$;
        \item $a_n = \left(1 - \frac{1}{3}\right) \left(1 - \frac{1}{6}\right) \cdots \left(1 - \frac{1}{n(n+1)/2}\right), \ n=2, 3, \ldots$;
        \item $a_n = \left(1 - \frac{1}{2^2}\right) \left(1 - \frac{1}{3^2}\right) \cdots \left(1 - \frac{1}{n^2}\right)$;
        \item $a_n = (1+q)(1+q^2)(1+q^4) \cdots (1+q^{2^m}), \ (|q|<1)$.
    \end{enumerate}
\end{exercise}

\begin{solution}
    \begin{enumerate}[(1)]
        \item $$\lim_{n \to \infty} a_n = \lim_{n \to \infty} \frac{4 + \frac{5}{n} + \frac{2}{n^2}}{3 + \frac{2}{n} + \frac{1}{n^2}} = \frac{\displaystyle \lim_{n \to \infty} 4 + \lim_{n \to \infty} \frac{5}{n} + \lim_{n \to \infty} \frac{2}{n^2}}{\displaystyle \lim_{n \to \infty} 3 + \lim_{n \to \infty} \frac{2}{n} + \lim_{n \to \infty} \frac{1}{n^2}} = \frac{4 + 0 + 0}{3 + 0 + 0} = \frac{4}{3}.$$
        \item $$a_n = \left( 1 - \frac{1}{2} \right) + \left( \frac{1}{2} - \frac{1}{3} \right) + \cdots + \left( \frac{1}{n-1} - \frac{1}{n} \right) = 1 - \frac{1}{n},$$
              $$\lim_{n \to \infty} a_n = \lim_{n \to \infty} 1 - \lim_{n \to \infty} \frac{1}{n} = 1 - 0 = 1.$$
        \item \begin{align*}
                  a_n = & \frac{2}{3} \cdot \frac{5}{6} \cdots \frac{(n^2 + n - 2)/2}{n(n+1)/2} = \frac{2}{3} \cdot \frac{5}{6} \cdots \frac{(n-1)(n+2)}{n(n+1)} = \frac{1\cdot 4}{2\cdot 3} \cdot \frac{2\cdot 5}{3\cdot 4} \cdots \frac{(n-1)(n+2)}{n(n+1)}
                  \\&= \frac{\left(1\cdot 2   \cdots (n-1) \right)\cdot\left(4\cdot 5 \cdots (n+2 )\right)}{\left(2\cdot 3  \cdots  (n) \right)\cdot\left(3\cdot 4 \cdots (n+1 )\right)}= \frac{1\cdot(n+2)}{n\cdot3} = \frac{n+2}{3n}
              ,\end{align*}
              $$\lim_{n \to \infty} a_n = \lim_{n \to \infty} \frac{n+2}{3n}= \lim_{n \to \infty}\frac{1}{3}+\frac{2}{3}\frac{1}{n} = \frac{1}{3}+\frac{2}{3}\cdot 0 = \frac{1}{3}.$$
        \item $$a_n = \frac{1 \cdot 3}{2 \cdot 2} \cdot \frac{2 \cdot 4}{3 \cdot 3} \cdots \frac{(n-1)(n+1)}{n \cdot n} = \frac{1}{2} \cdot \frac{n+1}{n} = \frac{n+1}{2n},$$
              $$\lim_{n \to \infty} a_n = \lim_{n \to \infty} \frac{n+1}{2n} = \frac{1 + \frac{1}{n}}{2} = \frac{1 + 0}{2} = \frac{1}{2}.$$
        \item $$a_n = \frac{(1-q)(1+q)(1+q^2)(1+q^4) \cdots (1+q^{2^n})}{1-q} = \frac{1-q^{2^{n+1}}}{1-q},$$
              $$\lim_{n \to \infty} a_n = \lim_{n \to \infty} \frac{1-q^{2^{n+1}}}{1-q} = \frac{\displaystyle 1 - \lim_{m \to \infty} q^{2^{n+1}}}{1-q} = \frac{1 - 0}{1-q} = \frac{1}{1-q}.$$
    \end{enumerate}
\end{solution}

\begin{exercise}[1.2.9]
    若 $a_n \ne 0 (n=1, 2, \ldots)$ 且 $\lim_{n\to\infty} a_n = a$, 能否断定 $\lim_{n\to\infty} \frac{a_n}{a_{n+1}} = 1$?
\end{exercise}

\begin{solution}
    不能. 例如 $a_n = \frac{1}{2^n}$, 则 $\lim_{n\to\infty} a_n = 0$, 但 $\lim_{n\to\infty} \frac{a_n}{a_{n+1}} = \lim_{n\to\infty} \frac{2^{n+1}}{2^n} = 2$.

    一个可能的错误做法是$$\lim_{n\to\infty} \frac{a_n}{a_{n+1}} = \frac{\lim_{n\to\infty} a_n}{\lim_{n\to\infty} a_{n+1}} = \frac{a}{a} = 1,$$ 但这是不允许的, 因为 $\lim_{n\to\infty} a_n$ 可能为 $0$.
\end{solution}

\begin{exercise}[1.2.10]
    若数列 $\{a_n\}, \{b_n\}$ 满足 $\lim_{n\to\infty} a_n \cdot b_n = 0$, 是否必有 $\lim_{n\to\infty} a_n = 0$ 或 $\lim_{n\to\infty} b_n = 0$?
    若还假设 $\lim_{n\to\infty} a_n = a$, 回答同样的问题.
\end{exercise}

\begin{solution}
    不一定. 例如 $a_n = \begin{cases}
            1, & n\text{为奇数} \\
            0, & n\text{为偶数} \\
        \end{cases}, b_n = \begin{cases}
            0, & n\text{为奇数} \\
            1, & n\text{为偶数} \\
        \end{cases}$, 则 $\lim_{n\to\infty} a_n \cdot b_n = \lim_{n\to\infty} 0=0$, 但 $\lim_{n\to\infty} a_n,\lim_{n\to\infty} b_n$均不存在.

    当 $\lim_{n\to\infty} a_n = a$时成立. 假设$a \ne 0$ 时, 则$\lim_{n\to\infty} b_n = \lim_{n\to\infty} \frac{a_n b_n}{a_n} = \frac{0}{a} = 0$.
\end{solution}

\begin{exercise}[1.2.11]
    若数列 $\{a_n\}$ 收敛, 数列 $\{b_n\}$ 发散, 则数列 $\{a_n \pm b_n\}, \{a_n \cdot b_n\}$ 的收敛性如何? 举例说明. 若数列 $\{a_n\}$ 与 $\{b_n\}$ 皆发散, 回答同样的问题.
\end{exercise}

\begin{solution}
    \begin{enumerate}
        \item $\{a_n\}$ 收敛, 数列 $\{b_n\}$ 发散,则
              \begin{enumerate}
                  \item  $\{a_n + b_n\}, \{a_n - b_n\}$ 都发散
                        可以采用反证法:若$\{a_n + b_n\}$收敛,由于 $\{a_n\}$ 收敛,容易知道$\{a_n + b_n-a_n\}=\{b_n\}$收敛,这与$\{b_n\}$ 发散矛盾,因此$\{a_n + b_n\}$发散, $\{a_n - b_n\}$同理可得 .
                  \item $\{a_n \cdot b_n\}$ 的收敛性不确定.
                        \begin{enumerate}
                            \item  $a_n = \frac{1}{n}, b_n = n$, 则 $a_n \cdot b_n = 1$ 收敛;
                            \item  $a_n = 1, b_n = n$, 则 $a_n \cdot b_n = n$ 发散.
                        \end{enumerate}
              \end{enumerate}
        \item  $\{a_n\}, \{b_n\}$ 都发散,则
              \begin{enumerate}
                  \item  $\{a_n + b_n\}$ 的收敛性不确定
                        \begin{enumerate}
                            \item  $a_n = n, b_n =-n$, 则 $a_n + b_n = 0$ 收敛.
                            \item  $a_n = n, b_n = n$, 则 $a_n + b_n = 2n$ 发散.
                        \end{enumerate}
                  \item  $\{a_n - b_n\}$ 的收敛性不确定
                        \begin{enumerate}
                            \item  $a_n = n+\frac{1}{n}, b_n = n$, 则 $a_n - b_n = \frac{1}{n},$ 收敛.
                            \item  $a_n = (-1)^{n}, b_n =(-1)^{n-1}$, 则 $a_n -b_n = 2\cdot(-1)^{n}$ 发散.
                        \end{enumerate}
                  \item $\{a_n \cdot b_n\}$ 的收敛性不确定.
                        \begin{enumerate}
                            \item  $a_n = \begin{cases}
                                          1, & n\text{为奇数} \\
                                          0, & n\text{为偶数} \\
                                      \end{cases}, b_n = \begin{cases}
                                          0, & n\text{为奇数} \\
                                          1, & n\text{为偶数} \\
                                      \end{cases}$, 则 $a_n \cdot b_n = 0$ 收敛.
                            \item  $a_n = n, b_n = (-1)^n$, 则 $a_n \cdot b_n = (-1)^n n$ 发散;
                        \end{enumerate}
              \end{enumerate}
    \end{enumerate}
\end{solution}

\begin{exercise}[1.2.12]
    下面的推理是否正确?
    \begin{enumerate}
        \item 设数列 $\{a_n\}: a_1 = 1, a_{n+1} = 2a_n - 1 \ (n=1, 2, 3, \ldots)$, 求 $\lim_{n\to\infty} a_n$.
              \par\noindent 解: 设 $\lim_{n\to\infty} a_n = a$, 在 $a_{n+1} = 2a_n - 1$ 两边取极限, 得 $a = 2a - 1$, 即 $a = 1$.

        \item
              \begin{align*}
                   & \lim_{n\to\infty} \left( \frac{1}{\sqrt{n^2+1}} + \frac{1}{\sqrt{n^2+2}} + \cdots + \frac{1}{\sqrt{n^2+n}} \right)                        \\
                   & = \lim_{n\to\infty} \frac{1}{\sqrt{n^2+1}} + \lim_{n\to\infty} \frac{1}{\sqrt{n^2+2}} + \cdots + \lim_{n\to\infty} \frac{1}{\sqrt{n^2+n}} \\
                   & = \underbrace{0 + 0 + \cdots + 0}_{\text{$n$个}} = 0.
              \end{align*}

        \item $\lim_{n\to\infty} \left(1 + \frac{1}{n}\right)^n = \left[\lim_{n\to\infty} \left(1 + \frac{1}{n}\right)\right]^n = 1^n = 1$.
    \end{enumerate}
\end{exercise}

\begin{solution}
    \begin{enumerate}
        \item 错误. 不能在未知数列是否收敛时,就假设极限存在并对递推公式两边取极限. 实际上,该数列的通项公式为 $a_n = 1$,所以 $\lim_{n\to\infty} a_n = 1$.
        \item 错误. 不能将一个数列的极限拆成无穷多个数列极限的和. 实际上
              $$
                  \frac{n}{\sqrt{n^2+n}}\les \frac{1}{\sqrt{n^2+1}} + \frac{1}{\sqrt{n^2+2}} + \cdots + \frac{1}{\sqrt{n^2+n}}
                  \les  \frac{n}{\sqrt{n^2+1}}.
              $$
              并有$$\lim_{n\to}\frac{n}{\sqrt{n^2+n}}=1,\lim_{n\to}\frac{n}{\sqrt{n^2+1}}=1.
              $$
              由夹逼准则$$
                  \lim_{n\to\infty} \left( \frac{1}{\sqrt{n^2+1}} + \frac{1}{\sqrt{n^2+2}} + \cdots + \frac{1}{\sqrt{n^2+n}} \right) = 1.
              $$
        \item 错误. 不能将一个数列的极限拆成无穷多个数列极限的积. 实际上
              $$
                  \lim_{n\to\infty} \left(1 + \frac{1}{n}\right)^n = \e.
              $$
    \end{enumerate}
\end{solution}

\begin{exercise}[1.2.13]
    设数列 $\{a_n\}$ 与 $\{b_n\}$ 分别收敛于 $a, b$. 若 $a > b$, 则从某一项开始, 有 $a_n > b_n$; 反之, 若从某项开始恒有 $a_n \ges b_n$, 则 $a \ges b$.
\end{exercise}

\begin{solution}
    这是保序性的直接推论.
\end{solution}

\begin{exercise}[1.2.14]
    设数列 $\{a_n\}, \{b_n\}$ 分别收敛于 $a$ 及 $b$. 记 $c_n = \max(a_n, b_n)$, $d_n = \min(a_n, b_n) \ (n=1, 2, \ldots)$. 证明
    $$
        \lim_{n\to\infty} c_n = \max(a, b), \quad \lim_{n\to\infty} d_n = \min(a, b).
    $$
\end{exercise}

\begin{solution}
    由$\max(x,y) = \frac{x+y+|x-y|}{2}, \min(x,y) = \frac{x+y-|x-y|}{2}$,以及数列极限的四则运算和绝对值运算可得.
\end{solution}

\begin{exercise}[1.2.15]
    求下列极限:
    \begin{enumerate}[(1)]
        \item $\lim_{n\to\infty} \left[\frac{1}{(n+1)^2} + \frac{1}{(n+2)^2} + \cdots + \frac{1}{(2n)^2}\right]$;
        \item $\lim_{n\to\infty} ((n+1)^k - n^k)$, 其中 $0 < k < 1$;
        \item $\lim_{n\to\infty} (\sqrt{2} \cdot \sqrt[4]{2} \cdot \sqrt[8]{2} \cdots \sqrt[2^n]{2})$;
        \item $\lim_{n\to\infty} \left(\sqrt{n^2 - n + 2} - n\right)$;
        \item $\lim_{n\to\infty} \left(\sqrt{\cos^2 1 + \cos^2 2 + \cdots + \cos^2 n}\right)$.
    \end{enumerate}
\end{exercise}

\begin{solution}
    \begin{enumerate}[(1)]
        \item 由于$$0\les \sum_{k=1}^{n} \frac{1}{(n+k)^2} \les \sum_{k=1}^{n} \frac{1}{n^2} = \frac{n}{n^2} = \frac{1}{n}.$$
              并且$$\lim_{n\to\infty} 0 = 0,\lim_{n\to\infty} \frac{1}{n} = 0.$$
              由夹逼准则$$\lim_{n\to\infty}\sum_{k=1}^{n} \frac{1}{(n+k)^2}=0.$$
        \item 由于$$0\les ((n+1)^k - n^k) = n^k \left( \left( 1 + \frac{1}{n} \right)^k - 1 \right) \les n^k \left( \left( 1 + \frac{1}{n} \right)^1 - 1 \right)  = n^{k-1} .$$
              并且$$\lim_{n\to\infty} 0=0,\lim_{n\to\infty} n^{k-1} = 0.$$
              由夹逼准则$$\lim_{n\to\infty} ((n+1)^k - n^k)  = 0.$$
        \item $$\lim_{n\to\infty} \prod_{k=1}^{n} \sqrt[2^k]{2} = \lim_{n\to\infty} 2^{\sum\limits_{k=1}^{n} \frac{1}{2^k}} =  2^{\lim\limits_{n\to\infty}\sum\limits_{k=1}^{n} \frac{1}{2^k}} =2^1 = 2.$$
        \item 由$\frac{1}{n} \ln(n^2 - n + 2) = O\left( \frac{\ln n}{n} \right) = o(1)$,故
              $$\lim_{n\to\infty} \sqrt[n]{n^2 - n + 2} = \lim_{n\to\infty} \e^{ \frac{1}{n} \ln \left(n^2 - n + 2\right)} =\e^{\lim\limits_{n\to\infty} \frac{1}{n} \ln \left(n^2 - n + 2\right)} = \e^0 = 1.$$

        \item 由于$$\sqrt[n]{\cos^2 1} \les \sqrt[n]{\cos^2 1 + \cos^2 2 + \cdots + \cos^2 n} \les \sqrt[n]{n}.$$
              并且$$\lim_{n\to\infty} \sqrt[n]{\cos^2 1} =1,\lim_{n\to\infty} \sqrt[n]{n} = 1.$$
              由夹逼准则$$\lim_{n\to\infty} \sqrt[n]{\cos^2 1 + \cos^2 2 + \cdots + \cos^2 n} = 1.$$
    \end{enumerate}
\end{solution}

\begin{exercise}[1.2.16]
    设 $a_1, a_2, \ldots, a_m$ 为 $m$ 个正数, 证明:
    $$
        \lim_{n\to\infty} \sqrt[n]{a_1^n + a_2^n + \cdots + a_m^n} = \max(a_1, a_2, \ldots, a_m).
    $$
\end{exercise}

\begin{solution}
    设 $a_k = \max \{a_1, a_2, \ldots, a_m \}$,则
    $$
        a_k = \sqrt[n]{a_k^n} \les \sqrt[n]{a_1^n + a_2^n + \cdots + a_m^n} \les \sqrt[n]{m a_k^n} = m^{\frac{1}{n}} a_k.
    $$
    由夹逼定理可得
    $$
        \lim_{n\to\infty} \sqrt[n]{a_1^n + a_2^n + \cdots + a_m^n} = a_k = \max(a_1, a_2, \ldots, a_m).
    $$
\end{solution}

\begin{exercise}[1.2.17]
    证明下列数列收敛:
    \begin{enumerate}[(1)]
        \item $a_n = \left(1 - \frac{1}{2}\right) \left(1 - \frac{1}{2^2}\right) \cdots \left(1 - \frac{1}{2^n}\right)$;
        \item $a_n = \frac{1}{3+1} + \frac{1}{3^2+1} + \cdots + \frac{1}{3^n+1}$;
        \item $a_n = \alpha_0 + \alpha_1 q + \cdots + \alpha_n q^n$, 其中 $|\alpha_k| \les M, (k=1, 2, \ldots)$, 而 $|q| < 1$;
        \item $a_n = \frac{\cos 1}{1 \cdot 2} + \frac{\cos 2}{2 \cdot 3} + \frac{\cos 3}{3 \cdot 4} + \cdots + \frac{\cos n}{n(n+1)}$.
    \end{enumerate}
\end{exercise}

\begin{proof}
    \begin{enumerate}[(1)]
        \item 由$1-\frac{1}{2^n} < 1$,可知$\{a_n\}$单调减,且$a_n > 0$,因此$\{a_n\}$收敛.
        \item 由$a_n < \sum_{k=1}^n \frac{1}{3^k} < \frac 12$,可知$\{a_n\}$有上界,且$a_n$单调递增,因此$\{a_n\}$收敛.
        \item 利用Cauchy收敛准则,对$\forall \ve >0$,取$N = \left\lfloor \log_{|q|} \frac{\ve(1-|q|)}{2M} \right\rfloor + 1$,则当$m>n>N$时,$$|a_m-a_n| = |\alpha_{n+1}q^{n+1}+\cdots+\alpha_m q^m| \les M(|q|^{n+1}+|q|^{n+2}+\cdots) = M\frac{|q|^{n+1}}{1-|q|} < \ve.$$
        \item 利用Cauchy收敛准则,对$\forall \ve >0$,取$N = \left\lfloor \frac{1}{\ve} \right\rfloor + 1$,则当$m>n>N$时,$$|a_m-a_n| = \left|\frac{\cos(n+1)}{(n+1)(n+2)}+\cdots+\frac{\cos m}{m(m+1)}\right| \les \sum_{k=n+1}^m \frac{1}{k(k+1)} = \frac{1}{n+1} - \frac{1}{m+1} < \frac{1}{n+1} < \ve.$$
    \end{enumerate}
\end{proof}

\begin{exercise}[1.2.18]
    证明下列数列收敛, 并求出其极限:
    \begin{enumerate}[(1)]
        \item $a_n = \frac{n}{c^n}, \ (c>1)$;
        \item $a_1 = \frac{c}{2}, \ a_{n+1} = \frac{c}{2} + \frac{a_n^2}{2} \ (0 \les c \les 1)$;
        \item $a > 0, a_0 > 0, a_{n+1} = \frac{1}{2} \left(a_n + \frac{a}{a_n}\right)$ (提示: 先证明 $a_n^2 \ges a$);
        \item $a_0 = 1, \ a_n = 1 + \frac{a_{n-1}}{a_{n-1} + 1}$;
        \item $a_n = \sin \sin \cdots \sin 1$ ($n$ 个 $\sin$).
    \end{enumerate}
\end{exercise}

\begin{solution}
    \begin{enumerate}[(1)]
        \item 由Stolz定理,有$$\lim_{n \to \infty} a_n = \lim_{n \to \infty} \frac{n}{c^n} = \lim_{n \to \infty} \frac{(n+1)-n}{c^{n+1}-c^n} = \lim_{n \to \infty} \frac{1}{c^n(c-1)} = 0.$$
        \item $$a_{n+1} - a_n = \frac{1}{2}(a_{n}-a_{n-1})(a_{n}+a_{n-1})$$
              由$a_2 - a_1 = \left( \frac{c}{2} \right)^2 > 0 $,可递归的得知$a_{n+1} - a_n > 0$,因此$\{a_n\}$单调增,且$a_1 < c$,归纳的可得$a_{n+1} < \frac{c}{2} + \frac{c^2}{2} < \frac{c}{2} + \frac{c}{2} = c$,因此$\{a_n\}$有上界,故$\{a_n\}$收敛.设$\lim_{n \to \infty} a_n = a$,则$a = \frac{c}{2} + \frac{a^2}{2} \Rightarrow a^2 - 2a + c = 0 \Rightarrow a = 1 \pm \sqrt{1-c}$,又由$a_n > 0$,可知$a = 1 - \sqrt{1-c}$.
        \item 由均值不等式,$$a_{n+1} = \left( \frac{1}{2} \left( a_n + \frac{a}{a_n} \right) \right)^2 \ges a$$
              于是
              $$a_{n+1} - a_n = \frac{a - a_n^2}{2a_n} \les 0$$
              因此$\{a_n\}$在$n \ges 1$时单调减,且有下界$\sqrt{a}$,因此$\{a_n\}$收敛.设$\lim_{n \to \infty} a_n = l$,则$l = \frac{1}{2} \left( l + \frac{a}{l} \right) $,解得$l= \sqrt{a}$.
        \item $$a_n - a_{n-1} = \frac{1+ a_{n-1} - a_{n-1}^2}{a_{n-1} + 1} $$
              $$1 + a_n - a_n^2 = 1 + 1 + \frac{a_{n-1}}{a_{n-1} + 1} - \left( 1 + \frac{a_{n-1}}{a_{n-1} + 1} \right)^2 = \frac{1 + a_{n-1} - a_{n-1}^2}{(a_{n-1} + 1)^2}$$
              由$1+a_0 -a_0^2 = 1 > 0$归纳的可得$1 + a_n - a_n^2 > 0$,因此$a_n - a_{n-1} > 0$,即$\{a_n\}$单调递增,且
              $1 + a_n - a_n^2 > 0 \Rightarrow a_n < \frac{1 + \sqrt{5}}{2}$有上界,
              因此$\{a_n\}$收敛,设$\lim_{n\to\infty}a_n=a$.递推式两侧取极限,得$a = 1 + \frac{a}{a+1} \Rightarrow a^2 - a - 1 = 0 \Rightarrow a = \frac{1 \pm \sqrt{5}}{2}$;由于$a_n>0$始终成立,故$a\ges 0$而$\frac{1 -\sqrt{5}}{2}<0$,故舍去这一值,进而得到$a=\frac{1 + \sqrt{5}}{2}.$
        \item $a_n = \sin a_{n-1} < a_{n-1}$,因此$\{a_n\}$单调减,且$a_n > 0$,因此$\{a_n\}$收敛.设$\lim_{n \to \infty} a_n = a$,则$a = \sin a \Rightarrow a = 0$.
    \end{enumerate}
\end{solution}

\begin{exercise}[1.2.19]
    设 $a_n \les a \les b_n \ (n=1, 2, \ldots)$, 且 $\lim_{n\to\infty} (a_n - b_n) = 0$. 求证: $\lim_{n\to\infty} a_n = a$, $\lim_{n\to\infty} b_n = a$.
\end{exercise}

\begin{solution}
    由$\lim_{n\to\infty} (a_n - b_n) = 0$,对$\forall \ve > 0$,存在$N\in \mathbb{N}^*$,使得当$n>N$时,$|a_n - b_n| < \ve$.又由$a_n \les a \les b_n$,可知$|a_n - a| = a -a_n \les b_n - a_n < \ve$,同理$|b_n - a| < \ve$.因此$\lim_{n\to\infty} a_n = a$, $\lim_{n\to\infty} b_n = a$.
\end{solution}

\begin{exercise}[1.2.20]
    证明: 若 $a_n > 0$, 且 $\lim_{n\to\infty} \frac{a_{n}}{a_{n+1}} = l > 1$, 则 $\lim_{n\to\infty} a_n = 0$.
\end{exercise}

\begin{solution}
    先证明一个引理:
    设$a_n > 0, n = 1, 2, \ldots$, 且$\lim_{n\to\infty} a_n = a$, 则$\lim_{n\to\infty} \sqrt[n]{a_1 a_2 \cdots a_n} = a$.

    证明如下
    \begin{enumerate}[(1)]
        \item $a=0$时,$$0 \les \sqrt[n]{a_1 a_2 \cdots a_n} \les \frac{a_1 + a_2 + \cdots + a_n}{n} $$
              \     同时,由Stolz定理,$$\lim_{n\to\infty}  \frac{a_1 + a_2 + \cdots + a_n}{n}=\lim_{n\to\infty}  \frac{a_n}{1}=0$$
              由夹逼定理,得证.
        \item $a>0$时,$$\frac{n}{\frac{1}{a_1} + \frac{1}{a_2} + \cdots + \frac{1}{a_n}} \les \sqrt[n]{a_1 a_2 \cdots a_n} \les \frac{a_1 + a_2 + \cdots + a_n}{n}.$$
              由Stolz定理,有$$\lim_{n\to\infty} \frac{n}{\frac{1}{a_1} + \frac{1}{a_2} + \cdots + \frac{1}{a_n}} = \lim_{n\to\infty} \frac{1}{\frac{1}{a_n}} = a,$$且$$\lim_{n\to\infty} \frac{a_1 + a_2 + \cdots + a_n}{n} = \lim_{n\to\infty} a_n = a.$$
              由夹逼定理,得证.
    \end{enumerate}

    回到本题,

    可知$\lim_{n\to\infty} \sqrt[n]{\frac{a_{n}}{a_{n-1}} \cdot \frac{a_{n-1}}{a_{n-2}} \cdots \frac{a_2}{a_1}} = \lim_{n\to\infty} \frac{a_n}{a_{n-1}} = \frac{1}{l} < 1$.因此$\exists r = \frac{1 + \frac{1}{l}}{2} \in (0,1)$,使得当$n$充分大时,$\sqrt[n]{\frac{a_{n}}{a_{n-1}} \cdot \frac{a_{n-1}}{a_{n-2}} \cdots \frac{a_2}{a_1}} < r$.由此可知,$$\frac{a_n}{a_{n-1}} \cdot \frac{a_{n-1}}{a_{n-2}} \cdots \frac{a_2}{a_1} < r^n,$$即$a_n < a_1 r^n.$因此$\lim_{n\to\infty} a_n = 0$.
\end{solution}

\begin{exercise}[1.2.21]
    设数列 $\{a_n\}$, $\{b_n\}$ 是正数列, 满足 $\frac{a_{n+1}}{a_n} \les \frac{b_{n+1}}{b_n}, \ n=1, 2, \ldots$. 求证: 若 $\{b_n\}$ 收敛, 则 $\{a_n\}$ 收敛.
\end{exercise}

\begin{solution}
    若$\lim_{n\to\infty} b_n = 0$,则由$a_n = a_1 \cdot \frac{a_2}{a_1} \cdot \frac{a_3}{a_2} \cdots \frac{a_n}{a_{n-1}} \les a_1 \cdot \frac{b_2}{b_1} \cdot \frac{b_3}{b_2} \cdots \frac{b_n}{b_{n-1}} = a_1 \cdot \frac{b_n}{b_1}$可知$\lim_{n\to\infty} a_n = 0$.
    若$\lim_{n\to\infty} b_n = b > 0$,由原式有$\frac{a_{n+1}}{b_{n+1}} \les \frac{a_n}{b_n}$,因此$\left\{ \frac{a_n}{b_n} \right\}$单调减,且$\frac{a_n}{b_n} > 0$,因此$\left\{ \frac{a_n}{b_n} \right\}$收敛,设$\lim_{n\to\infty} \frac{a_n}{b_n} = c$,则$\lim_{n\to\infty} a_n = \lim_{n\to\infty} b_n \cdot \lim_{n\to\infty} \frac{a_n}{b_n} = bc$.
\end{solution}

\begin{exercise}[1.2.22]
    利用极限 $\lim_{n\to\infty} \left(1 + \frac{1}{n}\right)^n = \e$, 求下列数列的极限:
    \begin{tasks}[label=(\arabic*)](2)
        \task $a_n = \left(1 + \frac{1}{2n+1}\right)^{2n+1}$;
        \task $a_n = \left(1 - \frac{1}{n-2}\right)^{n+1}$;
        \task $a_n = \left(\frac{1+n}{2+n}\right)^{n}$;
        \task $a_n = \left(1 + \frac{1}{n^3}\right)^{2n^3}$.
    \end{tasks}
\end{exercise}

简要说明:由$\lim_{n\to\infty} \left(1 + \frac{1}{n}\right)^n = \e$,故$\left\{ \left(1 + \frac{1}{n}\right)^n \right\}$的任意子列$\left\{ \left(1 + \frac{1}{n_k}\right)^{n_k} \right\}$也收敛于$\e$.因此,我们可以通过适当的变形,将题目中的数列变形为$\left(1 + \frac{1}{n_k}\right)^{n_k}$的形式,从而求出极限.对于类似于$\left(1 - \frac{1}{n}\right)^{-n}$的形式,可以考虑先通分再变形去掉指数的负号即可处理.

在此过程中下列命题也相同有用:
\begin{proposition*}
    设数列$\{a_n\}$收敛于$a$, $a_n > 0, a > 0$. $\{b_n\}$收敛于$b$.则$\lim_{n \to \infty} a_n^{b_n} = a^b$.
\end{proposition*}
请注意,这条结论对于$1^ \infty$型是不能直接使用的,即若$a_n \to 1, b_n \to \infty$,则不能直接说$a_n^{b_n} \to 1^\infty = 1$.但是对于$a_n \to a > 1, b_n \to \infty$,则可以直接说$a_n^{b_n} \to a^{+ \infty} = + \infty$;对于$a_n \to a < 1, b_n \to + \infty$,则可以直接说$a_n^{b_n} \to a^{+ \infty} = 0$.

\begin{solution}
    \begin{enumerate}[(1)]
        \item $\lim_{n\to\infty} \left(1 + \frac{1}{2n+1}\right)^{2n+1} = \left. \lim_{m\to\infty} \left(1 + \frac{1}{m}\right)^m \right|_{m=2n+1} = \e$;
        \item $\lim_{n\to\infty} \left(1 - \frac{1}{n-2}\right)^{n+1}  = \lim_{n\to\infty} \left(1+\frac{1}{n-3}\right)^{-n-1} =\lim_{n\to\infty} \left(1+\frac{1}{n-3}\right)^{(n-3)\cdot\left(-\frac{n+1}{n-3}\right)} =\e^{-1} $;
        \item $\lim_{n\to\infty} \left(\frac{1+n}{2+n}\right)^{n}=\lim_{n\to\infty} \left(1 + \frac{1}{n+1} \right)^{-n}=\lim_{n\to\infty} \left(1 + \frac{1}{n+1} \right)^{(n+1)\cdot\left(-\frac{n}{n+1}\right)} = \e^{-1}$;
        \item $\lim_{n\to\infty} \left(1 + \frac{1}{n^3}\right)^{2n^3}=\lim_{n\to\infty} \left(1 + \frac{1}{n^3}\right)^{n^3 \cdot 2} = \e^2$.
    \end{enumerate}
\end{solution}

\begin{exercise}[1.2.23]
    设 $\lim_{n\to\infty} a_n = \infty$, 且 $|b_n| \ges b > 0 \ (n=1, 2, \ldots)$, 则 $\lim_{n\to\infty} a_n b_n = \infty$.
\end{exercise}

\begin{solution}
    对$\forall M > 0$,由$\lim_{n\to\infty} a_n = \infty$,存在$N\in \mathbb{N}^*$,使得当$n>N$时,$|a_n |> \frac{M}{b}$.又由$|b_n| \ges b > 0$,可知$|a_n b_n| \ges |a_n| |b| > M$.因此$\lim_{n\to\infty} a_n b_n = \infty$.
\end{solution}

\begin{exercise}[1.2.24]
    确定 $n \to \infty$ 时, $\sqrt[n]{n!}$ 与 $n \sin \frac{n \pi}{2} \ (n \ges 1)$ 是否有界, 是否趋于无穷大.
\end{exercise}

\begin{solution}
    $\sqrt[n]{n!}$无界,且趋于无穷大.由均值不等式,$$\sqrt[n]{n!} \ges \frac{n}{\frac{1}{1} + \frac{1}{2} + \cdots + \frac{1}{n}}$$
    已知$\lim_{n \to \infty}  \frac{1}{\frac{1}{n}}  = +\infty$,由Stolz定理,
    $$\lim_{n \to \infty} \frac{n}{\frac{1}{1} + \frac{1}{2} + \cdots + \frac{1}{n}} 
        = \lim_{n \to \infty} \frac{1}{\frac1{n}} = +\infty.$$
    因此,$\lim_{n \to \infty} \sqrt[n]{n!} = +\infty$.

    \begin{remark}
        Stolz定理规范的思路要先说明$\lim_{n \to \infty} \frac{a_{n+1} - a_n}{b_{n+1} - b_n}$存在,然后才能说明$\lim_{n \to \infty} \frac{a_n}{b_n}$存在.为了方便,我们也会省去前面的部分,直接写$\lim_{n \to \infty} \frac{a_n}{b_n} = \lim_{n \to \infty} \frac{a_{n+1} - a_n}{b_{n+1} - b_n}$.
    \end{remark}

    $n \sin \frac{n \pi}{2}$无界,但是不趋于无穷大.当$n=4k+1$时,$n \sin \frac{n \pi}{2} = 4k+1$,趋于无穷大;当$n=4k+3$时,$n \sin \frac{n \pi}{2} = -(4k+3)$,趋于负无穷大;当$n$为偶数时,$n \sin \frac{n \pi}{2} = 0$.
\end{solution}

\begin{exercise}[1.2.25]
    设数列 $\{a_n\}$ 由 $a_1 = 1, a_{n+1} = a_n + \frac{1}{a_n} \ (n \ges 1)$ 定义, 证明: $a_n \to +\infty \ (n \to \infty)$.
\end{exercise}

\begin{solution}
    由$a_{n+1}^2 - a_n^2 = (a_{n+1} - a_n)(a_{n+1} + a_n) = \frac{1}{a_n}(a_n + a_n + \frac{1}{a_n}) = 2 + \frac{1}{a_n^2} > 2$,可知$a_n^2 > 2(n-1)$,因此$\lim_{n\to\infty} a_n = \infty$.
\end{solution}

\begin{exercise}[1.2.26]
    给出 $\frac{0}{0}$ 型 Stolz 定理的证明.
\end{exercise}

\begin{proposition*}[$\frac{0}{0}$ 型的 Stolz 定理]
    设 $\{a_n\}$ 和 $\{b_n\}$ 都是无穷小量, 其中 $\{a_n\}$ 还是严格单调减少数列, 又存在 (其中 $l$ 为有限或 $\pm\infty$)
    $$ \lim_{n \to \infty} \frac{b_{n+1}-b_n}{a_{n+1}-a_n} = l, $$
    则有
    $$ \lim_{n \to \infty} \frac{b_n}{a_n} = l. $$
\end{proposition*}

\begin{proof}
    \begin{enumerate}[(1)]
        \item 当$l$为有限值时,根据条件对 $\varepsilon > 0$ 存在 $N$, 使当 $n > N$ 时成立
              $$ \left|\frac{b_n-b_{n+1}}{a_n-a_{n+1}} - l \right| < \varepsilon. $$
              由于对每个 $n$ 都有 $a_n > a_{n+1}$, 这样就有
              $$ (l-\varepsilon)(a_n-a_{n+1}) < b_n - b_{n+1} < (l+\varepsilon)(a_n-a_{n+1}). $$
              任取 $m>n$, 并且将上述不等式中的 $n$ 换成 $n+1, \dots, \text{直到 } m-1$, 然后将所有这些不等式相加, 就得到
              $$ (l-\varepsilon)(a_n-a_m) < b_n - b_m < (l+\varepsilon)(a_n-a_m), $$
              以及
              $$ \left|\frac{b_n-b_m}{a_n-a_m} - l \right| < \varepsilon. $$
              令 $m \to \infty$, 并利用条件 $\lim_{m \to \infty} a_m = \lim_{m \to \infty} b_m = 0$, 就知道当 $n>N$ 时成立
              $$ \left|\frac{b_n}{a_n} - l \right| \les \varepsilon. $$
        \item $l = +\infty$ 时. 根据条件对任意 $M > 0$ 存在 $N$, 使当 $n > N$ 时成立
              $$ \frac{b_n-b_{n+1}}{a_n-a_{n+1}} > M. $$
              由于对每个 $n$ 都有 $a_n > a_{n+1}$, 这样就有
              $$ b_n - b_{n+1} > M(a_n-a_{n+1}). $$
              任取 $m>n$, 并且将上述不等式中的 $n$ 换成 $n+1, \dots, \text{直到 } m-1$, 然后将所有这些不等式相加, 就得到
              $$ b_n - b_m > M(a_n-a_m), $$
              以及
              $$ \frac{b_n-b_m}{a_n-a_m} > M. $$
              令 $m \to \infty$, 并利用条件 $\lim_{m \to \infty} a_m = \lim_{m \to \infty} b_m = 0$, 就知道当 $n>N$ 时成立
              $$ \frac{b_n}{a_n} > M.  $$
    \end{enumerate}
\end{proof}


\newpage
\section*{习题 1.3}

\begin{exercise}[1.3.1]\label{exe:1.3.1}
    按定义证明:
    \begin{tasks}[label=(\arabic*)](2)
        \task $\lim_{x \to -\infty} a^x = 0, (a > 1)$;
        \task $\lim_{x \to \infty} \frac{x-1}{x+1} = 1$;
        \task $\lim_{x \to -1} \frac{x^2-1}{x^2+x} = 2$;
        \task $\lim_{x \to 0^+} x^{1/q} = 0$ ($q$ 为正整数).
    \end{tasks}
\end{exercise}

\begin{solution}
    \begin{enumerate}[(1)]
        \item 对$\forall \ve > 0$,取$M = \log_a \ve$,则当$x < M$时,$|a^x - 0| = a^x < a^M = \ve$.
        \item 对$\forall \ve > 0$,取$M = \frac{2}{\ve}+1$,则当$|x| > \max\{M,1\}$时,$\left| \frac{x-1}{x+1} - 1 \right| = \left| \frac{-2}{x+1} \right| \les \frac{2}{|x|-1} <  \ve$.
        \item 对$\forall \ve > 0$,取$\delta = \min \left\{ \frac{1}{2}, \frac{\ve}{2} \right\}$,则当$0<|x+1| < \delta$时,$\left| \frac{x^2-1}{x^2+x} - 2 \right| = \left| \frac{-x^2 - 2x - 1}{x^2 + x} \right| = \left| \frac{x+1}{x} \right| < \frac{\delta}{1 / 2} \les  \ve$.
        \item 对$\forall \ve > 0$,取$\delta = \ve^q$,则当$0 < x < \delta$时,$|x^{1/q} - 0| = x^{1/q} < \delta^{1/q} = \ve$.
    \end{enumerate}
\end{solution}

\begin{exercise}[1.3.2]
    求下列极限:
    \begin{tasks}[label=(\arabic*)](2)
        \task $\lim_{x \to 1} \left(x^5 - 5x + 2 + \frac{1}{x}\right)$;
        \task $\lim_{x \to 1} \frac{x^n - 1}{x - 1}$ ($n$ 为正整数);
        \task $\lim_{x \to 1} \frac{x^2-1}{2x^2-x-1}$;
        \task $\lim_{x \to -\infty} \frac{(3x+6)^{70}(8x-5)^{20}}{(5x-1)^{90}}$.
    \end{tasks}
\end{exercise}

\begin{enumerate}[(1)]
    \item 由四则运算的极限可知,$$\lim_{x \to 1} \left(x^5 - 5x + 2 + \frac{1}{x}\right) = 1 - 5 + 2 + 1 = -1.$$
    \item $x^n - 1 = (x-1)(x^{n-1} + x^{n-2} + \cdots + x + 1)$,因此$$\lim_{x \to 1} \frac{x^n - 1}{x - 1} = \lim_{x \to 1} (x^{n-1} + x^{n-2} + \cdots + x + 1) = \sum_{k=0}^{n-1} 1 = n.$$
          请注意,这里$n$是常数,因此可以交换这$n$个极限与求和的顺序.
    \item $$\lim_{x \to 1} \frac{x^2-1}{2x^2-x-1} = \lim_{x \to 1} \frac{(x-1)(x+1)}{(x-1)(2x+1)} = \lim_{x \to 1} \frac{x+1}{2x+1} = \frac{2}{3}.$$
    \item $$\lim_{x \to -\infty} \frac{ \left( 3 + \dfrac{6}{x} \right)^{70} (8 - \dfrac{5}{x})^{20} }{ \left( 5 - \dfrac{1}{x} \right)^{90} } = \frac{ \left( 3 + \displaystyle \lim_{x \to -\infty} \dfrac{6}{x} \right)^{70} \left( 8 - \displaystyle  \lim_{x \to -\infty} \dfrac{5}{x} \right)^{20} }{ \left( 5 - \displaystyle  \lim_{x \to -\infty} \dfrac{1}{x} \right)^{90} } = \frac{3^{70} \cdot 8^{20}}{5^{90}}$$
          事实上,$\frac{3^{70} \cdot 8^{20}}{5^{90}} = \left( \frac{3^7 \cdot 2^6}{5^9} \right)^{10} =0.0000000000035726229189858259136514568727612$

          $46392142557535369616400676018940233797678923776.$
\end{enumerate}

\begin{exercise}[1.3.3]\label{exe:1.3.3}
    证明下列极限不存在:
    \begin{tasks}[label=(\arabic*)](2)
        \task $\lim_{x \to +\infty} \sin x$;
        \task $\lim_{x \to 0} \frac{|x|}{x}$.
    \end{tasks}
\end{exercise}

\begin{enumerate}[(1)]
    \item 用Cauchy收敛原理.对$\ve = \frac{1}{2}$,任取$M > 0$,总
          总存在$k = \llceil M / \pi \rrceil$,使得$x_1 = \left(k+\frac{1}{2}\right)\pi> M, x_2 = (k + 1)\pi > M (k \in \mathbb{N}^*)$,使得$|\sin x_1 - \sin x_2| = 1 > \ve$.因此极限不存在.
    \item 考虑两个单边极限,$$\lim_{x \to 0^+} \frac{|x|}{x} = \lim_{x \to 0^+} 1 = 1, \quad \lim_{x \to 0^-} \frac{|x|}{x} = \lim_{x \to 0^-} -1 = -1.$$
          而极限存在的充要条件是两个单边极限存在且相等,因此极限不存在.
\end{enumerate}

\begin{exercise}[1.3.4]
    设函数 $f(x)$ 在正无穷大处的极限为 $l$, 则对于任意趋于正无穷大的数列 $\{a_n\}$, 有 $\lim_{n \to \infty} f(a_n) = l$. 特别地 $\lim_{n \to \infty} f(n) = l$.
\end{exercise}

\begin{solution}
    [无穷版本的Heine定理 \ 解]对$\forall \ve > 0$,由$\lim_{x \to \infty} f(x) = l$,存在$M > 0$,使得当$x > M$时,$|f(x) - l| < \ve$.又由$\lim_{n \to \infty} a_n = +\infty$,存在$N\in \mathbb{N}^*$,使得当$n > N$时,$a_n > M$.因此当$n > N$时,$|f(a_n) - l| < \ve$.由此可知$\lim_{n \to \infty} f(a_n) = l$.特别地,取$a_n = n$,则$\lim_{n \to \infty} f(n) = l$.
\end{solution}

\begin{exercise}[1.3.5]
    讨论下列函数在 $x=0$ 处的极限.
    \begin{tasks}[label=(\arabic*)](2)
        \task $f(x) = [x]$;
        \task $f(x) = \operatorname{sgn} x$;
        \task $f(x) = \begin{cases} 2^x, & x > 0; \\ 0, & x = 0; \\ 1+x^2, & x < 0. \end{cases}$
        \task $f(x) = \begin{cases} \cos\frac{1}{x}, & x > 0; \\ x, & x \les 0. \end{cases}$
    \end{tasks}
\end{exercise}

\begin{solution}
    \begin{remark}
        教材中的符号$[x]$表示$x$的整数部分,即不大于$x$的最大整数.本题中,我们沿用此符号.其他地方,我们使用$\lfloor x \rfloor$表示对$x$向下取整,使用$\lceil x \rceil$表示对$x$向上取整.
    \end{remark}
    \begin{enumerate}[(1)]
        \item $\lim_{x \to 0^+} [x] = 0, \quad \lim_{x \to 0^-} [x] = -1$.因此极限不存在.
        \item $\lim_{x \to 0^+} \operatorname{sgn} x = 1, \quad \lim_{x \to 0^-} \operatorname{sgn} x = -1$.左右极限均存在,但不相等,因此极限不存在.
        \item $\lim_{x \to 0^+} 2^x = 1, \quad \lim_{x \to 0^-} (1+x^2) = 1$.因此极限存在,且$\lim_{x \to 0} f(x) = 1$.
        \item $\lim_{x \to 0^+} \cos\frac{1}{x}$不存在,因此右极限不存在.左极限$\lim_{x \to 0^-} x = 0$.函数在$x=0$处的极限不存在. \begin{remark}
                  $\lim_{x \to 0^+} \cos \frac{1}{x}$的极限过程等同于考虑$\lim_{x \to + \infty} \cos x$,而该极限不存在(与\hyperref[exe:1.3.3]{习题1.3.3(1)}同理).
              \end{remark}
    \end{enumerate}
\end{solution}

\begin{exercise}[1.3.6]
    求 $\lim_{n \to \infty} \cos\frac{x}{2} \cos\frac{x}{2^2} \cdots \cos\frac{x}{2^n}$.
\end{exercise}

\begin{solution}
    \begin{enumerate}[(1)]
        \item 当$\forall m \in \mathbb{N}^*,\sin \frac{x}{2^m} \ne 0$时,二倍角公式变形可得$\cos y = \frac{\sin 2y}{2 \sin y},\text{当}\sin y\neq 0$,反复利用可知$$\cos\frac{x}{2} \cos\frac{x}{2^2} \cdots \cos\frac{x}{2^n} = \frac{\sin x}{2 \sin \frac{x}{2}} \cdot \frac{\sin \frac{x}{2}}{2 \sin \frac{x}{2^2}} \cdots \frac{\sin \frac{x}{2^{n-1}}}{2 \sin \frac{x}{2^n}} = \frac{\sin x}{2^n \sin \frac{x}{2^n}}.$$
              因此$$\lim_{n \to \infty} \cos\frac{x}{2} \cos\frac{x}{2^2} \cdots \cos\frac{x}{2^n} = \lim_{n \to \infty} \frac{\sin x}{2^n \sin \frac{x}{2^n}} = \frac{\sin x}{x} \cdot \lim_{n \to \infty} \frac{\frac{x}{2^n}}{\sin\frac{x}{2^n}} = \frac{\sin x}{x}.$$
        \item 若存在$m_0 \ges 1$, $\sin \frac{x}{2^{m}} = 0$,有$\frac{x}{2^{m_0}}= k\pi,x = 2^{m_0}k \pi, k \in \mathbb{Z}$. 
        自然的推论是$\forall m\les m_0$,有$ \sin \frac{x}{2^m}=\sin (2^{m_0-m}k \pi)  =0$.
        
        此时根据是否存在最大的$m_0$,使得$\sin \frac{x}{2^{m_0}} = 0$可以分成两种情况:
              \begin{enumerate}
                  \item $x = 0$,则$\forall m\in \mathbb{N}^*$,有$\cos\frac{x}{2^m}= 1$,因此$\lim_{n \to \infty} \cos\frac{x}{2} \cos\frac{x}{2^2} \cdots \cos\frac{x}{2^n} = 1$;
                  \item $x \ne 0 \Leftrightarrow \exists m_0,\text{s.t.}  \sin \frac{x}{2^{m_0}} = 0,\sin \frac{x}{2^{m_0+1}} \neq 0,$也就是存在最大的$m_0$.
                  
                  因此可以得到$x = 2^{m_0}k \pi, k =2l+1,l\in\Z  $(如果$k$是偶数,那么与$\sin \frac{x}{2^{m_0+1}}=\sin \frac{k\pi}{2}  \neq 0$矛盾).
                  
                  此时$\cos \frac{x}{2^{m_0+1}} = \cos\frac{k\pi}{2} =\cos\left(l+\frac{1}{2}\right)\pi = 0$,因此$\lim_{n \to \infty} \cos\frac{x}{2} \cos\frac{x}{2^2} \cdots \cos\frac{x}{2^n}=0$.
                  
                  不过又由于$\sin x=0$同样成立,并且$x\neq 0$,因此可以把结果合并进$\frac{\sin x}{x}$.
              \end{enumerate}
    \end{enumerate}
    综上所述,$$\lim_{n \to \infty} \cos\frac{x}{2} \cos\frac{x}{2^2} \cdots \cos\frac{x}{2^n} = \begin{cases}
            \frac{\sin x}{x}, & x \ne 0; \\
            1,                & x = 0.
        \end{cases}
    $$

\end{solution}

\begin{exercise}[1.3.7]
    求证: $\lim_{n \to \infty} \left(\sin\frac{\alpha}{n^2} + \sin\frac{2\alpha}{n^2} + \dots + \sin\frac{n\alpha}{n^2}\right) = \frac{\alpha}{2}$.
\end{exercise}

\begin{solution}
    我们先证明如下事实:
    $$
        \sum_{k=1}^n \sin k \theta =\begin{cases}
            \frac{\sin\frac{n+1}{2}\theta\sin\frac{n}{2}\theta}{\sin \frac{\theta}{2}}, & \sin \frac{\theta}{2} \ne 0; \\
            0, & \sin \frac{\theta}{2} = 0.
        \end{cases}
    $$
    我们利用积化和差$$\sin \frac{\theta}{2}\sin k\theta = \frac{1}{2}\left(\cos\left(k-\frac{1}{2}\right)\theta-\cos\left(k+\frac{1}{2}\right)\theta\right)$$
    \begin{align*}
        &\sin \frac{\theta}{2}\left(\sum_{k=1}^n \sin k \theta \right)\\
        =&\sum_{k=1}^n \frac{1}{2}\left(\cos\left(k-\frac{1}{2}\right)\theta-\cos\left(k+\frac{1}{2}\right)\theta\right) \\
        =&\frac{1}{2}\left(\cos \frac{\theta}{2} -\cos \frac{3\theta}{2}+\cos \frac{3\theta}{2} -\cos \frac{5\theta}{2}+\cdots+ \cos\left(n-\frac{1}{2}\right)\theta-\cos\left(n+\frac{1}{2}\right)\theta\right) \\
        =&\frac{1}{2}\left(\cos \frac{\theta}{2} -\cos\left(n+\frac{1}{2}\right)\theta\right) \\
        =&\sin\frac{n+1}{2}\theta\sin\frac{n}{2}\theta.
    \end{align*}
    因此,当$ \sin \frac{\theta}{2}\neq 0$自然有$$
        \sum_{k=1}^n \sin k \theta =
            \frac{\sin\frac{n+1}{2}\theta\sin\frac{n}{2}\theta}{\sin \frac{\theta}{2}}.
    $$
    另一种情况是显然的,每一项都为$0$.
        
    回到本题,首先,如果$\alpha\neq 0$,那么这意味着存在充分大的$N$使得$n>N$,$0<\left|\frac{\alpha}{n^2}\right|<\pi$,此时,$\sin \frac{\alpha}{2n^2} \neq 0$.因此$n>N$时,
    \begin{align*}
        \sum_{k=1}^n \sin k \frac{\alpha}{n^2} =  \frac{\sin\frac{(n+1)\alpha}{2n^2}\sin\frac{n\alpha}{2n^2}}{\sin \frac{\alpha}{2n^2}}
    \end{align*}
    考虑$\sin x \sim x , (x \to 0)$,于是
    \begin{align*}
        \lim_{n\to\infty}\sum_{k=1}^n \sin k \frac{\alpha}{n^2} 
        =&\lim_{n \to \infty} \frac{\sin\frac{(n+1)\alpha}{2n^2}\sin\frac{n\alpha}{2n^2}}{\sin \frac{\alpha}{2n^2}}\\
        =&\lim_{n \to \infty} \frac{\frac{(n+1)\alpha}{2n^2}
        \cdot\frac{n\alpha}{2n^2}}{\frac{\alpha}{2n^2}}  \\
        =&\lim_{n \to \infty} \frac{(n+1)\alpha}{2n} \\
        =&\frac{\alpha}{2}.
    \end{align*}
    如果$\alpha = 0$,那么每一项都为$0$,极限自然为$0=\frac{\alpha}{2}$.
    
    综上所述,$$\lim_{n \to \infty} \left(\sin\frac{\alpha}{n^2} + \sin\frac{2\alpha}{n^2} + \dots + \sin\frac{n\alpha}{n^2}\right) = \frac{\alpha}{2}.$$

\end{solution}

\begin{exercise}[1.3.8]
    证明: 若 $\lim_{x \to \infty} f(x) = l$, 则 $\lim_{x \to 0} f\left(\frac{1}{x}\right)=l$, 反之亦正确. 叙述并证明, 当 $x \to +\infty$ 及 $x \to -\infty$ 时类似的结论. (应用本题结论, 可将极限过程为 $x \to \infty$ 的问题化为 $x \to 0$ 处理, 或者反过来. 例如, 我们有 $\lim_{x \to 0} (1+x)^{1/x} = \mathrm{e}$.)
\end{exercise}

\begin{solution}
    我们先给出这条命题的完整表述:
    \begin{proposition*}
        \begin{enumerate}[(1)]
            \item 若 $\lim_{x \to \infty} f(x) = l$, 则 $\lim_{x \to 0^+} f\left(\frac{1}{x}\right)=l$, 反之亦正确;
            \item 若 $\lim_{x \to +\infty} f(x) = l$, 则 $\lim_{x \to 0^+} f\left(\frac{1}{x}\right)=l$, 反之亦正确;
            \item 若 $\lim_{x \to -\infty} f(x) = l$, 则 $\lim_{x \to 0^-} f\left(\frac{1}{x}\right)=l$, 反之亦正确;
        \end{enumerate}
    \end{proposition*}
    \noindent
    \textbf{证明:}
    \begin{enumerate}[(1)]
        \item 由Heine定理,$\lim_{x \to \infty} f(x) = l \Rightarrow \forall \{x_n\},$若$\lim_{n \to \infty} x_n = \infty$则$\lim_{n \to \infty} f(x_n) = l$. $\Rightarrow$ $\forall \{ y_n \},$若$\lim_{n \to \infty} y_n = 0^+$,则$\lim_{n \to \infty} \frac{1}{y_n} = \infty$,则$\lim_{n \to \infty} f\left( \frac{1}{y_n} \right) = l$.由Heine定理可知$\lim_{x \to 0^+} f\left(\frac{1}{x}\right)=l$.

              反之,若$\lim_{x \to 0^+} f\left(\frac{1}{x}\right)=l$,由Heine定理,$\forall \{y_n\},$若$\lim_{n \to \infty} y_n = 0^+$,则$\lim_{n \to \infty} f\left( \frac{1}{y_n} \right) = l$. $\Rightarrow$ $\forall \{x_n\},$若$\lim_{n \to \infty} x_n = \infty$,则$\lim_{n \to \infty} \frac{1}{x_n} = 0^+$,则$\lim_{n \to \infty} f(x_n) = l$.由Heine定理可知$\lim_{x \to \infty} f(x) = l$.
        \item 由Heine定理,$\lim_{x \to +\infty} f(x) = l \Rightarrow \forall \{x_n\},$若$\lim_{n \to \infty} x_n = +\infty$则$\lim_{n \to \infty} f(x_n) = l$. $\Rightarrow$ $\forall \{ y_n \},$若$\lim_{n \to \infty} y_n = 0^+$,则$\lim_{n \to \infty} \frac{1}{y_n} = +\infty$,则$\lim_{n \to \infty} f\left( \frac{1}{y_n} \right) = l$.由Heine定理可知$\lim_{x \to 0^+} f\left(\frac{1}{x}\right)=l$.

              反之,若$\lim_{x \to 0^+} f\left(\frac{1}{x}\right)=l$,由Heine定理,$\forall \{y_n\},$若$\lim_{n \to \infty} y_n = 0^+$,则$\lim_{n \to \infty} f\left( \frac{1}{y_n} \right) = l$. $\Rightarrow$ $\forall \{x_n\},$若$\lim_{n \to \infty} x_n = +\infty$,则$\lim_{n \to \infty} \frac{1}{x_n} = 0^+$,则$\lim_{n \to \infty} f(x_n) = l$.由Heine定理可知$\lim_{x \to +\infty} f(x) = l$.

        \item 由Heine定理,$\lim_{x \to -\infty} f(x) = l \Rightarrow \forall \{x_n\},$若$\lim_{n \to \infty} x_n = -\infty$则$\lim_{n \to \infty} f(x_n) = l$. $\Rightarrow$ $\forall \{ y_n \},$若$\lim_{n \to \infty} y_n = 0^-$,则$\lim_{n \to \infty} \frac{1}{y_n} = -\infty$,则$\lim_{n \to \infty} f\left( \frac{1}{y_n} \right) = l$.由Heine定理可知$\lim_{x \to 0^-} f\left(\frac{1}{x}\right)=l$.

              反之,若$\lim_{x \to 0^-} f\left(\frac{1}{x}\right)=l$,由Heine定理,$\forall \{y_n\},$若$\lim_{n \to \infty} y_n = 0^-$,则$\lim_{n \to \infty} f\left( \frac{1}{y_n} \right) = l$. $\Rightarrow$ $\forall \{x_n\},$若$\lim_{n \to \infty} x_n = -\infty$,则$\lim_{n \to \infty} \frac{1}{x_n} = 0^-$,则$\lim_{n \to \infty} f(x_n) = l$.由Heine定理可知$\lim_{x \to -\infty} f(x) = l$.
    \end{enumerate}
\end{solution}

\begin{exercise}[1.3.9]
    求下列极限:
    \begin{tasks}[label=(\arabic*)](2)
        \task $\lim_{x \to 0} \frac{\tan 2x}{\sin 5x}$;
        \task $\lim_{x \to 0} \frac{\cos x - \cos 3x}{x^2}$;
        \task $\lim_{x \to +\infty} \left(\frac{x+1}{2x-1}\right)^x$;
        \task $\lim_{x \to \infty} \left(\frac{x^2+1}{x^2-1}\right)^{x^2}$.
    \end{tasks}
\end{exercise}

\begin{solution}
    \begin{enumerate}[(1)]
        \item $$\lim_{x \to 0} \frac{\tan 2x}{\sin 5x} = \lim_{x \to 0} \frac{\tan 2x}{2x} \cdot \frac{5x}{\sin 5x} \cdot \frac{2}{5} = 1 \cdot 1 \cdot \frac{2}{5} = \frac{2}{5}.$$
        \item 由和差化积,$$ \cos x - \cos 3x = 2 \sin 2x \sin x,$$因此$$\lim_{x \to 0} \frac{\cos x - \cos 3x}{x^2} = \lim_{x \to 0} \frac{2 \sin 2x \sin x}{x^2} = \lim_{x \to 0} \frac{\sin 2x}{2x} \cdot \frac{\sin x}{x} \cdot 4 = 1 \cdot 1 \cdot 4 = 4.$$
        \item 当$x>\frac{7}{2}$时,有$0<\frac{x+1}{2x-1}<\frac{3}{4}$恒成立,因此$$0\les\left(\frac{x+1}{2x-1}\right)^x\les\left(\frac{3}{4}\right)^x$$
        又由于$\lim_{x \to +\infty} \left(\frac{3}{4}\right)^x = 0$,由夹逼定理可知$$\lim_{x \to +\infty} \left(\frac{x+1}{2x-1}\right)^x = 0.$$
        \item $$
                  \lim_{x \to \infty} \left(1 + \frac{2}{x^2 - 1}\right)^{\frac{x^2 - 1}{2} \cdot \frac{2}{x^2 - 1} x^2} = \mathrm{e}^{\lim_{x \to \infty} \frac{2}{x^2 - 1} x^2} = \mathrm{e}^2
              $$
    \end{enumerate}
\end{solution}

\begin{exercise}[1.3.10]
    求下列极限.
    \begin{tasks}[label=(\arabic*)](2)
        \task $\lim_{x \to +\infty} \frac{\arctan x}{x}$;
        \task $\lim_{x \to 0} x^2 \sin\frac{1}{x}$;
        \task $\lim_{x \to 2} \frac{x^3-2x^2}{x-2}$;
        \task $\lim_{x \to \infty} (2x^2 - x + 1)$.
    \end{tasks}
\end{exercise}

\begin{solution}
    \begin{enumerate}[(1)]
        \item $\arctan x$在$x \to +\infty$时有界,而$x \to +\infty$时无界,因此$$\lim_{x \to +\infty} \frac{\arctan x}{x} = 0.$$具体而言,$$\lim_{x \to + \infty} \arctan x = \frac{\pi}{2} , \ \lim_{x \to -\infty} \arctan x = -\frac{\pi}{2}.$$
        \item 由夹逼定理,$$-x^2 \les x^2 \sin \frac{1}{x} \les x^2,$$且$\lim_{x \to 0} -x^2 = \lim_{x \to 0} x^2 = 0$,因此$$\lim_{x \to 0} x^2 \sin\frac{1}{x} = 0.$$
        \item $$\lim_{x \to 2} \frac{x^3-2x^2}{x-2} = \lim_{x \to 2} \frac{(x-2)x^2}{x-2} = \lim_{x \to 2}x^2  = 4.$$
        \item 证明: $$\lim_{x \to \infty} (2x^2 - x + 1) = +\infty.$$
              由$2x^2 - x + 1 = x^2 + (x - 1/2)^2 + 3/4 > x^2$,因此对$\forall M > 0$,取$N = \sqrt{M}$,则当$x > N$时,$2x^2 - x + 1 > x^2 > N^2 = M$.由此可知$$\lim_{x \to \infty} (2x^2 - x + 1) = +\infty.$$
    \end{enumerate}
\end{solution}

\begin{exercise}[1.3.11]
    按定义证明.
    \begin{tasks}[label=(\arabic*)](2)
        \task $\lim_{x \to +\infty} \log_a x = +\infty, (a > 1)$;
        \task $\lim_{x \to 0^+} \log_a x = -\infty, (a > 1)$;
        \task $\lim_{x \to \frac{\pi}{2}{}^- }\tan x = +\infty$;
        \task $\lim_{x \to 0^+} \mathrm{e}^{1/x} = +\infty$.
    \end{tasks}
\end{exercise}

\begin{solution}
    \begin{enumerate}[(1)]
        \item 对$\forall M > 0$,取$N = a^M$,则当$x > N$时,$\log_a x > \log_a N = M$.
        \item 对$\forall M < 0$,取$\delta = a^M$,则当$0 < x < \delta$时,$\log_a x < \log_a \delta = M$.
        \item 对$\forall M > 0$,取$\delta = \frac{\pi}{2} - \arctan M$,则当$\frac{\pi}{2} - \delta < x < \frac{\pi}{2}$时,$\tan x > \tan(\frac{\pi}{2} - \delta) = M$.
        \item 对$\forall M > 0$,取$\delta = \frac{1}{\ln M}$,则当$0 < x < \delta$时,$\mathrm{e}^{1/x} > \mathrm{e}^{1/\delta} = M$.
    \end{enumerate}
\end{solution}

\begin{exercise}[1.3.12]\label{exe:1.3.12}
    证明: 函数 $y=x\sin x$ 在 $(0, +\infty)$ 内无界, 但当 $x \to +\infty$ 时, 这个函数并不是无穷大量.
\end{exercise}

\begin{solution}
    $\forall M > 0$,存在$x_0 = (2k-1) \pi, k \in \mathbb{N}^*, 2k - 1 > M$,因此$y(x_0) = x_0 \sin x_0 = x_0 > M$.由此可知$y = x \sin x$在$(0, +\infty)$内无界.

    $\forall X > 0$,总存在$x_1 = 2k \pi, k \in \mathbb{N}^*, 2k \pi > X$,使得$y(x_1) = x_1 \sin x_1 = 0$.因此当$x \to +\infty$时,$y = x \sin x$并不是无穷大量.
\end{solution}

\begin{exercise}[1.3.13]
    函数 $y = \frac{1}{x}\cos\frac{1}{x}$ 在区间 $(0,1)$ 内是否有界? 又当 $x \to 0^+$ 时, 这个函数是否为无穷大量?
\end{exercise}

\begin{solution}
    考虑$0^+$处的$\frac{1}{x} \cos \frac{1}{x}$与考虑$+\infty$处的$x \cos x$是等价的.以与\hyperref[exe:1.3.12]{习题1.3.12}类似的方法可知,$y = x \cos x$在$(0, +\infty)$内无界,但当$x \to +\infty$时,$y = x \cos x$并不是无穷大量.因此,$y = \frac{1}{x} \cos \frac{1}{x}$在$(0, 1)$内无界,但当$x \to 0^+$时,$y = \frac{1}{x} \cos \frac{1}{x}$并不是无穷大量.
\end{solution}

\begin{exercise}[1.3.14]
    本题所涉及的函数极限有着鲜明的几何意义.

    记函数 $y=f(x)$ 所表示的曲线为 $C$. 若动点沿曲线无限远离原点时, 此动点与某一固定直线的距离趋于零, 则称该直线为曲线 $C$ 的一条渐近线.
    \begin{enumerate}[label=(\roman*)]
        \item \textit{垂直渐近线} \quad 易知(垂直于 $x$ 轴的)直线 $x=x_0$ 为曲线 $C$ 的渐近线的充分必要条件是
              $$ \lim_{x \to x_0^-} f(x) = \infty \quad \text{或} \quad \lim_{x \to x_0^+} f(x) = \infty. $$

        \item \textit{水平渐近线} \quad 易知(平行于 $x$ 轴的)直线 $y=b$ 为曲线 $C$ 的渐近线的充分必要条件是
              $$ \lim_{x \to +\infty} f(x) = b \quad \text{或} \quad \lim_{x \to -\infty} f(x) = b. $$

        \item \textit{斜渐近线} \quad 请读者证明, 方程为 $y=ax+b \, (a \ne 0)$ 的直线 $L$ 为曲线 $C$ 的渐近线的充分必要条件是
              $$ a = \lim_{x \to +\infty} \frac{f(x)}{x}, \quad b = \lim_{x \to +\infty} (f(x)-ax); $$
              或者
              $$ a = \lim_{x \to -\infty} \frac{f(x)}{x}, \quad b = \lim_{x \to -\infty} (f(x)-ax). $$
    \end{enumerate}
    这里自然要假定所说的极限都存在. (提示: 以 $x \to +\infty$ 为例, 设曲线 $C$ 及直线 $L$ 上的横坐标为 $x$ 的点分别为 $M,N$. 则 $M$ 至 $L$ 的距离, 是 $|MN|$ 的一个常数倍. 因此, 直线 $L$ 为曲线 $C$ 的渐近线, 等价于 $\lim_{x \to +\infty} (f(x) - (ax+b)) = 0$, 由此易得所说结果.)

    求下列曲线的渐近方程.
    \begin{tasks}[label=(\arabic*)](2)
        \task $y=x\ln\left(\mathrm{e} + \frac{1}{x}\right)$;
        \task $y=\frac{3x^2-2x+3}{x-1}$.
    \end{tasks}
\end{exercise}

\begin{solution}
    先证明,仅证明$+\infty$,另一种同理.正如提示所说,由于距离$d=\frac{f(x)-(ax+b)}{\sqrt{a^2+1}}$,因此$l$是渐近线,等价于$x\to+\infty$时$d$趋于$0$,等价于$f(x)-(ax+b)$趋于$0$.
    
    然后问题转化为了证明$$\lim_{x \to +\infty} (f(x) - (ax+b)) = 0 \Leftrightarrow a = \lim_{x \to +\infty} \frac{f(x)}{x}, \quad b = \lim_{x \to +\infty} (f(x)-ax).$$

    充分性:由$b = \lim_{x \to +\infty} (f(x)-ax)$可知,$$\lim_{x \to +\infty} (f(x) - (ax+b)) = 0.$$

    必要性:由$\lim_{x \to +\infty} (f(x) - (ax+b)) = 0$可知,$$\lim_{x \to +\infty} \frac{f(x)}{x} - a - \frac{b}{x} = 0, \quad \lim_{x \to +\infty} (f(x)-ax) = b$$因此$$\lim_{x \to +\infty} \frac{f(x)}{x} = a, \quad \lim_{x \to +\infty} (f(x)-ax) = b.$$



    \begin{enumerate}[(1)]
        \item
              \begin{enumerate}
                  \item 垂直渐近线, $x=- \frac{1}{\e}$: $\lim_{x \to (-\frac{1}{\e})^-} = -\frac{1}{\e} \lim_{y \to 0^+} \ln y = +\infty$;
                  \item 斜渐近线, $y=x + \frac{1}{\e}$: $\lim_{x \to \infty} \frac{y(x)}{x} = \lim_{x \to \infty} \ln(\e + \frac{1}{x}) = 1$, $\lim_{x \to  \infty} (y(x) - x) = \lim_{x \to \infty} x(\ln(\e + \frac{1}{x}) - 1) = \lim_{x \to \infty} x\ln(1 + \frac{1}{\e x}) = \lim_{x \to \infty} \frac{\ln(1 + \frac{1}{\e x})}{\frac{1}{x}} = \lim_{x \to  \infty} \frac{1 / \e x}{1 / x} = \frac{1}{\e}$($\pm\infty$两侧是同一条渐近线);
              \end{enumerate}
        \item
              \begin{enumerate}
                  \item 垂直渐近线,$x=1$: $\lim_{x \to 1} y(x) = \lim_{x \to 1} \frac{3x^2 - 2x + 3}{x-1} = \infty$;
                  \item 斜渐近线, $y=3x+1$:$\lim_{x \to \infty} \frac{y(x)}{x} = \lim_{x \to  \infty} \frac{3 - \frac{2}{x} + \frac{3}{x^2}}{1 - \frac{1}{x}} = 3$, $\lim_{x \to + \infty} (y(x) - 3x) = \lim_{x \to \infty} \frac{x+3}{x-1} = 1$;
              \end{enumerate}
    \end{enumerate}
\end{solution}

\begin{exercise}[1.3.15]
    证明: 在同一极限过程中等价的无穷小量有下列性质:
    \begin{enumerate}
        \item $\alpha(x) \sim \alpha(x)$ (自反性);
        \item 若 $\alpha(x) \sim \beta(x)$, 则 $\beta(x) \sim \alpha(x)$ (对称性);
        \item 若 $\alpha(x) \sim \beta(x)$, $\beta(x) \sim \gamma(x)$, 则 $\alpha(x) \sim \gamma(x)$ (传递性).
    \end{enumerate}
    (注意, (1) 中自然需假定 $\alpha(x)$ 不取零值; 而在 (2)、(3) 中, 条件蕴含着, 所说的无穷小量在极限过程中均不取零值.)
\end{exercise}

\begin{solution}
    解释一下,这里说的是(1)需要没有$\alpha(x)\equiv 0$这种情况.(2)(3)因为有"若xxx"的假设自然排除了这种情况.
    \begin{enumerate}[(1)]
        \item 显然, $\lim \frac{\alpha(x)}{\alpha(x)} = 1$,因此$\alpha(x) \sim \alpha(x)$.
        \item 由$\alpha(x) \sim \beta(x)$可知,$\lim \frac{\alpha(x)}{\beta(x)} = 1$,因此$\lim \frac{\beta(x)}{\alpha(x)} = 1$,即$\beta(x) \sim \alpha(x)$.
        \item 由$\alpha(x) \sim \beta(x)$, $\beta(x) \sim \gamma(x)$可知,$\lim \frac{\alpha(x)}{\beta(x)} = 1$, $\lim \frac{\beta(x)}{\gamma(x)} = 1$,因此$\lim \frac{\alpha(x)}{\gamma(x)} = \lim \frac{\alpha(x)}{\beta(x)} \cdot \lim \frac{\beta(x)}{\gamma(x)} = 1$,即$\alpha(x) \sim \gamma(x)$.
    \end{enumerate}
\end{solution}

\begin{exercise}[1.3.16]
    当 $x \to 0$ 时, 比较下列无穷小的阶:
    \begin{tasks}[label=(\arabic*)](2)
        \task $\tan x - \sin x$ 与 $x^3$;
        \task $x^3+x^2$ 与 $\sin^2 x$;
        \task $1-\cos x$ 与 $x^2$.
    \end{tasks}
\end{exercise}

\begin{solution}
    \begin{enumerate}[(1)]
        \item $$\tan x - \sin x = \frac{\sin x}{\cos x} - \sin x = \sin x \left( \frac{1}{\cos x} - 1 \right) = \sin x \cdot \frac{1 - \cos x}{\cos x}.$$
              由$\sin x \sim x, 1 - \cos x \sim \frac{x^2}{2}, \cos x \sim 1$,可知$$\tan x - \sin x \sim x \cdot \frac{x^2 / 2}{1} = \frac{x^3}{2}.$$
              因此,$$\tan x - \sin x \sim \frac{1}{2} x^3.$$
        \item $$\sin^2 x = (\sin x)^2 \sim x^2  (x \to 0).$$
              同时,$$x^3 + x^2=(x+1)x^2\sim x^2   (x \to 0).$$
              可得$$x^3 + x^2\sim \sin^2 x$$
        \item $$1 - \cos x = 2 \sin^2 \frac{x}{2} \sim 2 \cdot \left( \frac{x}{2} \right)^2 = \frac{x^2}{2} (x \to 0).$$
    \end{enumerate}
\end{solution}

\begin{exercise}[1.3.17]
    当 $x \to +\infty$ 时, 试比较下列无穷大量的阶:
    \begin{tasks}[label=(\arabic*)](2)
        \task $n$ 次多项式 $P_n(x)$ 与 $m$ 次多项式 $P_m(x)$ ($m,n$ 均为正整数);
        \task $x^\alpha$ 与 $x^\beta$ ($\alpha, \beta > 0$);
        \task $a^x$ 与 $b^x$ ($a, b > 1$).
    \end{tasks}
\end{exercise}
\begin{solution}
    \begin{enumerate}
        \item 
        $\lim_{x \to +\infty} \frac{P_n(x)}{P_m(x)} = \lim_{x \to +\infty} \frac{a_n x^n +a_{n-1} x^{n-1} \cdots}{b_m x^m +b_{m-1} x^{m-1} +\cdots} 
        = \lim_{x \to +\infty} \frac{x^n}{x^m}\frac{a_n  +a_{n-1} \frac{1}{x} +\cdots}{b_m + b_{m-1}\frac{1}{x} +\cdots}$ 
        
        $=\frac{a_n  }{b_m}\lim_{x \to +\infty}x^{n-m}=\begin{cases}
                    \frac{a_n}{b_m}, & n = m; \\
                    0, & n < m; \\
                    +\infty, & n > m.
                \end{cases} $,即得到$\begin{cases}
                    P_n(x) \sim P_m(x), & n = m; \\
                    P_m(x) \text{更高阶}, & n < m; \\
                    P_n(x) \text{更高阶}, & n > m.
                \end{cases} $.
        \item     利用$\lim_{x \to +\infty} \frac{x^\alpha}{x^\beta} = \lim_{x \to +\infty} x^{\alpha - \beta} = \begin{cases}
                        1, & \alpha = \beta; \\
                        0, & \alpha < \beta; \\
                        +\infty, & \alpha > \beta.
                    \end{cases} $可得$\begin{cases}
                    x^\alpha \sim x^\beta, & \alpha = \beta; \\
                    x^\beta \text{更高阶}, & \alpha < \beta; \\
                    x^\alpha \text{更高阶}, & \alpha > \beta.
                \end{cases}$
        \item 利用$\lim_{x \to +\infty} \frac{a^x}{b^x} = \lim_{x \to +\infty} \left( \frac{a}{b} \right)^x = \begin{cases}
                        1, & a = b; \\
                        0, & a < b; \\
                        +\infty, & a > b.
                    \end{cases} $可得$\begin{cases}
                    a^x \sim b^x, & a = b; \\
                    b^x \text{更高阶}, & a < b; \\
                    a^x \text{更高阶}, & a > b.
                \end{cases}$

    \end{enumerate}
\end{solution}

\begin{exercise}[1.3.18]
    试用等价无穷小量代换的方法计算下列极限.
    \begin{tasks}[label=(\arabic*)](2)
        \task $\lim_{x \to 0} \frac{\sin mx}{\sin nx}$ ($m, n$ 均为正整数);
        \task $\lim_{x \to 0} \frac{\tan ax}{x}$;
        \task $\lim_{x \to 0} \frac{\sqrt[n]{1+\sin x}-1}{\arctan x}$;
        \task $\lim_{x \to 0} \frac{\sqrt{2}-\sqrt{1+\cos x}}{\sin^2 x}$;
        \task $\lim_{x \to 0} \frac{\sqrt{1+x+x^2}-1}{\sin 2x}$;
        \task $\lim_{x \to 0} \frac{\sqrt{1+x^2}-1}{1-\cos x}$.
    \end{tasks}
\end{exercise}

\begin{solution}
    \begin{enumerate}[(1)]
        \item 由$\sin x \sim x$,可知$$\lim_{x \to 0} \frac{\sin mx}{\sin nx} = \lim_{x \to 0} \frac{mx}{nx} = \frac{m}{n}.$$
        \item 由$\tan x \sim x$,可知$a\neq0$时,$$\lim_{x \to 0} \frac{\tan ax}{x} = \lim_{x \to 0} \frac{ax}{x} = a.$$
        很显然该结果对$a=0$也成立.
        \item 由$(1+x)^\alpha - 1 \sim \alpha x$, $\arctan x \sim x$,可知$$\lim_{x \to 0} \frac{\sqrt[n]{1+\sin x}-1}{\arctan x} = \lim_{x \to 0} \frac{\frac{1}{n} \sin x}{x} = \frac{1}{n}.$$
        \item $$\frac{\sqrt{2}-\sqrt{1+\cos x}}{\sin^2 x} = \frac{(\sqrt{2}-\sqrt{1+\cos x})(\sqrt{2}+\sqrt{1+\cos x})}{\sin^2 x (\sqrt{2}+\sqrt{1+\cos x})} = \frac{1 - \cos x}{\sin^2 x (\sqrt{2}+\sqrt{1+\cos x})}.$$
              由$1 - \cos x \sim \frac{x^2}{2}, \sin x \sim x$,可知$$\lim_{x \to 0} \frac{\sqrt{2}-\sqrt{1+\cos x}}{\sin^2 x} = \lim_{x \to 0} \frac{\frac{x^2}{2}}{x^2 (\sqrt{2} + \sqrt{1 + \cos x})} = \frac{1}{4\sqrt{2}}.$$
        \item 由$(1+x)^\alpha - 1 \sim \alpha x$, $\sin x \sim x$,可知$$\lim_{x \to 0} \frac{\sqrt{1+x+x^2}-1}{\sin 2x} = \lim_{x \to 0} \frac{\frac{1}{2}(x + x^2)}{2x} = \frac{1}{4}.$$
        \item 由$(1+x)^\alpha - 1 \sim \alpha x$, $1 - \cos x \sim \frac{x^2}{2}$,可知$$\lim_{x \to 0} \frac{\sqrt{1+x^2}-1}{1-\cos x} = \lim_{x \to 0} \frac{\frac{1}{2} x^2}{\frac{x^2}{2}} = 1.$$
    \end{enumerate}
\end{solution}
\section{第1章综合习题}

\begin{exercise}[1.C.1]
    求下列数列的极限:
    \begin{enumerate}[(1)]
        \item $a_n = \frac{1}{2} \cdot \frac{3}{4} \cdots \frac{2n-1}{2n}$ (提示: $\frac{1}{2} \cdot \frac{3}{4} \cdots \frac{2n-1}{2n} \les \frac{1}{\sqrt{2n+1}}$);
        \item $a_n = \frac{10}{1} \cdot \frac{11}{3} \cdots \frac{n+9}{2n-1}$;
        \item 设 $a_1 > 1,  a_{n+1} = 2 - \frac{1}{a_n},  n=1, 2, \cdots$;
        \item 设 $a_1 = 3,  a_{n+1} = \frac{1}{1+a_n},  n=1, 2, \cdots$.
    \end{enumerate}
\end{exercise}

\begin{solution}
    \begin{enumerate}[(1)]
        \item 由$$(2n)^2 = 4n^2 \ges 4n^2 - 1 = (2n-1)(2n+1)$$可得$$\frac{2n-1}{2n} \les \frac{2n-1}{\sqrt{(2n-1)(2n+1)}} = \sqrt{\frac{2n-1}{2n+1}}.$$因此$$a_n = \frac{1}{2} \cdot \frac{3}{4} \cdots \frac{2n-1}{2n} \les \sqrt{\frac{1}{3} \cdot \frac{3}{5} \cdots \frac{2n-1}{2n+1}} = \sqrt{\frac{1}{2n+1}}.$$
              而$\lim_{n\to\infty} \sqrt{\frac{1}{2n+1}} = 0$,  故由夹逼定理可知$\lim_{n\to\infty} a_n = 0$;
        \item 由$\lim_{n \to \infty} \frac{n+9}{2n-1} = \frac{1}{2}$, 知$$\lim_{n\to\infty} \sqrt[n]{\frac{10}{1} \cdot \frac{11}{3} \cdots \frac{n+9}{2n-1}} = \lim_{n\to\infty} \frac{n+9}{2n-1} = \frac{1}{2}.$$
              因此$\lim_{n\to\infty} a_n = \lim_{n\to\infty} \left(\sqrt[n]{a_n}\right)^n = 0$;
        \item 由$a_1 > 1$, 以及若$a_n > 1$时,  $a_{n+1} = 2 - \frac{1}{a_n} > 1$, 归纳的可知$a_n > 1,  \forall n \in \mathbb{N}^+$. 所以数列有下界. 再用归纳法: 当 $n=1$ 时:
              $$ a_2 - a_1 = 2 - \left(\frac{1}{a_1} + a_1\right) \les 2-2=0,  $$
              推出 $a_2 \les a_1$. 假设对 $n$ 有 $a_n \les a_{n-1}$,  那么当 $n+1$ 时
              $$ a_{n+1} - a_n = \frac{1}{a_{n-1}} - \frac{1}{a_n} = \frac{a_n - a_{n-1}}{a_n a_{n-1}} \les 0. $$
              所以 $\{a_n\}$ 是单调减有下界数列,  因此收敛. 设 $\lim_{n \to \infty} a_n = a \ges 1$. 在
              $$ a_{n+1} = 2 - \frac{1}{a_n} $$
              两边取极限得
              $$ a = 2 - \frac{1}{a} \implies a^2 - 2a + 1 = 0 $$
              解得 $a=\pm 1$. 但 $a=-1$ 不合题意,  所以 $\lim_{n \to \infty} a_n = 1$.
        \item $a_{n+2} = \frac{1}{1+a_{n+1}} = \frac{1}{1+\frac{1}{1+a_n}} = \frac{1+a_n}{2+a_n}$.
              假如对任何 $n$,  有 $a_{2n} \ges a_{2n-2}$; $a_{2n+1} \les a_{2n-1}$,  那么对 $n+1$,  有
              $$ a_{2n+2} - a_{2n} = \frac{1}{1+a_{2n+1}} - \frac{1}{1+a_{2n-1}} = \frac{a_{2n-1}-a_{2n+1}}{1+a_{2n+1}a_{2n-1}} \ges 0 $$
              $$ a_{2n+3} - a_{2n+1} = \frac{1}{1+a_{2n+2}} - \frac{1}{1+a_{2n}} = \frac{a_{2n}-a_{2n+2}}{1+a_{2n+2}a_{2n}} \les 0 $$
              推出数列 $\{a_n\}$ 的子列 $\{a_{2n}\}$ 单调增有上界,  $\{a_{2n-1}\}$ 单调减有下界. 因此分别收敛. 对$$a_{2n+2} = \frac{1+a_{2n}}{2+a_{2n}}, $$两边取极限得
              $$ a = \frac{1+a}{2+a} \implies a^2 + a - 1 = 0 $$
              解得 $a = \frac{-1+\sqrt{5}}{2}$. 同理,  对$$a_{2n+3} = \frac{1+a_{2n+1}}{2+a_{2n+1}}, $$两边取极限得
              $$ b = \frac{1+b}{2+b} \implies b^2 + b - 1 = 0 $$
              解得 $b = \frac{-1+\sqrt{5}}{2}$.
              因此$\lim_{n \to \infty} a_{2n} = \lim_{n \to \infty} a_{2n-1}$,  故$\lim_{n \to \infty} a_n$存在,  且$\lim_{n \to \infty} a_n = \frac{-1+\sqrt{5}}{2}$.
    \end{enumerate}
\end{solution}

\begin{exercise}[1.C.2]
    设 $\{a_n\}$ 为单调递增的数列,  并且收敛于 $a$,  证明对一切 $n$ 有 $a_n < a$. (对单调递减且有极限的数列,  类似的结论成立.)
\end{exercise}

\begin{solution}
    反证法. 假设存在某个 $n_0$,  使得 $a_{n_0} > a$. 由数列单调递增的性质,  对一切 $n > n_0$ 有 $a_n \ges a_{n_0} > a$, 于是存在 $\varepsilon = \frac{a_{n_0} - a}{2} > 0$,  使得$\forall N$,  存在$ n = \max\{n_0,  N\} + 1 > N$,  使得
    $$ |a_n - a| = a_n - a \ges a_{n_0} - a = 2\varepsilon > \varepsilon,  $$
    这与数列收敛的定义矛盾.
\end{solution}

\begin{exercise}[1.C.3]
    证明下面的数列收敛:
    \begin{enumerate}[(1)]
        \item $a_n = 1 + \frac{1}{2^2} + \cdots + \frac{1}{n^2}$;
        \item $a_n = \left(1+\frac{1}{2}\right)\left(1+\frac{1}{2^2}\right)\cdots\left(1+\frac{1}{2^n}\right)$.
    \end{enumerate}
\end{exercise}

\begin{solution}
    \begin{enumerate}[(1)]
        \item 由数列定义可知 $\{a_n\}$ 单调递增. 又因为
              $$ a_n = 1 + \frac{1}{2^2} + \cdots + \frac{1}{n^2} \les 1 + \frac{1}{1 \cdot 2} + \frac{1}{2 \cdot 3} + \cdots + \frac{1}{(n-1)n} = 2 - \frac{1}{n} < 2,  $$
              所以 $\{a_n\}$ 有上界. 因此 $\{a_n\}$ 收敛;
        \item 由数列定义可知 $\{a_n\}$ 单调递增. 又因为
              $$a_n = \e^{\sum_{k=1} \ln \left(1 + \frac{1}{2^k}\right)} \les \e^{\sum_{k=1} \frac{1}{2^k}} = \e^1.$$
              所以 $\{a_n\}$ 有上界. 又由数列定义可知 $\{a_n\}$ 单调递增. 因此 $\{a_n\}$ 收敛.
    \end{enumerate}
\end{solution}

\begin{exercise}[1.C.4]
    试构造一个发散的数列 $\{a_n\}$,  满足条件: 对任意正数 $\varepsilon$,  存在正整数 $N$,  使当 $n > N$ 时,  有 $|a_{n+1}-a_n| < \varepsilon$.
\end{exercise}

\begin{solution}
    取$a_n = \sqrt{n}$. 则对任意 $\varepsilon > 0$,  存在正整数 $N > \frac{1}{4\varepsilon^2}$,  当 $n > N$ 时,  有
    $$ |a_{n+1}-a_n| = \sqrt{n+1} - \sqrt{n} = \frac{1}{\sqrt{n+1} + \sqrt{n}} < \frac{1}{2\sqrt{n}} < \varepsilon. $$
    但数列 $\{a_n\}$ 显然发散.
\end{solution}

\begin{exercise}[1.C.5]
    若数列 $\{a_n\}$ 满足: 存在常数 $M$,  使得对一切 $n$ 有
    $$ A_n = |a_2-a_1| + |a_3-a_2| + \cdots + |a_{n+1}-a_n| \les M. $$
    证明:
    \begin{enumerate}[(1)]
        \item 数列 $\{A_n\}$ 收敛;
        \item 数列 $\{a_n\}$ 也收敛.
    \end{enumerate}
\end{exercise}

\begin{solution}
    \begin{enumerate}[(1)]
        \item 由数列定义可知 $\{A_n\}$ 单调递增. 又因为对一切 $n$ 有 $A_n \les M$,  所以 $\{A_n\}$ 有上界. 因此 $\{A_n\}$ 收敛;
        \item 用Cauchy收敛准则证明. 由(1)知 $\{A_n\}$ 收敛,  因此对任意 $\varepsilon > 0$,  存在正整数 $N$,  $\forall n > N+1,  p > 0$,  有
              $$ |A_{n+p} - A_{n-1}| = |a_{n+1}-a_n| + |a_{n+2}-a_{n+1}| + \cdots + |a_{n+p}-a_{n+p-1}| < \varepsilon. $$
              由三角不等式可知
              $$ |a_{n+p} - a_n| \les |a_{n+1}-a_n| + |a_{n+2}-a_{n+1}| + \cdots + |a_{n+p}-a_{n+p-1}| < \varepsilon. $$
    \end{enumerate}
\end{solution}

\begin{exercise}[1.C.6]
    设 $\{a_n\}$ 是正严格递增数列. 求证: 若 $a_{n+1}-a_n$ 有界,  则对任意 $\alpha \in (0, 1)$ 有 $\lim_{n \to \infty} (a_{n+1}^\alpha - a_n^\alpha) = 0$. 并说明此结论的逆不对,  即,  存在正严格递增数列 $\{a_n\}$ 使得对任意 $\alpha \in (0, 1)$ 有 $\lim_{n \to \infty} (a_{n+1}^\alpha - a_n^\alpha) = 0$,  但是 $a_{n+1}-a_n$ 无界. (提示: 考虑 $a_n = n\ln n$.)
\end{exercise}

\begin{solution}
    \begin{enumerate}[(1)]
        \item 若$\{a_n\}$有界, 此时由于其严格单调, 故有极限, 记$\lim_{n\to\infty}a_n=l$, 可知$\lim_{n \to \infty} (a_{n+1}^\alpha - a_n^\alpha) = l^\alpha-l^\alpha=0.$
        \item 若$\{a_n\}$无界, 即$\lim_{n\to\infty}a_n=+\infty$, 设$\left|a_{n+1}-a_n\right| \les M$.
              \begin{align*}
                  0\les a_{n+1}^\alpha - a_n^\alpha & =a_n^\alpha\left(\left(\frac{a_{n+1}}{a_n}\right)^\alpha - 1\right) <a_n^\alpha\left(\frac{a_{n+1}}{a_n} - 1\right)=\frac{a_{n+1}-a_n}{a_n^{1-\alpha}} \les\frac{M}{a_n^{1-\alpha}}.
              \end{align*}
              同时, $$\lim_{n\to\infty}Ma_n^{\alpha-1}=0.$$
              因此由夹逼定理可知$\lim_{n \to \infty} (a_{n+1}^\alpha - a_n^\alpha) = 0$.
        \item 反之不对, 取$a_n=n\ln n$, 则\begin{align*}
                  a_{n+1}^\alpha-a_n^\alpha & =(n+1)^\alpha\ln^\alpha(n+1)-n^\alpha\ln^\alpha                       \\
                                            & <\left((n+1)^\alpha-n^\alpha\right)\ln^\alpha n                       \\
                                            & =n^\alpha\left(\left(1+\frac{1}{n}\right)^\alpha-1\right)\ln^\alpha n \\
                                            & <n^\alpha\left(\left(1+\frac{1}{n}\right)-1\right)\ln^\alpha n        \\
                                            & =n^{\alpha-1}\ln^\alpha n=\frac{\ln^\alpha n}{n^{1-\alpha}}.
              \end{align*}
              由于$$\lim_{n\to\infty}\frac{\ln^\alpha n}{n^{1-\alpha}}=0, $$因此由夹逼定理可知$\lim_{n \to \infty} (a_{n+1}^\alpha - a_n^\alpha) = 0$.但$$a_{n+1}-a_n=(n+1)\ln(n+1)-n\ln n=\ln (n+1)+n\ln\left(\frac{n+1}{n}\right)>\ln(n+1), $$显然无界.
    \end{enumerate}
\end{solution}

\begin{exercise}[1.C.7]
    设数列 $\{a_n\}$ 满足 $\lim_{n \to \infty} (a_{n+1}-a_n) = a$. 证明: $\lim_{n \to \infty} \frac{a_n}{n} = a$.
\end{exercise}

\begin{solution}
    由Stolz定理可知
    \begin{align*}
        \lim_{n \to \infty} \frac{a_n}{n} & = \lim_{n \to \infty} \frac{a_{n+1}-a_n}{(n+1)-n} = \lim_{n \to \infty} (a_{n+1}-a_n) = a.
    \end{align*}
\end{solution}

\begin{exercise}[1.C.8]\label{exe:1.C.8}
    证明: 若 $\lim_{n \to \infty} a_n = a$,  且 $a_n > 0$,  则 $\lim_{n \to \infty} \sqrt[n]{a_1a_2\cdots a_n} = a$.
\end{exercise}
\begin{solution}
    由Stolz定理可知
    \begin{align*}
        \lim_{n \to \infty} \sqrt[n]{a_1a_2\cdots a_n} = \lim_{n \to \infty}\e^{\frac{1}{n}\left(\ln a_1+\ln a_2+\cdots +\ln a_n\right)} = \e^{\lim\limits_{n \to \infty}\frac{\left(\ln a_1+\ln a_2+\cdots +\ln a_n\right)}{n}}\=\e^{\lim\limits_{n \to \infty}\frac{\ln a_n}{1}} =\e^{\ln a} =a.
    \end{align*}
\end{solution}

\begin{exercise}[1.C.9]\label{exe:1.C.9}
    证明: 若 $a_n > 0$,  且 $\lim_{n \to \infty} \frac{a_{n+1}}{a_n}$ 存在,  则 $\lim_{n \to \infty} \sqrt[n]{a_n}$ 也存在,  并且
    $$ \lim_{n \to \infty} \sqrt[n]{a_n} = \lim_{n \to \infty} \frac{a_{n+1}}{a_n}. $$
\end{exercise}

\begin{solution}
    设$b_n=\frac{a_{n}}{a_{n-1}}, (n>1);b_1=a_1$, 则$a_n=b_1b_2\cdots b_n$.由\hyperref[exe:1.C.8]{综合习题1.C.8}可知结果.
    直接Stolz也可以得到结果.
    \begin{align*}
        \lim_{n \to \infty} \sqrt[n]{a_n} & =\lim_{n \to \infty} \e{\frac{\ln a_n}{n}}=\e^{\lim\limits_{n \to \infty}\frac{\ln a_n}{n}}                               \\
                                          & =\e^{\lim\limits_{n \to \infty}\frac{\ln a_{n+1}-\ln a_{n}}{1}} =\e^{\lim\limits_{n \to \infty}\ln \frac{a_{n+1}}{a_{n}}} \\
                                          & =\e^{\ln \lim\limits_{n \to \infty}\frac{a_{n+1}}{a_{n}}} =\lim_{n \to \infty}\frac{a_{n+1}}{a_{n}}.
    \end{align*}
\end{solution}

\begin{exercise}[1.C.10]
    求下列极限:
    \begin{tasks}[label=(\arabic*)](2)
        \task $\lim_{n \to \infty} \frac{1+\sqrt{2}+\sqrt[3]{3}+\cdots+\sqrt[n]{n}}{n}$;
        \task $\lim_{n \to \infty} \frac{n}{\sqrt[n]{n!}}$.
    \end{tasks}
\end{exercise}

\begin{solution}
    \begin{enumerate}[(1)]
        \item 由Stolz定理可知
              \begin{align*}
                  \lim_{n \to \infty} \frac{1+\sqrt{2}+\sqrt[3]{3}+\cdots+\sqrt[n]{n}}{n} & = \lim_{n \to \infty} \sqrt[n]{n} = 1;
              \end{align*}
        \item 可以用\hyperref[exe:1.C.9]{综合习题1.C.9}来做, 记$a_n=\frac{n^n}{n!}$由于
              $$\lim_{n\to\infty}\frac{a_{n+1}}{a_n}=\lim_{n\to\infty}\frac{(n+1)^{n+1}\cdot n!}{n^n\cdot (n+1)!}=\lim_{n\to\infty}\frac{(n+1)^{n}}{n^n}=\lim_{n\to\infty}\left(1+\frac{1}{n}\right)^{n}=\e, $$
              因此$$\lim_{n \to \infty} \frac{n}{\sqrt[n]{n!}}=\lim_{n \to \infty}\sqrt[n]{\frac{n^n}{n!}}=\lim_{n \to \infty}\sqrt[n]{a_n}=\e.$$
    \end{enumerate}
\end{solution}

\begin{exercise}[1.C.11]
    已知 $\lim_{n \to \infty} a_n = a$,  求证 $\lim_{n \to \infty} \frac{a_1+2a_2+\cdots+na_n}{n^2} = \frac{a}{2}$.
\end{exercise}
\begin{solution}
    由 Stolz 定理,  有
    \begin{align*}
        \lim_{n \to \infty} \frac{a_1+2a_2+\cdots+na_n}{n^2} = \lim_{n \to \infty} \frac{na_n}{n^2 - (n-1)^2} = \lim_{n \to \infty} \frac{na_n}{2n-1} =\lim_{n \to \infty}\frac{n}{2n-1} \lim_{n \to \infty} a_n = \frac{a}{2}.
    \end{align*}
\end{solution}

\begin{exercise}[1.C.12]
    设 $\{a_n\}$ 且 $a_n \to a \in \mathbb{R}$,  又设 $\{b_n\}$ 是正数列,  $c_n = \frac{a_1b_1+a_2b_2+\cdots+a_nb_n}{b_1+b_2+\cdots+b_n}$. 求证:
    \begin{enumerate}[(1)]
        \item $\{c_n\}$ 收敛;
        \item 若 $(b_1+b_2+\cdots+b_n) \to +\infty$,  则 $\lim_{n \to \infty} c_n = a$.
    \end{enumerate}
\end{exercise}

\begin{solution}
    \begin{enumerate}[(1)]
        \item 记$B_n = b_1+b_2+\cdots+b_n$, 由$\lim_{n\to\infty} a_n = a$, 任取$\ve > 0$, 存在$K$, 当$k > K$时,$|a_k -a| < \ve$.

              当$n > K$, 有
              $$c_n - a = \frac{\sum_{k=1}^{n}(a_k - a)b_k}{B_n}= \frac{\sum_{k=1}^{K}(a_k - a)b_k}{B_n} + \frac{\sum_{k=K+1}^{n}(a_k - a)b_k}{B_n}.$$

              其中$$\left| \frac{\sum_{k=K+1}^{n}(a_k - a)b_k}{B_n}\right| \les \frac{\sum_{k=K+1}^{n}|a_k - a|b_k}{B_n} < \ve \frac{\sum_{k=K+1}^{n}b_k}{B_n} < \ve.$$

              因此$$\left|c_n - \left( a + \frac{\sum_{k=1}^{K}(a_k - a)b_k}{B_n}\right)\right| < \ve.$$

              而对于$$q_n := \frac{\sum_{k=1}^{K}(a_k - a)b_k}{B_n},$$
              $C := \sum_{k=1}^{K}(a_k - a)b_k$是仅与$K$有关,与$n$无关的常数,$B_n$单调增,因此$q_n$单调有界($C>0$时$q_n$单调减且$q_n > 0$, $C<0$时$q_n$单调增且$q_n < 0$), 故$q_n$收敛, 设$\lim_{n\to\infty}q_n = q$, 再取$N$,使得当$n,m > N$时,$|q_m - q_n| < \ve$, 则当$n,m > \max\{N,K\}$时,
              $$|c_m - c_n| \les \left| c_n - \left(a + q_n\right)\right| + \left| c_m - \left(a + q_m\right)\right| + |q_m - q_n| < 3\ve.$$
              因此由Cauchy收敛准则可知$c_n$收敛.
        \item 下给出两种方法,
              \begin{enumerate}
                  \item 由(1)中的过程,$q_n = \frac{C}{B_n}$, 由于$B_n \to +\infty$, $C$为常数, 因此$q_n \to 0$, 因此存在$N$, 使得当$n > N$时,$|q_n| < \ve$, 则当$n > \max\{N,K\}$时,
                        $$|c_n - a| \les \left| c_n - \left(a + q_n\right)\right| + |q_n| < \ve + |q_n| < 2\ve.$$
                  \item 由Stolz定理可知
                        \begin{align*}
                            \lim_{n \to \infty} c_n = \lim_{n \to \infty} \frac{a_1b_1+a_2b_2+\cdots+a_nb_n}{b_1+b_2+\cdots+b_n} = \lim_{n \to \infty} \frac{a_nb_n}{b_n} = \lim_{n \to \infty} a_n = a.
                        \end{align*}
              \end{enumerate}
    \end{enumerate}

    上述解答中给出了符合题目原意的证明,即先证明$c_n$收敛,然后在$B_n$无界时,再证明$\lim_{n \to \infty} c_n = a$.但对于这道题而言,还可以分类$B_n$有界和无界来讨论, 即先做(2),然后对$B_n$有界时,用Cauchy收敛准则证明$\left\{ \sum_{k=1}^{n}a_n b_n \right\}$收敛,即两种分类下以截然不同的方式来证明$c_n$收敛.
\end{solution}

\begin{remark}
    $a_n := \cdots$中$:=$表示定义. 如$a_n := \frac{1}{n}$表示我们新定义了一个数列$a_n$, 其通项公式为$a_n = \frac{1}{n}$.在上文中“$C := \sum_{k=1}^{K}(a_k - a)b_k$是仅与$K$有关,与$n$无关的常数.”表示: “记$C = \sum_{k=1}^{K}(a_k - a)b_k$, 则$C$是仅与$K$有关,与$n$无关的常数.”有的地方会写为$a_n \overset{\text{def}}{=} \cdots$.
\end{remark}

\begin{exercise}[1.C.13]
    证明: $\lim_{x \to +\infty} \left(1+\frac{1}{x^p}\right)^x = \begin{cases} 1,  & p > 1,  \\ \mathrm{e},  & p = 1,  \\ \infty,  & p < 1. \end{cases}$
\end{exercise}
\begin{solution}
    实际上题目中的无穷只能是$+\infty$.

    $p>0$时, $x^p\to+\infty$
    \begin{align*}
        \lim_{x \to +\infty} \left(1+\frac{1}{x^p}\right)^x & = \lim_{x \to +\infty} \left(1+\frac{1}{x^p}\right)^{x^p\cdot\frac{1}{x^{p-1}}} = \e^{\lim\limits_{x \to +\infty}\frac{1}{x^{p-1}}} = \begin{cases} 1,  & p > 1, \\
              \e, & p = 1, \\ +\infty,  & p < 1.\end{cases}
    \end{align*}
    $p\les 0$时, $x^p\to 0$, 则考虑$x>1$时,
    $$\lim_{x \to +\infty} \left(1+\frac{1}{x^p}\right)^x\ges\lim_{x \to +\infty}  2^x=+\infty.$$
\end{solution}
\begin{exercise}[1.C.14]
    设 $f(x)$ 为周期函数,  且 $\lim_{x \to \infty} f(x) = 0$,  证明 $f(x)$ 恒为零.
\end{exercise}
\begin{solution}
    设 $f(x)$ 的正周期为 $T>0$,  $\forall \ve >0, \exists N \in \N^*, $ 当 $|x|\ges N$ 时$|f(x)|<\ve$.

    因此对于$n=\left\lceil \frac{N}{T}\right\rceil$, 有$nT\ges N$, 故对于任意 $x \in [nT,  (n+1)T)$,  有$f(x)<\ve$.

    利用周期性可以得到$\forall x \in \mathbb{R},  |f(x)| < \ve$.
    由于 $\ve$ 是任意的正数,  所以 $f(x)$ 恒为零.
\end{solution}

\begin{exercise}[1.C.15]
    证明
    \begin{enumerate}[(1)]
        \item 函数 $f(x)$ 在 $x \to x_0^-$ 时有极限 $l$ 的充分必要条件是: 对于任意一个以 $x_0$ 为极限的单调递增数列 $\{a_n\} \,  (a_n \ne x_0)$,  都有 $\lim_{n \to \infty} f(a_n) = l$;
        \item 函数 $f(x)$ 在 $x \to x_0^+$ 时有极限 $l$ 的充分必要条件是: 对于任意一个以 $x_0$ 为极限的单调递减数列 $\{a_n\} \,  (a_n \ne x_0)$,  都有 $\lim_{n \to \infty} f(a_n) = l$.
    \end{enumerate}
\end{exercise}

\begin{solution}
    \begin{enumerate}[(1)]
        \item \begin{enumerate}
                  \item 必要性:考虑任意数列 $\{a_n\}$,  使得 $\lim_{n\to\infty}a_n = x_0$且 $\{a_n\}$ 单调递增, .

                        由于$\lim_{x \to x_0^-}f(x)=l$, 因此, $\forall \ve > 0,  \exists \delta > 0$,  当 $x_0 - \delta < x < x_0$ 时,  有 $|f(x) - l| < \ve$.

                        同时对于$\delta$, $\exists N \in \N^*$, 使得当 $n > N$ 时,  有 $|a_n - x_0| < \delta$, 即 $x_0 - \delta < a_n < x_0$.

                        因此我们有$m>N$时$|f(a_n)-l|<\ve$.即得到数列$\{f(a_n)\}$收敛到$l$.
                  \item 充分性: 反证, 若$x\to x_0^-$时$f(x)$的极限为$l$不成立, 即$\exists \ve > 0,  \forall \delta > 0,  \exists x_0 - \delta < x < x_0$,  使得$|f(x) - l| \ges \ve$.

                        因此我们依次构造 $\delta_1=1, \delta_n = \min\{\frac{1}{n}, x_0-a_{n-1}\}, (n>2)$, 则$\exists a_n, x_0-\delta_n<a_n<x_0$ , 使得$|f(a_n) - l| \ges \ve$.
                        即有$a_n>a_{n-1}$, 且$|x_0-a_n|<\frac{1}{n}$.这意味着$\{a_n\}$单调递增, 且$\lim_{n \to \infty} a_n = x_0$.

                        由于$|f(a_n) - l| \ges \ve$, 所以$\{f(a_n)\}$不收敛到$l$, 矛盾, 故充分性成立.
              \end{enumerate}
        \item 证明同理.具体而言:

              设$g(x) = f(-x)$, 则$f(x)$在$x\to x_0^-$时有极限$l$ $\Leftrightarrow$ $g(x)$在$x\to -x_0^+$时有极限$l$.由(1)可知, 这等价于对于任意一个以$-x_0$为极限的单调递增数列$\{b_n\} \,  (b_n \ne -x_0)$,  都有 $\lim_{n \to \infty} g(b_n) = l$.设$a_n=-b_n$, 则$\{a_n\}$是以$x_0$为极限的单调递减数列, 且$\lim_{n \to \infty} f(a_n) = l$.因此(2)得证.
    \end{enumerate}
\end{solution}

\begin{exercise}[1.C.16]
    设 $\xi$ 是一个无理数,  $a, b$ 是实数,  且 $a<b$. 求证: 存在整数 $m, n$ 使得 $m+n\xi \in (a, b)$,  即,  集合
    $$ S = \{m+n\xi \mid m,  n \in \mathbb{Z}\} $$
    在 $\mathbb{R}$ 稠密.
\end{exercise}
\begin{solution}
    稠密的定义:设$S\subset \R$, 若对任意$a, b\in\R, a<b$, 都有$S\cap(a, b)\neq\varnothing$, 则称$S$在$\R$中稠密.

    想法是这样的,我们为了找到某个$m+n\xi$落在$(a, b)$中,于是用$\xi$构造一个充分小的实数$\ve = m_0 + n_0 \xi \in \left( 0, {b-a} \right)$, 因为这个$\ve$够小,因此在$\R$的分割$\R=\bigcup_{l\in\Z}[l\ve, (l+1)\ve]$中,每一段的长度$\varepsilon$严格小于$b-a$.
    这样就能证明$\{l \ve \mid l\in\Z\} \cap (a, b) \neq \varnothing$, 也就是
    存在某个$l_0\in\Z$, 使得$l_0\ve\in(a, b)$.
    随后我们取$m=l_0m_0, n=l_0n_0$即有$m+n\xi=l_0\ve\in(a, b)$.

    \paragraph*{构造$\ve$}

    实际上, 对于$b-a>0$, 总存在$k\in\N^*$, 使得$\frac{1}{k}<{b-a}$.因此我们考虑构造一个满足$\ve<\frac{1}{k}, \ve\in S$即可.

    对于$l=1, 2, \cdots, k+1$, 我们考虑
    \begin{align*}
        n_l & =\left\lfloor l\xi \right\rfloor \\
        x_l & =l\xi-n_l \in S.
    \end{align*}
    $x_l$是$l\xi$的小数部分, 容易知道$x_l\in[0, 1)$, 并且$x_l$之间总是两两不同的, 否则$i\xi-n_i=j\xi-n_j, i\neq j$, 这意味着$\xi=\frac{n_i-n_j}{i-j}$, 这与$\xi$为无理数矛盾.

    因此对于$$[0, 1)=\bigcup_{j=1}^{k} \left[ \frac{j-1}{k}, \frac{j}{k} \right), $$
    这$k$个区间包括了$k+1$个不同实数$x_l$.因此总有一个区间内部存在同时两个实数, 记为$x_p, x_q \in S, p\neq q$, 不妨认为$x_q>x_p$.

    由$x_l$的构造 $x_p=p\xi-n_p, x_q=q\xi-n_q,$
    有$$ x_q-x_p=(q-p)\xi-(n_p-x_q)\in S,$$
    且由于$x_p, x_q$落在同一个区间内, 而区间长度为$\frac{1}{k}$,因此$0<x_q-x_p\les\frac{1}{k}< {b-a},$
    所以$x_q-x_p$满足我们对$\ve$的要求.我们取$$\ve=x_q-x_p.$$

    \paragraph*{构造$m, n$}

    我们先证明$\exists l_0\in\Z, \st l_0\ve\in(a, b)$:我们取$l_0 = \llceil \frac{b}{\ve} \rrceil - 1$,则$$l_0 \ve = \left(\llceil \frac{b}{\ve} \rrceil - 1\right)\ve < \left(\frac{b}{\ve} + 1 - 1\right)\ve = b.$$
    同时,由于$\ve < {b-a}$, 因此$$l_0 \ve = \left(\llceil \frac{b}{\ve} \rrceil - 1\right)\ve \ges \left(\frac{b}{\ve} - 1\right)\ve = b - \ve > b - (b-a) = a.$$

    因此取$l_0\ve\in(a, b)$, 因此$$m=l_0(n_q-n_p), n=l_0(q-p)$$
    即有$m+n\xi=l_0 (n_q-n_p)+l_0(q-p)\xi=l_0 \left( (q-p)\xi-(n_p-n_q) \right)=l_0\ve\in(a, b).$
\end{solution}

\newpage
% \chapter{连续函数的基本概念}
\section{习题 2.1}

\begin{exercise}[2.1.1]
    设函数 $f(x)$ 在点 $x_0$ 附近有定义, 且 $\lim_{h \to 0} [f(x_0+h)-f(x_0-h)]=0$. 问 $f(x)$ 是否必在 $x=x_0$ 处连续?
\end{exercise}

\begin{solution}
    不一定. 例如, 设
    $$ f(x) = \begin{cases} 0, & x \ne 0, \\ 1, & x=0. \end{cases} $$
    则 $\lim_{h \to 0} [f(0+h)-f(0-h)] = \lim_{h \to 0} (0-0)=0$, 但 $f(x)$ 在 $x=0$ 处不连续.
\end{solution}

\begin{exercise}[2.1.2]
    设对任意正数 $\varepsilon < \frac{b-a}{2}$, 函数 $f(x)$ 在 $[a+\varepsilon, b-\varepsilon]$ 上连续. 证明 $f(x)$ 在 $(a,b)$ 内连续.
\end{exercise}

\begin{solution}
    设 $x_0 \in (a,b)$, 则存在 $\varepsilon_0 = \min\left\{\frac{b-x_0}{2}, \frac{x_0-a}{2}\right\}> 0$, 使得 $a+\varepsilon_0 < x_0 < b-\varepsilon_0$. 因为 $f(x)$ 在 $[a+\varepsilon_0, b-\varepsilon_0]$ 上连续, 故 $f(x)$ 在 $x_0$ 处连续. 因此 $f(x)$ 在 $(a,b)$ 内连续.
\end{solution}

\begin{exercise}[2.1.3]
    设在点 $x=x_0$ 处, 函数 $f(x)$ 连续, 而 $g(x)$ 不连续, 问函数 $f(x) \pm g(x)$ 与 $f(x)g(x)$ 在点 $x_0$ 的连续性如何? 若 $f(x), g(x)$ 在 $x_0$ 处都不连续, 回答同样的问题.
\end{exercise}

\begin{solution}
    \begin{enumerate}
        \item $f(x) \pm g(x)$ 与 $f(x)g(x)$ 在点 $x_0$ 处均不连续. 反证: 若 $f(x)+g(x)$ 在点 $x_0$ 处连续, 则 $\lim_{x \to x_0} [f(x)+g(x)] = f(x_0)+g(x_0)$, 即 $\lim_{x \to x_0} g(x) = \lim_{x \to x_0} [f(x)+g(x)] - \lim_{x \to x_0} f(x) = f(x_0)+g(x_0)-f(x_0)=g(x_0)$, 矛盾. 同理可证 $f(x)-g(x)$ 在点 $x_0$ 处不连续.

              $f(x)g(x)$在$x_0$处连续性未知, 例如, 设
              $$ f(x) = x, \quad g(x) = \begin{cases} 1, & x \ne 0, \\ 0, & x=0; \end{cases} $$
                则 $f(x)g(x) \equiv x$, 在点 $x_0=0$ 处连续. 又例如, 设
                $$ f(x) = 1, \quad g(x) = \begin{cases} 1, & x \ne 0, \\ 0, & x=0; \end{cases} $$
                则 $f(x)g(x) = \begin{cases} 1, & x \ne 0, \\ 0, & x=0; \end{cases}$ 在点 $x_0=0$ 处不连续.

        \item 若 $f(x), g(x)$ 在 $x_0$ 处都不连续, 则 $f(x) \pm g(x)$ 与 $f(x)g(x)$ 在点 $x_0$ 处可能连续, 也可能不连续. 例如,
              \begin{enumerate}
                  \item $f(x) + g(x)$ 在点 $x_0$ 处连续: 设
                        $$ f(x) = \begin{cases} 0, & x \ne
              0,         \\ 1, & x=0;\end{cases} \quad g(x) = \begin{cases} 0, & x \ne 0, \\ -1, & x=0. \end{cases} $$
                        则 $f(x)+g(x) \equiv 0$, 在点 $x_0=0$ 处连续.
                  \item $f(x) + g(x)$ 在点 $x_0$ 处不连续: 设
                        $$ f(x) = \begin{cases} 0, & x \ne 0, \\ 1, & x=0; \end{cases} \quad g(x) = \begin{cases} 0, & x \ne 0, \\ 1, & x=0. \end{cases} $$
                        则 $f(x)+g(x) = \begin{cases} 0, & x \ne 0, \\ 2, & x=0; \end{cases}$ 在点 $x_0=0$ 处不连续.
                  \item $f(x)g(x)$ 在点 $x_0$ 处连续: 设
                        $$ f(x) = \begin{cases} 1, & x \ne 0, \\ 0, & x=0; \end{cases} \quad g(x) = \begin{cases} 0, & x \ne 0, \\ 1, & x=0. \end{cases} $$
                        则 $f(x)g(x) \equiv 0$, 在点 $x_0=0$ 处连续.
                  \item $f(x)g(x)$ 在点 $x_0$ 处不连续: 设
                        $$ f(x) = \begin{cases} 1, & x \ne 0, \\ 0, & x=0; \end{cases} \quad g(x) = \begin{cases} 1, & x \ne 0, \\ 0, & x=0. \end{cases} $$
                        则 $f(x)g(x) = \begin{cases} 1, & x \ne 0, \\ 0, & x=0; \end{cases}$ 在点 $x_0=0$ 处不连续.
              \end{enumerate}
    \end{enumerate}
\end{solution}

\begin{exercise}[2.1.4]
    \begin{enumerate}
        \item 设函数 $f(x)$ 在点 $x=x_0$ 处连续, 则函数 $|f(x)|$ 在点 $x=x_0$ 处也连续.
        \item 设函数 $f(x)$ 和 $g(x)$ 在一个区间 $I$ 上连续, 证明: 函数 $M(x)=\max(f(x), g(x))$ 及 $m(x)=\min(f(x), g(x))$ 在区间 $I$ 上均连续.
    \end{enumerate}
\end{exercise}

\begin{solution}
    \begin{enumerate}
        \item 因为 $f(x)$ 在点 $x=x_0$ 处连续, 故 $\lim_{x \to x_0} f(x) = f(x_0)$, 因此$\forall \varepsilon > 0$, $\exists \delta > 0$, 当 $|x-x_0| < \delta$ 时, 有 $|f(x)-f(x_0)| < \varepsilon$. 又因为
              $$ ||f(x)| - |f(x_0)|| \les |f(x)-f(x_0)| < \varepsilon, $$
              故 $\lim_{x \to x_0} |f(x)| = |f(x_0)|$, 即 $|f(x)|$ 在点 $x=x_0$ 处连续.
        \item 由 $M(x) = \frac{f(x)+g(x)}{2} + \frac{|f(x)-g(x)|}{2}$, $m(x) = \frac{f(x)+g(x)}{2} - \frac{|f(x)-g(x)|}{2}$ 可知, 只需证明 $|f(x)-g(x)|$ 在区间 $I$ 上连续. 因为 $f(x), g(x)$ 在区间 $I$ 上连续, 故 $f(x)-g(x)$ 在区间 $I$ 上连续. 由 (1) 可知, $|f(x)-g(x)|$ 在区间 $I$ 上连续.
    \end{enumerate}
\end{solution}

\begin{exercise}[2.1.5]
    证明: 存在这样的函数 $f(x)$, 处处不连续, 但函数 $|f(x)|$ 处处连续. (提示: 适当地修改 Dirichlet 函数可得出一个例子.)
\end{exercise}

\begin{solution}
    设
    $$ f(x) = \begin{cases} 1, & x \in \mathbb{Q}, \\ -1, & x \in \mathbb{R} \setminus \mathbb{Q}. \end{cases} $$
    则 $f(x)$ 处处不连续, 但 $|f(x)| \equiv 1$, 处处连续.
    \begin{proof}
        若$f(x)$在$x_0$处连续,则$\lim_{x \to x_0} f(x)$存在,取$\{a_n\} \subset \mathbb{Q}$, $\lim_{n \to \infty} a_n = x_0$,则$\lim_{n \to \infty} f(a_n) = 1$;取$\{b_n\} \subset \mathbb{R} \setminus \mathbb{Q}$, $\lim_{n \to \infty} b_n = x_0$,则$\lim_{n \to \infty} f(b_n) = -1$, $\lim_{x \to x_0} f(a_n) \ne \lim_{x \to x_0} f(b_n)$,矛盾.因此$f(x)$处处不连续.
    \end{proof}
\end{solution}

\begin{exercise}[2.1.6]
    指出下列函数的间断点,并说明其类型.
    \begin{tasks}[label=(\arabic*)](2)
        \task $f(x) = \frac{x+1}{x-2}$;
        \task $f(x) = \begin{cases} \frac{\sin x}{|x|}, & x \ne 0, \\ 1, & x=0; \end{cases}$
        \task $f(x) = \lfloor|\cos x|\rfloor$;
        \task $f(x) = \frac{1}{1+\mathrm{e}^{1/x}}$;
        \task $f(x) = \begin{cases} \frac{1}{x+7}, & -\infty < x < -7, \\ x, & -7 \les x \les 1, \\ (x-1)\sin\frac{1}{x-1}, & 1 < x < +\infty; \end{cases}$
        \task[]
        \task[(6)] $f(x) = \begin{cases} \frac{x^2-4}{x-2}, & x \ne 2, \\ 4, & x=2. \end{cases}$
    \end{tasks}
\end{exercise}

\begin{solution}
    \begin{enumerate}
        \item $\lim\limits_{x \to 2^-} f(x) = -\infty$,$\lim\limits_{x \to 2^+} f(x) = +\infty$.\\
        左右极限不存在,因此 $x=2$ 是第二类间断点.

        \item 
        $\lim\limits_{x \to 0^+} f(x) = \lim\limits_{x \to 0^+} \frac{\sin x}{x} = 1$,$\lim\limits_{x \to 0^-} f(x) = \lim\limits_{x \to 0^-} \left(-\frac{\sin x}{x}\right) = -1$.\\
        左右极限存在但不相等,因此 $x=0$ 是第一类间断点中的跳跃间断点.

        \item 可知$f(x)=\begin{cases}
            1,&x=k\pi,k\in\Z;\\
            0,&\text{其他}.
        \end{cases}$
        在 $x = k\pi$ 处,$\lim\limits_{x \to k\pi} f(x) = 0 \ne f(k\pi) = 1$.\\
        左右极限存在且相等,$x = k\pi$ ($k \in \mathbb{Z}$) 是第一类间断点中的可去间断点.

        \item 
        $\lim\limits_{x \to 0^+} f(x) = \lim\limits_{x \to 0^+} \frac{1}{1+\mathrm{e}^{1/x}} = 0$,
        $\lim\limits_{x \to 0^-} f(x) = \lim\limits_{x \to 0^-} \frac{1}{1+\mathrm{e}^{1/x}} = 1$.\\
        左右极限存在但不相等,因此 $x=0$ 是第一类间断点中的跳跃间断点.

        \item 
        \begin{itemize}
            \item 在 $x=-7$ 处:\\
            $\lim\limits_{x \to -7^-} f(x) = \lim\limits_{x \to -7^-} \frac{1}{x+7} = -\infty$,
            $\lim\limits_{x \to -7^+} f(x) = -7$.\\
            左极限不存在,因此 $x=-7$ 是第二类间断点.
            \item 在 $x=1$ 处:\\
            $\lim\limits_{x \to 1^-} f(x) = 1$,
            $\lim\limits_{x \to 1^+} f(x) = \lim\limits_{x \to 1^+} (x-1)\sin\frac{1}{x-1} = 0$,\\
            $f(1) = 1$.\\
            左右极限存在但不相等,因此 $x=1$ 是第一类间断点中的跳跃间断点.
        \end{itemize}

        \item 当 $x \ne 2$ 时,$f(x) = \frac{(x-2)(x+2)}{x-2} = x+2$.\\
        $\lim\limits_{x \to 2} f(x) = \lim\limits_{x \to 2} (x+2) = 4$,但 $f(2) = 4$,函数值与极限值相等.\\
        因此函数在 $x=2$ 处连续,无间断点.
    \end{enumerate}
\end{solution}

\begin{exercise}[2.1.7]
    试确定 $a$, 使得函数 $f(x) = \begin{cases} \mathrm{e}^x, & x < 0, \\ a+x, & x \ges 0 \end{cases}$ 在 $x=0$ 处连续.
\end{exercise}

\begin{solution}
    由 $f(0) = a$, $\lim_{x \to 0^-} f(x) = \mathrm{e}^0 = 1$, $\lim_{x \to 0^+} f(x) = a + 0 = a$ 可知, 当 $a=1$ 时, $\lim_{x \to 0} f(x) = f(0)=1$, 即 $f(x)$ 在 $x=0$ 处连续.
\end{solution}

\begin{exercise}[2.1.8]
    证明: 函数 $f(x) = \begin{cases} \frac{\mathrm{e}^{1/x}-\mathrm{e}^{-1/x}}{\mathrm{e}^{1/x}+\mathrm{e}^{-1/x}}, & x \ne 0, \\ 1, & x=0 \end{cases}$, 在点 $0$ 处右连续, 但不左连续.
\end{exercise}

\begin{solution}
    因为 $\lim_{x \to 0^+} f(x) = \lim_{x \to 0^+} \frac{\mathrm{e}^{1/x}-\mathrm{e}^{-1/x}}{\mathrm{e}^{1/x}+\mathrm{e}^{-1/x}} = \lim_{x \to 0^+} \frac{1-\mathrm{e}^{-2/x}}{1+\mathrm{e}^{-2/x}} = 1 = f(0)$, 故 $f(x)$ 在点 $0$ 处右连续. 又因为 $\lim_{x \to 0^-} f(x) = \lim_{x \to 0^-} \frac{\mathrm{e}^{1/x}-\mathrm{e}^{-1/x}}{\mathrm{e}^{1/x}+\mathrm{e}^{-1/x}} = \lim_{x \to 0^-} \frac{\mathrm{e}^{2/x}-1}{\mathrm{e}^{2/x}+1} = -1 \ne f(0)$, 故 $f(x)$ 在点 $0$ 处不左连续.
\end{solution}

\begin{exercise}[2.1.9]
    证明: 对每个实数 $x$, $\lim_{n \to \infty} \frac{1+x}{1+x^{2n}}$ 存在. 将该极限值记为 $f(x)$, 试讨论函数 $f(x)$ 的连续性.
\end{exercise}

\begin{solution}
    \begin{align*}
        f(x)=\lim_{n \to \infty}\frac{1+x}{1+x^{2n}}&=(1+x)\lim_{n \to \infty}\frac{1}{1+(x^2)^n}
        =\begin{cases}
            1+x,&|x|<1;\\
            \frac{1+x}{2},&x=\pm1;\\
            0,&|x|>1.
        \end{cases}
    \end{align*}
    即\begin{align*}
        f(x)=\begin{cases}
            1+x,&-1<x<1;\\
            1,&x=1;\\
            0,&x>1\text{或}x\les -1.
        \end{cases}
    \end{align*}
    因此,\\
    $\lim_{x \to -1^-} f(x) = \lim_{x \to -1^+} f(x) = 0 = f(-1)$, 故 $f(x)$ 在 $x=-1$ 处连续;\\
     $\lim_{x \to 1^-} f(x) = 2 \ne f(1)$, 故 $f(x)$ 在 $x=1$ 处不连续; 对于其他点, $f(x)$ 均连续.\\
    综上所述, $f(x)$ 在 $(-\infty, 1) \cup (1, +\infty)$ 上连续, 在 $x=1$ 处不连续.
\end{solution}

\begin{exercise}[2.1.10]
    证明: 若函数 $f(x)$ 在点 $x_0$ 连续, 则存在一个正数 $\delta$, 使得函数 $f(x)$ 在区间 $(x_0-\delta, x_0+\delta)$ 上有界. (这一结果称为连续函数的局部有界性.)
\end{exercise}

\begin{solution}
    因为 $f(x)$ 在点 $x_0$ 连续, 故 $\lim_{x \to x_0} f(x) = f(x_0)$, 即 $\forall \varepsilon > 0$, $\exists \delta > 0$, 当 $|x-x_0| < \delta$ 时, 有 $|f(x)-f(x_0)| < \varepsilon$. \\
    取 $\varepsilon = 1$, 则当 $|x-x_0| < \delta$ 时, 有 $|f(x)-f(x_0)| < 1$, 即 $-1 < f(x)-f(x_0) < 1$, 故 $f(x_0)-1 < f(x) < f(x_0)+1$. \\
    因此, 当 $x \in (x_0-\delta, x_0+\delta)$ 时, 有 $|f(x)| \les \max\{|f(x_0)-1|, |f(x_0)+1|\}$, 即 $f(x)$ 在区间 $(x_0-\delta, x_0+\delta)$ 上有界.
\end{solution}

\begin{exercise}[2.1.11]
    证明: 若函数 $f(x)$ 在点 $x_0$ 连续, 且 $f(x_0) \ne 0$, 则存在一个正数 $\delta$, 使得函数 $f(x)$ 在区间 $(x_0-\delta, x_0+\delta)$ 上与 $f(x_0)$ 同号. (这一结果称为连续函数的局部保号性) 进一步, 存在某个正数 $\gamma$, 使得 $f(x)$ 在这一区间中满足 $|f(x)| \ges \gamma$.
\end{exercise}

\begin{solution}
    因为 $f(x)$ 在点 $x_0$ 连续, 故 $\lim_{x \to x_0} f(x) = f(x_0)$, 即 $\forall \varepsilon > 0$, $\exists \delta > 0$, 当 $|x-x_0| < \delta$ 时, 有 $|f(x)-f(x_0)| < \varepsilon$. \\
    取 $\varepsilon = \frac{|f(x_0)|}{2}$, 则当 $|x-x_0| < \delta$ 时, 有 $|f(x)-f(x_0)| < \frac{|f(x_0)|}{2}$,即 $-\frac{|f(x_0)|}{2} < f(x)-f(x_0) < \frac{|f(x_0)|}{2}$, 故 $f(x_0)-\frac{|f(x_0)|}{2} < f(x) < f(x_0)+\frac{|f(x_0)|}{2}$. \\
    因为 $f(x_0) \ne 0$, 故当 $x \in (x_0-\delta, x_0+\delta)$ 时, 有
    \begin{itemize}
        \item $f(x_0) > 0$时$$ f(x) > f(x_0) - \frac{f(x_0)}{2} = \frac{f(x_0)}{2} > 0. $$
        \item $f(x_0) < 0$时$$ f(x) < f(x_0) - \frac{f(x_0)}{2} = \frac{f(x_0)}{2} < 0. $$
    \end{itemize}
    
    因此, 当 $x \in (x_0-\delta, x_0+\delta)$ 时, 有 $f(x)$ 与 $f(x_0)$ 同号. 进一步地, 取 $\gamma = \frac{|f(x_0)|}{2}$, 则当 $x \in (x_0-\delta, x_0+\delta)$ 时, 有 $|f(x)| \ges \gamma$.
\end{solution}

\begin{exercise}[2.1.12]
    证明: 若 $\lim_{x \to x_0} g(x) = a \ne g(x_0)$ (从而 $x_0$ 为 $g(x)$ 的可去间断点), $f(u)$ 在 $u=a$ 处连续, 则
    $$ \lim_{x \to x_0} f(g(x)) = f\left(\lim_{x \to x_0} g(x)\right) = f(a). $$
    (这一结论对其他五种极限过程也成立.)
\end{exercise}

\begin{solution}
    由于$\lim_{u \to a} f(u) = f(a)$, 故$\forall \varepsilon > 0$, $\exists \delta_1 > 0$, 当 $|u-a| < \delta_1$ 时, 有 $|f(u)-f(a)| < \varepsilon$. \\
    又因为 $\lim_{x \to x_0} g(x) = a$, 故对于$ \delta_1 > 0$, $\exists \delta_2 > 0$, 当 $|x-x_0| < \delta_2$ 时, 有 $|g(x)-a| < \delta_1$.\\
    因此, 当 $|x-x_0| < \delta_2$ 时, 有 $|g(x)-a| < \delta_1$, 即 $|f(g(x))-f(a)| < \varepsilon$.\\
    综上所述, $\lim_{x \to x_0} f(g(x)) = f(a)$.
\end{solution}

\begin{exercise}[2.1.13]
    证明: 若函数 $u(x), v(x)$ 在 $x_0$ 处连续, 且 $u(x_0) > 0$, 则函数 $u(x)^{v(x)}$ 也在点 $x_0$ 处连续.
\end{exercise}

\begin{proof}
    利用$\e^x$在$\R$上连续,$\ln x$在$(0,+\infty)$上连续,以及复合函数的极限可交换性,有
    $$\lim_{x \to x_0} u(x)^{v(x)} = \lim_{x \to x_0} \e^{v(x) \ln u(x)} = \e^{\lim\limits_{x \to x_0} v(x) \ln u(x)} = \e^{v(x_0) \ln u(x_0)} = u(x_0)^{v(x_0)}.$$
\end{proof}

\begin{exercise}[2.1.14]
    设 $f(x)$ 在 $\mathbb{R}$ 上连续, 且对于任意 $x$ 有 $f(2x)=f(x)$. 求证 $f(x)$ 是常数.
\end{exercise}

\begin{solution}
    即证:$f(x) \equiv f(0)$. 对于任意点$x_0 \in \R$,考虑任意以$x_0$为极限的数列,$\{x_n\}$,则由连续性$$f(x_0)=\lim_{n \to \infty} f(x_n) = \lim_{n \to \infty} f \left( \frac{x_n}{2^n} \cdot 2^n \right) = \lim_{n \to \infty} f\left(\frac{x_n}{2^n}\right) = f(0),$$且$f(x_0) = f(0).$由于$x_0$的任意性,故$f(x) \equiv f(0)$.
\end{solution}

\begin{exercise}[2.1.15]
    设 $f(x)$ 在 $\mathbb{R}$ 上连续, 且对于任意 $x,y$ 有 $f(x+y)=f(x)+f(y)$. 求证 $f(x)=cx$, 其中 $c$ 是常数.
\end{exercise}

\begin{solution}
    \begin{enumerate}
        \item 由$f(0) = f(0+0) = f(0)+f(0)$可知,$f(0) = 0$.
        \item 对$n\in \N$,$$f(nx)=f((n-1)x+x)=f((n-1)x)+f(x)=\cdots = \underset{n\text{个}}{\underbrace{f(x)+f(x)+\cdots +f(x)}}=nf(x).$$
        即对任意自然数$n\in \N,f(nx)=nf(x)$.
        \item $$f(-x) + f(x) = f(0) = 0\Rightarrow f(-x) = -f(x)$$
        即对任意整数$k\in \Z,f(kx)=kf(x)$.
        \item 对$m,n\in \Z,n\neq 0$,$$f\left(\frac{m}{n}x\right) = mf\left(\frac{1}{n}x\right) = m\cdot\frac{1}{n}\cdot nf\left(\frac{1}{n}x\right)=\frac{m}{n}f\left(x\right)$$
        即对任意有理数$r=\frac{m}{n}\in \Q,f(rx)=rf(x)$.
        \item 对$x \in \R$,则存在有理数列$\{r_n\}$使得$\lim_{n \to \infty} r_n = x,$因此,$$\lim_{n \to \infty} f(r_n) = \lim_{n \to \infty} r_n f(1) = x f(1).$$
        由于$f$在$x$处连续,故$$f(x)=f(\lim_{n \to \infty} r_n ) =\lim_{n \to \infty} f(r_n) = x f(1).$$
        取$c=f(1)$,即有$f(x)=xf(1)=cx.$
    \end{enumerate}
\end{solution}

\begin{exercise}[2.1.16]
    当 $x \to 0$ 时, 用 $\sin x \sim x, \tan x \sim x$ 证明 $\arcsin x \sim x, \arctan x \sim x$; 用 $\ln(1+x) \sim x$ 证明 $(\mathrm{e}^x-1) \sim x$.

    (上述的等价无穷小, 是微积分中非常基本的事实.)
\end{exercise}

\begin{solution}
    \begin{enumerate}
        \item $f(x) = \begin{cases}
            \frac{\sin x}{x},&x\neq 0\\
            1,&x=0
        \end{cases},$ 则 $f(0)=\lim_{x \to 0} f(x) = 1$,即$f(x)$在$x=0$连续.\\ 
        设 $g(x) = \arcsin x$, 则 $\lim_{x \to 0} g(x) = 0$. \\
        因此$$\lim_{x \to 0} \frac{x}{\arcsin x } = \lim_{x \to 0} \frac{\sin  \arcsin x}{\arcsin x }= \lim_{x \to 0} f(g(x)) = f \left( \lim_{x \to 0} g(x) \right) = f(0) = 1,$$ 
        即 $\arcsin x \sim x$;
        \item $f(x) = \begin{cases}
            \frac{\tan x}{x},&x\neq 0\\
            1,&x=0
        \end{cases},$则 $f(0)=\lim_{x \to 0} f(x) = 1$,即$f(x)$在$x=0$连续.\\ 
        设 $g(x) = \arctan x$, 则 $\lim_{x \to 0} g(x) = 0$.\\
         因此$$\lim_{x \to 0} \frac{x}{\arctan x} = \lim_{x \to 0} \frac{\tan \arctan x}{\arctan x}=\lim_{x \to 0} f(g(x)) = f \left( \lim_{x \to 0} g(x) \right) = f(0) = 1,$$
         即 $\arctan x \sim x$;
        \item $f(x) = \begin{cases}
            \frac{\ln(1+x)}{x},&x\neq 0\\
            1,&x=0
        \end{cases}$, 则 $f(0)=\lim_{x \to 0} f(x) = 1$,即$f(x)$在$x=0$连续.\\ 
         设 $g(x) = \e^x - 1$, 则 $\lim_{x \to 0} g(x) = 0$.\\
        因此$$\lim_{x \to 0} \frac{x}{\e^x-1}=\lim_{x \to 0} \frac{\ln(1+\e^{x}-1)}{\e^x-1} = \lim_{x \to 0} f(g(x)) = f \left( \lim_{x \to 0} g(x) \right) = f(0) = 1, $$即 $(\mathrm{e}^x - 1) \sim x$.
    \end{enumerate}
\end{solution}

\begin{exercise}[2.1.17]
    求极限:
    \begin{tasks}[label=(\arabic*)](2)
        \task $\lim_{x \to 0} \frac{\sqrt{1+x+x^2}-1}{\sin 2x}$;
        \task $\lim_{x \to 0} \frac{\sqrt{1+x^2}-1}{1-\cos x}$;
        \task $\lim_{x \to 0} \frac{(\sqrt[10]{1+\tan x}-1)(\sqrt{1+x}-1)}{2x\sin x}$;
        \task $\lim_{x \to 0} \frac{x \cdot \arcsin(\sin x)}{1-\cos x}$;
        \task $\lim_{x \to 0} \frac{1-\cos(1-\cos x)}{x^4}$;
        \task $\lim_{x \to -\infty} x(\sqrt{x^2+100}+x)$;
        \task $\lim_{x \to +\infty} (\sin\sqrt{x+1}-\sin\sqrt{x})$;
        \task $\lim_{x \to \infty} \sqrt{2-\frac{\sin x}{x}}$.
    \end{tasks}
\end{exercise}

\begin{solution}
    \begin{enumerate}
        \item $\sqrt{1+x+x^2}-1 \sim \frac{1}{2}(x+x^2) , \sin 2x \sim 2x,(x\to0)$, 故 $$\lim_{x \to 0} \frac{\sqrt{1+x+x^2}-1}{\sin 2x} = \lim_{x \to 0} \frac{\frac{1}{2}(x+x^2)}{2x} = \frac{1}{4};$$
        \item $\sqrt{1+x^2}-1 \sim \frac{1}{2}x^2 , 1-\cos x \sim \frac{1}{2}x^2,(x\to0)$, 故 $$\lim_{x \to 0} \frac{\sqrt{1+x^2}-1}{1-\cos x} = \lim_{x \to 0} \frac{\frac{1}{2}x^2}{\frac{1}{2}x^2} = 1;$$
        \item $\sqrt[10]{1+\tan x}-1 \sim \frac{1}{10}\tan x \sim \frac{1}{10}x , \sqrt{1+x}-1 \sim \frac{1}{2}x , 2x\sin x \sim 2x^2,(x\to0)$, 故 $$\lim_{x \to 0} \frac{(\sqrt[10]{1+\tan x}-1)(\sqrt{1+x}-1)}{2x\sin x} = \lim_{x \to 0} \frac{\frac{1}{10}x \cdot \frac{1}{2}x}{2x^2} = \frac{1}{40};$$
        \item $x \cdot \arcsin(\sin x) \sim x^2 , 1-\cos x \sim \frac{1}{2}x^2,(x\to0)$, 故 $$\lim_{x \to 0} \frac{x \cdot \arcsin(\sin x)}{1-\cos x} = \lim_{x \to 0} \frac{x^2}{\frac{1}{2}x^2} = 2;$$
        \item $1-\cos(1-\cos x) \sim \frac{1}{2}(1-\cos x)^2 \sim \frac{1}{2} \left( \frac{1}{2}x^2 \right)^2 = \frac{1}{8}x^4,(x\to0)$, 故 $$\lim_{x \to 0} \frac{1-\cos(1-\cos x)}{x^4} = \lim_{x \to 0} \frac{\frac{1}{8}x^4}{x^4} = \frac{1}{8};$$
        \item 令$y=\frac{1}{x}$,则$y\to 0^+$,\\
        并且$x=-\frac{1}{y}$,即有$x(\sqrt{x^2+100}+x)=-\frac{1}{y}\left(\sqrt{\frac{1}{y^2}+100}-\frac{1}{y}\right)=-\frac{\sqrt{1+100y^2}-1}{y^2}$\\
        而$\sqrt{1+100y^2}-1\sim \frac{1}{2}\cdot 100y^2=50y^2,(y\to 0),$故
        \begin{align*}
            \lim_{x \to -\infty}x(\sqrt{x^2+100}+x) =\lim_{y \to 0^+}-\frac{\sqrt{1+100y^2}-1}{y^2}=-\lim_{y \to 0^+}\frac{50y^2}{y^2}=-50
        \end{align*} 
        \item \begin{align*}
            \left| \sin \sqrt{x+1}-\sin\sqrt{x} \right| &=\left| 2 \cos \frac{\sqrt{x+1}+\sqrt{x}}{2} \sin \frac{\sqrt{x+1}-\sqrt{x}}{2} \right|  \\
            &=2 \left|\cos \frac{\sqrt{x+1}+\sqrt{x}}{2}\sin \frac{1}{2\left(\sqrt{x+1}+\sqrt{x}\right)}\right|\\
            &\les 2\cdot1\cdot  \frac{1}{2\left(\sqrt{x+1}+\sqrt{x}\right)}=\frac{1}{\sqrt{x+1}+\sqrt{x}}
        \end{align*}
         由$\lim_{x \to +\infty} \frac{1}{\sqrt{x+1}+\sqrt{x}} = 0$可知,$\lim_{x \to +\infty} (\sin\sqrt{x+1}-\sin\sqrt{x}) = 0$;
        \item $$\lim_{x \to \infty} \sqrt{2-\frac{\sin x}{x}} = \sqrt{2 - \lim_{x \to \infty} \frac{\sin x}{x}} = \sqrt{2 - 0} = \sqrt{2}.$$
    \end{enumerate}
\end{solution}

% \begin{exercise}[2.1.18]
%     函数 $\sinh x = \frac{\mathrm{e}^x - \mathrm{e}^{-x}}{2}$ 与 $\cosh x = \frac{\mathrm{e}^x + \mathrm{e}^{-x}}{2}$ 分别称为双曲正弦与双曲余弦 (统称为双曲函数), 它们均在定义域 $(-\infty, +\infty)$ 上连续. 证明以下各题. (可与三角函数的性质作比较.)
%     \begin{tasks}[label=(\arabic*)](2)
%         \task $\sinh(-x) = -\sinh x, \quad \cosh(-x) = \cosh x$;
%         \task $\cosh^2 x - \sinh^2 x = 1$;
%         \task $\sinh 2x = 2\sinh x \cosh x$;
%         \task $\cosh 2x = \sinh^2 x + \cosh^2 x$;
%         \task $\sinh(x \pm y) = \sinh x \cosh y \pm \cosh x \sinh y$;
%         \task $\cosh(x \pm y) = \cosh x \cosh y \pm \sinh x \sinh y$.
%     \end{tasks}
% \end{exercise}

% \begin{solution}
%     \begin{enumerate}
%         \item $\sinh(-x) = \frac{\mathrm{e}^{-x} - \mathrm{e}^x}{2} = -\sinh x$, $\cosh(-x) = \frac{\mathrm{e}^{-x} + \mathrm{e}^x}{2} = \cosh x$;
%         \item $\cosh^2 x - \sinh^2 x = \left( \frac{\mathrm{e}^x + \mathrm{e}^{-x}}{2} \right)^2 - \left( \frac{\mathrm{e}^x - \mathrm{e}^{-x}}{2} \right)^2 = \frac{\mathrm{e}^{2x} + 2 + \mathrm{e}^{-2x}}{4} - \frac{\mathrm{e}^{2x} - 2 + \mathrm{e}^{-2x}}{4} = 1$;
%         \item $\sinh 2x = \frac{\mathrm{e}^{2x} - \mathrm{e}^{-2x}}{2} = \frac{(\mathrm{e}^x)^2 - (\mathrm{e}^{-x})^2}{2} = \frac{(\mathrm{e}^x - \mathrm{e}^{-x})(\mathrm{e}^x + \mathrm{e}^{-x})}{2} = 2\sinh x \cosh x$;
%         \item $\cosh 2x = \frac{\mathrm{e}^{2x} + \mathrm{e}^{-2x}}{2} = \frac{(\mathrm{e}^x)^2 + (\mathrm{e}^{-x})^2}{2} = \frac{(\mathrm{e}^x - \mathrm{e}^{-x})^2 + (\mathrm{e}^x + \mathrm{e}^{-x})^2}{4} = \sinh^2 x + \cosh^2 x$;
%         \item $\sinh(x \pm y) = \frac{\mathrm{e}^{x \pm y} - \mathrm{e}^{-(x \pm y)}}{2} =
%                   \frac{\mathrm{e}^x \mathrm{e}^{\pm y} - \mathrm{e}^{-x} \mathrm{e}^{\mp y}}{2} = \frac{\mathrm{e}^x \mathrm{e}^{\pm y} - \mathrm{e}^{-x} \mathrm{e}^{\pm y} + \mathrm{e}^{-x} \mathrm{e}^{\pm y} - \mathrm{e}^{-x} \mathrm{e}^{\mp y}}{2} =
%                   \left( \frac{\mathrm{e}^x - \mathrm{e}^{-x}}{2} \right) \left( \frac{\mathrm{e}^{\pm y} + \mathrm{e}^{\mp y}}{2} \right) + \left( \frac{\mathrm{e}^x + \mathrm{e}^{-x}}{2} \right) \left( \frac{\mathrm{e}^{\pm y} - \mathrm{e}^{\mp y}}{2} \right) =
%                   \sinh x \cosh y \pm \cosh x \sinh y$;
%         \item $\cosh(x \pm y) =
%                   \frac{\mathrm{e}^{x \pm y} + \mathrm{e}^{-(x \pm y)}}{2} =\frac{\mathrm{e}^x \mathrm{e}^{\pm y} + \mathrm{e}^{-x} \mathrm{e}^{\mp y}}{2}
%                   =\frac{\mathrm{e}^x \mathrm{e}^{\pm y} + \mathrm{e}^{-x} \mathrm{e}^{\pm y} - \mathrm{e}^{-x} \mathrm{e}^{\pm y} + \mathrm{e}^{-x} \mathrm{e}^{\mp y}}{2}
%                   = \left( \frac{\mathrm{e}^x + \mathrm{e}^{-x}}{2} \right) \left( \frac{\mathrm{e}^{\pm y} + \mathrm{e}^{\mp y}}{2} \right) + \left( \frac{\mathrm{e}^x - \mathrm{e}^{-x}}{2} \right) \left( \frac{\mathrm{e}^{\pm y} - \mathrm{e}^{\mp y}}{2} \right)
%                   = \cosh x \cosh y \pm \sinh x \sinh y$.
%     \end{enumerate}
% \end{solution}

\newpage
% \section{习题 2.2}

\begin{exercise}[2.2.1]
    证明函数 $x \cdot 2^x - 1$ 在 $[0,1]$ 内有零点.
\end{exercise}

\begin{solution}
    设 $f(x) = x \cdot 2^x - 1$, 则 $f(0) = -1 < 0, f(1) = 1 > 0$. 又因 $f(x)$ 在 $[0,1]$ 上连续, 故由介值定理知, 存在 $x_0 \in (0,1)$, 使得 $f(x_0) = 0$.
\end{solution}

\begin{exercise}[2.2.2]
    证明函数 $x - a\sin x - b$ (其中 $a,b$ 为正数) 在 $(0, +\infty)$ 上有零点, 且零点不超过 $a+b$.
\end{exercise}

\begin{solution}
    设 $f(x) = x - a\sin x - b$, 则 $f(0) = -b < 0, f(a+b) = a + b - a\sin(a+b) - b = a(1-\sin(a+b)) \ges 0$. 又因 $f(x)$ 在 $[0, a+b]$ 上连续, 故由介值定理知, 存在 $x_0 \in (0, a+b)$, 使得 $f(x_0) = 0$. 又因对任意 $x > a+b$ 有 $f(x) = x - a\sin x - b > a + b - a \sin x - b \ges 0$, 故 $f(x)$ 在 $(0, +\infty)$ 上的零点不超过 $a+b$.
\end{solution}

\begin{exercise}[2.2.3]
    证明函数 $x - \sin(x+1)$ 有实零点.
\end{exercise}

\begin{solution}
    设 $f(x) = x - \sin(x+1)$, 由$ -1 \les \sin(x+1) \les 1$知, 
    则 $f(-2) \les -2 + 1 = -1 < 0, f(2) \ges 2 - 1 = 1 > 0$. 又因 $f(x)$ 在 $[-2,2]$ 上连续, 故由介值定理知, 存在 $x_0 \in (-2,2)$, 使得 $f(x_0) = 0$.
\end{solution}

\begin{exercise}[2.2.4]
    设函数 $f(x)$ 在 $[a,b]$ 上连续, 且值域就是 $[a,b]$. 证明 $f(x)$ 在 $[a,b]$ 上必有不动点, 即有 $x_0 \in [a,b]$, 使得 $f(x_0)=x_0$.
\end{exercise}

\begin{solution}
    函数$f(x)$的值域为$[a,b]$, 故存在 $x_1, x_2 \in [a,b]$, 使得 $f(x_1) = a, f(x_2) = b$. 设 $g(x) = f(x) - x$, 则 $g(x)$ 在 $[a,b]$ 上连续, 且 $g(x_1) = a - x_1 \les 0, g(x_2) = b - x_2 \ges 0$. 故由介值定理知, 存在 $x_0 \in [a,b]$, 使得 $g(x_0) = 0$, 即 $f(x_0) = x_0$.
\end{solution}

\begin{exercise}[2.2.5]
    设函数 $f(x), g(x)$ 在区间 $[a,b]$ 上连续, 且 $f(a) > g(a), f(b) < g(b)$. 试证: 存在 $x_0 \in (a,b)$, 使得 $f(x_0)=g(x_0)$.
\end{exercise}

\begin{solution}
    设 $h(x) = f(x) - g(x)$, 则 $h(x)$ 在 $[a,b]$ 上连续, 且 $h(a) = f(a) - g(a) > 0, h(b) = f(b) - g(b) < 0$. 故由介值定理知, 存在 $x_0 \in (a,b)$, 使得 $h(x_0) = 0$, 即 $f(x_0) = g(x_0)$.
\end{solution}

\begin{exercise}[2.2.6]
    设函数 $f(x)$ 在 $[0, 2a]$ 上连续, 且 $f(0)=f(2a)$. 证明: 在区间 $[0,a]$ 上存在某个 $x_0$, 使得 $f(x_0) = f(x_0+a)$.
\end{exercise}

\begin{solution}
    设 $g(x) = f(x) - f(x+a)$, 则 $g(x)$ 在 $[0,a]$ 上连续, 且 $g(0) = f(0) - f(a), g(a) = f(a) - f(2a) = f(a) - f(0) = -g(0)$. 因此$g(0) g(a) = -(g(0))^2 \les 0$. 由介值定理知, 存在 $x_0 \in [0,a]$, 使得 $g(x_0) = 0$, 即 $f(x_0) = f(x_0+a)$.
\end{solution}

\begin{exercise}[2.2.17]
    若 $f(x)$ 在 $[a, +\infty)$ 上连续, 且 $\lim_{x \to +\infty} f(x)$ 存在, 则 $f(x)$ 在 $[a, +\infty)$ 上或者有最大值, 或者有最小值.
\end{exercise}

\begin{solution}
    设 $\lim_{x \to +\infty} f(x) = A$, 则对任意 $\varepsilon > 0$, 存在 $M > a$, 使得当 $x > M$ 时, 有 $|f(x) - A| < \varepsilon$. 因此 $f(x)$ 在区间 $[M, +\infty)$ 上有界. 又因 $f(x)$ 在 $[a,M]$ 上连续, 故由 Weierstrass 定理知, $f(x)$ 在 $[a,M]$ 上有最大值与最小值. 记 $f(x)$ 在 $[a,M]$ 上的最大值为 $f(x_1)$, 最小值为 $f(x_2)$. 则 $f(x)$ 在 $[a,+\infty)$ 上的最大值为 $\max\{f(x_1), A+\varepsilon\}$, 最小值为 $\min\{f(x_2), A-\varepsilon\}$. 因此 $f(x)$ 在 $[a,+\infty)$ 上或者有最大值, 或者有最小值.
\end{solution}

\begin{exercise}[2.2.18]
    设函数 $f(x)$ 定义在区间 $[a,b]$ 上, 满足条件: $a \les f(x) \les b$ (对任意 $x \in [a,b]$), 且对 $[a,b]$ 中任意的 $x,y$ 有 $|f(x)-f(y)| \les k|x-y|$. 这里 $k$ 是常数, $0<k<1$. 证明:
    \begin{enumerate}
        \item 存在唯一的 $x_0 \in [a,b]$, 使得 $f(x_0)=x_0$.
        \item 任取 $x_1 \in [a,b]$, 并定义数列 $\{x_n\}: x_{n+1} = f(x_n), n=1,2,\dots$, 则 $\lim_{n \to \infty} x_n = x_0$.
        \item 给出一个在实轴上的连续函数, 使得对任意 $x \ne y$ 有 $|f(x)-f(y)| < |x-y|$, 但方程 $f(x)-x=0$ 无解.
    \end{enumerate}
\end{exercise}

\begin{exercise}[2.2.9]
    证明函数 $f(x) = \frac{1+x^2}{1+x^2+x^4}$ 在 $(-\infty, +\infty)$ 上有界.
\end{exercise}

\begin{exercise}[2.2.10]
    是否有满足下面条件的连续函数? 说明理由.
    \begin{enumerate}
        \item 定义域为 $[0,1]$, 值域为 $(0, +\infty)$;
        \item 定义域为 $[0,1]$, 值域为 $(0,1)$;
        \item 定义域为 $[0,1]$, 值域为 $[0,1]\cup[2,4]$;
        \item 定义域为 $(0,1)$, 值域为 $(2, +\infty)$.
    \end{enumerate}
\end{exercise}

\begin{exercise}[2.2.11]
    举例说明, 对任意正数 $\varepsilon < \frac{b-a}{2}$, 函数 $f(x)$ 在闭区间 $[a+\varepsilon, b-\varepsilon]$ 上有界, 不能保证 $f(x)$ 在开区间 $(a,b)$ 上有界. (比较习题 2.1 第 2 题.)
\end{exercise}

\begin{exercise}[2.2.12]
    设 $y=f(x)$ 在开区间 $I=(a,b)$ 上连续并严格单调. 证明 $y=f(x)$ 的值域 $f(I)$ 也是一个开区间.
\end{exercise}

\begin{exercise}[2.2.13]
    设函数 $f(x)$ 在有限区间 $(a,b)$ 上一致连续. 求证 $f(x)$ 在 $a$ 点的右极限和在 $b$ 点的左极限都存在.
\end{exercise}

\begin{exercise}[2.2.14]
    设函数 $f(x)$ 在 $(0, +\infty)$ 上一致连续, $\{a_n\}$ 是正收敛数列. 求证 $\{f(a_n)\}$ 也收敛. 又问假没 $f(x)$ 连续时, 结论是否还成立, 为什么?
\end{exercise}

\begin{exercise}[2.2.15]
    设函数 $f(x)$ 在 $(-\infty, +\infty)$ 上连续, $\{a_n\}$ 是收敛数列. 求证 $\{f(a_n)\}$ 也收敛.
\end{exercise}

\begin{exercise}[2.2.16]
    给出一个在 $(-\infty, +\infty)$ 上连续且有界但不一致连续的函数.
\end{exercise}

\newpage
% \section{第2章综合习题}

\begin{exercise}[2.C.1]
    证明: 函数 $f(x) = \begin{cases} 0, & x \text{ 为有理数}, \\ x, & x \text{ 为无理数} \end{cases}$ 仅在点 $x=0$ 处连续.
\end{exercise}

\begin{solution}
    设 $x_0 \in \mathbb{R}$, 则对任意 $\varepsilon > 0$, 存在有理数列 $\{r_n\}$ 与无理数列 $\{s_n\}$, 使得 $\lim_{n \to \infty} r_n = x_0, \lim_{n \to \infty} s_n = x_0$. 因此
    $$ \lim_{n \to \infty} f(r_n) = \lim_{n \to \infty} 0 = 0, \quad \lim_{n \to \infty} f(s_n) = \lim_{n \to \infty} s_n = x_0. $$
    当 $x_0 \ne 0$ 时, $\lim_{n \to \infty} f(r_n) = 0, \lim_{n \to \infty} f(s_n) = x_0$, 故 $\lim_{x \to x_0} f(x)$ 不存在, 因此 $f(x)$ 在 $x=x_0$ 处不连续.

    当$x_0 = 0$时,对任意$\varepsilon > 0$,取$\delta = \varepsilon$,则当$|x-0| < \delta$时,有$|f(x)-f(0)| = |f(x)| \les |x| < \delta = \varepsilon$.因此$\lim_{x \to 0} f(x) = f(0) = 0$,即$f(x)$在$x=0$处连续.
\end{solution}

\begin{exercise}[2.C.2]
    设 $x_1, x_2, \dots, x_n \in [0,1]$, 记 $f(x) = \frac{|x-x_1|+\dots+|x-x_n|}{n}$, 证明: 存在 $x_0 \in [0,1]$, 使得 $f(x_0) = \frac{1}{2}$.
\end{exercise}

\begin{solution}
    设 $g(x) = f(x) - \frac{1}{2}$, 则 $g(x)$ 在 $[0,1]$ 上连续, 且
    \begin{align*}
        f(0) + f(1) & = \frac{|0-x_1|+\dots+|0-x_n|}{n} + \frac{|1-x_1|+\dots+|1-x_n|}{n} \\
                    & = \frac{(x_1 + (1-x_1)) + \dots + (x_n + (1-x_n))}{n} = 1.
    \end{align*}
    因此 $g(0) + g(1) = f(0) + f(1) - 1 = 0$. 则$g(0) g(1) = -(g(0))^2 \les 0$. 结合零点定理可简单推知, 存在 $x_0 \in [0,1]$, 使得 $g(x_0) = 0$, 即 $f(x_0) = \frac{1}{2}$.
\end{solution}

\begin{exercise}[2.C.3]
    证明: 函数 $\frac{a_1}{x-\lambda_1} + \frac{a_2}{x-\lambda_2} + \frac{a_3}{x-\lambda_3}$ (其中 $a_1, a_2, a_3 > 0$, 且 $\lambda_1 < \lambda_2 < \lambda_3$) 在 $(\lambda_1, \lambda_2)$ 与 $(\lambda_2, \lambda_3)$ 内各有一个零点.
\end{exercise}

\begin{solution}
    [解]
    仅证明 $(\lambda_1, \lambda_2)$ 内有一个零点, $(\lambda_2, \lambda_3)$ 内的证明类似.

    由$\frac{a_2}{x-\lambda_2} + \frac{a_3}{x-\lambda_3}$在$\left[\lambda_1, \frac{\lambda_1+\lambda_2}{2}\right]$上连续,
    因此有界,即存在$M_1 > 0$,使得对任意$x \in \left[\lambda_1, \frac{\lambda_1+\lambda_2}{2}\right]$有$|\frac{a_2}{x-\lambda_2} + \frac{a_3}{x-\lambda_3}| \les M$.

    又由$\lim_{x \to \lambda_1^+} \frac{a_1}{x-\lambda_1} = +\infty$,因此存在$\delta_1 \in \left( 0, \frac{\lambda_2 - \lambda_1}{2} \right)$,使得对任意$x \in (\lambda_1, \lambda_1 + \delta_1)$有$\frac{a_1}{x-\lambda_1} > M_1$.因此,存在$x_1 \in (\lambda_1, \lambda_1 + \delta_1) \subset \left(\lambda_1, \frac{\lambda_1+\lambda_2}{2}\right)$,使得
    $$ \frac{a_1}{x_1-\lambda_1} + \frac{a_2}{x_1-\lambda_2} + \frac{a_3}{x_1-\lambda_3} = \frac{a_1}{x_1-\lambda_1} + \left( \frac{a_2}{x_1-\lambda_2} + \frac{a_3}{x_1-\lambda_3} \right) > M_1 - M_1 = 0. $$

    由$\frac{a_1}{x-\lambda_1} + \frac{a_3}{x-\lambda_3}$在$\left[\frac{\lambda_1+\lambda_2}{2}, \lambda_2\right]$上连续,
    因此有界,即存在$M_2 > 0$,使得对任意$x \in \left[\frac{\lambda_1+\lambda_2}{2}, \lambda_2\right]$有$\left|\frac{a_1}{x-\lambda_1} + \frac{a_3}{x-\lambda_3}\right| \les M_2$.

    又由$\lim_{x \to \lambda_2^-} \frac{a_2}{x-\lambda_2} = -\infty$,因此存在$\delta_2 \in \left( 0, \frac{\lambda_2 - \lambda_1}{2} \right)$,使得对任意$x \in (\lambda_2 - \delta_2, \lambda_2)$有$\frac{a_2}{x-\lambda_2} < -M_2$.因此,存在$x_2 \in (\lambda_2 - \delta_2, \lambda_2) \subset \left(\frac{\lambda_1+\lambda_2}{2}, \lambda_2\right)$,使得
    $$ \frac{a_1}{x_2-\lambda_1} + \frac{a_2}{x_2-\lambda_2} + \frac{a_3}{x_2-\lambda_3} = \left( \frac{a_1}{x_2-\lambda_1} + \frac{a_3}{x_2-\lambda_3} \right) + \frac{a_2}{x_2-\lambda_2} < M_2 - M_2 = 0. $$

    综上,存在$x_1, x_2 \in (\lambda_1, \lambda_2)$,使得对于函数$f(x) = \frac{a_1}{x-\lambda_1} + \frac{a_2}{x-\lambda_2} + \frac{a_3}{x-\lambda_3}$有$f(x_1) > 0, f(x_2) < 0$.由介值定理知,存在$x_0 \in (x_1, x_2) \subset (\lambda_1, \lambda_2)$,使得$f(x_0) = 0$.

    同时由于$\frac{a_1}{x-\lambda_1}, \frac{a_2}{x-\lambda_2}, \frac{a_3}{x-\lambda_3}$在$(\lambda_1, \lambda_2)$上单调递减,因此$f(x)$在$(\lambda_1, \lambda_2)$上单调递减.因此,零点$x_0$唯一.
\end{solution}

\begin{exercise}[2.C.4]
    设 $f(x)$ 是一个多项式, 则必存在一点 $x_0$, 使得 $|f(x_0)| \les |f(x)|$ 对任意实数 $x$ 成立.
\end{exercise}

\begin{solution}
    设 $f(x) = a_n x^n + a_{n-1} x^{n-1} + \dots + a_1 x + a_0$, 其中 $a_n \ne 0, n \ges 1$. 则
    $$ |f(x)| = |x|^n \left| a_n + \frac{a_{n-1}}{x} + \dots + \frac{a_1}{x^{n-1}} + \frac{a_0}{x^n} \right|. $$
    因为 $\lim_{x \to \infty} \left( a_n + \frac{a_{n-1}}{x} + \dots + \frac{a_1}{x^{n-1}} + \frac{a_0}{x^n} \right) = a_n$, 因此存在 $M > 0$, 使得当 $|x| > M$ 时, 有
    $$ \left| a_n + \frac{a_{n-1}}{x} + \dots + \frac{a_1}{x^{n-1}} + \frac{a_0}{x^n} \right| > \frac{|a_n|}{2}. $$
    由$\lim_{x \to \infty} \frac{|a_n|}{2} |x|^n = +\infty$,因此$\exists X > M$,使得当$|x| > X$时,有$\frac{|a_n|}{2} |x|^n > |f(0)|$.而由$|f(x)|$在$[-X,X]$上连续,故由最值性知,存在$x_0 \in [-X,X]$,使得$|f(x_0)| = \inf \{ f(x): x \in [-X,X] \}$.特别的,对任意$x \in [-X,X]$有$|f(x_0)| \les |f(0)|$.因此对于$|x| > X$时,有$|f(x)| > \frac{|a_n|}{2}|x|^n \ges |f(x_0)|$.综上,对任意$x \in \mathbb{R}$有$|f(x_0)| \les |f(x)|$.
\end{solution}

\begin{exercise}[2.C.5]
    设 $f(x)$ 在区间 $[0,1]$ 上连续, 且 $f(0)=f(1)$. 证明: 对任意正整数 $n$, 在区间 $\left[0, 1-\frac{1}{n}\right]$ 中有一点 $\xi$, 使得 $f(\xi) = f\left(\xi+\frac{1}{n}\right)$.
\end{exercise}

\begin{solution}
    设 $g(x) = f(x) - f\left(x+\frac{1}{n}\right)$, 则 $g(x)$ 在 $\left[0, 1-\frac{1}{n}\right]$ 上连续, 且
    \begin{align*}
        g(0) & = f(0) - f\left(\frac{1}{n}\right), \\
        g\left(\frac{1}{n}\right) & = f\left(\frac{1}{n}\right) - f\left(\frac{2}{n}\right), \\
        & \dots \\
        g\left(\frac{n-1}{n}\right) & = f\left(\frac{n-1}{n}\right) - f(1) = f\left(\frac{n-1}{n}\right) - f(0).
    \end{align*}
    因此$\frac{1}{n} \left( g(0) + g\left(\frac{1}{n}\right) + \dots + g\left(\frac{n-1}{n}\right) \right) = 0$.则$g(0), g\left(\frac{1}{n}\right), \dots, g\left(\frac{n-1}{n}\right)$中至少有一个非正, 另一个非负. 由介值定理知, 存在 $\xi \in \left[0, 1-\frac{1}{n}\right]$, 使得 $g(\xi) = 0$, 即 $f(\xi) = f\left(\xi+\frac{1}{n}\right)$.
\end{solution}

\begin{exercise}[2.C.6]
    证明: 存在一个实数 $x$, 满足 $x^5 + \frac{\cos x}{1+x^2+\sin^2 x} = 72$.
\end{exercise}

\begin{solution}
    设$f(x)=x^5 + \frac{\cos x}{1+x^2+\sin^2 x} ,$
    \begin{align*}
        f(3) &= 3^5 + \frac{\cos 3}{1+3^2+\sin^2 3}  \ges 243 \frac{1}{1+3^2-1} > 72,\\
        f(-3) &= (-3)^5 + \frac{\cos (-3)}{1+(-3)^2+\sin^2 (-3)} \les -243 + \frac{1}{1+(-3)^2-1} <72.
    \end{align*}
    由介值定理知, 存在 $x \in [-3,3]$, 使得 $f(x) = 72$.
\end{solution}

\begin{exercise}[2.C.7]
    若 $f(x)$ 在 $[a, +\infty)$ 上连续, 且 $\lim_{x \to +\infty} f(x)$ 存在, 则 $f(x)$ 在 $[a, +\infty)$ 上或者有最大值, 或者有最小值.
\end{exercise}

\begin{solution}
    记$$S = \sup \{ f(x): x \in [a, +\infty) \}, \ I = \inf \{ f(x): x \in [a, +\infty) \}, \ L = \lim_{x \to +\infty} f(x).$$
    \begin{enumerate}
        \item 若$S > L$,取$\varepsilon = \frac{S-L}{2} > 0$,则存在$X > a$,使得对任意$x > X$有$|f(x)-L| < \varepsilon$,即$L - \varepsilon < f(x) < L + \varepsilon$.因此,对任意$x > X$有$f(x) < L + \varepsilon = \frac{S+L}{2} < S$.因此$$\sup \{ f(x): x \in [a, X] \} = S,$$由闭区间上连续函数的最值性,知存在$x_0 \in [a,X]$,使得$f(x_0) = S$.
        \item 若$I < L$,取$\varepsilon = \frac{L-I}{2} > 0$,则存在$X > a$,使得对任意$x > X$有$|f(x)-L| < \varepsilon$,即$L - \varepsilon < f(x) < L + \varepsilon$.因此,对任意$x > X$有$f(x) > L - \varepsilon = \frac{I+L}{2} > I$.因此$$\inf \{ f(x): x \in [a, X] \} = I,$$由闭区间上连续函数的最值性,知存在$x_0 \in [a,X]$,使得$f(x_0) = I$.
        \item 若$S = L = I$,则$f(x) \equiv L$,即任取$x_0 \in [a,+\infty)$,均有$f(x_0) = L$同时为最大值和最小值.
    \end{enumerate}
    \begin{remark}
        一个只有极限没有最大值的例子是$f(x) = \arctan x, x \in [0,+\infty)$.
    \end{remark}
\end{solution}

\begin{exercise}[2.C.8]
    设函数 $f(x)$ 定义在区间 $[a,b]$ 上, 满足条件: $a \les f(x) \les b$ (对任意 $x \in [a,b]$), 且对 $[a,b]$ 中任意的 $x,y$ 有 $|f(x)-f(y)| \les k|x-y|$. 这里 $k$ 是常数, $0<k<1$. 证明:
    \begin{enumerate}
        \item 存在唯一的 $x_0 \in [a,b]$, 使得 $f(x_0)=x_0$.
        \item 任取 $x_1 \in [a,b]$, 并定义数列 $\{x_n\}: x_{n+1} = f(x_n), n=1,2,\dots$, 则 $\lim_{n \to \infty} x_n = x_0$.
        \item 给出一个在实轴上的连续函数, 使得对任意 $x \ne y$ 有 $|f(x)-f(y)| < |x-y|$, 但方程 $f(x)-x=0$ 无解.
    \end{enumerate}
\end{exercise}

\begin{solution}
    \begin{enumerate}
        \item 先证明$f(x)$在$[a,b]$上连续:
              设$x_0 \in [a,b]$,对任意$\ve > 0$,取$\delta = \frac{\ve}{k} > 0$,则对任意$x \in [a,b]$且$|x-x_0| < \delta$,有
              $$|f(x)-f(x_0)| \les k|x-x_0| < k\delta = \ve.$$

              设 $g(x) = f(x) - x$, 则 $g(x)$ 在 $[a,b]$ 上连续, 且 $g(a) = f(a) - a \ges 0, g(b) = f(b) - b \les 0$. 故结合零点定理可简单推知, 存在 $x_0 \in [a,b]$, 使得 $g(x_0) = 0$, 即 $f(x_0) = x_0$. 又因对任意 $x,y \in [a,b]$ 有
              $$|g(x)-g(y)| = |f(x)-f(y)-(x-y)| \ges |x-y| - |f(x)-f(y)| \ges (1-k)|x-y|,$$
              故若存在$x_1 \ne x_0$使得$f(x_1) = x_1$,则
              $$(1-k)|x_1-x_0| \les |g(x_1)-g(x_0)| = |f(x_1)-f(x_0)-(x_1-x_0)| = 0,$$
              即$x_1 = x_0$. 因此$x_0$唯一.
        \item \begin{enumerate}
                  \item 若$x_2 = x_1$,则由$f(x_1) = x_1$以及(1)中所述的唯一性,知$x_2 = x_1 = x_0$,则$x_3 = f(x_2) = f(x_0) = x_0$,依此类推,有$x_n = x_0$对任意$n \ges 1$成立.因此$\lim_{n \to \infty} x_n = x_0$.
                  \item 若$x_2 \ne x_1$,对任意 $n \ges 1$ 有
                        $$|x_{n+1} - x_n| = |f(x_n) - f(x_{n-1})| \les k|x_n - x_{n-1}| \les \dots \les k^{n-1}|x_2 - x_1|.$$
                        因此, 对任意 $m > n \ges 1$ 有
                        \begin{align*}
                            |x_m - x_n| & \les |x_m - x_{m-1}| + |x_{m-1} - x_{m-2}| + \dots + |x_{n+1} - x_n| \\
                                        & \les (k^{m-2} + k^{m-3} + \dots + k^{n-1})|x_2 - x_1|                \\
                                        & \les \frac{k^{n-1}}{1-k}|x_2 - x_1|.
                        \end{align*}
                        $\forall \ve > 0, \exists N = \left\lceil \log_k \frac{(1-k)\ve}{|x_2 - x_1|} \right\rceil$, 当$m,n > N$时, 有
                        $$|x_m - x_n| \les \frac{k^{n-1}}{1-k}|x_2 - x_1| < \ve.$$
                        故数列$\{x_n\}$为Cauchy列, 故存在$\lim_{n \to \infty} x_n:=a$存在. 又因$f(x)$在$[a,b]$上连续,对递推式两侧取极限,有
                        $$a = \lim_{n \to \infty} x_{n+1} = \lim_{n \to \infty} f(x_n) = f\left( \lim_{n \to \infty} x_n \right) = f(a).$$
                        由(1)中所述的唯一性,知$a = x_0$. 因此$\lim_{n \to \infty} x_n = x_0$.
              \end{enumerate}
        \item 一个不太严谨的思考过程:我想要构造一个满足$|f(x)-f(y)| < |x-y|$的函数, 我考虑了$|f'(x)|<1$的趋势为增的函数, 同时$f(x)-x$无解要求了$f(x)$应该是贴在$y=x$的(不妨设为)上方的, 于是我考虑了$f(x) = x + g(x)$,并假设$g(x)$可导, 其中如果$-1 < -g'(x) < 0$,那就能保证$f(x)$的导数满足要求.(当然上述思路中,对$g(x)$的选取过程都只是必要的)
              $$f(x) = x + \frac{1}{1 + \e^{x}}.$$
              \begin{enumerate}
                  \item 满足$|f(x)-f(y)| < |x-y|$:
                        设$x > y$,则
                        \begin{align*}
                            |f(x)-f(y)| & = f(x)-f(y) = (x-y) + \frac{1}{1+\e^x} - \frac{1}{1+\e^y} \\
                                        & = (x-y) - \frac{\e^x - \e^y}{(1+\e^x)(1+\e^y)}            \\
                                        & < x-y = |x-y|.
                        \end{align*}
                  \item 方程$f(x)-x=0$无解:
                        由$f(x)-x = \frac{1}{1+\e^x} > 0$,知$f(x)-x=0$无解.
              \end{enumerate}
    \end{enumerate}
\end{solution}

\begin{exercise}[2.C.9]
    证明: 对任意正整数 $n$, 方程 $x^n+x^{n-1}+\dots+x=1$ 恰有一个正根 $x_n$; 进一步证明数列 $\{x_n\}$ ($n \ges 1$) 收敛, 并求其极限.
\end{exercise}

\begin{solution}
    \begin{enumerate}
        \item 设$f_n(x) = x^n + x^{n-1} + \cdots + x - 1$, 则$f_n(x)$在$(0,1)$上连续, 且$f_n(0) = -1 < 0, f_n(1) = n-1 \ges 0$. 故结合零点定理可简单推知, 存在$x_n \in (0,1]$, 使得$f(x_n) = 0$. 由于$f_n(x)$的每一项都严格单调递增,故$f_n(x)$也严格单调递增,因此$x_n$是唯一解.
        \item 下证明数列$\{x_n\}$单调递减:
        若$x_{n+1} \ges x_n$, 由$x_n > 0$
        $$1 = x_{n+1}^{n+1} + x_{n+1}^n + \dots + x_{n+1} \ges x_n^{n+1} + x_n^n + \dots + x_n > x_n^n + x_n^{n-1} + \dots + x_n = 1,$$故矛盾,因此$x_{n+1} < x_n$. $\{x_n\}$单调递减,有$0$为下界, 故数列$\{x_n\}$收敛.
        \item 考虑$$1-x_n^n = (1-x_n)(1+x_n+x_n^2+\dots+x_n^{n-1}) = 1-x_n,$$在两边同时取极限之前,我们还得先考虑$\lim_{n \to \infty} x_n^n$.由于$1 = x_2 + x_2^2 \ges x_2 \cdot x_2 + x_2^2 = 2x_2^2 \Rightarrow x_2 \les \frac{1}{\sqrt{2}} < 1$, 对$x_n^n < x_2^n$, 由夹逼定理知$\lim_{n \to \infty} x_n^n = 0$. 因此,对数列$\{x_n\}$取极限,有
    $$1 = \lim_{n \to \infty} 1 - x_n^n = \lim_{n \to \infty} 1 - x_n \Rightarrow \lim_{n \to \infty} x_n = 0.$$
    \end{enumerate}

    
\end{solution}

\begin{exercise}[2.C.10]
    设 $a<b, f(x)$ 在 $[a,b]$ 上连续, 且对任意 $x \in [a,b)$ 存在 $y \in (x,b)$ 使得 $f(y)>f(x)$. 求证: $f(b)>f(a)$.
\end{exercise}

\begin{solution}
    考虑$f(x)$在$[a,b]$上的最大值$M = \max \{ f(x): x \in [a,b] \}$, 由闭区间上连续函数的最值性,知存在$x_0 \in [a,b]$,使得$f(x_0) = M$. 若$x_0 \ne b$,则由题设条件,存在$y \in (x_0,b)$,使得$f(y) > f(x_0) = M$,矛盾.因此$x_0 = b$,即$f(b) = M$,同时$f(a) < M$.因此$f(b) > f(a)$.
\end{solution}

\newpage
\end{document}

% emptyset to \varnothing
% \epsilon to \varepsilon
% \phi to \varphi
% 引用模板 \hyperref[ref:narasimhan2]{R. Narasimhan$^{[2]}$}
% \,\eqref{} to\eqref{}
% \dots to \cdots
% , to , 
% 。 to .
% ; to ;