\section{第2章综合习题}

\begin{exercise}[2.C.1]
    证明: 函数 $f(x) = \begin{cases} 0, & x \text{ 为有理数}, \\ x, & x \text{ 为无理数} \end{cases}$ 仅在点 $x=0$ 处连续.
\end{exercise}



\begin{exercise}[2.C.2]
    设 $x_1, x_2, \dots, x_n \in [0,1]$, 记 $f(x) = \frac{|x-x_1|+\dots+|x-x_n|}{n}$, 证明: 存在 $x_0 \in [0,1]$, 使得 $f(x_0) = \frac{1}{2}$.
\end{exercise}

\begin{exercise}[2.C.3]
    证明: 函数 $\frac{a_1}{x-\lambda_1} + \frac{a_2}{x-\lambda_2} + \frac{a_3}{x-\lambda_3}$ (其中 $a_1, a_2, a_3 > 0$, 且 $\lambda_1 < \lambda_2 < \lambda_3$) 在 $(\lambda_1, \lambda_2)$ 与 $(\lambda_2, \lambda_3)$ 内各有一个零点.
\end{exercise}

\begin{exercise}[2.C.4]
    设 $f(x)$ 是一个多项式, 则必存在一点 $x_0$, 使得 $|f(x_0)| \les |f(x)|$ 对任意实数 $x$ 成立.
\end{exercise}

\begin{exercise}[2.C.5]
    设 $f(x)$ 在区间 $[0,1]$ 上连续, 且 $f(0)=f(1)$. 证明: 对任意正整数 $n$, 在区间 $[0, 1-\frac{1}{n}]$ 中有一点 $\xi$, 使得 $f(\xi) = f(\xi+\frac{1}{n})$.
\end{exercise}

\begin{exercise}[2.C.6]
    证明: 存在一个实数 $x$, 满足 $x^5 + \frac{\cos x}{1+x^2+\sin^2 x} = 72$.
\end{exercise}

\begin{exercise}[2.C.7]
    若 $f(x)$ 在 $[a, +\infty)$ 上连续, 且 $\lim_{x \to +\infty} f(x)$ 存在, 则 $f(x)$ 在 $[a, +\infty)$ 上或者有最大值, 或者有最小值.
\end{exercise}

\begin{exercise}[2.C.8]
    设函数 $f(x)$ 定义在区间 $[a,b]$ 上, 满足条件: $a \les f(x) \les b$ (对任意 $x \in [a,b]$), 且对 $[a,b]$ 中任意的 $x,y$ 有 $|f(x)-f(y)| \les k|x-y|$. 这里 $k$ 是常数, $0<k<1$. 证明:
    \begin{enumerate}
        \item 存在唯一的 $x_0 \in [a,b]$, 使得 $f(x_0)=x_0$.
        \item 任取 $x_1 \in [a,b]$, 并定义数列 $\{x_n\}: x_{n+1} = f(x_n), n=1,2,\dots$, 则 $\lim_{n \to \infty} x_n = x_0$.
        \item 给出一个在实轴上的连续函数, 使得对任意 $x \ne y$ 有 $|f(x)-f(y)| < |x-y|$, 但方程 $f(x)-x=0$ 无解.
    \end{enumerate}
\end{exercise}

\begin{exercise}[2.C.9]
    证明: 对任意正整数 $n$, 方程 $x^n+x^{n-1}+\dots+x=1$ 恰有一个正根 $x_n$; 进一步证明数列 $\{x_n\}$ ($n \ges 1$) 收敛, 并求其极限.
\end{exercise}

\begin{exercise}[2.C.10]
    设 $a<b, f(x)$ 在 $[a,b]$ 上连续, 且对任意 $x \in [a,b)$ 存在 $y \in (x,b)$ 使得 $f(y)>f(x)$. 求证: $f(b)>f(a)$.
\end{exercise}

\newpage