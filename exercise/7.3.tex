\section{习题 7.3}

\begin{exercise}[7.3.1]
求下列幂级数的收敛半径.
\begin{tasks}(2)
    \task $\displaystyle \sum_{n=1}^\infty (-1)^{n+1} \frac{x^n}{n^2};$
    \task $\displaystyle \sum_{n=1}^\infty \frac{(n!)^2}{(2n)!} x^n;$
    \task $\displaystyle \sum_{n=1}^\infty 2^n x^{2n};$
    \task $\displaystyle \sum_{n=1}^\infty \frac{x^n}{a^n + b^n} (a > 0, b > 0);$
    \task $\displaystyle \sum_{n=1}^\infty \frac{(x-2)^{2n-1}}{(2n-1)!};$
    \task $\displaystyle \sum_{n=1}^\infty \frac{3^n + (-2)^n}{n} (x+1)^n;$
    \task $\displaystyle \sum_{n=1}^\infty \left( 1 + \frac{1}{2} + \dots + \frac{1}{n} \right) x^n;$
    \task $\displaystyle \sum_{n=1}^\infty \frac{x^{n^2}}{2^n}.$
\end{tasks}
\end{exercise}

\begin{solution}
    \begin{enumerate}
        \item $$L = \limsup_{n \to \infty} \sqrt[n]{\left| \frac{(-1)^{n+1}}{n^2} \right|} = \lim_{n \to \infty} \frac{1}{\sqrt[n]{n^2}} = 1,$$故收敛半径$R = 1$.
        \item $$L = \limsup_{n \to \infty} \sqrt[n]{\left| \frac{(n!)^2}{(2n)!} \right|} = \lim_{n \to \infty} \frac{(n!)^{2/n}}{(2n)!^{1/n}} = \lim_{n \to \infty} \frac{n^2}{(2n)(2n-1)} = \frac{1}{4},$$故收敛半径$R = 4$.
        \item $$L = \limsup_{n \to \infty} \sqrt[(2n)]{\left| 2^n \right|} = \lim_{n \to \infty} \sqrt[2n]{2^n} = \sqrt{2},$$故收敛半径$R = \frac{1}{\sqrt{2}}$.
        \item $$L = \limsup_{n \to \infty} \sqrt[n]{\left| \frac{1}{a^n + b^n} \right|} = \lim_{n \to \infty} \frac{1}{\sqrt[n]{a^n + b^n}} = \frac{1}{\max\{a, b\}},$$故收敛半径$R = \max\{a, b\}$.
        \item $$L = \limsup_{n \to \infty} \sqrt[2n-1]{\left| \frac{1}{(2n-1)!} \right|} = \lim_{n \to \infty} \frac{1}{\sqrt[2n-1]{(2n-1)!}} = 0,$$故收敛半径$R = +\infty$.
        \item $$L = \limsup_{n \to \infty} \sqrt[n]{\left| \frac{3^n + (-2)^n}{n} \right|} = \lim_{n \to \infty} \frac{\sqrt[n]{3^n + (-2)^n}}{\sqrt[n]{n}} = 3,$$故收敛半径$R = \frac{1}{3}$.
        \item $$L = \limsup_{n \to \infty} \sqrt[n]{\left| 1 + \frac{1}{2} + \dots + \frac{1}{n} \right|} = \lim_{n \to \infty} \sqrt[n]{\ln n} = 1,$$故收敛半径$R = 1$.
        \item $$L = \limsup_{n \to \infty} \sqrt[n]{\left| \frac{1}{2^n} \right|} = \lim_{n \to \infty} \frac{1}{\sqrt[n^2]{2^n}} = 1,$$故收敛半径$R = 1$.
    \end{enumerate}
\end{solution}

\begin{exercise}[7.3.2]
设 $\displaystyle f(x) = \sum_{n=0}^\infty a_n x^n$ 在 $|x| < R$ 时收敛, 若 $\displaystyle \sum_{n=0}^\infty \frac{a_n}{n+1} R^{n+1}$ 也收敛, 则
$$ \int_0^R f(x) \mathrm{d}x = \sum_{n=0}^\infty \frac{a_n}{n+1} R^{n+1} $$
(注意: 这里不管 $\displaystyle \sum_{n=0}^\infty a_n x^n$ 在 $x = R$ 是否收敛). 应用这个结果证明:
$$ \int_0^1 \frac{1}{1+x} \mathrm{d}x = \ln 2 = \sum_{n=1}^\infty (-1)^{n-1} \frac{1}{n}. $$
\end{exercise}

\begin{solution}
    \begin{enumerate}
        \item 先求幂级数$\frac{a_n}{n+1} R^{n+1}$的收敛半径:
        任取$x_0 \in (-R,R)$,存在$r:|x_0| < r < R$,使得幂级数$\sum_{n=0}^\infty a_n x^n$在$(-r,r)$内一致收敛.同时$|a_n r^n|< M$有界,因此 
        $$\left| \frac{a_n}{n+1} R^{n+1} \right| = \left| a_n r^n \cdot \frac{R^{n+1}}{(n+1) r^n} \right| \les M \cdot \frac{R^{n+1}}{(n+1) r^n}.$$
        由比较判别法可知幂级数$\sum_{n=0}^\infty \frac{a_n}{n+1} R^{n+1}$在$(-\frac{r}{R} R, \frac{r}{R} R)$内一致收敛,因此收敛半径$R' \ges R$.

        若$R' > R$,则存在$x_0: R < |x_0| < R'$,使得幂级数$\sum_{n=0}^\infty \frac{a_n}{n+1} R^{n+1}$在$(-|x_0|, |x_0|)$内一致收敛,矛盾,因此$R' = R$.

        作为幂级数的,$\sum_{n=0}^\infty \frac{a_n}{n+1} R^{n+1}$在$(-R,R)$内的任意闭子区间$[-r,r]$上一致收敛,所以
        $$ \int_0^r f(x) \mathrm{d}x = \sum_{n=0}^\infty \frac{a_n}{n+1} r^{n+1},$$
        由$r$的任意性知在$(0,R)$上成立.
    \end{enumerate}
\end{solution}

\begin{exercise}[7.3.3]
求下列幂级数的收敛区域及其和函数.
\begin{tasks}(2)
    \task $\displaystyle \sum_{n=0}^\infty (-1)^n \frac{x^{2n+1}}{2n + 1};$
    \task $\displaystyle \sum_{n=0}^\infty (n + 1) x^n;$
    \task $\displaystyle \sum_{n=1}^\infty n(n + 1) x^{n-1};$
    \task $\displaystyle \sum_{n=1}^\infty \frac{x^n}{n(n + 1)};$
    \task $\displaystyle \sum_{n=1}^\infty \frac{x^{2n-1}}{(2n - 1)!!}.$
\end{tasks}
\end{exercise}

\begin{solution}
    \begin{enumerate}
        \item $$L = \limsup_{n \to \infty} \sqrt[2n+1]{\left| (-1)^n \right|} = 1,$$故收敛半径$R = 1$.当$x=1$时,级数$\sum_{n=0}^\infty (-1)^n \frac{1}{2n + 1}$收敛;当$x=-1$时,级数$\sum_{n=0}^\infty (-1)^n \frac{-1}{2n + 1}$收敛.因此收敛区域为$[-1, 1]$.
        
              当$|x| < 1$时,$$S = \sum_{n=0}^\infty (-1)^n \frac{x^{2n+1}}{2n + 1} = \int_0^x \sum_{n=0}^\infty (-1)^n x^{2n} \mathrm{d}x = \int_0^x \frac{1}{1 + x^2} \mathrm{d}x = \arctan x.$$
        \item $$L = \limsup_{n \to \infty} \sqrt[n]{\left| n + 1 \right|} = 1,$$故收敛半径$R = 1$.当$x=1$时,级数$\sum_{n=0}^\infty (n + 1)$发散;当$x=-1$时,级数$\sum_{n=0}^\infty (-1)^n (n + 1)$发散.因此收敛区域为$(-1, 1)$.
              
              当$|x| < 1$时,$$S = \sum_{n=0}^\infty (n + 1) x^n = \frac{\mathrm{d}}{\mathrm{d}x} \sum_{n=0}^\infty x^{n+1} = \frac{\mathrm{d}}{\mathrm{d}x} \left( \frac{x}{1 - x} \right) = \frac{1}{(1 - x)^2}.$$
        \item $$L = \limsup_{n \to \infty} \sqrt[n-1]{\left| n(n + 1) \right|} = 1,$$故收敛半径$R = 1$.当$x=1$时,级数$\sum_{n=1}^\infty n(n + 1)$发散;当$x=-1$时,级数$\sum_{n=1}^\infty (-1)^{n-1} n(n + 1)$发散.因此收敛区域为$(-1, 1)$.
              
              当$|x| < 1$时,$$S = \sum_{n=1}^\infty n(n + 1) x^{n-1} = \frac{\mathrm{d}}{\mathrm{d}x} \sum_{n=1}^\infty (n + 1) x^n = \frac{\mathrm{d}}{\mathrm{d}x} \left( \frac{1}{(1 - x)^2} \right) = \frac{2}{(1 - x)^3}.$$
        \item $$L = \limsup_{n \to \infty} \sqrt[n]{\left| \frac{1}{n(n + 1)} \right|} = 1,$$故收敛半径$R = 1$.当$x=1$时,级数$\sum_{n=1}^\infty \frac{1}{n(n + 1)}$收敛;当$x=-1$时,级数$\sum_{n=1}^\infty \frac{(-1)^n}{n(n + 1)}$收敛.因此收敛区域为$[-1, 1]$.
              
              当$|x| < 1$时,\begin{align*}
                    S &= \sum_{n=1}^\infty \frac{x^n}{n(n + 1)} \\
                        &= \int_0^x \sum_{n=1}^\infty \frac{t^{n-1}}{n} \mathrm{d}t \\
                        &= \int_0^x \frac{-\ln(1 - t)}{t} \mathrm{d}t \\
                        &= \frac{(1 - x) \ln(1 - x) + x}{x}.
              \end{align*}
        \item $$L = \limsup_{n \to \infty} \sqrt[2n-1]{\left| \frac{1}{(2n - 1)!!} \right|} = 0,$$故收敛半径$R = +\infty$.
              
              当$x \in \mathbb{R}$时,注意到$y' = xy+1, y(0) = 0$,解得
              $$\sum_{n=1}^\infty \frac{x^{2n-1}}{(2n - 1)!!} = y = \mathrm{e}^{\frac{x^2}{2}} \int_0^x \mathrm{e}^{-\frac{t^2}{2}} \mathrm{d}t.$$
    \end{enumerate}
\end{solution}

\begin{exercise}[7.3.4]
求下列级数的和.
\begin{tasks}(2)
    \task $\displaystyle \sum_{n=2}^\infty \frac{1}{(n^2 - 1) 2^n};$
    \task $\displaystyle \sum_{n=0}^\infty \frac{(-1)^n (n^2 - n + 1)}{2^n};$
    \task $\displaystyle \sum_{n=0}^\infty \frac{(-1)^n}{3n + 1};$
    \task $\displaystyle \sum_{n=0}^\infty \frac{(n + 1)^2}{n!}.$
\end{tasks}
\end{exercise}

\begin{solution}
    \begin{enumerate}
        \item 考虑幂级数$$\sum_{n=2}^\infty \frac{x^n}{n^2 - 1},$$其收敛半径$R = 1$.当$|x| < 1$时,
              \begin{align*}
                    S &= \sum_{n=2}^\infty \frac{x^n}{n^2 - 1} \\
                        &= \frac{1}{2} \sum_{n=2}^\infty \left( \frac{1}{n-1} - \frac{1}{n+1} \right) x^n \\
                        &= \frac{1}{2} \left( x + \frac{x^2}{2} + \sum_{n=3}^\infty \left( \frac{1}{n-1} - \frac{1}{n+1} \right) x^n \right) \\
                        &= \frac{x^2 - 2x^2 \ln(1-x) + 2x + 2\ln(1-x)}{4x}.
                    \end{align*}
              因此$$\sum_{n=2}^\infty \frac{1}{(n^2 - 1) 2^n} = S|_{x=\frac{1}{2}} = \frac{5}{8} - \frac{3}{4} \ln 2.$$
        \item 考虑幂级数$$\sum_{n=0}^\infty x^n (n^2 - n+1),$$其收敛半径$R = 1$.当$|x| < 1$时,
              \begin{align*}
                    S &= \sum_{n=0}^\infty x^n (n^2 - n+1) \\
                        &= \sum_{n=0}^\infty n^2 x^n - \sum_{n=0}^\infty n x^n + \sum_{n=0}^\infty x^n \\
                        &= \frac{x(1+x)}{(1-x)^3} - \frac{x}{(1-x)^2} + \frac{1}{1-x} \\
                        &= \frac{3x^2-2x+1}{(1 - x)^3}.
              \end{align*}
              因此$$\sum_{n=0}^\infty \frac{(-1)^n (n^2 - n + 1)}{2^n} = S|_{x=-\frac{1}{2}} = \frac{22}{27}.$$
        \item 考虑幂级数$$\sum_{n=0}^\infty x^{3n+1},$$其收敛半径$R = 1$.当$|x| < 1$时,
              \begin{align*}
                    S &= \sum_{n=0}^\infty x^{3n+1} = \frac{x}{1 - x^3}.
              \end{align*}
              因此$$\sum_{n=0}^\infty \frac{(-1)^n}{3n + 1} = \int_0^{-1} \frac{x}{1 - x^3} \mathrm{d}x = \frac{1}{3} \ln 2 + \frac{\pi}{3\sqrt{3}}.$$
            
        \item 考虑幂级数$$\sum_{n=0}^\infty \frac{(n + 1)^2}{n!} x^n,$$其收敛半径$R = +\infty$.当$x \in \mathbb{R}$时,
              \begin{align*}
                    S &= \sum_{n=0}^\infty \frac{(n + 1)^2}{n!} x^n \\
                        &= x^2 \mathrm{e}^x + 3x \mathrm{e}^x + \mathrm{e}^x \\
                        &= (x^2 + 3x + 1) \mathrm{e}^x .
              \end{align*}
              因此$$\sum_{n=0}^\infty \frac{(n + 1)^2}{n!} = S|_{x=1} = 5\mathrm{e}.$$
        \end{enumerate}
\end{solution}

\begin{exercise}[7.3.5]
求下列函数在指定点处的 Taylor 展开式, 并给出收敛区域.
\begin{tasks}(2)
    \task $x^3 - 2x^2 + 5x - 7, x = 1;$
    \task $\mathrm{e}^x, x = a;$
    \task $\ln x, x = 1;$
    \task $\frac{1}{x^2 + 3x + 2}, x = -4;$
    \task $\ln(1 + x - 2x^2), x = 0;$
    \task $\cos x, x = \frac{\pi}{4}.$
\end{tasks}
\end{exercise}

\begin{solution}
    \begin{enumerate}
        \item 记$y = x - 1$,则$$x^3 - 2x^2 + 5x - 7 = (y+1)^3 - 2(y+1)^2 + 5(y+1) - 7 = y^3 + y^2 + 4y - 3.$$
              收敛区域为$\mathbb{R}$.
        \item $$\mathrm{e}^x = \mathrm{e}^a \sum_{n=0}^\infty \frac{(x - a)^n}{n!}.$$
              收敛区域为$\mathbb{R}$.
        \item 记$y = x - 1$,则$$\ln x = \ln(y + 1) = \sum_{n=1}^\infty (-1)^{n-1} \frac{y^n}{n}.$$
              收敛区域为$(-1, 1]$.
        \item 记$y = x + 4$,则
              \begin{align*}
                    \frac{1}{x^2 + 3x + 2} &= \frac{1}{(y-4)^2 + 3(y-4) + 2} \\
                        &= \frac{1}{y^2 - 5y + 6} =\sum_{n=0}^\infty \left( \frac{1}{2n + 1} - \frac{1}{3n + 1} \right) y^n \\
                        &= \sum_{n=0}^\infty \left( \frac{1}{2n + 1} - \frac{1}{3n + 1} \right) (x + 4)^n
                    \end{align*}
              收敛区域为$(-6,-2)$.
        \item $\ln(1 + x - 2x^2) = \ln(1 + 2x) + \ln(1 - x).$ 故
$$ f(x) = \sum_{n=0}^\infty \frac{-(-2)^n - 1}{n} x^n. $$
收敛区间为 $\left(-\frac{1}{2}, \frac{1}{2}\right].$
\item $$ \cos x = \frac{1}{\sqrt{2}} - \frac{1}{\sqrt{2}} \left(x - \frac{\pi}{4}\right) - \frac{1}{\sqrt{2}} \frac{(x - \frac{\pi}{4})^2}{2!} + \frac{1}{\sqrt{2}} \frac{(x - \frac{\pi}{4})^3}{3!} + \dots $$
符号规则以 $4$ 为周期: $+,-,-,+, \dots$. 收敛区间为 $\mathbb{R}.$
    \end{enumerate}
\end{solution}



\begin{exercise}[7.3.6]
求下列函数的 Maclaurin 展开式, 并给出收敛区域.
\begin{tasks}(2)
    \task $\sin^2 x;$
    \task $\arcsin x;$
    \task $\ln \sqrt{\frac{1+x}{1-x}};$
    \task $(1 + x) \ln(1 + x);$
    \task $\displaystyle \int_0^x \cos t^2 \mathrm{d}t;$
    \task $\displaystyle \int_0^x \frac{\sin t}{t} \mathrm{d}t;$
    \task $\displaystyle \int_0^x \mathrm{e}^{-t^2} \mathrm{d}t.$
\end{tasks}
\end{exercise}

\begin{solution}
    \begin{enumerate}
        \item[(1)]
        \begin{align*}
        \sin^2 x &= \frac{1 - \cos 2x}{2} = \frac{1}{2} - \frac{1}{2} \sum_{n=0}^\infty \frac{(-1)^n (2x)^{2n}}{(2n)!} \\
        &= \frac{1}{2} - \frac{1}{2} \left( 1 - \frac{(2x)^2}{2!} + \frac{(2x)^4}{4!} - \dots \right) \\
        &= \sum_{n=1}^\infty \frac{(-1)^{n+1} 2^{2n-1}}{(2n)!} x^{2n}.
        \end{align*}
        $$L = \lim_{n \to \infty} \left| \frac{a_{n+1}}{a_n} \right| = \lim_{n \to \infty} \frac{2^{2n+1}}{(2n+2)!} \frac{(2n)!}{2^{2n-1}} = \lim_{n \to \infty} \frac{4}{(2n+2)(2n+1)} = 0,$$故收敛半径$R = +\infty$,收敛区域为$(-\infty, +\infty)$.

        \item[(2)]
        \begin{align*}
        \arcsin x &= \int_0^x \frac{1}{\sqrt{1 - t^2}} \mathrm{d}t \\
        &= \sum_{n=0}^\infty \frac{(2n - 1)!!}{(2n)!!} \frac{1}{2n + 1} x^{2n + 1}.
        \end{align*}
        $$L = \limsup_{n \to \infty} \sqrt[2n+1]{\left| \frac{(2n - 1)!!}{(2n)!!} \frac{1}{2n + 1} \right|} = 1,$$故收敛半径$R = 1$,当$x = \pm 1$时,级数由比值判别法知收敛.因此收敛区域为$[-1, 1]$.

        \item[(3)]
        \begin{align*}
        \ln \sqrt{\frac{1 + x}{1 - x}} &= \frac{1}{2} [\ln(1 + x) - \ln(1 - x)] \\
        &= \frac{1}{2} \left[ \sum_{n=1}^\infty \frac{(-1)^{n-1}}{n} x^n - \sum_{n=1}^\infty \frac{(-1)^{n-1}}{n} (-x)^n \right] \\
        &= \frac{1}{2} \sum_{n=1}^\infty \frac{1}{n} [(-1)^{n-1} - (-1)^{2n-1}] x^n \\
        &= \frac{1}{2} \sum_{n=1}^\infty \frac{1}{n} [(-1)^{n-1} + 1] x^n \\
        &= \sum_{k=0}^\infty \frac{1}{2k + 1} x^{2k + 1}.
        \end{align*}
        $$L = \limsup_{k \to \infty} \sqrt[2k+1]{\left| \frac{1}{2k + 1} \right|} = 1,$$故收敛半径$R = 1$,当$x = 1$时,级数 $\sum \frac{1}{2k+1}$ 发散;当$x = -1$时,级数 $\sum \frac{-1}{2k+1}$ 发散.因此收敛区域为$(-1, 1)$.

        \item[(4)]
        \begin{align*}
        (1 + x) \ln(1 + x) &= (1 + x) \sum_{n=1}^\infty \frac{(-1)^{n-1}}{n} x^n \\
        &= \sum_{n=1}^\infty \frac{(-1)^{n-1}}{n} x^n + \sum_{n=1}^\infty \frac{(-1)^{n-1}}{n} x^{n+1} \\
        &= x + \sum_{n=2}^\infty \frac{(-1)^{n-1}}{n} x^n + \sum_{n=2}^\infty \frac{(-1)^{n-2}}{n-1} x^n \\
        &= x + \sum_{n=2}^\infty \left[ \frac{(-1)^{n-1}}{n} + \frac{(-1)^{n-2}}{n-1} \right] x^n \\
        &= x + \sum_{n=2}^\infty \frac{(-1)^n}{n(n-1)} x^n.
        \end{align*}
        $$L = \lim_{n \to \infty} \sqrt[n]{\frac{1}{n(n-1)}} = 1,$$故收敛半径$R = 1$,当$x = \pm 1$时,级数 $\sum \frac{1}{n(n-1)}$ 绝对收敛.因此收敛区域为$[-1, 1]$.

        \item[(5)]
        \begin{align*}
        \cos t^2 &= \sum_{n=0}^\infty \frac{(-1)^n}{(2n)!} (t^2)^{2n} = \sum_{n=0}^\infty \frac{(-1)^n}{(2n)!} t^{4n}. \\
        \int_0^x \cos t^2 \mathrm{d}t &= \int_0^x \sum_{n=0}^\infty \frac{(-1)^n}{(2n)!} t^{4n} \mathrm{d}t \\
        &= \sum_{n=0}^\infty \frac{(-1)^n}{(2n)!} \int_0^x t^{4n} \mathrm{d}t \\
        &= \sum_{n=0}^\infty \frac{(-1)^n}{(2n)! (4n + 1)} x^{4n + 1}.
        \end{align*}
        $$L = \lim_{n \to \infty} \sqrt[4n+1]{\left| \frac{1}{(2n)! (4n + 1)} \right|} = 0,$$故收敛半径$R = +\infty$,收敛区域为$(-\infty, +\infty)$.

        \item[(6)]
        \begin{align*}
        \frac{\sin t}{t} &= \frac{1}{t} \sum_{n=0}^\infty \frac{(-1)^n}{(2n + 1)!} t^{2n + 1} = \sum_{n=0}^\infty \frac{(-1)^n}{(2n + 1)!} t^{2n}. \\
        \int_0^x \frac{\sin t}{t} \mathrm{d}t &= \int_0^x \sum_{n=0}^\infty \frac{(-1)^n}{(2n + 1)!} t^{2n} \mathrm{d}t \\
        &= \sum_{n=0}^\infty \frac{(-1)^n}{(2n + 1)!} \int_0^x t^{2n} \mathrm{d}t \\
        &= \sum_{n=0}^\infty \frac{(-1)^n}{(2n + 1)! (2n + 1)} x^{2n + 1}.
        \end{align*}
        $$L = \lim_{n \to \infty} \sqrt[2n+1]{\left| \frac{1}{(2n + 1)! (2n + 1)} \right|} = 0,$$故收敛半径$R = +\infty$,收敛区域为$(-\infty, +\infty)$.

        \item[(7)]
        \begin{align*}
        \mathrm{e}^{-t^2} &= \sum_{n=0}^\infty \frac{(-1)^n}{n!} (-t^2)^n = \sum_{n=0}^\infty \frac{(-1)^n}{n!} t^{2n}. \\
        \int_0^x \mathrm{e}^{-t^2} \mathrm{d}t &= \int_0^x \sum_{n=0}^\infty \frac{(-1)^n}{n!} t^{2n} \mathrm{d}t \\
        &= \sum_{n=0}^\infty \frac{(-1)^n}{n!} \int_0^x t^{2n} \mathrm{d}t \\
        &= \sum_{n=0}^\infty \frac{(-1)^n}{n! (2n + 1)} x^{2n + 1}.
        \end{align*}
        $$L = \lim_{n \to \infty} \sqrt[2n+1]{\left| \frac{1}{n! (2n + 1)} \right|} = 0,$$故收敛半径$R = +\infty$,收敛区域为$(-\infty, +\infty)$.
    \end{enumerate}
\end{solution}

\begin{exercise}[7.3.7]
方程 $y + \lambda \sin y = x (\lambda \neq -1)$ 在 $x = 0$ 附近确定了一个隐函数 $y(x)$, 试求它的幂级数展开式中的前四项.
\end{exercise}

\begin{solution}
设 $y(x) = a_1 x + a_2 x^2 + a_3 x^3 + a_4 x^4 + \dots$ 是 Maclaurin 展开式.
由 $y(0) + \lambda \sin(y(0)) = 0$, 得 $y(0) = 0$.

对 $y + \lambda \sin y = x$ 求导:
$$ y'(1 + \lambda \cos y) = 1 $$
代入 $x=0$: $$y'(0)(1 + \lambda \cos 0) = 1 \quad \Rightarrow \quad y'(0) = \frac{1}{1 + \lambda}. $$

对 $y'(1 + \lambda \cos y) = 1$ 求导:
$$ y''(1 + \lambda \cos y) + y' (\lambda (-\sin y) y') = 0 $$
代入 $x=0$: $$y''(0)(1 + \lambda) + y'(0) (\lambda (-\sin 0) y'(0)) = 0 \quad \Rightarrow \quad y''(0) = 0. $$

对 $y'' (1 + \lambda \cos y) = \lambda (\sin y) (y')^2$ 求导:
$$ y''' (1 + \lambda \cos y) + y'' (\lambda (-\sin y) y') = \lambda (\cos y) (y')^3 + \lambda (\sin y) 2y' y'' $$
代入 $x=0$ (利用 $y(0)=0, y''(0)=0$):
$$ y'''(0) (1 + \lambda) = \lambda (\cos 0) (y'(0))^3 + 0 $$
$$ y'''(0) (1 + \lambda) = \lambda \left( \frac{1}{1 + \lambda} \right)^3 \quad \Rightarrow \quad y'''(0) = \frac{\lambda}{(1 + \lambda)^4}. $$

因此$y(x)$的 Maclaurin 展开式为:
\begin{align*}
y(x) &= y(0) + y'(0)x + \frac{y''(0)}{2!}x^2 + \frac{y'''(0)}{3!}x^3 + \dots \\
&= 0 + \left(\frac{1}{1 + \lambda}\right) x + 0 \cdot x^2 + \frac{1}{6} \left(\frac{\lambda}{(1 + \lambda)^4}\right) x^3 + \dots \\
&= \frac{1}{1 + \lambda} x + \frac{\lambda}{6(1 + \lambda)^4} x^3 + \dots
\end{align*}
\end{solution}



\newpage