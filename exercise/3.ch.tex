\section*{第 3 章综合习题}

\begin{exercise}[3.C.1]
    设 $f(x) = x(x+1)(x+2)\dots(x+n)$, 求 $f'(0)$.
\end{exercise}

\begin{exercise}[3.C.2]
    设奇函数 $f(x)$ 在 $(-\infty, +\infty)$ 上具有二阶连续导数, 记函数
    $$g(x) = \begin{cases} \frac{f(x)}{x}, & x \ne 0, \\ a, & x = 0. \end{cases}$$
    \begin{enumerate}
        \item 确定 $a$ 的值, 使 $g(x)$ 在 $(-\infty, +\infty)$ 上连续.
        \item 对 (1) 中确定的 $a$, 证明: $g(x)$ 在 $(-\infty, +\infty)$ 上可导, 且导函数连续.
    \end{enumerate}
\end{exercise}

\begin{exercise}[3.C.3]
    设 $\frac{a_0}{n+1} + \frac{a_1}{n} + \dots + a_n = 0$, 证明: 方程 $a_0 x^n + a_1 x^{n-1} + \dots + a_n = 0$ 在 $(0, 1)$ 内至少有一个根.
\end{exercise}

\begin{exercise}[3.C.4]
    若函数 $f(x)$ 在 $[a, b]$ 上连续, 在 $(a, b)$ 内可微, 且 $f(a) = f(b) = 0$, 证明: 存在 $\xi \in (a, b)$, 使 $f'(\xi) + f(\xi) = 0$.
\end{exercise}

\begin{exercise}[3.C.5]
    设 $f(x)$ 在区间 $I$ 上连续, 如果任给 $I$ 中两点 $x_1, x_2$, 有
    $$f\left(\frac{x_1 + x_2}{2}\right) \les \frac{f(x_1) + f(x_2)}{2},$$
    证明: $f(x)$ 是区间 $I$ 上的凸函数.
\end{exercise}

\begin{exercise}[3.C.6]
    设 $f(x)$ 是 $[0, 1]$ 上的二阶可微函数, $f(0) = f(1) = 0$. 证明: 存在 $\xi \in (0, 1)$, 使得
    $$f''(\xi) = \frac{2 f'(\xi)}{1 - \xi}.$$
\end{exercise}

\begin{exercise}[3.C.7]
    若函数 $f(x)$ 在 $[a, b]$ 上连续, 且 $f(a) = f(b) = 0$, $f'(a)f'(b) > 0$. 证明: 在 $(a, b)$ 内存在一点 $\xi$, 使得 $f'(\xi) = 0$.
\end{exercise}

\begin{exercise}[3.C.8]
    设 $f(x)$ 在 $[0, 1]$ 上可导, $f(0) = 1, f(1) = \frac{1}{2}$. 求证: 存在 $\xi \in (0, 1)$, 使得
    $$f^2(\xi) + f'(\xi) = 0.$$
\end{exercise}

\begin{exercise}[3.C.9]
    设函数 $f(x)$ 在 $[a, +\infty)$ 上二阶可微, 且满足 $f(a) > 0, f'(a) < 0$, 以及当 $x > a$ 时, $f''(x) \les 0$. 试证在区间 $(a, +\infty)$ 内, 函数 $f(x)$ 恰有一个零点.
\end{exercise}

\begin{exercise}[3.C.10]
    设函数 $f(x)$ 在 $[a, b]$ 上可导, $f'(x)$ 严格单调增. 若 $f(a) = f(b) = \lambda$, 证明: 对任意 $x \in (a, b)$, 有 $f(x) < \lambda$.
\end{exercise}

\begin{exercise}[3.C.11]
    函数 $\frac{\sin x^2}{x} (x > 0)$ 表明, 若函数 $f(x)$ 在 $(a, +\infty)$ 上可导, 且 $\lim_{x \to +\infty} f(x)$ 存在, 不能保证 $\lim_{x \to +\infty} f'(x)$ 存在. 证明: 若已知该极限存在, 则其值必然为零.
\end{exercise}

\begin{exercise}[3.C.12]
    设函数 $f(x)$ 在 $x > 0$ 时二阶可微, 且 $f''(x) < 0, f(0) = 0$. 证明: 对任意正数 $x_1, x_2$, 有 $f(x_1 + x_2) < f(x_1) + f(x_2)$.
\end{exercise}

\begin{exercise}[3.C.13]
    设函数 $f(x)$ 在 $x_0$ 处存在二阶导数, 求
    $$\lim_{h \to 0} \frac{f(x_0 + h) + f(x_0 - h) - 2f(x_0)}{h^2}.$$
\end{exercise}

\begin{exercise}[3.C.14]
    证明下列不等式.
    \begin{enumerate}
        \item 对任意实数 $x$, $\mathrm{e}^x \ges 1 + x + \frac{x^2}{2} + \frac{x^3}{6}$;
        \item 对 $x > 0$, $x - \frac{x^2}{2} \les \ln(1 + x) \les x - \frac{x^2}{2} + \frac{x^3}{3}$;
        \item 对 $0 < x < \frac{\pi}{2}$, $x - \frac{x^3}{6} < \sin x < x - \frac{x^3}{6} + \frac{x^5}{120}$;
        \item 对任意实数 $x, y$, 有 $2 \mathrm{e}^{\frac{x+y}{2}} \les \mathrm{e}^x + \mathrm{e}^y$.
    \end{enumerate}
\end{exercise}

\begin{exercise}[3.C.15]
    求 $\displaystyle \lim_{n \to \infty} \left(1 + \frac{1}{n^2}\right) \left(1 + \frac{2}{n^2}\right) \dots \left(1 + \frac{n}{n^2}\right)$.
\end{exercise}

\begin{exercise}[3.C.16]
    求 $\sqrt[n]{n} (n=1, 2, \dots)$ 的最大值.
\end{exercise}

\begin{exercise}[3.C.17]
    试给出函数 $x \cos x$ 在 $[0, \frac{\pi}{2}]$ 上的一个尽可能小的上界.
\end{exercise}

\begin{exercise}[3.C.18]
    设函数 $f(x)$ 在闭区间 $[-1, 1]$ 上具有三阶连续导数, 且 $f(-1) = 0, f(1) = 1, f'(0) = 0$. 证明: 存在 $\xi \in (-1, 1)$, 使得 $f'''(\xi) = 3$.
\end{exercise}

\begin{exercise}[3.C.19]
    设 $a > 1$, 函数 $f: (0, +\infty) \to (0, +\infty)$ 可微. 求证: 存在趋于无穷的正数列 $\{x_n\}$, 使得
    $$f'(x_n) < f(a x_n), \quad n=1, 2, \dots$$
\end{exercise}

\begin{exercise}[3.C.20]
    利用凸函数的性质证明 Hölder (赫尔德) 不等式: 设 $a_i, b_i, i = 1, 2, \dots, n$ 是正数, $p, q$ 是大于 $1$ 的正数, 且 $\frac{1}{p} + \frac{1}{q} = 1$, 则有
    $$\sum_{i=1}^n a_i b_i \les \left(\sum_{i=1}^n a_i^p\right)^{\frac{1}{p}} \left(\sum_{i=1}^n b_i^q\right)^{\frac{1}{q}}.$$
    (提示: 考虑函数 $f(x) = x^p$.)
\end{exercise}