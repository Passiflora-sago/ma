\section{第1章综合习题}

\begin{exercise}[1.C.1]
    求下列数列的极限:
    \begin{enumerate}[(1)]
        \item $a_n = \frac{1}{2} \cdot \frac{3}{4} \cdots \frac{2n-1}{2n}$ (提示: $\frac{1}{2} \cdot \frac{3}{4} \cdots \frac{2n-1}{2n} \les \frac{1}{\sqrt{2n+1}}$);
        \item $a_n = \frac{10}{1} \cdot \frac{11}{3} \cdots \frac{n+9}{2n-1}$;
        \item 设 $a_1 > 1,  a_{n+1} = 2 - \frac{1}{a_n},  n=1, 2, \cdots$;
        \item 设 $a_1 = 3,  a_{n+1} = \frac{1}{1+a_n},  n=1, 2, \cdots$.
    \end{enumerate}
\end{exercise}

\begin{solution}
    \begin{enumerate}[(1)]
        \item 由$$(2n)^2 = 4n^2 \ges 4n^2 - 1 = (2n-1)(2n+1)$$可得$$\frac{2n-1}{2n} \les \frac{2n-1}{\sqrt{(2n-1)(2n+1)}} = \sqrt{\frac{2n-1}{2n+1}}.$$因此$$a_n = \frac{1}{2} \cdot \frac{3}{4} \cdots \frac{2n-1}{2n} \les \sqrt{\frac{1}{3} \cdot \frac{3}{5} \cdots \frac{2n-1}{2n+1}} = \sqrt{\frac{1}{2n+1}}.$$
              而$\lim_{n\to\infty} \sqrt{\frac{1}{2n+1}} = 0$,  故由夹逼定理可知$\lim_{n\to\infty} a_n = 0$;
        \item 由$\lim_{n \to \infty} \frac{n+9}{2n-1} = \frac{1}{2}$, 知$$\lim_{n\to\infty} \sqrt[n]{\frac{10}{1} \cdot \frac{11}{3} \cdots \frac{n+9}{2n-1}} = \lim_{n\to\infty} \frac{n+9}{2n-1} = \frac{1}{2}.$$
              因此$\lim_{n\to\infty} a_n = \lim_{n\to\infty} \left(\sqrt[n]{a_n}\right)^n = 0$;
        \item 由$a_1 > 1$, 以及若$a_n > 1$时,  $a_{n+1} = 2 - \frac{1}{a_n} > 1$, 归纳的可知$a_n > 1,  \forall n \in \mathbb{N}^+$. 所以数列有下界. 再用归纳法: 当 $n=1$ 时:
              $$ a_2 - a_1 = 2 - \left(\frac{1}{a_1} + a_1\right) \les 2-2=0,  $$
              推出 $a_2 \les a_1$. 假设对 $n$ 有 $a_n \les a_{n-1}$,  那么当 $n+1$ 时
              $$ a_{n+1} - a_n = \frac{1}{a_{n-1}} - \frac{1}{a_n} = \frac{a_n - a_{n-1}}{a_n a_{n-1}} \les 0. $$
              所以 $\{a_n\}$ 是单调减有下界数列,  因此收敛. 设 $\lim_{n \to \infty} a_n = a \ges 1$. 在
              $$ a_{n+1} = 2 - \frac{1}{a_n} $$
              两边取极限得
              $$ a = 2 - \frac{1}{a} \implies a^2 - 2a + 1 = 0 $$
              解得 $a=\pm 1$. 但 $a=-1$ 不合题意,  所以 $\lim_{n \to \infty} a_n = 1$.
        \item $a_{n+2} = \frac{1}{1+a_{n+1}} = \frac{1}{1+\frac{1}{1+a_n}} = \frac{1+a_n}{2+a_n}$.
              假如对任何 $n$,  有 $a_{2n} \ges a_{2n-2}$; $a_{2n+1} \les a_{2n-1}$,  那么对 $n+1$,  有
              $$ a_{2n+2} - a_{2n} = \frac{1}{1+a_{2n+1}} - \frac{1}{1+a_{2n-1}} = \frac{a_{2n-1}-a_{2n+1}}{1+a_{2n+1}a_{2n-1}} \ges 0 $$
              $$ a_{2n+3} - a_{2n+1} = \frac{1}{1+a_{2n+2}} - \frac{1}{1+a_{2n}} = \frac{a_{2n}-a_{2n+2}}{1+a_{2n+2}a_{2n}} \les 0 $$
              推出数列 $\{a_n\}$ 的子列 $\{a_{2n}\}$ 单调增有上界,  $\{a_{2n-1}\}$ 单调减有下界. 因此分别收敛. 对$$a_{2n+2} = \frac{1+a_{2n}}{2+a_{2n}}, $$两边取极限得
              $$ a = \frac{1+a}{2+a} \implies a^2 + a - 1 = 0 $$
              解得 $a = \frac{-1+\sqrt{5}}{2}$. 同理,  对$$a_{2n+3} = \frac{1+a_{2n+1}}{2+a_{2n+1}}, $$两边取极限得
              $$ b = \frac{1+b}{2+b} \implies b^2 + b - 1 = 0 $$
              解得 $b = \frac{-1+\sqrt{5}}{2}$.
              因此$\lim_{n \to \infty} a_{2n} = \lim_{n \to \infty} a_{2n-1}$,  故$\lim_{n \to \infty} a_n$存在,  且$\lim_{n \to \infty} a_n = \frac{-1+\sqrt{5}}{2}$.
    \end{enumerate}
\end{solution}

\begin{exercise}[1.C.2]
    设 $\{a_n\}$ 为单调递增的数列,  并且收敛于 $a$,  证明对一切 $n$ 有 $a_n < a$. (对单调递减且有极限的数列,  类似的结论成立.)
\end{exercise}

\begin{solution}
    反证法. 假设存在某个 $n_0$,  使得 $a_{n_0} > a$. 由数列单调递增的性质,  对一切 $n > n_0$ 有 $a_n \ges a_{n_0} > a$, 于是存在 $\varepsilon = \frac{a_{n_0} - a}{2} > 0$,  使得$\forall N$,  存在$ n = \max\{n_0,  N\} + 1 > N$,  使得
    $$ |a_n - a| = a_n - a \ges a_{n_0} - a = 2\varepsilon > \varepsilon,  $$
    这与数列收敛的定义矛盾.
\end{solution}

\begin{exercise}[1.C.3]
    证明下面的数列收敛:
    \begin{enumerate}[(1)]
        \item $a_n = 1 + \frac{1}{2^2} + \cdots + \frac{1}{n^2}$;
        \item $a_n = \left(1+\frac{1}{2}\right)\left(1+\frac{1}{2^2}\right)\cdots\left(1+\frac{1}{2^n}\right)$.
    \end{enumerate}
\end{exercise}

\begin{solution}
    \begin{enumerate}[(1)]
        \item 由数列定义可知 $\{a_n\}$ 单调递增. 又因为
              $$ a_n = 1 + \frac{1}{2^2} + \cdots + \frac{1}{n^2} \les 1 + \frac{1}{1 \cdot 2} + \frac{1}{2 \cdot 3} + \cdots + \frac{1}{(n-1)n} = 2 - \frac{1}{n} < 2,  $$
              所以 $\{a_n\}$ 有上界. 因此 $\{a_n\}$ 收敛;
        \item 由数列定义可知 $\{a_n\}$ 单调递增. 又因为
              $$a_n = \e^{\sum_{k=1} \ln \left(1 + \frac{1}{2^k}\right)} \les \e^{\sum_{k=1} \frac{1}{2^k}} = \e^1.$$
              所以 $\{a_n\}$ 有上界. 又由数列定义可知 $\{a_n\}$ 单调递增. 因此 $\{a_n\}$ 收敛.
    \end{enumerate}
\end{solution}

\begin{exercise}[1.C.4]
    试构造一个发散的数列 $\{a_n\}$,  满足条件: 对任意正数 $\varepsilon$,  存在正整数 $N$,  使当 $n > N$ 时,  有 $|a_{n+1}-a_n| < \varepsilon$.
\end{exercise}

\begin{solution}
    取$a_n = \sqrt{n}$. 则对任意 $\varepsilon > 0$,  存在正整数 $N > \frac{1}{4\varepsilon^2}$,  当 $n > N$ 时,  有
    $$ |a_{n+1}-a_n| = \sqrt{n+1} - \sqrt{n} = \frac{1}{\sqrt{n+1} + \sqrt{n}} < \frac{1}{2\sqrt{n}} < \varepsilon. $$
    但数列 $\{a_n\}$ 显然发散.
\end{solution}

\begin{exercise}[1.C.5]
    若数列 $\{a_n\}$ 满足: 存在常数 $M$,  使得对一切 $n$ 有
    $$ A_n = |a_2-a_1| + |a_3-a_2| + \cdots + |a_{n+1}-a_n| \les M. $$
    证明:
    \begin{enumerate}[(1)]
        \item 数列 $\{A_n\}$ 收敛;
        \item 数列 $\{a_n\}$ 也收敛.
    \end{enumerate}
\end{exercise}

\begin{solution}
    \begin{enumerate}[(1)]
        \item 由数列定义可知 $\{A_n\}$ 单调递增. 又因为对一切 $n$ 有 $A_n \les M$,  所以 $\{A_n\}$ 有上界. 因此 $\{A_n\}$ 收敛;
        \item 用Cauchy收敛准则证明. 由(1)知 $\{A_n\}$ 收敛,  因此对任意 $\varepsilon > 0$,  存在正整数 $N$,  $\forall n > N+1,  p > 0$,  有
              $$ |A_{n+p} - A_{n-1}| = |a_{n+1}-a_n| + |a_{n+2}-a_{n+1}| + \cdots + |a_{n+p}-a_{n+p-1}| < \varepsilon. $$
              由三角不等式可知
              $$ |a_{n+p} - a_n| \les |a_{n+1}-a_n| + |a_{n+2}-a_{n+1}| + \cdots + |a_{n+p}-a_{n+p-1}| < \varepsilon. $$
    \end{enumerate}
\end{solution}

\begin{exercise}[1.C.6]
    设 $\{a_n\}$ 是正严格递增数列. 求证: 若 $a_{n+1}-a_n$ 有界,  则对任意 $\alpha \in (0, 1)$ 有 $\lim_{n \to \infty} (a_{n+1}^\alpha - a_n^\alpha) = 0$. 并说明此结论的逆不对,  即,  存在正严格递增数列 $\{a_n\}$ 使得对任意 $\alpha \in (0, 1)$ 有 $\lim_{n \to \infty} (a_{n+1}^\alpha - a_n^\alpha) = 0$,  但是 $a_{n+1}-a_n$ 无界. (提示: 考虑 $a_n = n\ln n$.)
\end{exercise}

\begin{solution}
    \begin{enumerate}[(1)]
        \item 若$\{a_n\}$有界, 此时由于其严格单调, 故有极限, 记$\lim_{n\to\infty}a_n=l$, 可知$\lim_{n \to \infty} (a_{n+1}^\alpha - a_n^\alpha) = l^\alpha-l^\alpha=0.$
        \item 若$\{a_n\}$无界, 即$\lim_{n\to\infty}a_n=+\infty$, 设$\left|a_{n+1}-a_n\right| \les M$.
              \begin{align*}
                  0\les a_{n+1}^\alpha - a_n^\alpha & =a_n^\alpha\left(\left(\frac{a_{n+1}}{a_n}\right)^\alpha - 1\right) <a_n^\alpha\left(\frac{a_{n+1}}{a_n} - 1\right)=\frac{a_{n+1}-a_n}{a_n^{1-\alpha}} \les\frac{M}{a_n^{1-\alpha}}.
              \end{align*}
              同时, $$\lim_{n\to\infty}Ma_n^{\alpha-1}=0.$$
              因此由夹逼定理可知$\lim_{n \to \infty} (a_{n+1}^\alpha - a_n^\alpha) = 0$.
        \item 反之不对, 取$a_n=n\ln n$, 则\begin{align*}
                  a_{n+1}^\alpha-a_n^\alpha & =(n+1)^\alpha\ln^\alpha(n+1)-n^\alpha\ln^\alpha                       \\
                                            & <\left((n+1)^\alpha-n^\alpha\right)\ln^\alpha n                       \\
                                            & =n^\alpha\left(\left(1+\frac{1}{n}\right)^\alpha-1\right)\ln^\alpha n \\
                                            & <n^\alpha\left(\left(1+\frac{1}{n}\right)-1\right)\ln^\alpha n        \\
                                            & =n^{\alpha-1}\ln^\alpha n=\frac{\ln^\alpha n}{n^{1-\alpha}}.
              \end{align*}
              由于$$\lim_{n\to\infty}\frac{\ln^\alpha n}{n^{1-\alpha}}=0, $$因此由夹逼定理可知$\lim_{n \to \infty} (a_{n+1}^\alpha - a_n^\alpha) = 0$.但$$a_{n+1}-a_n=(n+1)\ln(n+1)-n\ln n=\ln (n+1)+n\ln\left(\frac{n+1}{n}\right)>\ln(n+1), $$显然无界.
    \end{enumerate}
\end{solution}

\begin{exercise}[1.C.7]
    设数列 $\{a_n\}$ 满足 $\lim_{n \to \infty} (a_{n+1}-a_n) = a$. 证明: $\lim_{n \to \infty} \frac{a_n}{n} = a$.
\end{exercise}

\begin{solution}
    由Stolz定理可知
    \begin{align*}
        \lim_{n \to \infty} \frac{a_n}{n} & = \lim_{n \to \infty} \frac{a_{n+1}-a_n}{(n+1)-n} = \lim_{n \to \infty} (a_{n+1}-a_n) = a.
    \end{align*}
\end{solution}

\begin{exercise}[1.C.8]\label{exe:1.C.8}
    证明: 若 $\lim_{n \to \infty} a_n = a$,  且 $a_n > 0$,  则 $\lim_{n \to \infty} \sqrt[n]{a_1a_2\cdots a_n} = a$.
\end{exercise}
\begin{solution}
    由Stolz定理可知
    \begin{align*}
        \lim_{n \to \infty} \sqrt[n]{a_1a_2\cdots a_n} = \lim_{n \to \infty}\e^{\frac{1}{n}\left(\ln a_1+\ln a_2+\cdots +\ln a_n\right)} = \e^{\lim\limits_{n \to \infty}\frac{\left(\ln a_1+\ln a_2+\cdots +\ln a_n\right)}{n}}\=\e^{\lim\limits_{n \to \infty}\frac{\ln a_n}{1}} =\e^{\ln a} =a.
    \end{align*}
\end{solution}

\begin{exercise}[1.C.9]\label{exe:1.C.9}
    证明: 若 $a_n > 0$,  且 $\lim_{n \to \infty} \frac{a_{n+1}}{a_n}$ 存在,  则 $\lim_{n \to \infty} \sqrt[n]{a_n}$ 也存在,  并且
    $$ \lim_{n \to \infty} \sqrt[n]{a_n} = \lim_{n \to \infty} \frac{a_{n+1}}{a_n}. $$
\end{exercise}

\begin{solution}
    设$b_n=\frac{a_{n}}{a_{n-1}}, (n>1);b_1=a_1$, 则$a_n=b_1b_2\cdots b_n$.由\hyperref[exe:1.C.8]{综合习题1.C.8}可知结果.
    直接Stolz也可以得到结果.
    \begin{align*}
        \lim_{n \to \infty} \sqrt[n]{a_n} & =\lim_{n \to \infty} \e{\frac{\ln a_n}{n}}=\e^{\lim\limits_{n \to \infty}\frac{\ln a_n}{n}}                               \\
                                          & =\e^{\lim\limits_{n \to \infty}\frac{\ln a_{n+1}-\ln a_{n}}{1}} =\e^{\lim\limits_{n \to \infty}\ln \frac{a_{n+1}}{a_{n}}} \\
                                          & =\e^{\ln \lim\limits_{n \to \infty}\frac{a_{n+1}}{a_{n}}} =\lim_{n \to \infty}\frac{a_{n+1}}{a_{n}}.
    \end{align*}
\end{solution}

\begin{exercise}[1.C.10]
    求下列极限:
    \begin{tasks}[label=(\arabic*)](2)
        \task $\lim_{n \to \infty} \frac{1+\sqrt{2}+\sqrt[3]{3}+\cdots+\sqrt[n]{n}}{n}$;
        \task $\lim_{n \to \infty} \frac{n}{\sqrt[n]{n!}}$.
    \end{tasks}
\end{exercise}

\begin{solution}
    \begin{enumerate}[(1)]
        \item 由Stolz定理可知
              \begin{align*}
                  \lim_{n \to \infty} \frac{1+\sqrt{2}+\sqrt[3]{3}+\cdots+\sqrt[n]{n}}{n} & = \lim_{n \to \infty} \sqrt[n]{n} = 1;
              \end{align*}
        \item 可以用\hyperref[exe:1.C.9]{综合习题1.C.9}来做, 记$a_n=\frac{n^n}{n!}$由于
              $$\lim_{n\to\infty}\frac{a_{n+1}}{a_n}=\lim_{n\to\infty}\frac{(n+1)^{n+1}\cdot n!}{n^n\cdot (n+1)!}=\lim_{n\to\infty}\frac{(n+1)^{n}}{n^n}=\lim_{n\to\infty}\left(1+\frac{1}{n}\right)^{n}=\e, $$
              因此$$\lim_{n \to \infty} \frac{n}{\sqrt[n]{n!}}=\lim_{n \to \infty}\sqrt[n]{\frac{n^n}{n!}}=\lim_{n \to \infty}\sqrt[n]{a_n}=\e.$$
    \end{enumerate}
\end{solution}

\begin{exercise}[1.C.11]
    已知 $\lim_{n \to \infty} a_n = a$,  求证 $\lim_{n \to \infty} \frac{a_1+2a_2+\cdots+na_n}{n^2} = \frac{a}{2}$.
\end{exercise}
\begin{solution}
    由 Stolz 定理,  有
    \begin{align*}
        \lim_{n \to \infty} \frac{a_1+2a_2+\cdots+na_n}{n^2} = \lim_{n \to \infty} \frac{na_n}{n^2 - (n-1)^2} = \lim_{n \to \infty} \frac{na_n}{2n-1} =\lim_{n \to \infty}\frac{n}{2n-1} \lim_{n \to \infty} a_n = \frac{a}{2}.
    \end{align*}
\end{solution}

\begin{exercise}[1.C.12]
    设 $\{a_n\}$ 且 $a_n \to a \in \mathbb{R}$,  又设 $\{b_n\}$ 是正数列,  $c_n = \frac{a_1b_1+a_2b_2+\cdots+a_nb_n}{b_1+b_2+\cdots+b_n}$. 求证:
    \begin{enumerate}[(1)]
        \item $\{c_n\}$ 收敛;
        \item 若 $(b_1+b_2+\cdots+b_n) \to +\infty$,  则 $\lim_{n \to \infty} c_n = a$.
    \end{enumerate}
\end{exercise}

\begin{solution}
    \begin{enumerate}[(1)]
        \item 记$B_n = b_1+b_2+\cdots+b_n$, 由$\lim_{n\to\infty} a_n = a$, 任取$\ve > 0$, 存在$K$, 当$k > K$时,$|a_k -a| < \ve$.

              当$n > K$, 有
              $$c_n - a = \frac{\sum_{k=1}^{n}(a_k - a)b_k}{B_n}= \frac{\sum_{k=1}^{K}(a_k - a)b_k}{B_n} + \frac{\sum_{k=K+1}^{n}(a_k - a)b_k}{B_n}.$$

              其中$$\left| \frac{\sum_{k=K+1}^{n}(a_k - a)b_k}{B_n}\right| \les \frac{\sum_{k=K+1}^{n}|a_k - a|b_k}{B_n} < \ve \frac{\sum_{k=K+1}^{n}b_k}{B_n} < \ve.$$

              因此$$\left|c_n - \left( a + \frac{\sum_{k=1}^{K}(a_k - a)b_k}{B_n}\right)\right| < \ve.$$

              而对于$$q_n := \frac{\sum_{k=1}^{K}(a_k - a)b_k}{B_n},$$
              $C := \sum_{k=1}^{K}(a_k - a)b_k$是仅与$K$有关,与$n$无关的常数,$B_n$单调增,因此$q_n$单调有界($C>0$时$q_n$单调减且$q_n > 0$, $C<0$时$q_n$单调增且$q_n < 0$), 故$q_n$收敛, 设$\lim_{n\to\infty}q_n = q$, 再取$N$,使得当$n,m > N$时,$|q_m - q_n| < \ve$, 则当$n,m > \max\{N,K\}$时,
              $$|c_m - c_n| \les \left| c_n - \left(a + q_n\right)\right| + \left| c_m - \left(a + q_m\right)\right| + |q_m - q_n| < 3\ve.$$
              因此由Cauchy收敛准则可知$c_n$收敛.
        \item 下给出两种方法,
              \begin{enumerate}
                  \item 由(1)中的过程,$q_n = \frac{C}{B_n}$, 由于$B_n \to +\infty$, $C$为常数, 因此$q_n \to 0$, 因此存在$N$, 使得当$n > N$时,$|q_n| < \ve$, 则当$n > \max\{N,K\}$时,
                        $$|c_n - a| \les \left| c_n - \left(a + q_n\right)\right| + |q_n| < \ve + |q_n| < 2\ve.$$
                  \item 由Stolz定理可知
                        \begin{align*}
                            \lim_{n \to \infty} c_n = \lim_{n \to \infty} \frac{a_1b_1+a_2b_2+\cdots+a_nb_n}{b_1+b_2+\cdots+b_n} = \lim_{n \to \infty} \frac{a_nb_n}{b_n} = \lim_{n \to \infty} a_n = a.
                        \end{align*}
              \end{enumerate}
    \end{enumerate}

    上述解答中给出了符合题目原意的证明,即先证明$c_n$收敛,然后在$B_n$无界时,再证明$\lim_{n \to \infty} c_n = a$.但对于这道题而言,还可以分类$B_n$有界和无界来讨论, 即先做(2),然后对$B_n$有界时,用Cauchy收敛准则证明$\left\{ \sum_{k=1}^{n}a_n b_n \right\}$收敛,即两种分类下以截然不同的方式来证明$c_n$收敛.
\end{solution}

\begin{remark}
    $a_n := \cdots$中$:=$表示定义. 如$a_n := \frac{1}{n}$表示我们新定义了一个数列$a_n$, 其通项公式为$a_n = \frac{1}{n}$.在上文中“$C := \sum_{k=1}^{K}(a_k - a)b_k$是仅与$K$有关,与$n$无关的常数.”表示: “记$C = \sum_{k=1}^{K}(a_k - a)b_k$, 则$C$是仅与$K$有关,与$n$无关的常数.”有的地方会写为$a_n \overset{\text{def}}{=} \cdots$.
\end{remark}

\begin{exercise}[1.C.13]
    证明: $\lim_{x \to +\infty} \left(1+\frac{1}{x^p}\right)^x = \begin{cases} 1,  & p > 1,  \\ \mathrm{e},  & p = 1,  \\ \infty,  & p < 1. \end{cases}$
\end{exercise}
\begin{solution}
    实际上题目中的无穷只能是$+\infty$.

    $p>0$时, $x^p\to+\infty$
    \begin{align*}
        \lim_{x \to +\infty} \left(1+\frac{1}{x^p}\right)^x & = \lim_{x \to +\infty} \left(1+\frac{1}{x^p}\right)^{x^p\cdot\frac{1}{x^{p-1}}} = \e^{\lim\limits_{x \to +\infty}\frac{1}{x^{p-1}}} = \begin{cases} 1,  & p > 1, \\
              \e, & p = 1, \\ +\infty,  & p < 1.\end{cases}
    \end{align*}
    $p\les 0$时, $x^p\to 0$, 则考虑$x>1$时,
    $$\lim_{x \to +\infty} \left(1+\frac{1}{x^p}\right)^x\ges\lim_{x \to +\infty}  2^x=+\infty.$$
\end{solution}
\begin{exercise}[1.C.14]
    设 $f(x)$ 为周期函数,  且 $\lim_{x \to \infty} f(x) = 0$,  证明 $f(x)$ 恒为零.
\end{exercise}
\begin{solution}
    设 $f(x)$ 的正周期为 $T>0$,  $\forall \ve >0, \exists N \in \N^*, $ 当 $|x|\ges N$ 时$|f(x)|<\ve$.

    因此对于$n=\left\lceil \frac{N}{T}\right\rceil$, 有$nT\ges N$, 故对于任意 $x \in [nT,  (n+1)T)$,  有$f(x)<\ve$.

    利用周期性可以得到$\forall x \in \mathbb{R},  |f(x)| < \ve$.
    由于 $\ve$ 是任意的正数,  所以 $f(x)$ 恒为零.
\end{solution}

\begin{exercise}[1.C.15]
    证明
    \begin{enumerate}[(1)]
        \item 函数 $f(x)$ 在 $x \to x_0^-$ 时有极限 $l$ 的充分必要条件是: 对于任意一个以 $x_0$ 为极限的单调递增数列 $\{a_n\} \,  (a_n \ne x_0)$,  都有 $\lim_{n \to \infty} f(a_n) = l$;
        \item 函数 $f(x)$ 在 $x \to x_0^+$ 时有极限 $l$ 的充分必要条件是: 对于任意一个以 $x_0$ 为极限的单调递减数列 $\{a_n\} \,  (a_n \ne x_0)$,  都有 $\lim_{n \to \infty} f(a_n) = l$.
    \end{enumerate}
\end{exercise}

\begin{solution}
    \begin{enumerate}[(1)]
        \item \begin{enumerate}
                  \item 必要性:考虑任意数列 $\{a_n\}$,  使得 $\lim_{n\to\infty}a_n = x_0$且 $\{a_n\}$ 单调递增, .

                        由于$\lim_{x \to x_0^-}f(x)=l$, 因此, $\forall \ve > 0,  \exists \delta > 0$,  当 $x_0 - \delta < x < x_0$ 时,  有 $|f(x) - l| < \ve$.

                        同时对于$\delta$, $\exists N \in \N^*$, 使得当 $n > N$ 时,  有 $|a_n - x_0| < \delta$, 即 $x_0 - \delta < a_n < x_0$.

                        因此我们有$m>N$时$|f(a_n)-l|<\ve$.即得到数列$\{f(a_n)\}$收敛到$l$.
                  \item 充分性: 反证, 若$x\to x_0^-$时$f(x)$的极限为$l$不成立, 即$\exists \ve > 0,  \forall \delta > 0,  \exists x_0 - \delta < x < x_0$,  使得$|f(x) - l| \ges \ve$.

                        因此我们依次构造 $\delta_1=1, \delta_n = \min\{\frac{1}{n}, x_0-a_{n-1}\}, (n>2)$, 则$\exists a_n, x_0-\delta_n<a_n<x_0$ , 使得$|f(a_n) - l| \ges \ve$.
                        即有$a_n>a_{n-1}$, 且$|x_0-a_n|<\frac{1}{n}$.这意味着$\{a_n\}$单调递增, 且$\lim_{n \to \infty} a_n = x_0$.

                        由于$|f(a_n) - l| \ges \ve$, 所以$\{f(a_n)\}$不收敛到$l$, 矛盾, 故充分性成立.
              \end{enumerate}
        \item 证明同理.具体而言:

              设$g(x) = f(-x)$, 则$f(x)$在$x\to x_0^-$时有极限$l$ $\Leftrightarrow$ $g(x)$在$x\to -x_0^+$时有极限$l$.由(1)可知, 这等价于对于任意一个以$-x_0$为极限的单调递增数列$\{b_n\} \,  (b_n \ne -x_0)$,  都有 $\lim_{n \to \infty} g(b_n) = l$.设$a_n=-b_n$, 则$\{a_n\}$是以$x_0$为极限的单调递减数列, 且$\lim_{n \to \infty} f(a_n) = l$.因此(2)得证.
    \end{enumerate}
\end{solution}

\begin{exercise}[1.C.16]
    设 $\xi$ 是一个无理数,  $a, b$ 是实数,  且 $a<b$. 求证: 存在整数 $m, n$ 使得 $m+n\xi \in (a, b)$,  即,  集合
    $$ S = \{m+n\xi \mid m,  n \in \mathbb{Z}\} $$
    在 $\mathbb{R}$ 稠密.
\end{exercise}
\begin{solution}
    稠密的定义:设$S\subset \R$, 若对任意$a, b\in\R, a<b$, 都有$S\cap(a, b)\neq\varnothing$, 则称$S$在$\R$中稠密.

    想法是这样的,我们为了找到某个$m+n\xi$落在$(a, b)$中,于是用$\xi$构造一个充分小的实数$\ve = m_0 + n_0 \xi \in \left( 0, {b-a} \right)$.因为这个$\ve$够小,所以能证明
    存在某个$l_0\in\Z$, 使得$l_0\ve\in(a, b)$,直观理解如\autoref{fig:1.C.16} 所示.
\begin{figure}[htbp]
\centering

\begin{center}
\begin{tikzpicture}[>=Stealth, thick, scale=1.2]

% 数轴
\draw[-] (-2.5,0) -- (2.5,0);

% 刻度点坐标
\coordinate (P1) at (-2,0);
\coordinate (P2) at (-1,0);
\coordinate (P3) at (0,0);
\coordinate (P4) at (1,0);
\coordinate (P5) at (2,0);
\coordinate (A) at (-0.35,0);
\coordinate (B) at (0.85,0);

% AB → 正常刻度线(上方)
\foreach \x in {A,B}{
\draw (\x) -- ++(0,0.1);
}

% P1…P5 → 箭头+方框标注在下方
\foreach \x in {P1,P2,P3,P4,P5}{
% 小箭头竖线
\draw (\x) -- ++(0,-0.2);
}

% P 组的方框标注
\node[draw, fill=white, inner sep=2pt, below=12pt of P2] (box1) {$(l_0-1)\varepsilon$};
\node[draw, fill=white, inner sep=2pt, below=12pt of P3] (box2) {$l_0\varepsilon$};
\node[draw, fill=white, inner sep=2pt, below=12pt of P4] (box3) {$\left\lceil\dfrac{b}{\varepsilon}\right\rceil\varepsilon$};

% 箭头连接方框→刻度
\draw[-{Triangle[scale=1.1]}] (box1.north) -- (P2);
\draw[-{Triangle[scale=1.1]}] (box2.north) -- (P3);
\draw[-{Triangle[scale=1.1]}] (box3.north) -- (P4);

% AB 上方文字
\node[above] at (A) {$a$};
\node[above] at (B) {$b$};

% 间距 ε
\draw[|-|] (P1) ++(0,+0.5) -- ++(1,0)
    node[midway,fill=white,inner sep=1pt] {$\varepsilon$};

\end{tikzpicture}

\begin{tikzpicture}[>=Stealth, thick, scale=1.2]

% 数轴
\draw[-] (-2.5,0) -- (2.5,0);

% 刻度点坐标
\coordinate (P1) at (-2,0);
\coordinate (P2) at (-1,0);
\coordinate (P3) at (0,0);
\coordinate (P4) at (1,0);
\coordinate (P5) at (2,0);
\coordinate (A) at (-1.55,0);
\coordinate (B) at (0.75,0);

% AB → 正常刻度线(上方)
\foreach \x in {A,B}{
\draw (\x) -- ++(0,0.1);
}

% P1…P5 → 箭头+方框标注在下方
\foreach \x in {P1,P2,P3,P4,P5}{
% 小箭头竖线
\draw (\x) -- ++(0,-0.2);
}

% P 组的方框标注
\node[draw, fill=white, inner sep=2pt, below=12pt of P2] (box1) {$(l_0-1)\varepsilon$};
\node[draw, fill=white, inner sep=2pt, below=12pt of P3] (box2) {$l_0\varepsilon$};
\node[draw, fill=white, inner sep=2pt, below=12pt of P4] (box3) {$\left\lceil\dfrac{b}{\varepsilon}\right\rceil\varepsilon$};

% 箭头连接方框→刻度
\draw[-{Triangle[scale=1.1]}] (box1.north) -- (P2);
\draw[-{Triangle[scale=1.1]}] (box2.north) -- (P3);
\draw[-{Triangle[scale=1.1]}] (box3.north) -- (P4);

% AB 上方文字
\node[above] at (A) {$a$};
\node[above] at (B) {$b$};

% 间距 ε
\draw[|-|] (P1) ++(0,+0.5) -- ++(1,0)
    node[midway,fill=white,inner sep=1pt] {$\varepsilon$};

\end{tikzpicture}

\begin{tikzpicture}[>=Stealth, thick, scale=1.2]

% 数轴
\draw[-] (-2.5,0) -- (2.5,0);

% 刻度点坐标
\coordinate (P1) at (-2,0);
\coordinate (P2) at (-1,0);
\coordinate (P3) at (0,0);
\coordinate (P4) at (1,0);
\coordinate (P5) at (2,0);
\coordinate (A) at (-0.45,0);
\coordinate (B) at (1,0);

% AB → 正常刻度线(上方)
\foreach \x in {A,B}{
\draw (\x) -- ++(0,0.1);
}

% P1…P5 → 箭头+方框标注在下方
\foreach \x in {P1,P2,P3,P4,P5}{
% 小箭头竖线
\draw (\x) -- ++(0,-0.2);
}

% P 组的方框标注
\node[draw, fill=white, inner sep=2pt, below=12pt of P2] (box1) {$(l_0-1)\varepsilon$};
\node[draw, fill=white, inner sep=2pt, below=12pt of P3] (box2) {$l_0\varepsilon$};
\node[draw, fill=white, inner sep=2pt, below=12pt of P4] (box3) {$\left\lceil\dfrac{b}{\varepsilon}\right\rceil\varepsilon$};

% 箭头连接方框→刻度
\draw[-{Triangle[scale=1.1]}] (box1.north) -- (P2);
\draw[-{Triangle[scale=1.1]}] (box2.north) -- (P3);
\draw[-{Triangle[scale=1.1]}] (box3.north) -- (P4);

% AB 上方文字
\node[above] at (A) {$a$};
\node[above] at (B) {$b$};

% 间距 ε
\draw[|-|] (P1) ++(0,+0.5) -- ++(1,0)
    node[midway,fill=white,inner sep=1pt] {$\varepsilon$};

\end{tikzpicture}
\end{center}

    \caption{$a, b$之间的区间长度大于$\varepsilon$, 因此存在某个$l_0\in\Z$, 使得$l_0\varepsilon\in(a, b)$.}这里的思路和\hyperref[exe:1.1.2]{习题1.1.2}中证明两个无理数之间存在有理数的思路是类似的.
    \label{fig:1.C.16}
\end{figure}


    随后我们取$m=l_0m_0, n=l_0n_0$即有$m+n\xi=l_0\ve\in(a, b)$.

    \paragraph*{构造$\ve$}

    实际上, 对于$b-a>0$, 总存在$k\in\N^*$, 使得$\frac{1}{k}<{b-a}$.因此我们考虑构造一个满足$\ve<\frac{1}{k}, \ve\in S$即可.

    对于$l=1, 2, \cdots, k+1$, 我们考虑
    \begin{align*}
        n_l & =\left\lfloor l\xi \right\rfloor \\
        x_l & =l\xi-n_l \in S.
    \end{align*}
    $x_l$是$l\xi$的小数部分, 容易知道$x_l\in[0, 1)$, 并且$x_l$之间总是两两不同的, 否则$i\xi-n_i=j\xi-n_j, i\neq j$, 这意味着$\xi=\frac{n_i-n_j}{i-j}$, 这与$\xi$为无理数矛盾.

    因此对于$$[0, 1)=\bigcup_{j=1}^{k} \left[ \frac{j-1}{k}, \frac{j}{k} \right), $$
    这$k$个区间包括了$k+1$个不同实数$x_l$.因此总有一个区间内部存在同时两个实数, 记为$x_p, x_q \in S, p\neq q$, 不妨认为$x_q>x_p$.

    由$x_l$的构造 $x_p=p\xi-n_p, x_q=q\xi-n_q,$
    有$$ x_q-x_p=(q-p)\xi-(n_p-x_q)\in S,$$
    且由于$x_p, x_q$落在同一个区间内, 而区间长度为$\frac{1}{k}$,因此$0<x_q-x_p\les\frac{1}{k}< {b-a},$
    所以$x_q-x_p$满足我们对$\ve$的要求.我们取$$\ve=x_q-x_p.$$

    \paragraph*{构造$m, n$}

    我们先证明$\exists l_0\in\Z, \st l_0\ve\in(a, b)$:我们取$l_0 = \llceil \frac{b}{\ve} \rrceil - 1$,则$$l_0 \ve = \left(\llceil \frac{b}{\ve} \rrceil - 1\right)\ve < \left(\frac{b}{\ve} + 1 - 1\right)\ve = b.$$
    同时,由于$\ve < {b-a}$, 因此$$l_0 \ve = \left(\llceil \frac{b}{\ve} \rrceil - 1\right)\ve \ges \left(\frac{b}{\ve} - 1\right)\ve = b - \ve > b - (b-a) = a.$$
    因此$l_0\ve\in(a, b)$.
    
    于是令$$m=l_0(n_q-n_p), n=l_0(q-p)$$
    即有$m+n\xi=l_0 (n_q-n_p)+l_0(q-p)\xi=l_0 \left( (q-p)\xi-(n_p-n_q) \right)=l_0\ve\in(a, b).$
\end{solution}





\newpage