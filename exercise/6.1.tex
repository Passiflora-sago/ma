\chapter{常微分方程初步}
\section{习题 6.1}

\begin{exercise}[6.1.1]
    求下列可分离变量的微分方程.
    \begin{tasks}(2)
        \task $(1 + x^2)\mathrm{d}y = y \mathrm{d}x;$
        \task $y' = \mathrm{e}^{x-y};$
        \task $xy' + y = y^2;$
        \task $yy' = \frac{1 - 2x}{y}.$
    \end{tasks}
\end{exercise}

\begin{solution}
    \begin{enumerate}
        \item 不难验证$y\equiv0$是方程的解. 当$y\neq0$时,
              \begin{align*}
                           & (1 + x^2)\mathrm{d}y = y \mathrm{d}x                    \\
                  \implies & \frac{1}{y} \mathrm{d}y = \frac{1}{1 + x^2} \mathrm{d}x \\
                  \implies & \ln |y| = \arctan x + C                                 \\
                  \implies & y = C' \mathrm{e}^{\arctan x}, \quad C' = \e^C
              \end{align*}
              综上, 方程的解为
              $$y = C \e^{\arctan x}$$
        \item \begin{align*}
                           & y' = \mathrm{e}^{x-y}                                    \\
                  \implies & \mathrm{e}^y \mathrm{d}y = \mathrm{e}^x \mathrm{d}x      \\
                  \implies & \mathrm{e}^y = \mathrm{e}^x + C ,\quad C > -\mathrm{e}^x \\
                  \implies & y = \ln (\mathrm{e}^x + C)
              \end{align*}
        \item 不难验证$y\equiv 0,1$是方程的解. 当$y\neq0,1$时,
              \begin{align*}
                           & xy' + y = y^2                                      \\
                  \implies & \frac{\mathrm{d}y}{y(y-1)} = \frac{\mathrm{d}x}{x} \\
                  \implies & \ln \left| \frac{y-1}{y} = \ln |x| + C \right.     \\
                  \implies & \frac{y-1}{y} = C' x, \quad C' = \e^C \neq 0       \\
                  \implies & y = \frac{1}{1 - C' x}
              \end{align*}
              综上, 方程的解为
              $$y = 0; \quad y = \frac{1}{1 - C x}$$
        \item \begin{align*}
                           & yy' = \frac{1 - 2x}{y}                               \\
                  \implies & y^2 \mathrm{d}y = (1 - 2x) \mathrm{d}x               \\
                  \implies & \frac{1}{3} y^3 = x - x^2 + C                        \\
                  \implies & y = \sqrt[3]{3x - 3x^2 + C}.
              \end{align*}
              且由题设得$y \ne 0$,因此方程的所有解为:
              %   \begin{align*}
              %            & \left\{y =  \sqrt[3]{3x - 3x^2 + C} \mid C < -\frac{3}{4} \right\}                                                    \\
              %       \cup & \left\{y =  \sqrt[3]{3x - 3x^2 + C} \mid C \ges -\frac{3}{4}, x \in \left(-\infty, \frac{1}{2} \right) ,y < 0\right\} \\
              %       \cup & \left\{y =  \sqrt[3]{3x - 3x^2 + C} \mid C \ges -\frac{3}{4}, x \in \left(\frac{1}{2}, +\infty \right) ,y < 0\right\} \\
              %       \cup & \left\{y = \sqrt[3]{3x - 3x^2 + C} \mid C \ges -\frac{3}{4} , y > 0 \right\}
              %   \end{align*}
              $$\frac{3x-3x^2+C}{y^3} = 1$$
              \begin{remark}
                  后面为了方便,我们也将所有解写成:
                  $$y^3 = 3x - 3x^2 + C, \quad y \ne 0.$$
              \end{remark}
    \end{enumerate}
\end{solution}

\begin{exercise}[6.1.2]
    求下列微分方程.
    \begin{tasks}(2)
        \task $y' = \frac{y^2}{x^2} - 2;$
        \task $y' = \frac{y}{x} + \frac{x}{y};$
        \task $\displaystyle \frac{\mathrm{d}x}{x^2 - xy + y^2} = \frac{\mathrm{d}y}{2y^2 - xy};$
        \task $(x^2 + 3y^2)\mathrm{d}x - 2xy\mathrm{d}y = 0.$
    \end{tasks}
\end{exercise}

\begin{solution}
    \begin{enumerate}
        \item 记$p = \frac{y}{x}$, 则$y = px$, $y' = p'x + p$. 代入方程得
              \begin{align*}
                           & p'x + p = p^2 - 2                                                                                      \\
                  \implies & \frac{\mathrm{d}p}{p^2 - p - 2} = \frac{\mathrm{d}x}{x}                                                \\
                  \implies & \int \frac13 \left( \frac{1}{p - 2} - \frac{1}{p + 1} \right) \mathrm{d}p = \int \frac{\mathrm{d}x}{x} \\
                  \implies & \ln |p - 2| - \ln |p + 1| = 3\ln |x| + C                                                               \\
                  \implies & \frac{p - 2}{p + 1} = C' x^3, \quad C' = \e^C > 0                                                      \\
                  \implies & \frac{y - 2x}{x+y} = C' x^3                                                                            \\
                  \implies & y = \frac{2x + C' x^4}{1 - C' x^3}
              \end{align*}
              同时注意到$y - 2x = \frac{3C'x^4}{1-C' x^3}$, 函数被分为$y-2x \ges 0, y - 2x < 0$两支,这两支的解是独立的,因此综上所述
              $$y = \frac{2x + C_1 x^4}{1 - C_1 x^3}, C_1 \in \mathbb{R}, y - 2x \ges 0; \quad y = \frac{2x + C_2 x^4}{1 - C_2 x^3}, C_2 \in \mathbb{R}, y - 2x < 0.$$
              \begin{remark}
                  为了方便,我们也写为
                  $$y = \frac{2x + C x^4}{1 - C x^3},\quad x \ne 0$$
              \end{remark}
        \item 记$p = \frac{y}{x}$, 则$y = px$, $y' = p'x + p$. 代入方程得
              \begin{align*}
                           & p'x + p = p + \frac{1}{p}                  \\
                  \implies & p' = \frac{1}{p x}                         \\
                  \implies & p \mathrm{d}p = \frac{\mathrm{d}x}{x}      \\
                  \implies & \frac{1}{2} p^2 = \ln |x| + C              \\
                  \implies & y^2 = x^2 (2 \ln |x| + C'), \quad C' = 2C.
              \end{align*}
              由于微分方程在$x = 0$处不连续, $x$正负半轴的解是独立的,
              $$y^2 = x^2 (2 \ln x + C_1), x > 0; \quad y^2 = x^2 (2 \ln (-x) + C_2), x < 0.$$
              或者$$y^2 = x^2 (2 \ln |x| + C), x \ne 0.$$
        \item 记$p = \frac{y}{x}$, 则$y = px$, $ y' = p'x + p$. $y = x,y=2x$显然为方程的解,$u \ne 0,1,2$时代入方程得
              \begin{align*}
                           & \frac{1}{x^2 - x(px) + (px)^2} \mathrm{d}x = \frac{1}{2(px)^2 - x(px)} \mathrm{d}(px)                                            \\
                  \implies & \frac{\dif x}{x} = \frac{p^2 -p +1}{p(p-1)(p-2)} \dif p                                                                          \\
                  \implies & \int \frac{\dif x}{x} = \int \left( - \frac{1}{2} \frac{\dif p}{p} + \frac{\dif p}{p-1} - \frac{3}{2} \frac{\dif p}{p-2} \right) \\
                  \implies & x^{-1} p^\frac12 (p-1)^{-1} (p-2)^\frac32 = C                                                                                         \\
                  \implies & C'(y-x)^2 =y(y-2x)^3, \quad C' = C^2
              \end{align*}
              综上, 方程的解为
              $$ C_1 (y - x)^2 = y (y - 2x)^3, y > 0, \quad C_2 (y - x)^2 = y (y - 2x)^3, y < 0, \quad y = x, y \ne 0, \quad y = 2x, y \ne 0$$
              或者 
              $$ C (y - x)^2 = y (y - 2x)^3, y \ne 0, \quad y = x, y \ne 0, \quad y = 2x, y \ne 0$$
              或者 
              $$ C (y - x)^2 = y (y - 2x)^3, y \ne 0, \quad y = x, y \ne 0$$
        \item 记$p = \frac{y}{x}$, 则$y = px$, $ y' = p'x + p$. 代入方程得$x \ne 0$时,
              \begin{align*}
                           & (x^2 + 3 (px)^2) \mathrm{d}x - 2x(px) \mathrm{d}(px)       \\
                  \implies & \frac{2p}{1+p^2} \dif p = \frac{1}{x} \dif x               \\
                  \implies & \ln\left(1+p^2\right) = \ln |x| + C                        \\
                  \implies & 1 + \left(\frac{y}{x}\right)^2 = C' x, \quad C' = \e^C > 0 \\
                  \implies & x^2 + y^2 = C' x^3
              \end{align*}
              综上, 方程的解为
              $$x^2 + y^2 = C x^3$$
    \end{enumerate}
\end{solution}

\begin{exercise}[6.1.3]
    证明: 形如
    $$ \frac{\mathrm{d}y}{\mathrm{d}x} = f \left( \frac{a_1 x + b_1 y + c_1}{a_2 x + b_2 y + c_2} \right) $$
    的方程, 可通过代换化为齐次方程. (提示: 若方程组
    $$ \begin{cases} a_1 x + b_1 y + c_1 = 0, \\ a_2 x + b_2 y + c_2 = 0 \end{cases} $$
    有非零解 $x_0, y_0$ (即 $x_0, y_0$ 不全为零), 则可令 $u = x - x_0, v = y - y_0$; 其他情形更易于处理.)

    求下面方程:
    \begin{tasks}(2)
        \task $\displaystyle \frac{\mathrm{d}y}{\mathrm{d}x} = \frac{x+y+3}{x-y+1};$
        \task $\displaystyle \frac{\mathrm{d}y}{\mathrm{d}x} = \frac{2x+4y+3}{x+2y+1}.$
    \end{tasks}
\end{exercise}

\begin{solution}
    对线性方程组
    $$ \begin{cases} a_1 x + b_1 y + c_1 = 0, \\ a_2 x + b_2 y + c_2 = 0 \end{cases} $$
    的系数行列式 $D = a_1 b_2 - a_2 b_1$ 进行讨论.

    \begin{enumerate}
        \item $D \neq 0$:此时方程组有唯一解 $(x_0, y_0)$. 令 $u = x - x_0, v = y - y_0$, 则 $\mathrm{d}u = \mathrm{d}x, \mathrm{d}v = \mathrm{d}y$, 且 $\frac{\mathrm{d}y}{\mathrm{d}x} = \frac{\mathrm{d}v}{\mathrm{d}u}$.
              原方程变为:
              $$ \frac{\mathrm{d}v}{\mathrm{d}u} = f \left( \frac{a_1 (u + x_0) + b_1 (v + y_0) + c_1}{a_2 (u + x_0) + b_2 (v + y_0) + c_2} \right) $$
              因为 $(x_0, y_0)$ 是方程组的解, 所以 $a_1 x_0 + b_1 y_0 + c_1 = 0$ 且 $a_2 x_0 + b_2 y_0 + c_2 = 0$.
              上式化为:
              $$ \frac{\mathrm{d}v}{\mathrm{d}u} = f \left( \frac{a_1 u + b_1 v}{a_2 u + b_2 v} \right) = f \left( \frac{a_1 + b_1 (v/u)}{a_2 + b_2 (v/u)} \right) $$
              这是一个关于 $u, v$ 的齐次方程.

        \item $D = 0$:此时 $a_1 b_2 = a_2 b_1$. 我们可以设 $a_1 = k a_2, b_1 = k b_2$ (假设 $a_2, b_2$ 不全为零).
              \begin{itemize}
                  \item 如果 $c_1 = k c_2$,  $\frac{a_1 x + b_1 y + c_1}{a_2 x + b_2 y + c_2} = k$ 为常数. 方程变为 $\frac{\mathrm{d}y}{\mathrm{d}x} = f(k)$, 其解为 $y = f(k)x + C$.
                  \item 如果 $c_1 \neq k c_2$,
                        令 $z = a_2 x + b_2 y$, 则 $\frac{\mathrm{d}z}{\mathrm{d}x} = a_2 + b_2 \frac{\mathrm{d}y}{\mathrm{d}x}$.
                        原方程的右边可以写成 $z$ 的函数:
                        $$ \frac{\mathrm{d}y}{\mathrm{d}x} = f \left( \frac{k(a_2 x + b_2 y) + c_1}{a_2 x + b_2 y + c_2} \right) = f \left( \frac{k z + c_1}{z + c_2} \right) $$
                        代入 $\frac{\mathrm{d}z}{\mathrm{d}x}$ 的表达式:
                        $$ \frac{1}{b_2} \left( \frac{\mathrm{d}z}{\mathrm{d}x} - a_2 \right) = f \left( \frac{k z + c_1}{z + c_2} \right) \implies \frac{\mathrm{d}z}{\mathrm{d}x} = a_2 + b_2 f \left( \frac{k z + c_1}{z + c_2} \right) $$
                        这是一个变量分离方程 $\frac{\mathrm{d}z}{a_2 + b_2 f(\dots)} = \mathrm{d}x$.
              \end{itemize}
    \end{enumerate}
    综上, 该类方程总能通过代换化为齐次方程或变量分离方程.

    \begin{enumerate}[(1)]
        \item 解方程组 $\begin{cases} x+y+3=0 \\ x-y+1=0 \end{cases}$, 得 $x=-2, y=-1$.
              令 $u = x+2, v = y+1$. 则 $\frac{\mathrm{d}v}{\mathrm{d}u} = \frac{(u-2)+(v-1)+3}{(u-2)-(v-1)+1} = \frac{u+v}{u-v}$.
              这是齐次方程. 令 $p = v/u$, 则 $v=pu, \frac{\mathrm{d}v}{\mathrm{d}u} = u \frac{\mathrm{d}p}{\mathrm{d}u} + p$.
              $$ u \frac{\mathrm{d}p}{\mathrm{d}u} + p = \frac{u+pu}{u-pu} = \frac{1+p}{1-p} \implies u \frac{\mathrm{d}p}{\mathrm{d}u} = \frac{1+p}{1-p} - p = \frac{1+p-p+p^2}{1-p} = \frac{1+p^2}{1-p} $$
              分离变量: $\frac{1-p}{1+p^2} \mathrm{d}p = \frac{\mathrm{d}u}{u}$.
              积分: $$\displaystyle \int \frac{1-p}{1+p^2} \mathrm{d}p = \int \frac{1}{1+p^2} \mathrm{d}p - \int \frac{p}{1+p^2} \mathrm{d}p = \int \frac{\mathrm{d}u}{u}$$得到
              $$ \arctan(p) - \frac{1}{2} \ln(1+p^2) = \ln|u| + C $$
              代回 $p=v/u$:
              $$ \arctan\left(\frac{v}{u}\right) - \frac{1}{2} \ln\left(1+\frac{v^2}{u^2}\right) = \ln|u| + C $$
              $$ \arctan\left(\frac{v}{u}\right) = \frac{1}{2} \ln(u^2+v^2) - \frac{1}{2}\ln(u^2) + \ln|u| + C = \frac{1}{2} \ln(u^2+v^2) + C $$
              代回 $u=x+2, v=y+1$:
              $$ \arctan\left(\frac{y+1}{x+2}\right) - \frac{1}{2} \ln((x+2)^2 + (y+1)^2) = C $$
                即,微分方程的解为
                $$y+1 = (x+2) \tan\left( \frac{1}{2} \ln((x+2)^2 + (y+1)^2) + C \right), x - y + 1 \ne 0$$
        \item 这里 $a_1=2, b_1=4, a_2=1, b_2=2$. $D = 2 \cdot 2 - 4 \cdot 1 = 0$. 令 $z = x+2y$. 则 $\frac{\mathrm{d}z}{\mathrm{d}x} = 1 + 2 \frac{\mathrm{d}y}{\mathrm{d}x}$.
              原方程变为 $\frac{\mathrm{d}y}{\mathrm{d}x} = \frac{2(x+2y)+3}{x+2y+1} = \frac{2z+3}{z+1}$.
              代入 $\frac{\mathrm{d}z}{\mathrm{d}x}$ 的表达式:
              $$ \frac{\mathrm{d}z}{\mathrm{d}x} = 1 + 2\left(\frac{2z+3}{z+1}\right) = \frac{z+1+4z+6}{z+1} = \frac{5z+7}{z+1} $$
              分离变量: $\frac{z+1}{5z+7} \mathrm{d}z = \mathrm{d}x$.
              积分: $$\displaystyle \int \frac{z+1}{5z+7} \mathrm{d}z = \int \frac{\frac{1}{5}(5z+7) - \frac{7}{5} + 1}{5z+7} \mathrm{d}z = \int \left(\frac{1}{5} - \frac{2/5}{5z+7}\right) \mathrm{d}z = \int \mathrm{d}x$$得到
              $$ \frac{1}{5}z - \frac{2}{25} \ln|5z+7| = x + C $$
              代回 $z=x+2y$:
              $$ \frac{1}{5}(x+2y) - \frac{2}{25} \ln|5(x+2y)+7| = x + C $$
              整理得:
              $$ 5(x+2y) - 2 \ln|5x+10y+7| = 25x + C' $$
              $$ 10y - 20x - 2 \ln|5x+10y+7| = C' $$
              $$ 10x - 5y + \ln|5x+10y+7| = C'' $$
              即,微分方程的解为
              $$ 10x - 5y + \ln|5x+10y+7| = C, x+2y+1 \ne 0$$
    \end{enumerate}
\end{solution}

\begin{exercise}[6.1.4]
    求下列线性方程和 Bernoulli 方程的解.
    \begin{tasks}(2)
        \task $(1 + x^2)y' - 2xy = (1 + x^2)^2;$
        \task $y' + \frac{1-2x}{x} = 1;$
        \task $y' = \frac{y}{x + y^3};$
        \task $y' + \frac{y}{x} = y^2 \ln x;$
        \task $y' = y \tan x + y^2 \cos x;$
        \task $y - y' \cos x = y^2 (1 - \sin x) \cos x.$
    \end{tasks}
\end{exercise}

\begin{solution}
    \begin{enumerate}
        \item 转为标准形式:$$y' + \left(-\frac{2x}{1+x^2}\right) y = 1 + x^2.$$
              则$\int P(x) \dif x = \int -\frac{2x}{1+x^2} \dif x = -\ln(1+x^2)$.
              根据线性方程通解公式:
              \begin{align*}
                  y & = \e^{-\int P(x) \dif x} \left( \int Q(x) \e^{\int P(x) \dif x} \dif x + C \right) \\
                    & = (1+x^2) \left( \int (1+x^2) \frac{1}{1+x^2} \dif x + C \right)                   \\
                    & = (1+x^2) (x+C).
              \end{align*}
              综上, 方程的解为
              $$y = (x + C)(1+x^2)$$
        \item 原方程可写为 $y' = 1 - \frac{1-2x}{x} = 1 - \frac{1}{x} + 2 = 3 - \frac{1}{x}$.
              分离变量得 $\dif y = \left(3 - \frac{1}{x}\right) \dif x$. 积分得
              $$y = 3x - \ln|x| + C$$
        \item 将方程视为 $x$ 关于 $y$ 的函数. 原方程可写为 $\frac{\mathrm{d}x}{\mathrm{d}y} = \frac{x+y^3}{y} = \frac{x}{y} + y^2$.
              整理得: $$\frac{\mathrm{d}x}{\mathrm{d}y} - \frac{1}{y} x = y^2.$$
              这是一个关于 $x$ 的线性方程. $P(y) = -\frac{1}{y}, Q(y) = y^2$.
              $\int P(y) \dif y = -\ln|y|$.
              通解为:
              \begin{align*}
                  x & = \e^{-(-\ln|y|)} \left( \int y^2 \e^{-\ln|y|} \dif y + C \right) \\
                    & = |y| \left( \int y^2 \frac{1}{|y|} \dif y + C \right)            \\
                    & = y \left( \int y \dif y + C \right) \quad (\text{假设 } y > 0)   \\
                    & = y \left( \frac{1}{2}y^2 + C \right) = \frac{y^3}{2} + Cy.
              \end{align*}
              解为 $x = \frac{y^3}{2} + Cy, x+y^3 \ne 0$
        \item 这是 Bernoulli 方程. $y \equiv 0$ 是一个解. 当 $y \neq 0$ 时, 方程两边同除以 $y^2$:
              $$y^{-2} y' + \frac{1}{x} y^{-1} = \ln x.$$
              作代换 $u = y^{-1}$, 则 $u' = -y^{-2}y'$. 方程变为:
              $$-u' + \frac{1}{x} u = \ln x \implies u' - \frac{1}{x} u = -\ln x.$$
              这是一个线性方程. $\int P(x) \dif x = -\ln|x|$.
              \begin{align*}
                  u & = \e^{-(-\ln|x|)} \left( \int (-\ln x) \e^{-\ln|x|} \dif x + C \right)         \\
                    & = |x| \left( \int (-\ln x) \frac{1}{|x|} \dif x + C \right)                    \\
                    & = x \left( -\int \frac{\ln x}{x} \dif x + C \right) \quad (\text{假设 } x > 0) \\
                    & = x \left( -\frac{1}{2}(\ln x)^2 + C \right).
              \end{align*}
              所以 $y^{-1} = x \left( C - \frac{1}{2}(\ln x)^2 \right)$, 即 $y = \frac{1}{x \left( C - \frac{1}{2}(\ln x)^2 \right)}$.
              通解为 
              $$y = 0 \quad \text{ 和 }\quad 1 = y x \left( C - \frac{1}{2}(\ln x)^2 \right).$$
        \item 这是 Bernoulli 方程. $y \equiv 0$ 是一个解. 当 $y \neq 0$ 时, 方程写为 $y' - (\tan x)y = (\cos x)y^2$.
              两边同除以 $y^2$: $y^{-2}y' - (\tan x)y^{-1} = \cos x$.
              令 $u = y^{-1}$, 则 $u' = -y^{-2}y'$. 方程变为:
              $$-u' - (\tan x)u = \cos x \implies u' + (\tan x)u = -\cos x.$$
              这是一个线性方程. $\int \tan x \dif x = -\ln|\cos x|$.
              \begin{align*}
                  u & = \e^{-(-\ln|\cos x|)} \left( \int (-\cos x) \e^{-\ln|\cos x|} \dif x + C \right) \\
                    & = |\cos x| \left( \int (-\cos x) \frac{1}{|\cos x|} \dif x + C \right).
              \end{align*}
              假设 $\cos x > 0$, 则 $u = \cos x \left( \int (-1) \dif x + C \right) = \cos x (-x+C)$.
              所以 $y^{-1} = (C-x)\cos x$, 即 $y = \frac{1}{(C-x)\cos x}$.
              通解为 $$y = 0 ,\text{ 和 } 1 = y \cos x (C - x).$$
        \item 这是 Bernoulli 方程. $y \equiv 0$ 是一个解. 当 $y \neq 0$ 时, 将方程整理为标准形式:
              $$ y' \cos x = y - y^2(1-\sin x)\cos x \implies y' - (\sec x)y = -(1-\sin x)y^2. $$
              两边同除以 $y^2$: $y^{-2}y' - (\sec x)y^{-1} = -(1-\sin x)$.
              令 $u = y^{-1}$, 则 $u' = -y^{-2}y'$. 方程变为:
              $$-u' - (\sec x)u = -(1-\sin x) \implies u' + (\sec x)u = 1-\sin x.$$
               这是一个线性方程. $\int \sec x \dif x = \ln|\sec x + \tan x|$.
              \begin{align*}
                  u & = \e^{-\ln|\sec x + \tan x|} \left( \int (1-\sin x) \e^{\ln|\sec x + \tan x|} \dif x + C \right) \\
                    & = \frac{1}{|\sec x + \tan x|} \left( \int (1-\sin x)|\sec x + \tan x| \dif x + C \right).
              \end{align*}
              假设 $\sec x + \tan x > 0$, 则
              \begin{align*}
                  \int (1-\sin x)(\sec x + \tan x) \dif x & = \int (\sec x + \tan x - \tan x - \sin x \tan x) \dif x                                 \\
                                                          & = \int (\sec x - \frac{\sin^2 x}{\cos x}) \dif x = \int \frac{1-\sin^2 x}{\cos x} \dif x \\
                                                          & = \int \cos x \dif x = \sin x.
              \end{align*}
              所以 $u = \frac{1}{\sec x + \tan x}(\sin x + C)$.

            \begin{remark}
                此处只需要考虑 $\sec x + \tan x > 0$ 的情形, 从另一个角度来理解是这样的:
\begin{align*}
                \diff \left( (\tan x + \sec x) u \right) &=\left( (\sec x + \tan x )u' + \sec x(\sec x + \tan x) u \right)\dif x\\
                &= \left( (\sec x + \tan x) (1-\sin x) \right) \dif x
              \end{align*}
因此$$(\tan x + \sec x) u = \int (\sec x + \tan x)(1-\sin x) \dif x + C = \sin x + C.$$
             
            \end{remark}
              故微分方程的解为
              $$y = 0 \quad \text{ 和 } \quad y (\sin x + C) = \sec x + \tan x.$$
    \end{enumerate}
\end{solution}

\begin{exercise}[6.1.5]
    求下列方程满足初值条件的特解.
    \begin{tasks}(2)
        \task $y' = \frac{y}{x} \ln \frac{y}{x}, y(1) = 1;$
        \task $y' + \frac{y}{x} = \frac{\sin x}{x}, y(\pi) = 1.$
    \end{tasks}
\end{exercise}

\begin{solution}
    \begin{enumerate}
        \item 这是齐次方程. 令 $u = y/x$, 则 $y' = u'x + u$.
              方程变为 $$u'x = u(\ln u - 1).$$
              分离变量: $$\frac{\mathrm{d}u}{u(\ln u - 1)} = \frac{\mathrm{d}x}{x}.$$
              积分得 $$\ln u - 1 = Cx.$$
              所以 $u = \e^{1+Cx}$, 即 $y = x \e^{1+Cx}$.
              代入初值 $y(1)=1$, 得 $1 = 1 \cdot \e^{1+C}$, 解得 $C=-1$.
              故特解为 $$y = x \e^{1-x}, x > 0.$$
        \item 这是线性方程. 积分因子为 $I(x) = \e^{\int \frac{1}{x} \mathrm{d}x} = \e^{\ln x} = x$ (因 $x>0$).
              通解为
              \begin{align*}
                  y & = \frac{1}{x} \left( \int x \cdot \frac{\sin x}{x} \mathrm{d}x + C \right)        \\
                    & = \frac{1}{x} \left( \int \sin x \mathrm{d}x + C \right) = \frac{-\cos x + C}{x}.
              \end{align*}
              代入初值 $y(\pi)=1$, 得 $1 = \frac{-\cos\pi + C}{\pi} = \frac{1+C}{\pi}$.
              解得 $C = \pi - 1$.
              故特解为 $$y = \frac{-\cos x + \pi - 1}{x}, x > 0.$$
    \end{enumerate}
\end{solution}

\begin{exercise}[6.1.6]
    求解下列微分方程.
    \begin{tasks}(2)
        \task $y' + x = \sqrt{x^2 + y};$
        \task $y' = \cos(x - y);$
        \task $y' - \mathrm{e}^{x-y} + \mathrm{e}^x = 0;$
        \task $y' \sin y + x \cos y + x = 0.$
    \end{tasks}
\end{exercise}

\begin{solution}
    \begin{enumerate}
        \item 令$u^2 = x^2 + y$, 则$y = u^2 - x^2$, $y' = 2u u' - 2x$. 代入方程得
              \begin{align*}
                           & y' + x = \sqrt{x^2 + y}                                                                                         \\
                  \implies & 2u u' - 2x + x = u                                                                                              \\
                  \implies & 2vx (x v' + v) = x + vx (u = vx)                                                                                \\
                  \implies & (1-p)^{-2} = C|x|\\
                  \implies & y = -2x^4 + C x^6
              \end{align*}
        \item 令$u = x - y$, 则$y = x - u$, $y' = 1 - u'$. 代入方程得
              \begin{align*}
                           & y' = \cos(x - y)                                  \\
                  \implies & 1 - u' = \cos u                                   \\
                  \implies & u' = 1 - \cos u                                   \\
                  \implies & \frac{\dif u}{1 - \cos u} = \mathrm{d}x           \\
                  \implies & \int \frac{\dif u}{1 - \cos u} = \int \mathrm{d}x
              \end{align*}
              注意到 $\frac{1}{1-\cos u} = \frac{1}{2} \csc^2\left(\frac{u}{2}\right)$, 所以
              $$ \int \frac{\dif u}{1 - \cos u} = \int \frac{1}{2} \csc^2\left(\frac{u}{2}\right) \dif u = -2 \cot\left(\frac{u}{2}\right) + C. $$
              故方程的通解为
              $$ -2 \cot\left(\frac{x-y}{2}\right) = x + C. $$
        \item 令$u =\e^{y}, u' = \e^y y'$,代入方程得 
        \begin{align*}
            & y' - \mathrm{e}^{x-y} + \mathrm{e}^x = 0 \\
            \implies & \e^y y' - \e^x + \e^x \e^y = 0 \\
            \implies & u' - \e^x + \e^x u = 0 \\
            \implies & \frac{1}{1-u} \dif u = \e^x \dif x \\
            \implies & -\ln|1-u| = \e^x + C \\
            \implies & 1 - \e^y = C' \e^{-\e^x}
        \end{align*}
        \item 直接分离变量,并验证$y = (2k+1) \pi,k \in \Z$为方程的特解,
        因此\begin{align*}
            &\ln | 1+ \cos y| = \frac{1}{2} x^2 + C \\
            \implies & 1 + \cos y = C' \e^{\frac{1}{2} x^2} \\
            \end{align*}
    \end{enumerate}
\end{solution}

\begin{exercise}[6.1.7]
    试用常数变易法导出 Bernoulli 方程的通解.
\end{exercise}

\begin{solution}
    Bernoulli 方程的一般形式为:
    $$ \frac{\mathrm{d}y}{\mathrm{d}x} + P(x)y = Q(x)y^n, \quad (n \neq 0, 1) $$
    我们使用常数变易法来求解.

    \begin{enumerate}
        \item \textbf{求解对应的齐次线性方程}

              我们先考虑与 Bernoulli 方程相关的齐次线性方程:
              $$ \frac{\mathrm{d}y}{\mathrm{d}x} + P(x)y = 0 $$
              这是一个可分离变量的方程:
              $$ \frac{\mathrm{d}y}{y} = -P(x)\mathrm{d}x $$
              两边积分, 得 $\ln|y| = -\int P(x)\mathrm{d}x + C_0$, 其通解为:
              $$ y = C {-\int P(x)\mathrm{d}x} $$
              其中 $C$ 是任意常数.

        \item \textbf{使用常数变易法}

              现在, 我们假设原 Bernoulli 方程的解具有与上述齐次解相似的形式, 但将常数 $C$ 替换为一个关于 $x$ 的函数 $C(x)$.
              设原方程的解为:
              $$ y(x) = C(x) \e^{-\int P(x)\mathrm{d}x} $$
              我们求其导数:
              $$ y'(x) = C'(x) \e^{-\int P(x)\mathrm{d}x} + C(x) \e^{-\int P(x)\mathrm{d}x} \cdot (-P(x)) $$
              注意到 $y = C(x) \e^{-\int P(x)\mathrm{d}x}$, 上式可以写成:
              $$ y'(x) = C'(x) \e^{-\int P(x)\mathrm{d}x} - P(x)y $$

        \item \textbf{代入原方程并求解 $C(x)$}

              将 $y'$ 的表达式代入原 Bernoulli 方程 $y' + P(x)y = Q(x)y^n$:
              $$ \left( C'(x) \e^{-\int P(x)\mathrm{d}x} - P(x)y \right) + P(x)y = Q(x)y^n $$
              化简得:
              $$ C'(x) \e^{-\int P(x)\mathrm{d}x} = Q(x)y^n $$
              现在, 将 $y = C(x) \e^{-\int P(x)\mathrm{d}x}$ 代入上式的右边:
              $$ C'(x) \e^{-\int P(x)\mathrm{d}x} = Q(x) \left( C(x) \e^{-\int P(x)\mathrm{d}x} \right)^n = Q(x) [C(x)]^n \e^{-n\int P(x)\mathrm{d}x} $$
              整理以求解 $C'(x)$:
              $$ C'(x) = Q(x) [C(x)]^n \e^{-n\int P(x)\mathrm{d}x} \e^{\int P(x)\mathrm{d}x} = Q(x) [C(x)]^n \e^{(1-n)\int P(x)\mathrm{d}x} $$
              这是一个关于 $C(x)$ 的可分离变量方程. 我们分离变量:
              $$ \frac{\mathrm{d}C}{[C(x)]^n} = Q(x) \e^{(1-n)\int P(x)\mathrm{d}x} \mathrm{d}x $$
              两边积分 (假设 $n \neq 1$):
              $$ \int [C(x)]^{-n} \mathrm{d}C = \int Q(x) \e^{(1-n)\int P(x)\mathrm{d}x} \mathrm{d}x $$
              $$ \frac{1}{1-n} [C(x)]^{1-n} = \int Q(x) \e^{(1-n)\int P(x)\mathrm{d}x} \mathrm{d}x + C_1 $$
              解出 $[C(x)]^{1-n}$:
              $$ [C(x)]^{1-n} = (1-n) \left( \int Q(x) \e^{(1-n)\int P(x)\mathrm{d}x} \mathrm{d}x + C_1 \right) $$

        \item \textbf{得到通解}

              从 $y = C(x) \e^{-\int P(x)\mathrm{d}x}$ 可得 $C(x) = y \e^{\int P(x)\mathrm{d}x}$.
              代入上式:
              $$ \left( y \e^{\int P(x)\mathrm{d}x} \right)^{1-n} = (1-n) \left( \int Q(x) \e^{(1-n)\int P(x)\mathrm{d}x} \mathrm{d}x + C_1 \right) $$
              $$ y^{1-n} \e^{(1-n)\int P(x)\mathrm{d}x} = (1-n) \int Q(x) \e^{(1-n)\int P(x)\mathrm{d}x} \mathrm{d}x + C' $$
              两边同乘以 $e^{-(1-n)\int P(x)\mathrm{d}x}$:
              $$ y^{1-n} = \e^{-(1-n)\int P(x)\mathrm{d}x} \left( (1-n) \int Q(x) \e^{(1-n)\int P(x)\mathrm{d}x} \mathrm{d}x + C' \right) $$
              这与通过标准代换 $u = y^{1-n}$ 得到的解是等价的.
    \end{enumerate}
\end{solution}

\begin{exercise}[6.1.8]
    一条曲线过点 $(2, 3)$, 其在坐标轴间的任意切线段被切点平分, 求这条曲线.
\end{exercise}

\begin{solution}
    设曲线上任意一点为 $(x, y)$, 该点的切线方程为 $Y - y = y'(X - x)$.
    切线与 $X$ 轴的交点 (令 $Y=0$): $X = x - \frac{y}{y'}$.
    切线与 $Y$ 轴的交点 (令 $X=0$): $Y = y - xy'$.
    这两个交点分别是 $(x - \frac{y}{y'}, 0)$ 和 $(0, y - xy')$.

    根据题意, 这两个交点所构成的线段的中点是切点 $(x, y)$ 本身.
    所以我们有中点坐标公式:
    $$ x = \frac{(x - \frac{y}{y'}) + 0}{2} \implies 2x = x - \frac{y}{y'} \implies x = -\frac{y}{y'} $$
    $$ y = \frac{0 + (y - xy')}{2} \implies 2y = y - xy' \implies y = -xy' $$
    从第一个方程可以得到微分方程:
    $$ \frac{\mathrm{d}y}{\mathrm{d}x} = y' = -\frac{y}{x} $$
    (第二个方程 $y = -x(-\frac{y}{x}) = y$ 与第一个方程是相容的).

    这是一个可分离变量的方程.
    $$ \frac{\mathrm{d}y}{y} = -\frac{\mathrm{d}x}{x} $$
    两边积分:
    $$ \int \frac{1}{y} \mathrm{d}y = - \int \frac{1}{x} \mathrm{d}x \implies \ln|y| = -\ln|x| + C_1 $$
    $$ \ln|y| + \ln|x| = C_1 \implies \ln|xy| = C_1 $$
    $$ |xy| = \e^{C_1} \implies xy = C \quad (C \text{为任意非零常数}) $$
    曲线经过点 $(2, 3)$, 代入方程确定常数 $C$:
    $$ 2 \cdot 3 = C \implies C = 6 $$
    因此, 所求的曲线方程为 $xy = 6$.
\end{solution}

\begin{exercise}[6.1.9]
    设函数 $f(x)$ 处处连续, 且 $\displaystyle f(x) = \int_0^x f(t) \mathrm{d}t$ (对 $x \in \mathbb{R}$), 求 $f(x)$.
\end{exercise}

\begin{solution}
    给定方程 $\displaystyle f(x) = \int_0^x f(t) \mathrm{d}t$.
    由于 $f(x)$ 连续, 根据微积分基本定理, 积分的变上限函数 $\displaystyle \int_0^x f(t) \mathrm{d}t$ 是可导的, 且其导数为 $f(x)$.
    对方程两边关于 $x$ 求导, 我们得到:
    $$ f'(x) = \frac{\mathrm{d}}{\mathrm{d}x} \int_0^x f(t) \mathrm{d}t = f(x) $$
    我们得到了一个微分方程 $f'(x) = f(x)$, 即 $\frac{\mathrm{d}y}{\mathrm{d}x} = y$ (令 $y=f(x)$).
    这是一个可分离变量的方程, 其通解为:
    $$ y = C \e^x $$
    所以 $f(x) = C \e^x$.

    现在我们需要确定常数 $C$.
    将 $x=0$ 代入原积分方程 $\displaystyle f(x) = \int_0^x f(t) \mathrm{d}t$:
    $$ f(0) = \int_0^0 f(t) \mathrm{d}t = 0 $$
    再将 $x=0$ 代入通解 $f(x) = C \e^x$:
    $$ f(0) = C \e^0 = C \cdot 1 = C $$
    因此, 我们得到 $C=0$.
    所以, 函数 $f(x)$ 的唯一解是 $f(x) = 0 \cdot \e^x = 0$.
    即 $f(x)$ 是零函数.
\end{solution}

\begin{exercise}[6.1.10]
    已知锸的衰变速率与锸的现存量成正比 (比例常数为 $k$). 设开始时锸的量为 $a$, 问 $t$ 时刻锸的量 $x(t)$ 为多少?
\end{exercise}

\begin{solution}
    设 $t$ 时刻锸的量为 $x(t)$.
    衰变速率即为 $\frac{\mathrm{d}x}{\mathrm{d}t}$.
    根据题意, 衰变速率与现存量成正比, 比例常数为 $k$. 因为是衰变, 所以速率为负.
    $$ \frac{\mathrm{d}x}{\mathrm{d}t} = -k x(t) \quad (k > 0) $$
    这是一个可分离变量的微分方程.
    $$ \frac{\mathrm{d}x}{x} = -k \mathrm{d}t $$
    两边积分:
    $$ \int \frac{1}{x} \mathrm{d}x = \int -k \mathrm{d}t \implies \ln|x| = -kt + C_1 $$
    由于 $x(t)$ 代表物质的量, $x(t) > 0$, 所以 $|x|=x$.
    $$ \ln x = -kt + C_1 \implies x(t) = \e^{-kt + C_1} = \e^{C_1} \e^{-kt} $$
    令 $C = \e^{C_1}$, 则通解为 $x(t) = C \e^{-kt}$.

    我们有初始条件: 开始时 ($t=0$) 锸的量为 $a$.
    即 $x(0) = a$.
    代入通解:
    $$ x(0) = C \e^{-k \cdot 0} = C \e^0 = C $$
    所以 $C = a$.
    因此, $t$ 时刻锸的量为:
    $$ x(t) = a \e^{-kt} $$
\end{solution}

\begin{exercise}[6.1.11]
    一汽艇以速度 $v = 10 \text{ km/h}$ 在静水上运动, 它的发动机在开足马力后关掉, 经过 $20 \text{ s}$ 后, 汽艇的速度降低为 $v_1 = 6 \text{ km/h}$. 设水对汽艇运动的阻力与汽艇速度成正比, 试求:
    \begin{tasks}(1)
        \task 发动机停止 $2 \text{ min}$ 后汽艇的速度;
        \task 发动机停止 $1 \text{ min}$ 后汽艇所走的路程.
    \end{tasks}
\end{exercise}

\begin{solution}
    设汽艇的质量为 $m$, 速度为 $v(t)$. 阻力 $F_r$ 与速度成正比, $F_r = -kv$ ($k>0$ 为阻力系数).
    根据牛顿第二定律 $F=ma$, 我们有:
    $$ m \frac{\mathrm{d}v}{\mathrm{d}t} = -kv $$
    这是一个可分离变量的方程.
    $$ \frac{\mathrm{d}v}{v} = -\frac{k}{m} \mathrm{d}t $$
    两边积分: $\ln v = -\frac{k}{m}t + C_1$.
    通解为 $v(t) = \e^{C_1} \e^{-\frac{k}{m}t} = C \e^{-\frac{k}{m}t}$.

    我们统一单位. 速度单位用 m/s, 时间单位用 s.
    $v_0 = 10 \text{ km/h} = 10 \cdot \frac{1000}{3600} \text{ m/s} = \frac{25}{9} \text{ m/s}$.
    $v_1 = 6 \text{ km/h} = 6 \cdot \frac{1000}{3600} \text{ m/s} = \frac{5}{3} \text{ m/s}$.

    初始条件: $t=0$ 时, $v(0) = v_0 = \frac{25}{9}$.
    代入通解: $v(0) = C \e^0 = C \implies C = \frac{25}{9}$.
    所以速度函数为 $v(t) = \frac{25}{9} \e^{-\frac{k}{m}t}$.

    利用 $t=20$s 时的数据求 $\frac{k}{m}$:
    $v(20) = \frac{5}{3}$.
    $$ \frac{5}{3} = \frac{25}{9} \e^{-\frac{k}{m} \cdot 20} \implies \e^{-20\frac{k}{m}} = \frac{5}{3} \cdot \frac{9}{25} = \frac{3}{5} $$
    $$ -20\frac{k}{m} = \ln\left(\frac{3}{5}\right) \implies \frac{k}{m} = -\frac{1}{20}\ln\left(\frac{3}{5}\right) = \frac{1}{20}\ln\left(\frac{5}{3}\right) $$
    所以速度的精确表达式为 $v(t) = \frac{25}{9} \e^{-\frac{t}{20}\ln(\frac{5}{3})} = \frac{25}{9} \left(\frac{3}{5}\right)^{\frac{t}{20}}$.

    \begin{enumerate}[(1)]
        \item 求 $t = 2 \text{ min} = 120 \text{ s}$ 后的速度.
              $$ v(120) = \frac{25}{9} \left(\frac{3}{5}\right)^{\frac{120}{20}} = \frac{25}{9} \left(\frac{3}{5}\right)^6 = \frac{5^2}{3^2} \cdot \frac{3^6}{5^6} = \frac{3^4}{5^4} = \left(\frac{3}{5}\right)^4 = \frac{81}{625} \text{ m/s} $$
              换算成 km/h: $\frac{81}{625} \cdot \frac{3600}{1000} = \frac{81}{625} \cdot \frac{18}{5} = \frac{1458}{3125} \approx 0.46656 \text{ km/h}$.

        \item 求 $t = 1 \text{ min} = 60 \text{ s}$ 后所走的路程 $s(60)$.
              路程是速度对时间的积分: $s(t) = \int_0^t v(\tau) \mathrm{d}\tau$.
              $$ s(60) = \int_0^{60} \frac{25}{9} \e^{-\frac{k}{m}\tau} \mathrm{d}\tau = \frac{25}{9} \left[ -\frac{m}{k} \e^{-\frac{k}{m}\tau} \right]_0^{60} $$
              我们知道 $\frac{k}{m} = \frac{1}{20}\ln\left(\frac{5}{3}\right)$, 所以 $\frac{m}{k} = \frac{20}{\ln(5/3)}$.
              \begin{align*}
                  s(60) & = \frac{25}{9} \left( -\frac{m}{k} \right) \left( \e^{-\frac{k}{m} \cdot 60} - \e^0 \right)                                        \\
                        & = -\frac{25}{9} \frac{20}{\ln(5/3)} \left( \left(\frac{3}{5}\right)^{\frac{60}{20}} - 1 \right)                                    \\
                        & = -\frac{500}{9\ln(5/3)} \left( \left(\frac{3}{5}\right)^3 - 1 \right)                                                             \\
                        & = -\frac{500}{9\ln(5/3)} \left( \frac{27}{125} - 1 \right) = -\frac{500}{9\ln(5/3)} \left( -\frac{98}{125} \right)                 \\
                        & = \frac{500 \cdot 98}{9 \cdot 125 \ln(5/3)} = \frac{4 \cdot 125 \cdot 98}{9 \cdot 125 \ln(5/3)} = \frac{392}{9\ln(5/3)} \text{ m}.
              \end{align*}
              $\ln(5/3) \approx \ln(1.667) \approx 0.5108$.
              $s(60) \approx \frac{392}{9 \cdot 0.5108} \approx \frac{392}{4.5972} \approx 85.27 \text{ m}$.
    \end{enumerate}
\end{solution}

\begin{exercise}[6.1.12]
    求解下列二阶方程.
    \begin{tasks}(2)
        \task $xy'' = y';$
        \task $y'' = \frac{y'}{x} + x;$
        \task $y'' = y' + x;$
        \task $y'' + (y')^2 = 2\mathrm{e}^{-y}.$
    \end{tasks}
\end{exercise}

\begin{solution}
    \begin{enumerate}
        \item 令 $p = y'$, 则 $y'' = p'$. 方程变为 $xp' = p$.
              这是一个可分离变量方程: $\frac{\mathrm{d}p}{p} = \frac{\mathrm{d}x}{x}$.
              积分得 $\ln|p| = \ln|x| + C_1$, 所以 $p = C_1' x$.
              即 $y' = C_1' x$.
              再次积分: $y = \int C_1' x \mathrm{d}x = \frac{1}{2}C_1' x^2 + C_2$.
              令 $C_1 = C_1'/2$, 解为 $y = C_1 x^2 + C_2$.

        \item 令 $p = y'$, 则 $y'' = p'$. 方程变为 $p' = \frac{p}{x} + x$, 即 $p' - \frac{1}{x}p = x$.
              这是一个线性方程. $\int P(x)\mathrm{d}x = \int -\frac{1}{x}\mathrm{d}x = -\ln|x|$.
              通解为:
              \begin{align*}
                  p & = \e^{-(-\ln|x|)} \left( \int x \e^{-\ln|x|} \mathrm{d}x + C_1 \right) \\
                    & = |x| \left( \int x \frac{1}{|x|} \mathrm{d}x + C_1 \right)            \\
                    & = x \left( \int \mathrm{d}x + C_1 \right) \quad (\text{假设 } x>0)     \\
                    & = x(x+C_1) = x^2 + C_1 x.
              \end{align*}
              即 $y' = x^2 + C_1 x$.
              再次积分: $y = \int (x^2 + C_1 x) \mathrm{d}x = \frac{1}{3}x^3 + \frac{1}{2}C_1 x^2 + C_2$.

        \item 令 $p = y'$, 则 $y'' = p'$. 方程变为 $p' = p + x$, 即 $p' - p = x$.
              这是一个线性方程. $\int P(x)\mathrm{d}x = \int -1 \mathrm{d}x = -x$.
              通解为:
              \begin{align*}
                  p & = \e^{-(-x)} \left( \int x \e^{-x} \mathrm{d}x + C_1 \right)     \\
                    & = \e^x \left( -(x+1)e^{-x} + C_1 \right) \quad (\text{分部积分}) \\
                    & = -(x+1) + C_1 \e^x.
              \end{align*}
              即 $y' = C_1 \e^x - x - 1$.
              再次积分: $y = \int (C_1 \e^x - x - 1) \mathrm{d}x = C_1 \e^x - \frac{1}{2}x^2 - x + C_2$.

        \item 方程不显含 $x$. 令 $p = y'$, 则 $y'' = \frac{\mathrm{d}p}{\mathrm{d}x} = \frac{\mathrm{d}p}{\mathrm{d}y}\frac{\mathrm{d}y}{\mathrm{d}x} = p\frac{\mathrm{d}p}{\mathrm{d}y}$.
              方程变为 $p\frac{\mathrm{d}p}{\mathrm{d}y} + p^2 = 2e^{-y}$.
              令 $u = p^2$, 则 $\frac{\mathrm{d}u}{\mathrm{d}y} = 2p\frac{\mathrm{d}p}{\mathrm{d}y}$. 方程化为:
              $$ \frac{1}{2}\frac{\mathrm{d}u}{\mathrm{d}y} + u = 2e^{-y} \implies \frac{\mathrm{d}u}{\mathrm{d}y} + 2u = 4e^{-y}. $$
              这是一个关于 $u(y)$ 的线性方程. $\int P(y)\mathrm{d}y = \int 2\mathrm{d}y = 2y$.
              通解为:
              \begin{align*}
                  u & = \e^{-2y} \left( \int 4e^{-y} \e^{2y} \mathrm{d}y + C_1 \right)                                        \\
                    & = \e^{-2y} \left( \int 4e^y \mathrm{d}y + C_1 \right) = \e^{-2y} (4e^y + C_1) = 4e^{-y} + C_1 \e^{-2y}.
              \end{align*}
              因为 $u=p^2=(y')^2$, 所以 $(y')^2 = 4e^{-y} + C_1 \e^{-2y}$.
              $$ y' = \pm \sqrt{4e^{-y} + C_1 \e^{-2y}} $$
              $$ \int \frac{\mathrm{d}y}{\sqrt{4e^{-y} + C_1 \e^{-2y}}} = \pm \int \mathrm{d}x \implies \int \frac{e^y \mathrm{d}y}{\sqrt{4e^y + C_1}} = \pm x + C_2. $$
              令 $w = \sqrt{4e^y+C_1}$, 则 $w^2 = 4e^y+C_1$, $2w\mathrm{d}w = 4e^y\mathrm{d}y$.
              $$ \int \frac{w/2 \mathrm{d}w}{w} = \frac{1}{2}w = \frac{1}{2}\sqrt{4e^y+C_1}. $$
              所以 $\frac{1}{2}\sqrt{4e^y+C_1} = \pm x + C_2$.
              $$ \sqrt{4e^y+C_1} = \pm 2x + 2C_2 \implies 4e^y+C_1 = (C_1' \pm 2x)^2. $$
              解为 $4e^y = (C_1 \pm 2x)^2 - C_2$.或者写为
              $$\e^y = x^2 + C_1 x + C_2.$$
    \end{enumerate}
\end{solution}

\begin{exercise}[6.1.13]
    求下列二阶方程满足初值条件的特解.
    \begin{tasks}(2)
        \task $y'' = \frac{y'}{x} + \frac{x^2}{y'}, y(1) = 1, y'(1) = 0;$
        \task $y^3 y'' = -1, y(1) = 1, y'(1) = 0.$
    \end{tasks}
\end{exercise}

\begin{solution}
    \begin{enumerate}
        \item 方程在 $y'=0$ 时无定义. 我们对方程变形: $y'y'' - \frac{(y')^2}{x} = x^2$.
              令 $u = (y')^2$, 则 $u' = 2y'y''$. 方程变为:
              $$ \frac{1}{2}u' - \frac{u}{x} = x^2 \implies u' - \frac{2}{x}u = 2x^2. $$
              这是一个线性方程. $\int P(x)\mathrm{d}x = \int -\frac{2}{x}\mathrm{d}x = -2\ln x = \ln(x^{-2})$.
              通解为:
              \begin{align*}
                  u & = \e^{-\ln(x^{-2})} \left( \int 2x^2 \e^{\ln(x^{-2})} \mathrm{d}x + C \right)                                        \\
                    & = x^2 \left( \int 2x^2 \cdot x^{-2} \mathrm{d}x + C \right) = x^2 \left( \int 2 \mathrm{d}x + C \right) = x^2(2x+C).
              \end{align*}
              所以 $(y')^2 = 2x^3 + Cx^2$.
              使用初值条件 $y'(1)=0$: $$0^2 = 2(1)^3 + C(1)^2 \implies 2+C=0 \implies C=-2.$$
              于是 $(y')^2 = 2x^3 - 2x^2 = 2x^2(x-1)$.
              $y' = \pm \sqrt{2} x \sqrt{x-1}$.
              再次积分:
              $$ y = \int \pm \sqrt{2} x \sqrt{x-1} \mathrm{d}x $$
              令 $t = \sqrt{x-1}$, 则 $x=t^2+1, \mathrm{d}x=2t\mathrm{d}t$.
              \begin{align*}
                  \int x\sqrt{x-1} \mathrm{d}x & = \int (t^2+1)t \cdot 2t \mathrm{d}t = \int (2t^4 + 2t^2) \mathrm{d}t                          \\
                                               & = \frac{2}{5}t^5 + \frac{2}{3}t^3 + C' = \frac{2}{5}(x-1)^{5/2} + \frac{2}{3}(x-1)^{3/2} + C'.
              \end{align*}
              所以 $y = \pm \sqrt{2} \left( \frac{2}{5}(x-1)^{5/2} + \frac{2}{3}(x-1)^{3/2} \right) + C_2$.
              使用初值条件 $y(1)=1$:
              $$1 = \pm \sqrt{2} (0+0) + C_2 \implies C_2 = 1.$$
              特解为 $y = 1 \pm \frac{2\sqrt{2}}{15}(3(x-1)+5)(x-1)^{3/2} = 1 \pm \frac{2\sqrt{2}}{15}(3x+2)(x-1)^{3/2}$.

        \item 方程不显含 $x$. 令 $p = y'$, 则 $y'' = p\frac{\mathrm{d}p}{\mathrm{d}y}$.
              方程变为 $$y^3 p\frac{\mathrm{d}p}{\mathrm{d}y} = -1$$
              分离变量: $$p \mathrm{d}p = -y^{-3} \mathrm{d}y$$
              积分后整理得$$p^2 = \frac{1}{y^2} + 2C_1,$$
              即 $$(y')^2 = \frac{1}{y^2} + C.$$
              使用初值条件 $y(1)=1, y'(1)=0$:
              $0^2 = \frac{1}{1^2} + C \implies C = -1$.
              所以 $$(y')^2 = \frac{1}{y^2} - 1 = \frac{1-y^2}{y^2} \Rightarrow y' = \pm \frac{\sqrt{1-y^2}}{y}.$$
              分离变量后积分: $$\int \frac{y}{\sqrt{1-y^2}} \mathrm{d}y = -\sqrt{1-y^2}$$
              所以 $$-\sqrt{1-y^2} = \pm x + C_2.$$
              使用初值条件 $y(1)=1$
              $$-\sqrt{1-1^2} = 0 = \pm 1 + C_2 \implies C_2 = \mp 1$$
              所以 $$-\sqrt{1-y^2} = \pm x \mp 1 = \pm(x-1)$$
              两边平方: $$1-y^2 = (x-1)^2 \Rightarrow y^2 + (x-1)^2 = 1$$
              由于 $y(1)=1>0$, 我们取上半圆,即特解为$y = \sqrt{1-(x-1)^2}$.
    \end{enumerate}
\end{solution}

\newpage