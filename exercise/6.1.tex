\section{习题 6.1}

\begin{exercise}[6.1.1]
    求下列可分离变量的微分方程.
    \begin{tasks}(2)
        \task $(1 + x^2)\mathrm{d}y = y \mathrm{d}x;$
        \task $y' = \mathrm{e}^{x-y};$
        \task $xy' + y = y^2;$
        \task $yy' = \frac{1 - 2x}{y}.$
    \end{tasks}
\end{exercise}

\begin{solution}
    \begin{enumerate}
        \item 不难验证$y\equiv0$是方程的解. 当$y\neq0$时,
              \begin{align*}
                           & (1 + x^2)\mathrm{d}y = y \mathrm{d}x                    \\
                  \implies & \frac{1}{y} \mathrm{d}y = \frac{1}{1 + x^2} \mathrm{d}x \\
                  \implies & \ln |y| = \arctan x + C                                 \\
                  \implies & y = C' \mathrm{e}^{\arctan x}, \quad C' = \e^C > 0.
              \end{align*}
              综上, 方程的解为
              $$y = C \e^{\arctan x},\quad C \in \mathbb{R}.$$
        \item \begin{align*}
                           & y' = \mathrm{e}^{x-y}                                    \\
                  \implies & \mathrm{e}^y \mathrm{d}y = \mathrm{e}^x \mathrm{d}x      \\
                  \implies & \mathrm{e}^y = \mathrm{e}^x + C ,\quad C > -\mathrm{e}^x \\
                  \implies & y = \ln (\mathrm{e}^x + C), \quad C > -\mathrm{e}^x.
              \end{align*}
        \item 不难验证$y\equiv 0,1$是方程的解. 当$y\neq0,1$时,
              \begin{align*}
                           & xy' + y = y^2                                      \\
                  \implies & \frac{\mathrm{d}y}{y(y-1)} = \frac{\mathrm{d}x}{x} \\
                  \implies & \ln \left| \frac{y-1}{y} = \ln |x| + C \right.     \\
                  \implies & \frac{y-1}{y} = C' x, \quad C' = \e^C \neq 0       \\
                  \implies & y = \frac{1}{1 - C' x}, \quad C' \neq 0.
              \end{align*}
              综上, 方程的解为
              $$y = 0; \quad y = \frac{1}{1 - C x}, \quad C \in \mathbb{R}$$
        \item \begin{align*}
                           & yy' = \frac{1 - 2x}{y}                               \\
                  \implies & y^2 \mathrm{d}y = (1 - 2x) \mathrm{d}x               \\
                  \implies & \frac{1}{3} y^3 = x - x^2 + C                        \\
                  \implies & y = \sqrt[3]{3x - 3x^2 + C}, \quad C \in \mathbb{R}.
              \end{align*}
              且由题设得$y \ne 0$,因此方程的所有解为:
              %   \begin{align*}
              %            & \left\{y =  \sqrt[3]{3x - 3x^2 + C} \mid C < -\frac{3}{4} \right\}                                                    \\
              %       \cup & \left\{y =  \sqrt[3]{3x - 3x^2 + C} \mid C \ges -\frac{3}{4}, x \in \left(-\infty, \frac{1}{2} \right) ,y < 0\right\} \\
              %       \cup & \left\{y =  \sqrt[3]{3x - 3x^2 + C} \mid C \ges -\frac{3}{4}, x \in \left(\frac{1}{2}, +\infty \right) ,y < 0\right\} \\
              %       \cup & \left\{y = \sqrt[3]{3x - 3x^2 + C} \mid C \ges -\frac{3}{4} , y > 0 \right\}
              %   \end{align*}
              $$y = \sqrt[3]{3x - 3x^2 + C}, x \in \mathcal{D_C} = \{ x \in \R \mid 3x - 3x^2 + C \ne 0 \}, C \in \mathbb{R}.$$
              \begin{remark}
                  也可以将所有解写为:
                  $$\frac{3x-3x^2+C}{y^3} = 1, \quad C \in \mathbb{R}.$$
                  后面为了方便,我们也写成:
                  $$y^3 = 3x - 3x^2 + C, \quad C \in \mathbb{R}, y \ne 0.$$
              \end{remark}
    \end{enumerate}
\end{solution}

\begin{exercise}[6.1.2]
    求下列微分方程.
    \begin{tasks}(2)
        \task $y' = \frac{y^2}{x^2} - 2;$
        \task $y' = \frac{y}{x} + \frac{x}{y};$
        \task $\displaystyle \frac{\mathrm{d}x}{x^2 - xy + y^2} = \frac{\mathrm{d}y}{2y^2 - xy};$
        \task $(x^2 + 3y^2)\mathrm{d}x - 2xy\mathrm{d}y = 0.$
    \end{tasks}
\end{exercise}

\begin{solution}
    \begin{enumerate}
        \item 记$p = \frac{y}{x}$, 则$y = px$, $y' = p'x + p$. 代入方程得
              \begin{align*}
                           & p'x + p = p^2 - 2                                                                                      \\
                  \implies & \frac{\mathrm{d}p}{p^2 - p - 2} = \frac{\mathrm{d}x}{x}                                                \\
                  \implies & \int \frac13 \left( \frac{1}{p - 2} - \frac{1}{p + 1} \right) \mathrm{d}p = \int \frac{\mathrm{d}x}{x} \\
                  \implies & \ln |p - 2| - \ln |p + 1| = 3\ln |x| + C                                                               \\
                  \implies & \frac{p - 2}{p + 1} = C' x^3, \quad C' = \e^C > 0                                                      \\
                  \implies & \frac{y - 2x}{x+y} = C' x^3                                                                            \\
                  \implies & y = \frac{2x + C' x^4}{1 - C' x^3}
              \end{align*}
              同时注意到$y - 2x = \frac{3C'x^4}{1-C' x^3}$, 函数被分为$y-2x \ges 0, y - 2x < 0$两支,这两支的解是独立的,因此综上所述
              $$y = \frac{2x + C_1 x^4}{1 - C_1 x^3}, C_1 \in \mathbb{R}, y - 2x \ges 0; \quad y = \frac{2x + C_2 x^4}{1 - C_2 x^3}, C_2 \in \mathbb{R}, y - 2x < 0.$$
              \begin{remark}
                  为了方便,我们还是写为
                  $$y = \frac{2x + C x^4}{1 - C x^3},\quad C \in \mathbb{R}, x \ne 0$$
              \end{remark}
        \item 记$p = \frac{y}{x}$, 则$y = px$, $y' = p'x + p$. 代入方程得
              \begin{align*}
                           & p'x + p = p + \frac{1}{p}                  \\
                  \implies & p' = \frac{1}{p x}                         \\
                  \implies & p \mathrm{d}p = \frac{\mathrm{d}x}{x}      \\
                  \implies & \frac{1}{2} p^2 = \ln |x| + C              \\
                  \implies & y^2 = x^2 (2 \ln |x| + C'), \quad C' = 2C.
              \end{align*}
              由于微分方程在$x = 0$处不连续, $x$正负半轴的解是独立的,
              $$y^2 = x^2 (2 \ln x + C_1), x > 0; \quad y^2 = x^2 (2 \ln (-x) + C_2), x < 0.$$
              我们还是允许如下的表示$$y^2 = x^2 (2 \ln |x| + C), C \in \mathbb{R}, x \ne 0.$$
        \item 记$p = \frac{y}{x}$, 则$y = px$, $ y' = p'x + p$. 代入方程得
              \begin{align*}
                           & \frac{1}{x^2 - x(px) + (px)^2} \mathrm{d}x = \frac{1}{2(px)^2 - x(px)} \mathrm{d}(px)                                            \\
                  \implies & \frac{\dif x}{x} = \frac{p^2 -p +1}{p(p-1)(p-2)} \dif p                                                                          \\
                  \implies & \int \frac{\dif x}{x} = \int \left( - \frac{1}{2} \frac{\dif p}{p} + \frac{\dif p}{p-1} - \frac{3}{2} \frac{\dif p}{p-2} \right) \\
                  \implies & x p^\frac12 (p-1)^{-1} (p-2)^\frac32 = C                                                                                         \\
                  \implies & (y-x)^2 = C' xy (y-2x)^3, \quad C' = C^2 > 0
              \end{align*}
              综上, 方程的解为
              $$ (y - x)^2 = C xy (y - 2x)^3, \quad C > 0,2y^2-xy \ne 0.$$
        \item 记$p = \frac{y}{x}$, 则$y = px$, $ y' = p'x + p$. 代入方程得$x \ne 0$时,
              \begin{align*}
                           & (x^2 + 3 (px)^2) \mathrm{d}x - 2x(px) \mathrm{d}(px)       \\
                  \implies & \frac{2p}{1+p^2} \dif p = \frac{1}{x} \dif x               \\
                  \implies & \ln\left(1+p^2\right) = \ln |x| + C                        \\
                  \implies & 1 + \left(\frac{y}{x}\right)^2 = C' x, \quad C' = \e^C > 0 \\
                  \implies & x^2 + y^2 = C' x^3
              \end{align*}
              综上, 方程的解为
              $$x^2 + y^2 = C x^3, \quad C \in \mathbb{R}$$
    \end{enumerate}
\end{solution}

\begin{exercise}[6.1.3]
    证明: 形如
    $$ \frac{\mathrm{d}y}{\mathrm{d}x} = f \left( \frac{a_1 x + b_1 y + c_1}{a_2 x + b_2 y + c_2} \right) $$
    的方程, 可通过代换化为齐次方程. (提示: 若方程组
    $$ \begin{cases} a_1 x + b_1 y + c_1 = 0, \\ a_2 x + b_2 y + c_2 = 0 \end{cases} $$
    有非零解 $x_0, y_0$ (即 $x_0, y_0$ 不全为零), 则可令 $u = x - x_0, v = y - y_0$; 其他情形更易于处理.)

    求下面方程:
    \begin{tasks}(2)
        \task $\displaystyle \frac{\mathrm{d}y}{\mathrm{d}x} = \frac{x+y+3}{x-y+1};$
        \task $\displaystyle \frac{\mathrm{d}y}{\mathrm{d}x} = \frac{2x+4y+3}{x+2y+1}.$
    \end{tasks}
\end{exercise}

\begin{solution}
\begin{enumerate}
\item 方程组
$$\begin{cases} x + y + 3 = 0, \\ x - y + 1 = 0 \end{cases}$$
有非零解$x_0 = -2, y_0 = -1$. 令$u = x +2 ,v =y + 1$,
则\begin{align*}
    & \frac{\dif v}{\dif u} = \frac{u+v}{u-v} \\
    \implies & u \dif u + v \dif v = 0 \\
\end{align*}
\end{enumerate}
\end{solution}

\begin{exercise}[6.1.4]
    求下列线性方程和 Bernoulli 方程的解.
    \begin{tasks}(2)
        \task $(1 + x^2)y' - 2xy = (1 + x^2)^2;$
        \task $y' + \frac{1-2x}{x^2} y = 1;$
        \task $y' = \frac{y}{x + y^3};$
        \task $y' + \frac{y}{x} = y^2 \ln x;$
        \task $y' = y \tan x + y^2 \cos x;$
        \task $y - y' \cos x = y^2 (1 - \sin x) \cos x.$
    \end{tasks}
\end{exercise}

\begin{exercise}[6.1.5]
    求下列方程满足初值条件的特解.
    \begin{tasks}(2)
        \task $y' = \frac{y}{x} \ln \frac{y}{x}, y(1) = 1;$
        \task $y' + \frac{y}{x} = \frac{\sin x}{x}, y(\pi) = 1.$
    \end{tasks}
\end{exercise}

\begin{exercise}[6.1.6]
    求解下列微分方程.
    \begin{tasks}(2)
        \task $y' + x = \sqrt{x^2 + y};$
        \task $y' = \cos(x - y);$
        \task $y' - \mathrm{e}^{x-y} + \mathrm{e}^x = 0;$
        \task $y' \sin y + x \cos y + x = 0.$
    \end{tasks}
\end{exercise}

\begin{exercise}[6.1.7]
    试用常数变易法导出 Bernoulli 方程的通解.
\end{exercise}

\begin{exercise}[6.1.8]
    一条曲线过点 $(2, 3)$, 其在坐标轴间的任意切线段被切点平分, 求这条曲线.
\end{exercise}

\begin{exercise}[6.1.9]
    设函数 $f(x)$ 处处连续, 且 $\displaystyle f(x) = \int_0^x f(t) \mathrm{d}t$ (对 $x \in \mathbb{R}$), 求 $f(x)$.
\end{exercise}

\begin{exercise}[6.1.10]
    已知锸的衰变速率与锸的现存量成正比 (比例常数为 $k$). 设开始时锸的量为 $a$, 问 $t$ 时刻锸的量 $x(t)$ 为多少?
\end{exercise}

\begin{exercise}[6.1.11]
    一汽艇以速度 $v = 10 \text{ km/h}$ 在静水上运动, 它的发动机在开足马力后关掉, 经过 $20 \text{ s}$ 后, 汽艇的速度降低为 $v_1 = 6 \text{ km/h}$. 设水对汽艇运动的阻力与汽艇速度成正比, 试求:
    \begin{tasks}(1)
        \task 发动机停止 $2 \text{ min}$ 后汽艇的速度;
        \task 发动机停止 $1 \text{ min}$ 后汽艇所走的路程.
    \end{tasks}
\end{exercise}

\begin{exercise}[6.1.12]
    求解下列二阶方程.
    \begin{tasks}(2)
        \task $xy'' = y';$
        \task $y'' = \frac{y'}{x} + x;$
        \task $y'' = y' + x;$
        \task $y'' + (y')^2 = 2\mathrm{e}^{-y}.$
    \end{tasks}
\end{exercise}

\begin{exercise}[6.1.13]
    求下列二阶方程满足初值条件的特解.
    \begin{tasks}(2)
        \task $y'' = \frac{y'}{x} + \frac{x^2}{y'}, y(1) = 1, y'(1) = 0;$
        \task $y^3 y'' = -1, y(1) = 1, y'(1) = 0.$
    \end{tasks}
\end{exercise}