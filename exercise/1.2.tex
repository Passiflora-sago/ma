\section{习题 1.2}

\begin{exercise}[1.2.1]
    用定义证明下面的结论:
    \begin{tasks}[label=(\arabic*)](2)
        \task $\lim_{n\to\infty} \frac{n}{5+3n} = \frac{1}{3}$;
        \task $\lim_{n\to\infty} \frac{\sin n}{n} = 0$;
        \task $\lim_{n\to\infty} (-1)^n \frac{1}{\sqrt{n+1}} = 0$;
        \task $\lim_{n\to\infty} \frac{n!}{n^n} = 0$.
    \end{tasks}
\end{exercise}

\begin{solution}
    \begin{enumerate}[(1)]
        \item $\forall \varepsilon > 0$, 取$N = \llceil \frac{5}{9\varepsilon} \rrceil$, 则当 $n > N$ 时, 有
              $$
                  \left| \frac{n}{5 + 3n} - \frac{1}{3} \right| = \left| \frac{3n - (5 + 3n)}{3(5 + 3n)} \right| = \frac{5}{3(5 + 3n)} < \frac{5}{9n} < \varepsilon.
              $$
        \item $\forall \varepsilon > 0$, 取$N = \llceil \frac{1}{\varepsilon} \rrceil$, 则当 $n > N$ 时, 有
              $$
                  \left| \frac{\sin n}{n} - 0 \right| = \frac{|\sin n|}{n} \les \frac{1}{n} < \varepsilon.
              $$
        \item $\forall \varepsilon > 0$, 取$N = \llceil \frac{1}{\varepsilon^2} - 1 \rrceil$, 则当 $n > N$ 时, 有
              $$
                  \left| (-1)^n \frac{1}{\sqrt{n + 1}} - 0 \right| = \frac{1}{\sqrt{n + 1}} < \varepsilon.
              $$
        \item $\forall \varepsilon > 0$, 取$N = \llceil \frac{1}{\varepsilon} \rrceil$, 则当 $n > N$ 时, 有
              $$
                  \left| \frac{n!}{n^n} - 0 \right| = \frac{n!}{n^n} = \frac{1}{n} \cdot \frac{2}{n} \cdots \frac{n - 1}{n} \cdot \frac{n}{n} < \frac{1}{n} < \varepsilon.
              $$
    \end{enumerate}
\end{solution}

\begin{exercise}[1.2.2]
    若数列 $\{a_n\}$ ($n \ges 1$) 满足条件: 任给正数 $\varepsilon$, 存在正整数 $N$, 使得当 $n > N$ 时, 有 $|a_n - a| < M\varepsilon$ (其中 $M$ 为常数), 则 $\{a_n\}$ 必以 $a$ 为极限.
\end{exercise}

$M$为常数指的是$M$不依赖于$\ve$和$n$. 例如$M=2, M=1000$等都是常数.也就是说, 上述$(2)$ 其实等价于
$\forall M > 0, \forall \ve >0, \exists N\in \mathbb{N}^*, \forall n>N $都有 $|a_n-a|<M \ve$ 成立.

\begin{exercise}[1.2.3]
    证明: 当且仅当 $\lim_{n\to\infty} (a_n - a) = 0$ 时, 有 $\lim_{n\to\infty} a_n = a$. (数列极限的许多证明问题, 都可用同样的方法处理.)
\end{exercise}

\begin{solution}
    充分性: 由$\lim_{n \to \infty} (a_n - a) = 0$,则$\forall \ve > 0, \exists N \in \mathbb{N}^*, \forall n > N$都有$|a_n - a| < \ve$成立.因此$\lim_{n \to \infty} a_n = a$.

    必要性: 由$\lim_{n \to \infty} a_n = a$,则$\forall \ve > 0, \exists N \in \mathbb{N}^*, \forall n > N$都有$|a_n - a| < \ve$成立.因此$\lim_{n \to \infty} (a_n - a) = 0$.
\end{solution}

\begin{exercise}[1.2.4]
    证明: 若 $\lim_{n\to\infty} a_n = a$, 则 $\lim_{n\to\infty} |a_n| = |a|$; 反之不一定成立 (试举例说明). 但若 $\lim_{n\to\infty} |a_n| = 0$, 则有 $\lim_{n\to\infty} a_n = 0$.
\end{exercise}

\begin{solution}
    由 $\lim_{n\to\infty} a_n = a \Rightarrow \forall \ve > 0, \exists N \in \mathbb{N}^*$, 当 $n > N$ 时, 有 $|a_n - a| < \ve$. 则
    $$
        ||a_n| - |a|| \les |a_n - a| < \ve.
    $$
    即 $\lim_{n\to\infty} |a_n| = |a|$.

    反之不一定成立,如数列 $a_n = (-1)^n$, 则 $\lim_{n\to\infty} |a_n| = 1$, 但 $\{a_n\}$ 发散.

    若 $\lim_{n\to\infty} |a_n| = 0$, 则 $\forall \ve > 0, \exists N \in \mathbb{N}^*$, 当 $n > N$ 时, 有 $||a_n| - 0| < \ve$. 则
    $$
        |a_n - 0| = |a_n| < \ve.
    $$
    即 $\lim_{n\to\infty} a_n = 0$.
\end{solution}

\begin{exercise}[1.2.5]
    证明: 若 $\lim_{n\to\infty} a_n = 0$, 又 $|b_n| \les M$, ($n=1,2,\cdots$), 则 $\lim_{n\to\infty} a_n b_n = 0$.
\end{exercise}

\begin{solution}
    由 $\lim_{n\to\infty} a_n = 0 \Rightarrow \forall \ve > 0, \exists N \in \mathbb{N}^*$, 当 $n > N$ 时, 有 $|a_n - 0| < \frac{\ve}{M}$. 则
    $$
        |a_n b_n - 0| = |a_n| |b_n| < \frac{\ve}{M} \cdot M = \ve.
    $$
    即 $\lim_{n\to\infty} a_n b_n = 0$.
\end{solution}

\begin{exercise}[1.2.6]
    证明: 若数列 $\{a_n\}$ 满足 $\lim_{k\to\infty} a_{2k+1} = a$, 及 $\lim_{k\to\infty} a_{2k} = a$, 则 $\lim_{n\to\infty} a_n = a$.
\end{exercise}

\begin{solution}
    按已知条件 \(\forall \varepsilon > 0, \exists N_1 > 0\),当 \(n > N_1\) 时 \(|x_{2n} - a| < \varepsilon\).
    又 \(\exists N_2 > 0\),当 \(n > N_2\) 时 \(|x_{2n+1} - a| < \varepsilon\).于是令\( N = \max\{2N_1, 2N_2 + 1\}, \)
    则 \(n > N\) 时恒有 \(|x_n - a| < \varepsilon\).故 \(\lim_{n \to \infty} x_n = a\).

\end{solution}

\begin{exercise}[1.2.7]
    证明下列数列不收敛:
    \begin{tasks}[label=(\arabic*)](2)
        \task $a_n = (-1)^n \frac{n}{n+1}$;
        \task $a_n = 5 \left(1 - \frac{2}{n}\right) + (-1)^n$.
    \end{tasks}
\end{exercise}

\begin{solution}
    \begin{enumerate}[(1)]
        \item 取$a_{2n} = \frac{2n}{2n+1}, a_{2n+1} = -\frac{2n+1}{2n+2}$,则$\lim_{n \to \infty} a_{2n} = 1, \lim_{n \to \infty} a_{2n+1} = -1$,而如果$\{a_n\}$收敛,则$\lim_{n \to \infty} a_{2n} = \lim_{n \to \infty} a_{2n+1}$,矛盾.
        \item 取$a_{2n} = 5 \left(1 - \frac{1}{n}\right)+1, a_{2n+1} = 5 \left(1 - \frac{2}{2n+1}\right)-1$,则$\lim_{n \to \infty} a_{2n} = 6, \lim_{n \to \infty} a_{2n+1} = 4$,而如果$\{a_n\}$收敛,则$\lim_{n \to \infty} a_{2n} = \lim_{n \to \infty} a_{2n+1}$,矛盾.
    \end{enumerate}
\end{solution}

\begin{exercise}[1.2.8]
    求下列极限:
    \begin{enumerate}[(1)]
        \item $a_n = \frac{4n^2 + 5n + 2}{3n^2 + 2n + 1}$;
        \item $a_n = \frac{1}{1 \cdot 2} + \frac{1}{2 \cdot 3} + \cdots + \frac{1}{(n-1)n}$;
        \item $a_n = \left(1 - \frac{1}{3}\right) \left(1 - \frac{1}{6}\right) \cdots \left(1 - \frac{1}{n(n+1)/2}\right), \ n=2, 3, \ldots$;
        \item $a_n = \left(1 - \frac{1}{2^2}\right) \left(1 - \frac{1}{3^2}\right) \cdots \left(1 - \frac{1}{n^2}\right)$;
        \item $a_n = (1+q)(1+q^2)(1+q^4) \cdots (1+q^{2^m}), \ (|q|<1)$.
    \end{enumerate}
\end{exercise}

\begin{solution}
    \begin{enumerate}[(1)]
        \item $$\lim_{n \to \infty} a_n = \lim_{n \to \infty} \frac{4 + \frac{5}{n} + \frac{2}{n^2}}{3 + \frac{2}{n} + \frac{1}{n^2}} = \frac{\displaystyle \lim_{n \to \infty} 4 + \lim_{n \to \infty} \frac{5}{n} + \lim_{n \to \infty} \frac{2}{n^2}}{\displaystyle \lim_{n \to \infty} 3 + \lim_{n \to \infty} \frac{2}{n} + \lim_{n \to \infty} \frac{1}{n^2}} = \frac{4 + 0 + 0}{3 + 0 + 0} = \frac{4}{3}.$$
        \item $$a_n = \left( 1 - \frac{1}{2} \right) + \left( \frac{1}{2} - \frac{1}{3} \right) + \cdots + \left( \frac{1}{n-1} - \frac{1}{n} \right) = 1 - \frac{1}{n},$$
              $$\lim_{n \to \infty} a_n = \lim_{n \to \infty} 1 - \lim_{n \to \infty} \frac{1}{n} = 1 - 0 = 1.$$
        \item \begin{align*}
                  a_n = & \frac{2}{3} \cdot \frac{5}{6} \cdots \frac{(n^2 + n - 2)/2}{n(n+1)/2} = \frac{2}{3} \cdot \frac{5}{6} \cdots \frac{(n-1)(n+2)}{n(n+1)} = \frac{1\cdot 4}{2\cdot 3} \cdot \frac{2\cdot 5}{3\cdot 4} \cdots \frac{(n-1)(n+2)}{n(n+1)}
                  \\&= \frac{\left(1\cdot 2   \cdots (n-1) \right)\cdot\left(4\cdot 5 \cdots (n+2 )\right)}{\left(2\cdot 3  \cdots  (n) \right)\cdot\left(3\cdot 4 \cdots (n+1 )\right)}= \frac{1\cdot(n+2)}{n\cdot3} = \frac{n+2}{3n}
              ,\end{align*}
              $$\lim_{n \to \infty} a_n = \lim_{n \to \infty} \frac{n+2}{3n}= \lim_{n \to \infty}\frac{1}{3}+\frac{2}{3}\frac{1}{n} = \frac{1}{3}+\frac{2}{3}\cdot 0 = \frac{1}{3}.$$
        \item $$a_n = \frac{1 \cdot 3}{2 \cdot 2} \cdot \frac{2 \cdot 4}{3 \cdot 3} \cdots \frac{(n-1)(n+1)}{n \cdot n} = \frac{1}{2} \cdot \frac{n+1}{n} = \frac{n+1}{2n},$$
              $$\lim_{n \to \infty} a_n = \lim_{n \to \infty} \frac{n+1}{2n} = \frac{1 + \frac{1}{n}}{2} = \frac{1 + 0}{2} = \frac{1}{2}.$$
        \item $$a_n = \frac{(1-q)(1+q)(1+q^2)(1+q^4) \cdots (1+q^{2^n})}{1-q} = \frac{1-q^{2^{n+1}}}{1-q},$$
              $$\lim_{n \to \infty} a_n = \lim_{n \to \infty} \frac{1-q^{2^{n+1}}}{1-q} = \frac{\displaystyle 1 - \lim_{m \to \infty} q^{2^{n+1}}}{1-q} = \frac{1 - 0}{1-q} = \frac{1}{1-q}.$$
    \end{enumerate}
\end{solution}

\begin{exercise}[1.2.9]
    若 $a_n \ne 0 (n=1, 2, \ldots)$ 且 $\lim_{n\to\infty} a_n = a$, 能否断定 $\lim_{n\to\infty} \frac{a_n}{a_{n+1}} = 1$?
\end{exercise}

\begin{solution}
    不能. 例如 $a_n = \frac{1}{2^n}$, 则 $\lim_{n\to\infty} a_n = 0$, 但 $\lim_{n\to\infty} \frac{a_n}{a_{n+1}} = \lim_{n\to\infty} \frac{2^{n+1}}{2^n} = 2$.

    一个可能的错误做法是$$\lim_{n\to\infty} \frac{a_n}{a_{n+1}} = \frac{\lim_{n\to\infty} a_n}{\lim_{n\to\infty} a_{n+1}} = \frac{a}{a} = 1,$$ 但这是不允许的, 因为 $\lim_{n\to\infty} a_n$ 可能为 $0$.
\end{solution}

\begin{exercise}[1.2.10]
    若数列 $\{a_n\}, \{b_n\}$ 满足 $\lim_{n\to\infty} a_n \cdot b_n = 0$, 是否必有 $\lim_{n\to\infty} a_n = 0$ 或 $\lim_{n\to\infty} b_n = 0$?
    若还假设 $\lim_{n\to\infty} a_n = a$, 回答同样的问题.
\end{exercise}

\begin{solution}
    不一定. 例如 $a_n = \begin{cases}
            1, & n\text{为奇数} \\
            0, & n\text{为偶数} \\
        \end{cases}, b_n = \begin{cases}
            0, & n\text{为奇数} \\
            1, & n\text{为偶数} \\
        \end{cases}$, 则 $\lim_{n\to\infty} a_n \cdot b_n = \lim_{n\to\infty} 0=0$, 但 $\lim_{n\to\infty} a_n,\lim_{n\to\infty} b_n$均不存在.

    当 $\lim_{n\to\infty} a_n = a$时成立. 假设$a \ne 0$ 时, 则$\lim_{n\to\infty} b_n = \lim_{n\to\infty} \frac{a_n b_n}{a_n} = \frac{0}{a} = 0$.
\end{solution}

\begin{exercise}[1.2.11]
    若数列 $\{a_n\}$ 收敛, 数列 $\{b_n\}$ 发散, 则数列 $\{a_n \pm b_n\}, \{a_n \cdot b_n\}$ 的收敛性如何? 举例说明. 若数列 $\{a_n\}$ 与 $\{b_n\}$ 皆发散, 回答同样的问题.
\end{exercise}

\begin{solution}
    \begin{enumerate}
        \item $\{a_n\}$ 收敛, 数列 $\{b_n\}$ 发散,则
              \begin{enumerate}
                  \item  $\{a_n + b_n\}, \{a_n - b_n\}$ 都发散
                        可以采用反证法:若$\{a_n + b_n\}$收敛,由于 $\{a_n\}$ 收敛,容易知道$\{a_n + b_n-a_n\}=\{b_n\}$收敛,这与$\{b_n\}$ 发散矛盾,因此$\{a_n + b_n\}$发散, $\{a_n - b_n\}$同理可得 .
                  \item $\{a_n \cdot b_n\}$ 的收敛性不确定.
                        \begin{enumerate}
                            \item  $a_n = \frac{1}{n}, b_n = n$, 则 $a_n \cdot b_n = 1$ 收敛;
                            \item  $a_n = 1, b_n = n$, 则 $a_n \cdot b_n = n$ 发散.
                        \end{enumerate}
              \end{enumerate}
        \item  $\{a_n\}, \{b_n\}$ 都发散,则
              \begin{enumerate}
                  \item  $\{a_n + b_n\}$ 的收敛性不确定
                        \begin{enumerate}
                            \item  $a_n = n, b_n =-n$, 则 $a_n + b_n = 0$ 收敛.
                            \item  $a_n = n, b_n = n$, 则 $a_n + b_n = 2n$ 发散.
                        \end{enumerate}
                  \item  $\{a_n - b_n\}$ 的收敛性不确定
                        \begin{enumerate}
                            \item  $a_n = n+\frac{1}{n}, b_n = n$, 则 $a_n - b_n = \frac{1}{n},$ 收敛.
                            \item  $a_n = (-1)^{n}, b_n =(-1)^{n-1}$, 则 $a_n -b_n = 2\cdot(-1)^{n}$ 发散.
                        \end{enumerate}
                  \item $\{a_n \cdot b_n\}$ 的收敛性不确定.
                        \begin{enumerate}
                            \item  $a_n = \begin{cases}
                                          1, & n\text{为奇数} \\
                                          0, & n\text{为偶数} \\
                                      \end{cases}, b_n = \begin{cases}
                                          0, & n\text{为奇数} \\
                                          1, & n\text{为偶数} \\
                                      \end{cases}$, 则 $a_n \cdot b_n = 0$ 收敛.
                            \item  $a_n = n, b_n = (-1)^n$, 则 $a_n \cdot b_n = (-1)^n n$ 发散;
                        \end{enumerate}
              \end{enumerate}
    \end{enumerate}
\end{solution}

\begin{exercise}[1.2.12]
    下面的推理是否正确?
    \begin{enumerate}
        \item 设数列 $\{a_n\}: a_1 = 1, a_{n+1} = 2a_n - 1 \ (n=1, 2, 3, \ldots)$, 求 $\lim_{n\to\infty} a_n$.
              \par\noindent 解: 设 $\lim_{n\to\infty} a_n = a$, 在 $a_{n+1} = 2a_n - 1$ 两边取极限, 得 $a = 2a - 1$, 即 $a = 1$.

        \item
              \begin{align*}
                   & \lim_{n\to\infty} \left( \frac{1}{\sqrt{n^2+1}} + \frac{1}{\sqrt{n^2+2}} + \cdots + \frac{1}{\sqrt{n^2+n}} \right)                        \\
                   & = \lim_{n\to\infty} \frac{1}{\sqrt{n^2+1}} + \lim_{n\to\infty} \frac{1}{\sqrt{n^2+2}} + \cdots + \lim_{n\to\infty} \frac{1}{\sqrt{n^2+n}} \\
                   & = \underbrace{0 + 0 + \cdots + 0}_{\text{$n$个}} = 0.
              \end{align*}

        \item $\lim_{n\to\infty} \left(1 + \frac{1}{n}\right)^n = \left[\lim_{n\to\infty} \left(1 + \frac{1}{n}\right)\right]^n = 1^n = 1$.
    \end{enumerate}
\end{exercise}

\begin{solution}
    \begin{enumerate}
        \item 错误. 不能在未知数列是否收敛时,就假设极限存在并对递推公式两边取极限. 实际上,该数列的通项公式为 $a_n = 1$,所以 $\lim_{n\to\infty} a_n = 1$.
        \item 错误. 不能将一个数列的极限拆成无穷多个数列极限的和. 实际上
              $$
                  \frac{n}{\sqrt{n^2+n}}\les \frac{1}{\sqrt{n^2+1}} + \frac{1}{\sqrt{n^2+2}} + \cdots + \frac{1}{\sqrt{n^2+n}}
                  \les  \frac{n}{\sqrt{n^2+1}}.
              $$
              并有$$\lim_{n\to}\frac{n}{\sqrt{n^2+n}}=1,\lim_{n\to}\frac{n}{\sqrt{n^2+1}}=1.
              $$
              由夹逼准则$$
                  \lim_{n\to\infty} \left( \frac{1}{\sqrt{n^2+1}} + \frac{1}{\sqrt{n^2+2}} + \cdots + \frac{1}{\sqrt{n^2+n}} \right) = 1.
              $$
        \item 错误. 不能将一个数列的极限拆成无穷多个数列极限的积. 实际上
              $$
                  \lim_{n\to\infty} \left(1 + \frac{1}{n}\right)^n = \e.
              $$
    \end{enumerate}
\end{solution}

\begin{exercise}[1.2.13]
    设数列 $\{a_n\}$ 与 $\{b_n\}$ 分别收敛于 $a, b$. 若 $a > b$, 则从某一项开始, 有 $a_n > b_n$; 反之, 若从某项开始恒有 $a_n \ges b_n$, 则 $a \ges b$.
\end{exercise}

\begin{solution}
    这是保序性的直接推论.
\end{solution}

\begin{exercise}[1.2.14]
    设数列 $\{a_n\}, \{b_n\}$ 分别收敛于 $a$ 及 $b$. 记 $c_n = \max(a_n, b_n)$, $d_n = \min(a_n, b_n) \ (n=1, 2, \ldots)$. 证明
    $$
        \lim_{n\to\infty} c_n = \max(a, b), \quad \lim_{n\to\infty} d_n = \min(a, b).
    $$
\end{exercise}

\begin{solution}
    由$\max(x,y) = \frac{x+y+|x-y|}{2}, \min(x,y) = \frac{x+y-|x-y|}{2}$,以及数列极限的四则运算和绝对值运算可得.
\end{solution}

\begin{exercise}[1.2.15]
    求下列极限:
    \begin{enumerate}[(1)]
        \item $\lim_{n\to\infty} \left[\frac{1}{(n+1)^2} + \frac{1}{(n+2)^2} + \cdots + \frac{1}{(2n)^2}\right]$;
        \item $\lim_{n\to\infty} ((n+1)^k - n^k)$, 其中 $0 < k < 1$;
        \item $\lim_{n\to\infty} (\sqrt{2} \cdot \sqrt[4]{2} \cdot \sqrt[8]{2} \cdots \sqrt[2^n]{2})$;
        \item $\lim_{n\to\infty} \left(\sqrt{n^2 - n + 2} - n\right)$;
        \item $\lim_{n\to\infty} \left(\sqrt{\cos^2 1 + \cos^2 2 + \cdots + \cos^2 n}\right)$.
    \end{enumerate}
\end{exercise}

\begin{solution}
    \begin{enumerate}[(1)]
        \item 由于$$0\les \sum_{k=1}^{n} \frac{1}{(n+k)^2} \les \sum_{k=1}^{n} \frac{1}{n^2} = \frac{n}{n^2} = \frac{1}{n}.$$
              并且$$\lim_{n\to\infty} 0 = 0,\lim_{n\to\infty} \frac{1}{n} = 0.$$
              由夹逼准则$$\lim_{n\to\infty}\sum_{k=1}^{n} \frac{1}{(n+k)^2}=0.$$
        \item 由于$$0\les ((n+1)^k - n^k) = n^k \left( \left( 1 + \frac{1}{n} \right)^k - 1 \right) \les n^k \left( \left( 1 + \frac{1}{n} \right)^1 - 1 \right)  = n^{k-1} .$$
              并且$$\lim_{n\to\infty} 0=0,\lim_{n\to\infty} n^{k-1} = 0.$$
              由夹逼准则$$\lim_{n\to\infty} ((n+1)^k - n^k)  = 0.$$
        \item $$\lim_{n\to\infty} \prod_{k=1}^{n} \sqrt[2^k]{2} = \lim_{n\to\infty} 2^{\sum\limits_{k=1}^{n} \frac{1}{2^k}} =  2^{\lim\limits_{n\to\infty}\sum\limits_{k=1}^{n} \frac{1}{2^k}} =2^1 = 2.$$
        \item 由$\frac{1}{n} \ln(n^2 - n + 2) = O\left( \frac{\ln n}{n} \right) = o(1)$,故
              $$\lim_{n\to\infty} \sqrt[n]{n^2 - n + 2} = \lim_{n\to\infty} \e^{ \frac{1}{n} \ln \left(n^2 - n + 2\right)} =\e^{\lim\limits_{n\to\infty} \frac{1}{n} \ln \left(n^2 - n + 2\right)} = \e^0 = 1.$$

        \item 由于$$\sqrt[n]{\cos^2 1} \les \sqrt[n]{\cos^2 1 + \cos^2 2 + \cdots + \cos^2 n} \les \sqrt[n]{n}.$$
              并且$$\lim_{n\to\infty} \sqrt[n]{\cos^2 1} =1,\lim_{n\to\infty} \sqrt[n]{n} = 1.$$
              由夹逼准则$$\lim_{n\to\infty} \sqrt[n]{\cos^2 1 + \cos^2 2 + \cdots + \cos^2 n} = 1.$$
    \end{enumerate}
\end{solution}

\begin{exercise}[1.2.16]
    设 $a_1, a_2, \ldots, a_m$ 为 $m$ 个正数, 证明:
    $$
        \lim_{n\to\infty} \sqrt[n]{a_1^n + a_2^n + \cdots + a_m^n} = \max(a_1, a_2, \ldots, a_m).
    $$
\end{exercise}

\begin{solution}
    设 $a_k = \max \{a_1, a_2, \ldots, a_m \}$,则
    $$
        a_k = \sqrt[n]{a_k^n} \les \sqrt[n]{a_1^n + a_2^n + \cdots + a_m^n} \les \sqrt[n]{m a_k^n} = m^{\frac{1}{n}} a_k.
    $$
    由夹逼定理可得
    $$
        \lim_{n\to\infty} \sqrt[n]{a_1^n + a_2^n + \cdots + a_m^n} = a_k = \max(a_1, a_2, \ldots, a_m).
    $$
\end{solution}

\begin{exercise}[1.2.17]
    证明下列数列收敛:
    \begin{enumerate}[(1)]
        \item $a_n = \left(1 - \frac{1}{2}\right) \left(1 - \frac{1}{2^2}\right) \cdots \left(1 - \frac{1}{2^n}\right)$;
        \item $a_n = \frac{1}{3+1} + \frac{1}{3^2+1} + \cdots + \frac{1}{3^n+1}$;
        \item $a_n = \alpha_0 + \alpha_1 q + \cdots + \alpha_n q^n$, 其中 $|\alpha_k| \les M, (k=1, 2, \ldots)$, 而 $|q| < 1$;
        \item $a_n = \frac{\cos 1}{1 \cdot 2} + \frac{\cos 2}{2 \cdot 3} + \frac{\cos 3}{3 \cdot 4} + \cdots + \frac{\cos n}{n(n+1)}$.
    \end{enumerate}
\end{exercise}

\begin{proof}
    \begin{enumerate}[(1)]
        \item 由$1-\frac{1}{2^n} < 1$,可知$\{a_n\}$单调减,且$a_n > 0$,因此$\{a_n\}$收敛.
        \item 由$a_n < \sum_{k=1}^n \frac{1}{3^k} < \frac 12$,可知$\{a_n\}$有上界,且$a_n$单调递增,因此$\{a_n\}$收敛.
        \item 利用Cauchy收敛准则,对$\forall \ve >0$,取$N = \left\lfloor \log_{|q|} \frac{\ve(1-|q|)}{2M} \right\rfloor + 1$,则当$m>n>N$时,$$|a_m-a_n| = |\alpha_{n+1}q^{n+1}+\cdots+\alpha_m q^m| \les M(|q|^{n+1}+|q|^{n+2}+\cdots) = M\frac{|q|^{n+1}}{1-|q|} < \ve.$$
        \item 利用Cauchy收敛准则,对$\forall \ve >0$,取$N = \left\lfloor \frac{1}{\ve} \right\rfloor + 1$,则当$m>n>N$时,$$|a_m-a_n| = \left|\frac{\cos(n+1)}{(n+1)(n+2)}+\cdots+\frac{\cos m}{m(m+1)}\right| \les \sum_{k=n+1}^m \frac{1}{k(k+1)} = \frac{1}{n+1} - \frac{1}{m+1} < \frac{1}{n+1} < \ve.$$
    \end{enumerate}
\end{proof}

\begin{exercise}[1.2.18]
    证明下列数列收敛, 并求出其极限:
    \begin{enumerate}[(1)]
        \item $a_n = \frac{n}{c^n}, \ (c>1)$;
        \item $a_1 = \frac{c}{2}, \ a_{n+1} = \frac{c}{2} + \frac{a_n^2}{2} \ (0 \les c \les 1)$;
        \item $a > 0, a_0 > 0, a_{n+1} = \frac{1}{2} \left(a_n + \frac{a}{a_n}\right)$ (提示: 先证明 $a_n^2 \ges a$);
        \item $a_0 = 1, \ a_n = 1 + \frac{a_{n-1}}{a_{n-1} + 1}$;
        \item $a_n = \sin \sin \cdots \sin 1$ ($n$ 个 $\sin$).
    \end{enumerate}
\end{exercise}

\begin{solution}
    \begin{enumerate}[(1)]
        \item 由Stolz定理,有$$\lim_{n \to \infty} a_n = \lim_{n \to \infty} \frac{n}{c^n} = \lim_{n \to \infty} \frac{(n+1)-n}{c^{n+1}-c^n} = \lim_{n \to \infty} \frac{1}{c^n(c-1)} = 0.$$
        \item $$a_{n+1} - a_n = \frac{1}{2}(a_{n}-a_{n-1})(a_{n}+a_{n-1})$$
              由$a_2 - a_1 = \left( \frac{c}{2} \right)^2 > 0 $,可递归的得知$a_{n+1} - a_n > 0$,因此$\{a_n\}$单调增,且$a_1 < c$,归纳的可得$a_{n+1} < \frac{c}{2} + \frac{c^2}{2} < \frac{c}{2} + \frac{c}{2} = c$,因此$\{a_n\}$有上界,故$\{a_n\}$收敛.设$\lim_{n \to \infty} a_n = a$,则$a = \frac{c}{2} + \frac{a^2}{2} \Rightarrow a^2 - 2a + c = 0 \Rightarrow a = 1 \pm \sqrt{1-c}$,又由$a_n > 0$,可知$a = 1 - \sqrt{1-c}$.
        \item 由均值不等式,$$a_{n+1} = \left( \frac{1}{2} \left( a_n + \frac{a}{a_n} \right) \right)^2 \ges a$$
              于是
              $$a_{n+1} - a_n = \frac{a - a_n^2}{2a_n} \les 0$$
              因此$\{a_n\}$在$n \ges 1$时单调减,且有下界$\sqrt{a}$,因此$\{a_n\}$收敛.设$\lim_{n \to \infty} a_n = l$,则$l = \frac{1}{2} \left( l + \frac{a}{l} \right) $,解得$l= \sqrt{a}$.
        \item $$a_n - a_{n-1} = \frac{1+ a_{n-1} - a_{n-1}^2}{a_{n-1} + 1} $$
              $$1 + a_n - a_n^2 = 1 + 1 + \frac{a_{n-1}}{a_{n-1} + 1} - \left( 1 + \frac{a_{n-1}}{a_{n-1} + 1} \right)^2 = \frac{1 + a_{n-1} - a_{n-1}^2}{(a_{n-1} + 1)^2}$$
              由$1+a_0 -a_0^2 = 1 > 0$归纳的可得$1 + a_n - a_n^2 > 0$,因此$a_n - a_{n-1} > 0$,即$\{a_n\}$单调递增,且
              $1 + a_n - a_n^2 > 0 \Rightarrow a_n < \frac{1 + \sqrt{5}}{2}$有上界,
              因此$\{a_n\}$收敛,设$\lim_{n\to\infty}a_n=a$.递推式两侧取极限,得$a = 1 + \frac{a}{a+1} \Rightarrow a^2 - a - 1 = 0 \Rightarrow a = \frac{1 \pm \sqrt{5}}{2}$;由于$a_n>0$始终成立,故$a\ges 0$而$\frac{1 -\sqrt{5}}{2}<0$,故舍去这一值,进而得到$a=\frac{1 + \sqrt{5}}{2}.$
        \item $a_n = \sin a_{n-1} < a_{n-1}$,因此$\{a_n\}$单调减,且$a_n > 0$,因此$\{a_n\}$收敛.设$\lim_{n \to \infty} a_n = a$,则$a = \sin a \Rightarrow a = 0$.
    \end{enumerate}
\end{solution}

\begin{exercise}[1.2.19]
    设 $a_n \les a \les b_n \ (n=1, 2, \ldots)$, 且 $\lim_{n\to\infty} (a_n - b_n) = 0$. 求证: $\lim_{n\to\infty} a_n = a$, $\lim_{n\to\infty} b_n = a$.
\end{exercise}

\begin{solution}
    由$\lim_{n\to\infty} (a_n - b_n) = 0$,对$\forall \ve > 0$,存在$N\in \mathbb{N}^*$,使得当$n>N$时,$|a_n - b_n| < \ve$.又由$a_n \les a \les b_n$,可知$|a_n - a| = a -a_n \les b_n - a_n < \ve$,同理$|b_n - a| < \ve$.因此$\lim_{n\to\infty} a_n = a$, $\lim_{n\to\infty} b_n = a$.
\end{solution}

\begin{exercise}[1.2.20]
    证明: 若 $a_n > 0$, 且 $\lim_{n\to\infty} \frac{a_{n}}{a_{n+1}} = l > 1$, 则 $\lim_{n\to\infty} a_n = 0$.
\end{exercise}

\begin{solution}
    先证明一个引理:
    设$a_n > 0, n = 1, 2, \ldots$, 且$\lim_{n\to\infty} a_n = a$, 则$\lim_{n\to\infty} \sqrt[n]{a_1 a_2 \cdots a_n} = a$.

    证明如下
    \begin{enumerate}[(1)]
        \item $a=0$时,$$0 \les \sqrt[n]{a_1 a_2 \cdots a_n} \les \frac{a_1 + a_2 + \cdots + a_n}{n} $$
              \     同时,由Stolz定理,$$\lim_{n\to\infty}  \frac{a_1 + a_2 + \cdots + a_n}{n}=\lim_{n\to\infty}  \frac{a_n}{1}=0$$
              由夹逼定理,得证.
        \item $a>0$时,$$\frac{n}{\frac{1}{a_1} + \frac{1}{a_2} + \cdots + \frac{1}{a_n}} \les \sqrt[n]{a_1 a_2 \cdots a_n} \les \frac{a_1 + a_2 + \cdots + a_n}{n}.$$
              由Stolz定理,有$$\lim_{n\to\infty} \frac{n}{\frac{1}{a_1} + \frac{1}{a_2} + \cdots + \frac{1}{a_n}} = \lim_{n\to\infty} \frac{1}{\frac{1}{a_n}} = a,$$且$$\lim_{n\to\infty} \frac{a_1 + a_2 + \cdots + a_n}{n} = \lim_{n\to\infty} a_n = a.$$
              由夹逼定理,得证.
    \end{enumerate}

    回到本题,

    可知$\lim_{n\to\infty} \sqrt[n]{\frac{a_{n}}{a_{n-1}} \cdot \frac{a_{n-1}}{a_{n-2}} \cdots \frac{a_2}{a_1}} = \lim_{n\to\infty} \frac{a_n}{a_{n-1}} = \frac{1}{l} < 1$.因此$\exists r = \frac{1 + \frac{1}{l}}{2} \in (0,1)$,使得当$n$充分大时,$\sqrt[n]{\frac{a_{n}}{a_{n-1}} \cdot \frac{a_{n-1}}{a_{n-2}} \cdots \frac{a_2}{a_1}} < r$.由此可知,$$\frac{a_n}{a_{n-1}} \cdot \frac{a_{n-1}}{a_{n-2}} \cdots \frac{a_2}{a_1} < r^n,$$即$a_n < a_1 r^n.$因此$\lim_{n\to\infty} a_n = 0$.
\end{solution}

\begin{exercise}[1.2.21]
    设数列 $\{a_n\}$, $\{b_n\}$ 是正数列, 满足 $\frac{a_{n+1}}{a_n} \les \frac{b_{n+1}}{b_n}, \ n=1, 2, \ldots$. 求证: 若 $\{b_n\}$ 收敛, 则 $\{a_n\}$ 收敛.
\end{exercise}

\begin{solution}
    若$\lim_{n\to\infty} b_n = 0$,则由$a_n = a_1 \cdot \frac{a_2}{a_1} \cdot \frac{a_3}{a_2} \cdots \frac{a_n}{a_{n-1}} \les a_1 \cdot \frac{b_2}{b_1} \cdot \frac{b_3}{b_2} \cdots \frac{b_n}{b_{n-1}} = a_1 \cdot \frac{b_n}{b_1}$可知$\lim_{n\to\infty} a_n = 0$.
    若$\lim_{n\to\infty} b_n = b > 0$,由原式有$\frac{a_{n+1}}{b_{n+1}} \les \frac{a_n}{b_n}$,因此$\left\{ \frac{a_n}{b_n} \right\}$单调减,且$\frac{a_n}{b_n} > 0$,因此$\left\{ \frac{a_n}{b_n} \right\}$收敛,设$\lim_{n\to\infty} \frac{a_n}{b_n} = c$,则$\lim_{n\to\infty} a_n = \lim_{n\to\infty} b_n \cdot \lim_{n\to\infty} \frac{a_n}{b_n} = bc$.
\end{solution}

\begin{exercise}[1.2.22]
    利用极限 $\lim_{n\to\infty} \left(1 + \frac{1}{n}\right)^n = \e$, 求下列数列的极限:
    \begin{tasks}[label=(\arabic*)](2)
        \task $a_n = \left(1 + \frac{1}{2n+1}\right)^{2n+1}$;
        \task $a_n = \left(1 - \frac{1}{n-2}\right)^{n+1}$;
        \task $a_n = \left(\frac{1+n}{2+n}\right)^{n}$;
        \task $a_n = \left(1 + \frac{1}{n^3}\right)^{2n^3}$.
    \end{tasks}
\end{exercise}

简要说明:由$\lim_{n\to\infty} \left(1 + \frac{1}{n}\right)^n = \e$,故$\left\{ \left(1 + \frac{1}{n}\right)^n \right\}$的任意子列$\left\{ \left(1 + \frac{1}{n_k}\right)^{n_k} \right\}$也收敛于$\e$.因此,我们可以通过适当的变形,将题目中的数列变形为$\left(1 + \frac{1}{n_k}\right)^{n_k}$的形式,从而求出极限.对于类似于$\left(1 - \frac{1}{n}\right)^{-n}$的形式,可以考虑先通分再变形去掉指数的负号即可处理.

在此过程中下列命题也相同有用:
\begin{proposition*}
    设数列$\{a_n\}$收敛于$a$, $a_n > 0, a > 0$. $\{b_n\}$收敛于$b$.则$\lim_{n \to \infty} a_n^{b_n} = a^b$.
\end{proposition*}
请注意,这条结论对于$1^ \infty$型是不能直接使用的,即若$a_n \to 1, b_n \to \infty$,则不能直接说$a_n^{b_n} \to 1^\infty = 1$.但是对于$a_n \to a > 1, b_n \to \infty$,则可以直接说$a_n^{b_n} \to a^{+ \infty} = + \infty$;对于$a_n \to a < 1, b_n \to + \infty$,则可以直接说$a_n^{b_n} \to a^{+ \infty} = 0$.

\begin{solution}
    \begin{enumerate}[(1)]
        \item $\lim_{n\to\infty} \left(1 + \frac{1}{2n+1}\right)^{2n+1} = \left. \lim_{m\to\infty} \left(1 + \frac{1}{m}\right)^m \right|_{m=2n+1} = \e$;
        \item $\lim_{n\to\infty} \left(1 - \frac{1}{n-2}\right)^{n+1}  = \lim_{n\to\infty} \left(1+\frac{1}{n-3}\right)^{-n-1} =\lim_{n\to\infty} \left(1+\frac{1}{n-3}\right)^{(n-3)\cdot\left(-\frac{n+1}{n-3}\right)} =\e^{-1} $;
        \item $\lim_{n\to\infty} \left(\frac{1+n}{2+n}\right)^{n}=\lim_{n\to\infty} \left(1 + \frac{1}{n+1} \right)^{-n}=\lim_{n\to\infty} \left(1 + \frac{1}{n+1} \right)^{(n+1)\cdot\left(-\frac{n}{n+1}\right)} = \e^{-1}$;
        \item $\lim_{n\to\infty} \left(1 + \frac{1}{n^3}\right)^{2n^3}=\lim_{n\to\infty} \left(1 + \frac{1}{n^3}\right)^{n^3 \cdot 2} = \e^2$.
    \end{enumerate}
\end{solution}

\begin{exercise}[1.2.23]
    设 $\lim_{n\to\infty} a_n = \infty$, 且 $|b_n| \ges b > 0 \ (n=1, 2, \ldots)$, 则 $\lim_{n\to\infty} a_n b_n = \infty$.
\end{exercise}

\begin{solution}
    对$\forall M > 0$,由$\lim_{n\to\infty} a_n = \infty$,存在$N\in \mathbb{N}^*$,使得当$n>N$时,$|a_n |> \frac{M}{b}$.又由$|b_n| \ges b > 0$,可知$|a_n b_n| \ges |a_n| |b| > M$.因此$\lim_{n\to\infty} a_n b_n = \infty$.
\end{solution}

\begin{exercise}[1.2.24]
    确定 $n \to \infty$ 时, $\sqrt[n]{n!}$ 与 $n \sin \frac{n \pi}{2} \ (n \ges 1)$ 是否有界, 是否趋于无穷大.
\end{exercise}

\begin{solution}
    $\sqrt[n]{n!}$无界,且趋于无穷大.由均值不等式,$$\sqrt[n]{n!} \ges \frac{n}{\frac{1}{1} + \frac{1}{2} + \cdots + \frac{1}{n}}$$
    已知$\lim_{n \to \infty}  \frac{1}{\frac{1}{n}}  = +\infty$,由Stolz定理,
    $$\lim_{n \to \infty} \frac{n}{\frac{1}{1} + \frac{1}{2} + \cdots + \frac{1}{n}} 
        = \lim_{n \to \infty} \frac{1}{\frac1{n}} = +\infty.$$
    因此,$\lim_{n \to \infty} \sqrt[n]{n!} = +\infty$.

    \begin{remark}
        Stolz定理规范的思路要先说明$\lim_{n \to \infty} \frac{a_{n+1} - a_n}{b_{n+1} - b_n}$存在,然后才能说明$\lim_{n \to \infty} \frac{a_n}{b_n}$存在.为了方便,我们也会省去前面的部分,直接写$\lim_{n \to \infty} \frac{a_n}{b_n} = \lim_{n \to \infty} \frac{a_{n+1} - a_n}{b_{n+1} - b_n}$.
    \end{remark}

    $n \sin \frac{n \pi}{2}$无界,但是不趋于无穷大.当$n=4k+1$时,$n \sin \frac{n \pi}{2} = 4k+1$,趋于无穷大;当$n=4k+3$时,$n \sin \frac{n \pi}{2} = -(4k+3)$,趋于负无穷大;当$n$为偶数时,$n \sin \frac{n \pi}{2} = 0$.
\end{solution}

\begin{exercise}[1.2.25]
    设数列 $\{a_n\}$ 由 $a_1 = 1, a_{n+1} = a_n + \frac{1}{a_n} \ (n \ges 1)$ 定义, 证明: $a_n \to +\infty \ (n \to \infty)$.
\end{exercise}

\begin{solution}
    由$a_{n+1}^2 - a_n^2 = (a_{n+1} - a_n)(a_{n+1} + a_n) = \frac{1}{a_n}(a_n + a_n + \frac{1}{a_n}) = 2 + \frac{1}{a_n^2} > 2$,可知$a_n^2 > 2(n-1)$,因此$\lim_{n\to\infty} a_n = \infty$.
\end{solution}

\begin{exercise}[1.2.26]
    给出 $\frac{0}{0}$ 型 Stolz 定理的证明.
\end{exercise}

\begin{proposition*}[$\frac{0}{0}$ 型的 Stolz 定理]
    设 $\{a_n\}$ 和 $\{b_n\}$ 都是无穷小量, 其中 $\{a_n\}$ 还是严格单调减少数列, 又存在 (其中 $l$ 为有限或 $\pm\infty$)
    $$ \lim_{n \to \infty} \frac{b_{n+1}-b_n}{a_{n+1}-a_n} = l, $$
    则有
    $$ \lim_{n \to \infty} \frac{b_n}{a_n} = l. $$
\end{proposition*}

\begin{proof}
    \begin{enumerate}[(1)]
        \item 当$l$为有限值时,根据条件对 $\varepsilon > 0$ 存在 $N$, 使当 $n > N$ 时成立
              $$ \left|\frac{b_n-b_{n+1}}{a_n-a_{n+1}} - l \right| < \varepsilon. $$
              由于对每个 $n$ 都有 $a_n > a_{n+1}$, 这样就有
              $$ (l-\varepsilon)(a_n-a_{n+1}) < b_n - b_{n+1} < (l+\varepsilon)(a_n-a_{n+1}). $$
              任取 $m>n$, 并且将上述不等式中的 $n$ 换成 $n+1, \dots, \text{直到 } m-1$, 然后将所有这些不等式相加, 就得到
              $$ (l-\varepsilon)(a_n-a_m) < b_n - b_m < (l+\varepsilon)(a_n-a_m), $$
              以及
              $$ \left|\frac{b_n-b_m}{a_n-a_m} - l \right| < \varepsilon. $$
              令 $m \to \infty$, 并利用条件 $\lim_{m \to \infty} a_m = \lim_{m \to \infty} b_m = 0$, 就知道当 $n>N$ 时成立
              $$ \left|\frac{b_n}{a_n} - l \right| \les \varepsilon. $$
        \item $l = +\infty$ 时. 根据条件对任意 $M > 0$ 存在 $N$, 使当 $n > N$ 时成立
              $$ \frac{b_n-b_{n+1}}{a_n-a_{n+1}} > M. $$
              由于对每个 $n$ 都有 $a_n > a_{n+1}$, 这样就有
              $$ b_n - b_{n+1} > M(a_n-a_{n+1}). $$
              任取 $m>n$, 并且将上述不等式中的 $n$ 换成 $n+1, \dots, \text{直到 } m-1$, 然后将所有这些不等式相加, 就得到
              $$ b_n - b_m > M(a_n-a_m), $$
              以及
              $$ \frac{b_n-b_m}{a_n-a_m} > M. $$
              令 $m \to \infty$, 并利用条件 $\lim_{m \to \infty} a_m = \lim_{m \to \infty} b_m = 0$, 就知道当 $n>N$ 时成立
              $$ \frac{b_n}{a_n} > M.  $$
    \end{enumerate}
\end{proof}


\newpage