\chapter{连续函数的基本概念}
\section{习题 2.1}

\begin{exercise}[2.1.1]
    设函数 $f(x)$ 在点 $x_0$ 附近有定义, 且 $\lim_{h \to 0} [f(x_0+h)-f(x_0-h)]=0$. 问 $f(x)$ 是否必在 $x=x_0$ 处连续?
\end{exercise}

\begin{solution}
    不一定. 例如, 设
    $$ f(x) = \begin{cases} 0, & x \ne 0, \\ 1, & x=0. \end{cases} $$
    则 $\lim_{h \to 0} [f(0+h)-f(0-h)] = \lim_{h \to 0} (0-0)=0$, 但 $f(x)$ 在 $x=0$ 处不连续.
\end{solution}

\begin{exercise}[2.1.2]
    设对任意正数 $\varepsilon < \frac{b-a}{2}$, 函数 $f(x)$ 在 $[a+\varepsilon, b-\varepsilon]$ 上连续. 证明 $f(x)$ 在 $(a,b)$ 内连续.
\end{exercise}

\begin{solution}
    设 $x_0 \in (a,b)$, 则存在 $\varepsilon_0 = \min\left\{\frac{b-x_0}{2}, \frac{x_0-a}{2}\right\}> 0$, 使得 $a+\varepsilon_0 < x_0 < b-\varepsilon_0$. 因为 $f(x)$ 在 $[a+\varepsilon_0, b-\varepsilon_0]$ 上连续, 故 $f(x)$ 在 $x_0$ 处连续. 因此 $f(x)$ 在 $(a,b)$ 内连续.
\end{solution}

\begin{exercise}[2.1.3]
    设在点 $x=x_0$ 处, 函数 $f(x)$ 连续, 而 $g(x)$ 不连续, 问函数 $f(x) \pm g(x)$ 与 $f(x)g(x)$ 在点 $x_0$ 的连续性如何? 若 $f(x), g(x)$ 在 $x_0$ 处都不连续, 回答同样的问题.
\end{exercise}

\begin{solution}
    \begin{enumerate}
        \item $f(x) \pm g(x)$ 与 $f(x)g(x)$ 在点 $x_0$ 处均不连续. 反证: 若 $f(x)+g(x)$ 在点 $x_0$ 处连续, 则 $\lim_{x \to x_0} [f(x)+g(x)] = f(x_0)+g(x_0)$, 即 $\lim_{x \to x_0} g(x) = \lim_{x \to x_0} [f(x)+g(x)] - \lim_{x \to x_0} f(x) = f(x_0)+g(x_0)-f(x_0)=g(x_0)$, 矛盾. 同理可证 $f(x)-g(x)$ 在点 $x_0$ 处不连续.

              $f(x)g(x)$在$x_0$处连续性未知, 例如, 设
              $$ f(x) = x, \quad g(x) = \begin{cases} 1, & x \ne 0, \\ 0, & x=0; \end{cases} $$
                则 $f(x)g(x) \equiv x$, 在点 $x_0=0$ 处连续. 又例如, 设
                $$ f(x) = 1, \quad g(x) = \begin{cases} 1, & x \ne 0, \\ 0, & x=0; \end{cases} $$
                则 $f(x)g(x) = \begin{cases} 1, & x \ne 0, \\ 0, & x=0; \end{cases}$ 在点 $x_0=0$ 处不连续.

        \item 若 $f(x), g(x)$ 在 $x_0$ 处都不连续, 则 $f(x) \pm g(x)$ 与 $f(x)g(x)$ 在点 $x_0$ 处可能连续, 也可能不连续. 例如,
              \begin{enumerate}
                  \item $f(x) + g(x)$ 在点 $x_0$ 处连续: 设
                        $$ f(x) = \begin{cases} 0, & x \ne
              0,         \\ 1, & x=0;\end{cases} \quad g(x) = \begin{cases} 0, & x \ne 0, \\ -1, & x=0. \end{cases} $$
                        则 $f(x)+g(x) \equiv 0$, 在点 $x_0=0$ 处连续.
                  \item $f(x) + g(x)$ 在点 $x_0$ 处不连续: 设
                        $$ f(x) = \begin{cases} 0, & x \ne 0, \\ 1, & x=0; \end{cases} \quad g(x) = \begin{cases} 0, & x \ne 0, \\ 1, & x=0. \end{cases} $$
                        则 $f(x)+g(x) = \begin{cases} 0, & x \ne 0, \\ 2, & x=0; \end{cases}$ 在点 $x_0=0$ 处不连续.
                  \item $f(x)g(x)$ 在点 $x_0$ 处连续: 设
                        $$ f(x) = \begin{cases} 1, & x \ne 0, \\ 0, & x=0; \end{cases} \quad g(x) = \begin{cases} 0, & x \ne 0, \\ 1, & x=0. \end{cases} $$
                        则 $f(x)g(x) \equiv 0$, 在点 $x_0=0$ 处连续.
                  \item $f(x)g(x)$ 在点 $x_0$ 处不连续: 设
                        $$ f(x) = \begin{cases} 1, & x \ne 0, \\ 0, & x=0; \end{cases} \quad g(x) = \begin{cases} 1, & x \ne 0, \\ 0, & x=0. \end{cases} $$
                        则 $f(x)g(x) = \begin{cases} 1, & x \ne 0, \\ 0, & x=0; \end{cases}$ 在点 $x_0=0$ 处不连续.
              \end{enumerate}
    \end{enumerate}
\end{solution}

\begin{exercise}[2.1.4]
    \begin{enumerate}
        \item 设函数 $f(x)$ 在点 $x=x_0$ 处连续, 则函数 $|f(x)|$ 在点 $x=x_0$ 处也连续.
        \item 设函数 $f(x)$ 和 $g(x)$ 在一个区间 $I$ 上连续, 证明: 函数 $M(x)=\max(f(x), g(x))$ 及 $m(x)=\min(f(x), g(x))$ 在区间 $I$ 上均连续.
    \end{enumerate}
\end{exercise}

\begin{solution}
    \begin{enumerate}
        \item 因为 $f(x)$ 在点 $x=x_0$ 处连续, 故 $\lim_{x \to x_0} f(x) = f(x_0)$, 因此$\forall \varepsilon > 0$, $\exists \delta > 0$, 当 $|x-x_0| < \delta$ 时, 有 $|f(x)-f(x_0)| < \varepsilon$. 又因为
              $$ ||f(x)| - |f(x_0)|| \les |f(x)-f(x_0)| < \varepsilon, $$
              故 $\lim_{x \to x_0} |f(x)| = |f(x_0)|$, 即 $|f(x)|$ 在点 $x=x_0$ 处连续.
        \item 由 $M(x) = \frac{f(x)+g(x)}{2} + \frac{|f(x)-g(x)|}{2}$, $m(x) = \frac{f(x)+g(x)}{2} - \frac{|f(x)-g(x)|}{2}$ 可知, 只需证明 $|f(x)-g(x)|$ 在区间 $I$ 上连续. 因为 $f(x), g(x)$ 在区间 $I$ 上连续, 故 $f(x)-g(x)$ 在区间 $I$ 上连续. 由 (1) 可知, $|f(x)-g(x)|$ 在区间 $I$ 上连续.
    \end{enumerate}
\end{solution}

\begin{exercise}[2.1.5]
    证明: 存在这样的函数 $f(x)$, 处处不连续, 但函数 $|f(x)|$ 处处连续. (提示: 适当地修改 Dirichlet 函数可得出一个例子.)
\end{exercise}

\begin{solution}
    设
    $$ f(x) = \begin{cases} 1, & x \in \mathbb{Q}, \\ -1, & x \in \mathbb{R} \setminus \mathbb{Q}. \end{cases} $$
    则 $f(x)$ 处处不连续, 但 $|f(x)| \equiv 1$, 处处连续.
    \begin{proof}
        若$f(x)$在$x_0$处连续,则$\lim_{x \to x_0} f(x)$存在,取$\{a_n\} \subset \mathbb{Q}$, $\lim_{n \to \infty} a_n = x_0$,则$\lim_{n \to \infty} f(a_n) = 1$;取$\{b_n\} \subset \mathbb{R} \setminus \mathbb{Q}$, $\lim_{n \to \infty} b_n = x_0$,则$\lim_{n \to \infty} f(b_n) = -1$, $\lim_{x \to x_0} f(a_n) \ne \lim_{x \to x_0} f(b_n)$,矛盾.因此$f(x)$处处不连续.
    \end{proof}
\end{solution}

\begin{exercise}[2.1.6]
    指出下列函数的间断点,并说明其类型.
    \begin{tasks}[label=(\arabic*)](2)
        \task $f(x) = \frac{x+1}{x-2}$;
        \task $f(x) = \begin{cases} \frac{\sin x}{|x|}, & x \ne 0, \\ 1, & x=0; \end{cases}$
        \task $f(x) = \lfloor|\cos x|\rfloor$;
        \task $f(x) = \frac{1}{1+\mathrm{e}^{1/x}}$;
        \task $f(x) = \begin{cases} \frac{1}{x+7}, & -\infty < x < -7, \\ x, & -7 \les x \les 1, \\ (x-1)\sin\frac{1}{x-1}, & 1 < x < +\infty; \end{cases}$
        \task[]
        \task[(6)] $f(x) = \begin{cases} \frac{x^2-4}{x-2}, & x \ne 2, \\ 4, & x=2. \end{cases}$
    \end{tasks}
\end{exercise}

\begin{solution}
    \begin{enumerate}
        \item $\lim\limits_{x \to 2^-} f(x) = -\infty$,$\lim\limits_{x \to 2^+} f(x) = +\infty$.\\
        左右极限不存在,因此 $x=2$ 是第二类间断点.

        \item 
        $\lim\limits_{x \to 0^+} f(x) = \lim\limits_{x \to 0^+} \frac{\sin x}{x} = 1$,$\lim\limits_{x \to 0^-} f(x) = \lim\limits_{x \to 0^-} \left(-\frac{\sin x}{x}\right) = -1$.\\
        左右极限存在但不相等,因此 $x=0$ 是第一类间断点中的跳跃间断点.

        \item 可知$f(x)=\begin{cases}
            1,&x=k\pi,k\in\Z;\\
            0,&\text{其他}.
        \end{cases}$
        在 $x = k\pi$ 处,$\lim\limits_{x \to k\pi} f(x) = 0 \ne f(k\pi) = 1$.\\
        左右极限存在且相等,$x = k\pi$ ($k \in \mathbb{Z}$) 是第一类间断点中的可去间断点.

        \item 
        $\lim\limits_{x \to 0^+} f(x) = \lim\limits_{x \to 0^+} \frac{1}{1+\mathrm{e}^{1/x}} = 0$,
        $\lim\limits_{x \to 0^-} f(x) = \lim\limits_{x \to 0^-} \frac{1}{1+\mathrm{e}^{1/x}} = 1$.\\
        左右极限存在但不相等,因此 $x=0$ 是第一类间断点中的跳跃间断点.

        \item 
        \begin{itemize}
            \item 在 $x=-7$ 处:\\
            $\lim\limits_{x \to -7^-} f(x) = \lim\limits_{x \to -7^-} \frac{1}{x+7} = -\infty$,
            $\lim\limits_{x \to -7^+} f(x) = -7$.\\
            左极限不存在,因此 $x=-7$ 是第二类间断点.
            \item 在 $x=1$ 处:\\
            $\lim\limits_{x \to 1^-} f(x) = 1$,
            $\lim\limits_{x \to 1^+} f(x) = \lim\limits_{x \to 1^+} (x-1)\sin\frac{1}{x-1} = 0$,\\
            $f(1) = 1$.\\
            左右极限存在但不相等,因此 $x=1$ 是第一类间断点中的跳跃间断点.
        \end{itemize}

        \item 当 $x \ne 2$ 时,$f(x) = \frac{(x-2)(x+2)}{x-2} = x+2$.\\
        $\lim\limits_{x \to 2} f(x) = \lim\limits_{x \to 2} (x+2) = 4$,但 $f(2) = 4$,函数值与极限值相等.\\
        因此函数在 $x=2$ 处连续,无间断点.
    \end{enumerate}
\end{solution}

\begin{exercise}[2.1.7]
    试确定 $a$, 使得函数 $f(x) = \begin{cases} \mathrm{e}^x, & x < 0, \\ a+x, & x \ges 0 \end{cases}$ 在 $x=0$ 处连续.
\end{exercise}

\begin{solution}
    由 $f(0) = a$, $\lim_{x \to 0^-} f(x) = \mathrm{e}^0 = 1$, $\lim_{x \to 0^+} f(x) = a + 0 = a$ 可知, 当 $a=1$ 时, $\lim_{x \to 0} f(x) = f(0)=1$, 即 $f(x)$ 在 $x=0$ 处连续.
\end{solution}

\begin{exercise}[2.1.8]
    证明: 函数 $f(x) = \begin{cases} \frac{\mathrm{e}^{1/x}-\mathrm{e}^{-1/x}}{\mathrm{e}^{1/x}+\mathrm{e}^{-1/x}}, & x \ne 0, \\ 1, & x=0 \end{cases}$, 在点 $0$ 处右连续, 但不左连续.
\end{exercise}

\begin{solution}
    因为 $\lim_{x \to 0^+} f(x) = \lim_{x \to 0^+} \frac{\mathrm{e}^{1/x}-\mathrm{e}^{-1/x}}{\mathrm{e}^{1/x}+\mathrm{e}^{-1/x}} = \lim_{x \to 0^+} \frac{1-\mathrm{e}^{-2/x}}{1+\mathrm{e}^{-2/x}} = 1 = f(0)$, 故 $f(x)$ 在点 $0$ 处右连续. 又因为 $\lim_{x \to 0^-} f(x) = \lim_{x \to 0^-} \frac{\mathrm{e}^{1/x}-\mathrm{e}^{-1/x}}{\mathrm{e}^{1/x}+\mathrm{e}^{-1/x}} = \lim_{x \to 0^-} \frac{\mathrm{e}^{2/x}-1}{\mathrm{e}^{2/x}+1} = -1 \ne f(0)$, 故 $f(x)$ 在点 $0$ 处不左连续.
\end{solution}

\begin{exercise}[2.1.9]
    证明: 对每个实数 $x$, $\lim_{n \to \infty} \frac{1+x}{1+x^{2n}}$ 存在. 将该极限值记为 $f(x)$, 试讨论函数 $f(x)$ 的连续性.
\end{exercise}

\begin{solution}
    \begin{align*}
        f(x)=\lim_{n \to \infty}\frac{1+x}{1+x^{2n}}&=(1+x)\lim_{n \to \infty}\frac{1}{1+(x^2)^n}
        =\begin{cases}
            1+x,&|x|<1;\\
            \frac{1+x}{2},&x=\pm1;\\
            0,&|x|>1.
        \end{cases}
    \end{align*}
    即\begin{align*}
        f(x)=\begin{cases}
            1+x,&-1<x<1;\\
            1,&x=1;\\
            0,&x>1\text{或}x\les -1.
        \end{cases}
    \end{align*}
    因此,\\
    $\lim_{x \to -1^-} f(x) = \lim_{x \to -1^+} f(x) = 0 = f(-1)$, 故 $f(x)$ 在 $x=-1$ 处连续;\\
     $\lim_{x \to 1^-} f(x) = 2 \ne f(1)$, 故 $f(x)$ 在 $x=1$ 处不连续; 对于其他点, $f(x)$ 均连续.\\
    综上所述, $f(x)$ 在 $(-\infty, 1) \cup (1, +\infty)$ 上连续, 在 $x=1$ 处不连续.
\end{solution}

\begin{exercise}[2.1.10]
    证明: 若函数 $f(x)$ 在点 $x_0$ 连续, 则存在一个正数 $\delta$, 使得函数 $f(x)$ 在区间 $(x_0-\delta, x_0+\delta)$ 上有界. (这一结果称为连续函数的局部有界性.)
\end{exercise}

\begin{solution}
    因为 $f(x)$ 在点 $x_0$ 连续, 故 $\lim_{x \to x_0} f(x) = f(x_0)$, 即 $\forall \varepsilon > 0$, $\exists \delta > 0$, 当 $|x-x_0| < \delta$ 时, 有 $|f(x)-f(x_0)| < \varepsilon$. \\
    取 $\varepsilon = 1$, 则当 $|x-x_0| < \delta$ 时, 有 $|f(x)-f(x_0)| < 1$, 即 $-1 < f(x)-f(x_0) < 1$, 故 $f(x_0)-1 < f(x) < f(x_0)+1$. \\
    因此, 当 $x \in (x_0-\delta, x_0+\delta)$ 时, 有 $|f(x)| \les \max\{|f(x_0)-1|, |f(x_0)+1|\}$, 即 $f(x)$ 在区间 $(x_0-\delta, x_0+\delta)$ 上有界.
\end{solution}

\begin{exercise}[2.1.11]
    证明: 若函数 $f(x)$ 在点 $x_0$ 连续, 且 $f(x_0) \ne 0$, 则存在一个正数 $\delta$, 使得函数 $f(x)$ 在区间 $(x_0-\delta, x_0+\delta)$ 上与 $f(x_0)$ 同号. (这一结果称为连续函数的局部保号性) 进一步, 存在某个正数 $\gamma$, 使得 $f(x)$ 在这一区间中满足 $|f(x)| \ges \gamma$.
\end{exercise}

\begin{solution}
    因为 $f(x)$ 在点 $x_0$ 连续, 故 $\lim_{x \to x_0} f(x) = f(x_0)$, 即 $\forall \varepsilon > 0$, $\exists \delta > 0$, 当 $|x-x_0| < \delta$ 时, 有 $|f(x)-f(x_0)| < \varepsilon$. \\
    取 $\varepsilon = \frac{|f(x_0)|}{2}$, 则当 $|x-x_0| < \delta$ 时, 有 $|f(x)-f(x_0)| < \frac{|f(x_0)|}{2}$,即 $-\frac{|f(x_0)|}{2} < f(x)-f(x_0) < \frac{|f(x_0)|}{2}$, 故 $f(x_0)-\frac{|f(x_0)|}{2} < f(x) < f(x_0)+\frac{|f(x_0)|}{2}$. \\
    因为 $f(x_0) \ne 0$, 故当 $x \in (x_0-\delta, x_0+\delta)$ 时, 有
    \begin{itemize}
        \item $f(x_0) > 0$时$$ f(x) > f(x_0) - \frac{f(x_0)}{2} = \frac{f(x_0)}{2} > 0. $$
        \item $f(x_0) < 0$时$$ f(x) < f(x_0) - \frac{f(x_0)}{2} = \frac{f(x_0)}{2} < 0. $$
    \end{itemize}
    
    因此, 当 $x \in (x_0-\delta, x_0+\delta)$ 时, 有 $f(x)$ 与 $f(x_0)$ 同号. 进一步地, 取 $\gamma = \frac{|f(x_0)|}{2}$, 则当 $x \in (x_0-\delta, x_0+\delta)$ 时, 有 $|f(x)| \ges \gamma$.
\end{solution}

\begin{exercise}[2.1.12]
    证明: 若 $\lim_{x \to x_0} g(x) = a \ne g(x_0)$ (从而 $x_0$ 为 $g(x)$ 的可去间断点), $f(u)$ 在 $u=a$ 处连续, 则
    $$ \lim_{x \to x_0} f(g(x)) = f\left(\lim_{x \to x_0} g(x)\right) = f(a). $$
    (这一结论对其他五种极限过程也成立.)
\end{exercise}

\begin{solution}
    由于$\lim_{u \to a} f(u) = f(a)$, 故$\forall \varepsilon > 0$, $\exists \delta_1 > 0$, 当 $|u-a| < \delta_1$ 时, 有 $|f(u)-f(a)| < \varepsilon$. \\
    又因为 $\lim_{x \to x_0} g(x) = a$, 故对于$ \delta_1 > 0$, $\exists \delta_2 > 0$, 当 $|x-x_0| < \delta_2$ 时, 有 $|g(x)-a| < \delta_1$.\\
    因此, 当 $|x-x_0| < \delta_2$ 时, 有 $|g(x)-a| < \delta_1$, 即 $|f(g(x))-f(a)| < \varepsilon$.\\
    综上所述, $\lim_{x \to x_0} f(g(x)) = f(a)$.
\end{solution}

\begin{exercise}[2.1.13]
    证明: 若函数 $u(x), v(x)$ 在 $x_0$ 处连续, 且 $u(x_0) > 0$, 则函数 $u(x)^{v(x)}$ 也在点 $x_0$ 处连续.
\end{exercise}

\begin{proof}
    利用$\e^x$在$\R$上连续,$\ln x$在$(0,+\infty)$上连续,以及复合函数的极限可交换性,有
    $$\lim_{x \to x_0} u(x)^{v(x)} = \lim_{x \to x_0} \e^{v(x) \ln u(x)} = \e^{\lim\limits_{x \to x_0} v(x) \ln u(x)} = \e^{v(x_0) \ln u(x_0)} = u(x_0)^{v(x_0)}.$$
\end{proof}

\begin{exercise}[2.1.14]
    设 $f(x)$ 在 $\mathbb{R}$ 上连续, 且对于任意 $x$ 有 $f(2x)=f(x)$. 求证 $f(x)$ 是常数.
\end{exercise}

\begin{solution}
    即证:$f(x) \equiv f(0)$. 对于任意点$x_0 \in \R$,考虑任意以$x_0$为极限的数列,$\{x_n\}$,则由连续性$$f(x_0)=\lim_{n \to \infty} f(x_n) = \lim_{n \to \infty} f \left( \frac{x_n}{2^n} \cdot 2^n \right) = \lim_{n \to \infty} f\left(\frac{x_n}{2^n}\right) = f(0),$$且$f(x_0) = f(0).$由于$x_0$的任意性,故$f(x) \equiv f(0)$.
\end{solution}

\begin{exercise}[2.1.15]
    设 $f(x)$ 在 $\mathbb{R}$ 上连续, 且对于任意 $x,y$ 有 $f(x+y)=f(x)+f(y)$. 求证 $f(x)=cx$, 其中 $c$ 是常数.
\end{exercise}

\begin{solution}
    \begin{enumerate}
        \item 由$f(0) = f(0+0) = f(0)+f(0)$可知,$f(0) = 0$.
        \item 对$n\in \N$,$$f(nx)=f((n-1)x+x)=f((n-1)x)+f(x)=\cdots = \underset{n\text{个}}{\underbrace{f(x)+f(x)+\cdots +f(x)}}=nf(x).$$
        即对任意自然数$n\in \N,f(nx)=nf(x)$.
        \item $$f(-x) + f(x) = f(0) = 0\Rightarrow f(-x) = -f(x)$$
        即对任意整数$k\in \Z,f(kx)=kf(x)$.
        \item 对$m,n\in \Z,n\neq 0$,$$f\left(\frac{m}{n}x\right) = mf\left(\frac{1}{n}x\right) = m\cdot\frac{1}{n}\cdot nf\left(\frac{1}{n}x\right)=\frac{m}{n}f\left(x\right)$$
        即对任意有理数$r=\frac{m}{n}\in \Q,f(rx)=rf(x)$.
        \item 对$x \in \R$,则存在有理数列$\{r_n\}$使得$\lim_{n \to \infty} r_n = x,$因此,$$\lim_{n \to \infty} f(r_n) = \lim_{n \to \infty} r_n f(1) = x f(1).$$
        由于$f$在$x$处连续,故$$f(x)=f(\lim_{n \to \infty} r_n ) =\lim_{n \to \infty} f(r_n) = x f(1).$$
        取$c=f(1)$,即有$f(x)=xf(1)=cx.$
    \end{enumerate}
\end{solution}

\begin{exercise}[2.1.16]
    当 $x \to 0$ 时, 用 $\sin x \sim x, \tan x \sim x$ 证明 $\arcsin x \sim x, \arctan x \sim x$; 用 $\ln(1+x) \sim x$ 证明 $(\mathrm{e}^x-1) \sim x$.

    (上述的等价无穷小, 是微积分中非常基本的事实.)
\end{exercise}

\begin{solution}
    \begin{enumerate}
        \item $f(x) = \begin{cases}
            \frac{\sin x}{x},&x\neq 0\\
            1,&x=0
        \end{cases},$ 则 $f(0)=\lim_{x \to 0} f(x) = 1$,即$f(x)$在$x=0$连续.\\ 
        设 $g(x) = \arcsin x$, 则 $\lim_{x \to 0} g(x) = 0$. \\
        因此$$\lim_{x \to 0} \frac{x}{\arcsin x } = \lim_{x \to 0} \frac{\sin  \arcsin x}{\arcsin x }= \lim_{x \to 0} f(g(x)) = f \left( \lim_{x \to 0} g(x) \right) = f(0) = 1,$$ 
        即 $\arcsin x \sim x$;
        \item $f(x) = \begin{cases}
            \frac{\tan x}{x},&x\neq 0\\
            1,&x=0
        \end{cases},$则 $f(0)=\lim_{x \to 0} f(x) = 1$,即$f(x)$在$x=0$连续.\\ 
        设 $g(x) = \arctan x$, 则 $\lim_{x \to 0} g(x) = 0$.\\
         因此$$\lim_{x \to 0} \frac{x}{\arctan x} = \lim_{x \to 0} \frac{\tan \arctan x}{\arctan x}=\lim_{x \to 0} f(g(x)) = f \left( \lim_{x \to 0} g(x) \right) = f(0) = 1,$$
         即 $\arctan x \sim x$;
        \item $f(x) = \begin{cases}
            \frac{\ln(1+x)}{x},&x\neq 0\\
            1,&x=0
        \end{cases}$, 则 $f(0)=\lim_{x \to 0} f(x) = 1$,即$f(x)$在$x=0$连续.\\ 
         设 $g(x) = \e^x - 1$, 则 $\lim_{x \to 0} g(x) = 0$.\\
        因此$$\lim_{x \to 0} \frac{x}{\e^x-1}=\lim_{x \to 0} \frac{\ln(1+\e^{x}-1)}{\e^x-1} = \lim_{x \to 0} f(g(x)) = f \left( \lim_{x \to 0} g(x) \right) = f(0) = 1, $$即 $(\mathrm{e}^x - 1) \sim x$.
    \end{enumerate}
\end{solution}

\begin{exercise}[2.1.17]
    求极限:
    \begin{tasks}[label=(\arabic*)](2)
        \task $\lim_{x \to 0} \frac{\sqrt{1+x+x^2}-1}{\sin 2x}$;
        \task $\lim_{x \to 0} \frac{\sqrt{1+x^2}-1}{1-\cos x}$;
        \task $\lim_{x \to 0} \frac{(\sqrt[10]{1+\tan x}-1)(\sqrt{1+x}-1)}{2x\sin x}$;
        \task $\lim_{x \to 0} \frac{x \cdot \arcsin(\sin x)}{1-\cos x}$;
        \task $\lim_{x \to 0} \frac{1-\cos(1-\cos x)}{x^4}$;
        \task $\lim_{x \to -\infty} x(\sqrt{x^2+100}+x)$;
        \task $\lim_{x \to +\infty} (\sin\sqrt{x+1}-\sin\sqrt{x})$;
        \task $\lim_{x \to \infty} \sqrt{2-\frac{\sin x}{x}}$.
    \end{tasks}
\end{exercise}

\begin{solution}
    \begin{enumerate}
        \item $\sqrt{1+x+x^2}-1 \sim \frac{1}{2}(x+x^2) , \sin 2x \sim 2x,(x\to0)$, 故 $$\lim_{x \to 0} \frac{\sqrt{1+x+x^2}-1}{\sin 2x} = \lim_{x \to 0} \frac{\frac{1}{2}(x+x^2)}{2x} = \frac{1}{4};$$
        \item $\sqrt{1+x^2}-1 \sim \frac{1}{2}x^2 , 1-\cos x \sim \frac{1}{2}x^2,(x\to0)$, 故 $$\lim_{x \to 0} \frac{\sqrt{1+x^2}-1}{1-\cos x} = \lim_{x \to 0} \frac{\frac{1}{2}x^2}{\frac{1}{2}x^2} = 1;$$
        \item $\sqrt[10]{1+\tan x}-1 \sim \frac{1}{10}\tan x \sim \frac{1}{10}x , \sqrt{1+x}-1 \sim \frac{1}{2}x , 2x\sin x \sim 2x^2,(x\to0)$, 故 $$\lim_{x \to 0} \frac{(\sqrt[10]{1+\tan x}-1)(\sqrt{1+x}-1)}{2x\sin x} = \lim_{x \to 0} \frac{\frac{1}{10}x \cdot \frac{1}{2}x}{2x^2} = \frac{1}{40};$$
        \item $x \cdot \arcsin(\sin x) \sim x^2 , 1-\cos x \sim \frac{1}{2}x^2,(x\to0)$, 故 $$\lim_{x \to 0} \frac{x \cdot \arcsin(\sin x)}{1-\cos x} = \lim_{x \to 0} \frac{x^2}{\frac{1}{2}x^2} = 2;$$
        \item $1-\cos(1-\cos x) \sim \frac{1}{2}(1-\cos x)^2 \sim \frac{1}{2} \left( \frac{1}{2}x^2 \right)^2 = \frac{1}{8}x^4,(x\to0)$, 故 $$\lim_{x \to 0} \frac{1-\cos(1-\cos x)}{x^4} = \lim_{x \to 0} \frac{\frac{1}{8}x^4}{x^4} = \frac{1}{8};$$
        \item 令$y=\frac{1}{x}$,则$y\to 0^+$,\\
        并且$x=-\frac{1}{y}$,即有$x(\sqrt{x^2+100}+x)=-\frac{1}{y}\left(\sqrt{\frac{1}{y^2}+100}-\frac{1}{y}\right)=-\frac{\sqrt{1+100y^2}-1}{y^2}$\\
        而$\sqrt{1+100y^2}-1\sim \frac{1}{2}\cdot 100y^2=50y^2,(y\to 0),$故
        \begin{align*}
            \lim_{x \to -\infty}x(\sqrt{x^2+100}+x) =\lim_{y \to 0^+}-\frac{\sqrt{1+100y^2}-1}{y^2}=-\lim_{y \to 0^+}\frac{50y^2}{y^2}=-50
        \end{align*} 
        \item \begin{align*}
            \left| \sin \sqrt{x+1}-\sin\sqrt{x} \right| &=\left| 2 \cos \frac{\sqrt{x+1}+\sqrt{x}}{2} \sin \frac{\sqrt{x+1}-\sqrt{x}}{2} \right|  \\
            &=2 \left|\cos \frac{\sqrt{x+1}+\sqrt{x}}{2}\sin \frac{1}{2\left(\sqrt{x+1}+\sqrt{x}\right)}\right|\\
            &\les 2\cdot1\cdot  \frac{1}{2\left(\sqrt{x+1}+\sqrt{x}\right)}=\frac{1}{\sqrt{x+1}+\sqrt{x}}
        \end{align*}
         由$\lim_{x \to +\infty} \frac{1}{\sqrt{x+1}+\sqrt{x}} = 0$可知,$\lim_{x \to +\infty} (\sin\sqrt{x+1}-\sin\sqrt{x}) = 0$;
        \item $$\lim_{x \to \infty} \sqrt{2-\frac{\sin x}{x}} = \sqrt{2 - \lim_{x \to \infty} \frac{\sin x}{x}} = \sqrt{2 - 0} = \sqrt{2}.$$
    \end{enumerate}
\end{solution}

% \begin{exercise}[2.1.18]
%     函数 $\sinh x = \frac{\mathrm{e}^x - \mathrm{e}^{-x}}{2}$ 与 $\cosh x = \frac{\mathrm{e}^x + \mathrm{e}^{-x}}{2}$ 分别称为双曲正弦与双曲余弦 (统称为双曲函数), 它们均在定义域 $(-\infty, +\infty)$ 上连续. 证明以下各题. (可与三角函数的性质作比较.)
%     \begin{tasks}[label=(\arabic*)](2)
%         \task $\sinh(-x) = -\sinh x, \quad \cosh(-x) = \cosh x$;
%         \task $\cosh^2 x - \sinh^2 x = 1$;
%         \task $\sinh 2x = 2\sinh x \cosh x$;
%         \task $\cosh 2x = \sinh^2 x + \cosh^2 x$;
%         \task $\sinh(x \pm y) = \sinh x \cosh y \pm \cosh x \sinh y$;
%         \task $\cosh(x \pm y) = \cosh x \cosh y \pm \sinh x \sinh y$.
%     \end{tasks}
% \end{exercise}

% \begin{solution}
%     \begin{enumerate}
%         \item $\sinh(-x) = \frac{\mathrm{e}^{-x} - \mathrm{e}^x}{2} = -\sinh x$, $\cosh(-x) = \frac{\mathrm{e}^{-x} + \mathrm{e}^x}{2} = \cosh x$;
%         \item $\cosh^2 x - \sinh^2 x = \left( \frac{\mathrm{e}^x + \mathrm{e}^{-x}}{2} \right)^2 - \left( \frac{\mathrm{e}^x - \mathrm{e}^{-x}}{2} \right)^2 = \frac{\mathrm{e}^{2x} + 2 + \mathrm{e}^{-2x}}{4} - \frac{\mathrm{e}^{2x} - 2 + \mathrm{e}^{-2x}}{4} = 1$;
%         \item $\sinh 2x = \frac{\mathrm{e}^{2x} - \mathrm{e}^{-2x}}{2} = \frac{(\mathrm{e}^x)^2 - (\mathrm{e}^{-x})^2}{2} = \frac{(\mathrm{e}^x - \mathrm{e}^{-x})(\mathrm{e}^x + \mathrm{e}^{-x})}{2} = 2\sinh x \cosh x$;
%         \item $\cosh 2x = \frac{\mathrm{e}^{2x} + \mathrm{e}^{-2x}}{2} = \frac{(\mathrm{e}^x)^2 + (\mathrm{e}^{-x})^2}{2} = \frac{(\mathrm{e}^x - \mathrm{e}^{-x})^2 + (\mathrm{e}^x + \mathrm{e}^{-x})^2}{4} = \sinh^2 x + \cosh^2 x$;
%         \item $\sinh(x \pm y) = \frac{\mathrm{e}^{x \pm y} - \mathrm{e}^{-(x \pm y)}}{2} =
%                   \frac{\mathrm{e}^x \mathrm{e}^{\pm y} - \mathrm{e}^{-x} \mathrm{e}^{\mp y}}{2} = \frac{\mathrm{e}^x \mathrm{e}^{\pm y} - \mathrm{e}^{-x} \mathrm{e}^{\pm y} + \mathrm{e}^{-x} \mathrm{e}^{\pm y} - \mathrm{e}^{-x} \mathrm{e}^{\mp y}}{2} =
%                   \left( \frac{\mathrm{e}^x - \mathrm{e}^{-x}}{2} \right) \left( \frac{\mathrm{e}^{\pm y} + \mathrm{e}^{\mp y}}{2} \right) + \left( \frac{\mathrm{e}^x + \mathrm{e}^{-x}}{2} \right) \left( \frac{\mathrm{e}^{\pm y} - \mathrm{e}^{\mp y}}{2} \right) =
%                   \sinh x \cosh y \pm \cosh x \sinh y$;
%         \item $\cosh(x \pm y) =
%                   \frac{\mathrm{e}^{x \pm y} + \mathrm{e}^{-(x \pm y)}}{2} =\frac{\mathrm{e}^x \mathrm{e}^{\pm y} + \mathrm{e}^{-x} \mathrm{e}^{\mp y}}{2}
%                   =\frac{\mathrm{e}^x \mathrm{e}^{\pm y} + \mathrm{e}^{-x} \mathrm{e}^{\pm y} - \mathrm{e}^{-x} \mathrm{e}^{\pm y} + \mathrm{e}^{-x} \mathrm{e}^{\mp y}}{2}
%                   = \left( \frac{\mathrm{e}^x + \mathrm{e}^{-x}}{2} \right) \left( \frac{\mathrm{e}^{\pm y} + \mathrm{e}^{\mp y}}{2} \right) + \left( \frac{\mathrm{e}^x - \mathrm{e}^{-x}}{2} \right) \left( \frac{\mathrm{e}^{\pm y} - \mathrm{e}^{\mp y}}{2} \right)
%                   = \cosh x \cosh y \pm \sinh x \sinh y$.
%     \end{enumerate}
% \end{solution}

\newpage