\section*{习题 1.3}

\begin{exercise}[1.3.1]\label{exe:1.3.1}
    按定义证明:
    \begin{tasks}[label=(\arabic*)](2)
        \task $\lim_{x \to -\infty} a^x = 0, (a > 1)$;
        \task $\lim_{x \to \infty} \frac{x-1}{x+1} = 1$;
        \task $\lim_{x \to -1} \frac{x^2-1}{x^2+x} = 2$;
        \task $\lim_{x \to 0^+} x^{1/q} = 0$ ($q$ 为正整数).
    \end{tasks}
\end{exercise}

\begin{solution}
    \begin{enumerate}[(1)]
        \item 对$\forall \ve > 0$,取$M = \log_a \ve$,则当$x < M$时, $|a^x - 0| = a^x < a^M = \ve$.
        \item 对$\forall \ve > 0$,取$M = \frac{2}{\ve}+1$,则当$|x| > \max\{M,1\}$时, $\left| \frac{x-1}{x+1} - 1 \right| = \left| \frac{-2}{x+1} \right| \les \frac{2}{|x|-1} <  \ve$.
        \item 对$\forall \ve > 0$,取$\delta = \min \left\{ \frac{1}{2}, \frac{\ve}{2} \right\}$,则当$0<|x+1| < \delta$时, $\left| \frac{x^2-1}{x^2+x} - 2 \right| = \left| \frac{-x^2 - 2x - 1}{x^2 + x} \right| = \left| \frac{x+1}{x} \right| < \frac{\delta}{1 / 2} \les  \ve$.
        \item 对$\forall \ve > 0$,取$\delta = \ve^q$,则当$0 < x < \delta$时, $|x^{1/q} - 0| = x^{1/q} < \delta^{1/q} = \ve$.
    \end{enumerate}
\end{solution}

\begin{exercise}[1.3.2]
    求下列极限:
    \begin{tasks}[label=(\arabic*)](2)
        \task $\lim_{x \to 1} \left(x^5 - 5x + 2 + \frac{1}{x}\right)$;
        \task $\lim_{x \to 1} \frac{x^n - 1}{x - 1}$ ($n$ 为正整数);
        \task $\lim_{x \to 1} \frac{x^2-1}{2x^2-x-1}$;
        \task $\lim_{x \to -\infty} \frac{(3x+6)^{70}(8x-5)^{20}}{(5x-1)^{90}}$.
    \end{tasks}
\end{exercise}

\begin{enumerate}[(1)]
    \item 由四则运算的极限可知, $$\lim_{x \to 1} \left(x^5 - 5x + 2 + \frac{1}{x}\right) = 1 - 5 + 2 + 1 = -1.$$
    \item $x^n - 1 = (x-1)(x^{n-1} + x^{n-2} + \cdots + x + 1)$,因此$$\lim_{x \to 1} \frac{x^n - 1}{x - 1} = \lim_{x \to 1} (x^{n-1} + x^{n-2} + \cdots + x + 1) = \sum_{k=0}^{n-1} 1 = n.$$
          请注意,这里$n$是常数,因此可以交换这$n$个极限与求和的顺序.
    \item $$\lim_{x \to 1} \frac{x^2-1}{2x^2-x-1} = \lim_{x \to 1} \frac{(x-1)(x+1)}{(x-1)(2x+1)} = \lim_{x \to 1} \frac{x+1}{2x+1} = \frac{2}{3}.$$
    \item $$\lim_{x \to -\infty} \frac{ \left( 3 + \dfrac{6}{x} \right)^{70} (8 - \dfrac{5}{x})^{20} }{ \left( 5 - \dfrac{1}{x} \right)^{90} } = \frac{ \left( 3 + \displaystyle \lim_{x \to -\infty} \dfrac{6}{x} \right)^{70} \left( 8 - \displaystyle  \lim_{x \to -\infty} \dfrac{5}{x} \right)^{20} }{ \left( 5 - \displaystyle  \lim_{x \to -\infty} \dfrac{1}{x} \right)^{90} } = \frac{3^{70} \cdot 8^{20}}{5^{90}}$$
          事实上, $\frac{3^{70} \cdot 8^{20}}{5^{90}} = \left( \frac{3^7 \cdot 2^6}{5^9} \right)^{10} =0.0000000000035726229189858259136514568727612$

          $46392142557535369616400676018940233797678923776.$
\end{enumerate}

\begin{exercise}[1.3.3]\label{exe:1.3.3}
    证明下列极限不存在:
    \begin{tasks}[label=(\arabic*)](2)
        \task $\lim_{x \to +\infty} \sin x$;
        \task $\lim_{x \to 0} \frac{|x|}{x}$.
    \end{tasks}
\end{exercise}

\begin{enumerate}[(1)]
    \item 用Cauchy收敛原理.对$\ve = \frac{1}{2}$,任取$M > 0$,总
          总存在$k = \llceil M / \pi \rrceil$,使得$x_1 = \left(k+\frac{1}{2}\right)\pi> M, x_2 = (k + 1)\pi > M (k \in \mathbb{N}^*)$,使得$|\sin x_1 - \sin x_2| = 1 > \ve$.因此极限不存在.
    \item 考虑两个单边极限, $$\lim_{x \to 0^+} \frac{|x|}{x} = \lim_{x \to 0^+} 1 = 1, \quad \lim_{x \to 0^-} \frac{|x|}{x} = \lim_{x \to 0^-} -1 = -1.$$
          而极限存在的充要条件是两个单边极限存在且相等,因此极限不存在.
\end{enumerate}

\begin{exercise}[1.3.4]
    设函数 $f(x)$ 在正无穷大处的极限为 $l$, 则对于任意趋于正无穷大的数列 $\{a_n\}$, 有 $\lim_{n \to \infty} f(a_n) = l$. 特别地 $\lim_{n \to \infty} f(n) = l$.
\end{exercise}

\begin{solution}
    [(无穷版本的Heine定理) \ 解]对$\forall \ve > 0$,由$\lim_{x \to \infty} f(x) = l$,存在$M > 0$,使得当$x > M$时, $|f(x) - l| < \ve$.又由$\lim_{n \to \infty} a_n = +\infty$,存在$N\in \mathbb{N}^*$,使得当$n > N$时, $a_n > M$.因此当$n > N$时, $|f(a_n) - l| < \ve$.由此可知$\lim_{n \to \infty} f(a_n) = l$.特别地,取$a_n = n$,则$\lim_{n \to \infty} f(n) = l$.
\end{solution}

\begin{exercise}[1.3.5]
    讨论下列函数在 $x=0$ 处的极限.
    \begin{tasks}[label=(\arabic*)](2)
        \task $f(x) = [x]$;
        \task $f(x) = \operatorname{sgn} x$;
        \task $f(x) = \begin{cases} 2^x, & x > 0; \\ 0, & x = 0; \\ 1+x^2, & x < 0. \end{cases}$
        \task $f(x) = \begin{cases} \cos\frac{1}{x}, & x > 0; \\ x, & x \les 0. \end{cases}$
    \end{tasks}
\end{exercise}

\begin{solution}
    \begin{remark}
        教材中的符号$[x]$表示$x$的整数部分,即不大于$x$的最大整数.本题中,我们沿用此符号.其他地方,我们使用$\lfloor x \rfloor$表示对$x$向下取整,使用$\lceil x \rceil$表示对$x$向上取整.
    \end{remark}
    \begin{enumerate}[(1)]
        \item $\lim_{x \to 0^+} [x] = 0, \quad \lim_{x \to 0^-} [x] = -1$.因此极限不存在.
        \item $\lim_{x \to 0^+} \operatorname{sgn} x = 1, \quad \lim_{x \to 0^-} \operatorname{sgn} x = -1$.左右极限均存在,但不相等,因此极限不存在.
        \item $\lim_{x \to 0^+} 2^x = 1, \quad \lim_{x \to 0^-} (1+x^2) = 1$.因此极限存在,且$\lim_{x \to 0} f(x) = 1$.
        \item $\lim_{x \to 0^+} \cos\frac{1}{x}$不存在,因此右极限不存在.左极限$\lim_{x \to 0^-} x = 0$.函数在$x=0$处的极限不存在. \begin{remark}
                  $\lim_{x \to 0^+} \cos \frac{1}{x}$的极限过程等同于考虑$\lim_{x \to + \infty} \cos x$,而该极限不存在(与\hyperref[exe:1.3.3]{习题1.3.3(1)}同理).
              \end{remark}
    \end{enumerate}
\end{solution}

\begin{exercise}[1.3.6]
    求 $\lim_{n \to \infty} \cos\frac{x}{2} \cos\frac{x}{2^2} \cdots \cos\frac{x}{2^n}$.
\end{exercise}

\begin{solution}
    \begin{enumerate}[(1)]
        \item 当$\forall m \in \mathbb{N}^*,\sin \frac{x}{2^m} \ne 0$时,二倍角公式变形可得$\cos y = \frac{\sin 2y}{2 \sin y},\text{当}\sin y\neq 0$,反复利用可知$$\cos\frac{x}{2} \cos\frac{x}{2^2} \cdots \cos\frac{x}{2^n} = \frac{\sin x}{2 \sin \frac{x}{2}} \cdot \frac{\sin \frac{x}{2}}{2 \sin \frac{x}{2^2}} \cdots \frac{\sin \frac{x}{2^{n-1}}}{2 \sin \frac{x}{2^n}} = \frac{\sin x}{2^n \sin \frac{x}{2^n}}.$$
              因此$$\lim_{n \to \infty} \cos\frac{x}{2} \cos\frac{x}{2^2} \cdots \cos\frac{x}{2^n} = \lim_{n \to \infty} \frac{\sin x}{2^n \sin \frac{x}{2^n}} = \frac{\sin x}{x} \cdot \lim_{n \to \infty} \frac{\frac{x}{2^n}}{\sin\frac{x}{2^n}} = \frac{\sin x}{x}.$$
        \item 若存在$m_0 \ges 1$, $\sin \frac{x}{2^{m}} = 0$,有$\frac{x}{2^{m_0}}= k\pi,x = 2^{m_0}k \pi, k \in \mathbb{Z}$. 
        自然的推论是$\forall m\les m_0$,有$ \sin \frac{x}{2^m}=\sin (2^{m_0-m}k \pi)  =0$.
        
        此时根据是否存在最大的$m_0$,使得$\sin \frac{x}{2^{m_0}} = 0$可以分成两种情况:
              \begin{enumerate}
                  \item $x = 0$,则$\forall m\in \mathbb{N}^*$,有$\cos\frac{x}{2^m}= 1$,因此$\lim_{n \to \infty} \cos\frac{x}{2} \cos\frac{x}{2^2} \cdots \cos\frac{x}{2^n} = 1$;
                  \item $x \ne 0 \Leftrightarrow \exists m_0,\text{s.t.}  \sin \frac{x}{2^{m_0}} = 0,\sin \frac{x}{2^{m_0+1}} \neq 0, $也就是存在最大的$m_0$.
                  
                  因此可以得到$x = 2^{m_0}k \pi, k =2l+1,l\in\Z  $(如果$k$是偶数,那么与$\sin \frac{x}{2^{m_0+1}}=\sin \frac{k\pi}{2}  \neq 0$矛盾).
                  
                  此时$\cos \frac{x}{2^{m_0+1}} = \cos\frac{k\pi}{2} =\cos\left(l+\frac{1}{2}\right)\pi = 0$,因此$\lim_{n \to \infty} \cos\frac{x}{2} \cos\frac{x}{2^2} \cdots \cos\frac{x}{2^n}=0$.
                  
                  不过又由于$\sin x=0$同样成立,并且$x\neq 0$,因此可以把结果合并进$\frac{\sin x}{x}$.
              \end{enumerate}
    \end{enumerate}
    综上所述, $$\lim_{n \to \infty} \cos\frac{x}{2} \cos\frac{x}{2^2} \cdots \cos\frac{x}{2^n} = \begin{cases}
            \frac{\sin x}{x}, & x \ne 0; \\
            1,                & x = 0.
        \end{cases}
    $$

\end{solution}

\begin{exercise}[1.3.7]
    求证: $\lim_{n \to \infty} \left(\sin\frac{\alpha}{n^2} + \sin\frac{2\alpha}{n^2} + \dots + \sin\frac{n\alpha}{n^2}\right) = \frac{\alpha}{2}$.
\end{exercise}

\begin{solution}
    我们先证明如下事实:
    $$
        \sum_{k=1}^n \sin k \theta =\begin{cases}
            \frac{\sin\frac{n+1}{2}\theta\sin\frac{n}{2}\theta}{\sin \frac{\theta}{2}}, & \sin \frac{\theta}{2} \ne 0; \\
            0, & \sin \frac{\theta}{2} = 0.
        \end{cases}
    $$
    我们利用积化和差$$\sin \frac{\theta}{2}\sin k\theta = \frac{1}{2}\left(\cos\left(k-\frac{1}{2}\right)\theta-\cos\left(k+\frac{1}{2}\right)\theta\right)$$
    \begin{align*}
        \sin \frac{\theta}{2}\left(\sum_{k=1}^n \sin k \theta \right)=&\sum_{k=1}^n \frac{1}{2}\left(\cos\left(k-\frac{1}{2}\right)\theta-\cos\left(k+\frac{1}{2}\right)\theta\right) \\
        =&\frac{1}{2}\left(\cos \frac{\theta}{2} -\cos \frac{3\theta}{2}+\cos \frac{3\theta}{2} -\cos \frac{5\theta}{2}+\cdots+ \cos\left(n-\frac{1}{2}\right)\theta-\cos\left(n+\frac{1}{2}\right)\theta\right) \\
        =&\frac{1}{2}\left(\cos \frac{\theta}{2} -\cos\left(n+\frac{1}{2}\right)\theta\right) \\
        =&\sin\frac{n+1}{2}\theta\sin\frac{n}{2}\theta.
    \end{align*}
    因此,当$ \sin \frac{\theta}{2}\neq 0$自然有$$
        \sum_{k=1}^n \sin k \theta =
            \frac{\sin\frac{n+1}{2}\theta\sin\frac{n}{2}\theta}{\sin \frac{\theta}{2}}.
    $$
    另一种情况是显然的,每一项都为$0$.
        
    回到本题,首先,如果$\alpha\neq 0$,那么这意味着存在充分大的$N$使得$n>N$, $0<\left|\frac{\alpha}{n^2}\right|<\pi$,此时, $\sin \frac{\alpha}{2n^2} \neq 0$.因此$n>N$时,
    \begin{align*}
        \sum_{k=1}^n \sin k \frac{\alpha}{n^2} =  \frac{\sin\frac{(n+1)\alpha}{2n^2}\sin\frac{n\alpha}{2n^2}}{\sin \frac{\alpha}{2n^2}}
    \end{align*}
    考虑$\sin x \sim x , (x \to 0)$,于是
    \begin{align*}
        \lim_{n\to\infty}\sum_{k=1}^n \sin k \frac{\alpha}{n^2} 
        =\lim_{n \to \infty} \frac{\sin\frac{(n+1)\alpha}{2n^2}\sin\frac{n\alpha}{2n^2}}{\sin \frac{\alpha}{2n^2}}
        =\lim_{n \to \infty} \frac{\frac{(n+1)\alpha}{2n^2}
        \cdot\frac{n\alpha}{2n^2}}{\frac{\alpha}{2n^2}}  =\lim_{n \to \infty} \frac{(n+1)\alpha}{2n}  = \frac{\alpha}{2}.
    \end{align*}
    如果$\alpha = 0$,那么每一项都为$0$,极限自然为$0=\frac{\alpha}{2}$.
    
    综上所述, $$\lim_{n \to \infty} \left(\sin\frac{\alpha}{n^2} + \sin\frac{2\alpha}{n^2} + \dots + \sin\frac{n\alpha}{n^2}\right) = \frac{\alpha}{2}.$$

\end{solution}

\begin{exercise}[1.3.8]
    证明: 若 $\lim_{x \to \infty} f(x) = l$, 则 $\lim_{x \to 0} f\left(\frac{1}{x}\right)=l$, 反之亦正确. 叙述并证明, 当 $x \to +\infty$ 及 $x \to -\infty$ 时类似的结论. (应用本题结论, 可将极限过程为 $x \to \infty$ 的问题化为 $x \to 0$ 处理, 或者反过来. 例如, 我们有 $\lim_{x \to 0} (1+x)^{1/x} = \mathrm{e}$.)
\end{exercise}

\begin{solution}
    我们先给出这条命题的完整表述:
    \begin{proposition*}
        \begin{enumerate}[(1)]
            \item 若 $\lim_{x \to \infty} f(x) = l$, 则 $\lim_{x \to 0^+} f\left(\frac{1}{x}\right)=l$, 反之亦正确;
            \item 若 $\lim_{x \to +\infty} f(x) = l$, 则 $\lim_{x \to 0^+} f\left(\frac{1}{x}\right)=l$, 反之亦正确;
            \item 若 $\lim_{x \to -\infty} f(x) = l$, 则 $\lim_{x \to 0^-} f\left(\frac{1}{x}\right)=l$, 反之亦正确;
        \end{enumerate}
    \end{proposition*}
    \noindent
    \textbf{证明:}
    \begin{enumerate}[(1)]
        \item 由Heine定理, $\lim_{x \to \infty} f(x) = l \Rightarrow \forall \{x_n\}, $若$\lim_{n \to \infty} x_n = \infty$则$\lim_{n \to \infty} f(x_n) = l$. $\Rightarrow$ $\forall \{ y_n \}, $若$\lim_{n \to \infty} y_n = 0^+$,则$\lim_{n \to \infty} \frac{1}{y_n} = \infty$,则$\lim_{n \to \infty} f\left( \frac{1}{y_n} \right) = l$.由Heine定理可知$\lim_{x \to 0^+} f\left(\frac{1}{x}\right)=l$.

              反之,若$\lim_{x \to 0^+} f\left(\frac{1}{x}\right)=l$,由Heine定理, $\forall \{y_n\}, $若$\lim_{n \to \infty} y_n = 0^+$,则$\lim_{n \to \infty} f\left( \frac{1}{y_n} \right) = l$. $\Rightarrow$ $\forall \{x_n\}, $若$\lim_{n \to \infty} x_n = \infty$,则$\lim_{n \to \infty} \frac{1}{x_n} = 0^+$,则$\lim_{n \to \infty} f(x_n) = l$.由Heine定理可知$\lim_{x \to \infty} f(x) = l$.
        \item 由Heine定理, $\lim_{x \to +\infty} f(x) = l \Rightarrow \forall \{x_n\}, $若$\lim_{n \to \infty} x_n = +\infty$则$\lim_{n \to \infty} f(x_n) = l$. $\Rightarrow$ $\forall \{ y_n \}, $若$\lim_{n \to \infty} y_n = 0^+$,则$\lim_{n \to \infty} \frac{1}{y_n} = +\infty$,则$\lim_{n \to \infty} f\left( \frac{1}{y_n} \right) = l$.由Heine定理可知$\lim_{x \to 0^+} f\left(\frac{1}{x}\right)=l$.

              反之,若$\lim_{x \to 0^+} f\left(\frac{1}{x}\right)=l$,由Heine定理, $\forall \{y_n\}, $若$\lim_{n \to \infty} y_n = 0^+$,则$\lim_{n \to \infty} f\left( \frac{1}{y_n} \right) = l$. $\Rightarrow$ $\forall \{x_n\}, $若$\lim_{n \to \infty} x_n = +\infty$,则$\lim_{n \to \infty} \frac{1}{x_n} = 0^+$,则$\lim_{n \to \infty} f(x_n) = l$.由Heine定理可知$\lim_{x \to +\infty} f(x) = l$.

        \item 由Heine定理, $\lim_{x \to -\infty} f(x) = l \Rightarrow \forall \{x_n\}, $若$\lim_{n \to \infty} x_n = -\infty$则$\lim_{n \to \infty} f(x_n) = l$. $\Rightarrow$ $\forall \{ y_n \}, $若$\lim_{n \to \infty} y_n = 0^-$,则$\lim_{n \to \infty} \frac{1}{y_n} = -\infty$,则$\lim_{n \to \infty} f\left( \frac{1}{y_n} \right) = l$.由Heine定理可知$\lim_{x \to 0^-} f\left(\frac{1}{x}\right)=l$.

              反之,若$\lim_{x \to 0^-} f\left(\frac{1}{x}\right)=l$,由Heine定理, $\forall \{y_n\}, $若$\lim_{n \to \infty} y_n = 0^-$,则$\lim_{n \to \infty} f\left( \frac{1}{y_n} \right) = l$. $\Rightarrow$ $\forall \{x_n\}, $若$\lim_{n \to \infty} x_n = -\infty$,则$\lim_{n \to \infty} \frac{1}{x_n} = 0^-$,则$\lim_{n \to \infty} f(x_n) = l$.由Heine定理可知$\lim_{x \to -\infty} f(x) = l$.
    \end{enumerate}
\end{solution}

\begin{exercise}[1.3.9]
    求下列极限:
    \begin{tasks}[label=(\arabic*)](2)
        \task $\lim_{x \to 0} \frac{\tan 2x}{\sin 5x}$;
        \task $\lim_{x \to 0} \frac{\cos x - \cos 3x}{x^2}$;
        \task $\lim_{x \to +\infty} \left(\frac{x+1}{2x-1}\right)^x$;
        \task $\lim_{x \to \infty} \left(\frac{x^2+1}{x^2-1}\right)^{x^2}$.
    \end{tasks}
\end{exercise}

\begin{solution}
    \begin{enumerate}[(1)]
        \item $$\lim_{x \to 0} \frac{\tan 2x}{\sin 5x} = \lim_{x \to 0} \frac{\tan 2x}{2x} \cdot \frac{5x}{\sin 5x} \cdot \frac{2}{5} = 1 \cdot 1 \cdot \frac{2}{5} = \frac{2}{5}.$$
        \item 由和差化积, $$ \cos x - \cos 3x = 2 \sin 2x \sin x, $$因此$$\lim_{x \to 0} \frac{\cos x - \cos 3x}{x^2} = \lim_{x \to 0} \frac{2 \sin 2x \sin x}{x^2} = \lim_{x \to 0} \frac{\sin 2x}{2x} \cdot \frac{\sin x}{x} \cdot 4 = 1 \cdot 1 \cdot 4 = 4.$$
        \item 当$x>\frac{7}{2}$时,有$0<\frac{x+1}{2x-1}<\frac{3}{4}$恒成立,因此$$0\les\left(\frac{x+1}{2x-1}\right)^x\les\left(\frac{3}{4}\right)^x$$
        又由于$\lim_{x \to +\infty} \left(\frac{3}{4}\right)^x = 0$,由夹逼定理可知$$\lim_{x \to +\infty} \left(\frac{x+1}{2x-1}\right)^x = 0.$$
        \item $$
                  \lim_{x \to \infty} \left(1 + \frac{2}{x^2 - 1}\right)^{\frac{x^2 - 1}{2} \cdot \frac{2}{x^2 - 1} x^2} = \mathrm{e}^{\lim_{x \to \infty} \frac{2}{x^2 - 1} x^2} = \mathrm{e}^2
              $$
    \end{enumerate}
\end{solution}

\begin{exercise}[1.3.10]
    求下列极限.
    \begin{tasks}[label=(\arabic*)](2)
        \task $\lim_{x \to +\infty} \frac{\arctan x}{x}$;
        \task $\lim_{x \to 0} x^2 \sin\frac{1}{x}$;
        \task $\lim_{x \to 2} \frac{x^3-2x^2}{x-2}$;
        \task $\lim_{x \to \infty} (2x^2 - x + 1)$.
    \end{tasks}
\end{exercise}

\begin{solution}
    \begin{enumerate}[(1)]
        \item $\arctan x$在$x \to +\infty$时有界,而$x \to +\infty$时无界,因此$$\lim_{x \to +\infty} \frac{\arctan x}{x} = 0.$$具体而言, $$\lim_{x \to + \infty} \arctan x = \frac{\pi}{2} , \ \lim_{x \to -\infty} \arctan x = -\frac{\pi}{2}.$$
        \item 由夹逼定理, $$-x^2 \les x^2 \sin \frac{1}{x} \les x^2, $$且$\lim_{x \to 0} -x^2 = \lim_{x \to 0} x^2 = 0$,因此$$\lim_{x \to 0} x^2 \sin\frac{1}{x} = 0.$$
        \item $$\lim_{x \to 2} \frac{x^3-2x^2}{x-2} = \lim_{x \to 2} \frac{(x-2)x^2}{x-2} = \lim_{x \to 2}x^2  = 4.$$
        \item 证明: $$\lim_{x \to \infty} (2x^2 - x + 1) = +\infty.$$
              由$2x^2 - x + 1 = x^2 + (x - 1/2)^2 + 3/4 > x^2$,因此对$\forall M > 0$,取$N = \sqrt{M}$,则当$x > N$时, $2x^2 - x + 1 > x^2 > N^2 = M$.由此可知$$\lim_{x \to \infty} (2x^2 - x + 1) = +\infty.$$
    \end{enumerate}
\end{solution}

\begin{exercise}[1.3.11]
    按定义证明.
    \begin{tasks}[label=(\arabic*)](2)
        \task $\lim_{x \to +\infty} \log_a x = +\infty, (a > 1)$;
        \task $\lim_{x \to 0^+} \log_a x = -\infty, (a > 1)$;
        \task $\lim_{x \to \frac{\pi}{2}{}^- }\tan x = +\infty$;
        \task $\lim_{x \to 0^+} \mathrm{e}^{1/x} = +\infty$.
    \end{tasks}
\end{exercise}

\begin{solution}
    \begin{enumerate}[(1)]
        \item 对$\forall M > 0$,取$N = a^M$,则当$x > N$时, $\log_a x > \log_a N = M$.
        \item 对$\forall M < 0$,取$\delta = a^M$,则当$0 < x < \delta$时, $\log_a x < \log_a \delta = M$.
        \item 对$\forall M > 0$,取$\delta = \frac{\pi}{2} - \arctan M$,则当$\frac{\pi}{2} - \delta < x < \frac{\pi}{2}$时, $\tan x > \tan(\frac{\pi}{2} - \delta) = M$.
        \item 对$\forall M > 0$,取$\delta = \frac{1}{\ln M}$,则当$0 < x < \delta$时, $\mathrm{e}^{1/x} > \mathrm{e}^{1/\delta} = M$.
    \end{enumerate}
\end{solution}

\begin{exercise}[1.3.12]\label{exe:1.3.12}
    证明: 函数 $y=x\sin x$ 在 $(0, +\infty)$ 内无界, 但当 $x \to +\infty$ 时, 这个函数并不是无穷大量.
\end{exercise}

\begin{solution}
    $\forall M > 0$,存在$x_0 = (2k-1) \pi, k \in \mathbb{N}^*, 2k - 1 > M$,因此$y(x_0) = x_0 \sin x_0 = x_0 > M$.由此可知$y = x \sin x$在$(0, +\infty)$内无界.

    $\forall X > 0$,总存在$x_1 = 2k \pi, k \in \mathbb{N}^*, 2k \pi > X$,使得$y(x_1) = x_1 \sin x_1 = 0$.因此当$x \to +\infty$时, $y = x \sin x$并不是无穷大量.
\end{solution}

\begin{exercise}[1.3.13]
    函数 $y = \frac{1}{x}\cos\frac{1}{x}$ 在区间 $(0,1)$ 内是否有界? 又当 $x \to 0^+$ 时, 这个函数是否为无穷大量?
\end{exercise}

\begin{solution}
    考虑$0^+$处的$\frac{1}{x} \cos \frac{1}{x}$与考虑$+\infty$处的$x \cos x$是等价的.以与\hyperref[exe:1.3.12]{习题1.3.12}类似的方法可知, $y = x \cos x$在$(0, +\infty)$内无界,但当$x \to +\infty$时, $y = x \cos x$并不是无穷大量.因此, $y = \frac{1}{x} \cos \frac{1}{x}$在$(0, 1)$内无界,但当$x \to 0^+$时, $y = \frac{1}{x} \cos \frac{1}{x}$并不是无穷大量.
\end{solution}

\begin{exercise}[1.3.14]
    本题所涉及的函数极限有着鲜明的几何意义.

    记函数 $y=f(x)$ 所表示的曲线为 $C$. 若动点沿曲线无限远离原点时, 此动点与某一固定直线的距离趋于零, 则称该直线为曲线 $C$ 的一条渐近线.
    \begin{enumerate}[label=(\roman*)]
        \item \textit{垂直渐近线} \quad 易知(垂直于 $x$ 轴的)直线 $x=x_0$ 为曲线 $C$ 的渐近线的充分必要条件是
              $$ \lim_{x \to x_0^-} f(x) = \infty \quad \text{或} \quad \lim_{x \to x_0^+} f(x) = \infty. $$

        \item \textit{水平渐近线} \quad 易知(平行于 $x$ 轴的)直线 $y=b$ 为曲线 $C$ 的渐近线的充分必要条件是
              $$ \lim_{x \to +\infty} f(x) = b \quad \text{或} \quad \lim_{x \to -\infty} f(x) = b. $$

        \item \textit{斜渐近线} \quad 请读者证明, 方程为 $y=ax+b \, (a \ne 0)$ 的直线 $L$ 为曲线 $C$ 的渐近线的充分必要条件是
              $$ a = \lim_{x \to +\infty} \frac{f(x)}{x}, \quad b = \lim_{x \to +\infty} (f(x)-ax); $$
              或者
              $$ a = \lim_{x \to -\infty} \frac{f(x)}{x}, \quad b = \lim_{x \to -\infty} (f(x)-ax). $$
    \end{enumerate}
    这里自然要假定所说的极限都存在. (提示: 以 $x \to +\infty$ 为例, 设曲线 $C$ 及直线 $L$ 上的横坐标为 $x$ 的点分别为 $M,N$. 则 $M$ 至 $L$ 的距离, 是 $|MN|$ 的一个常数倍. 因此, 直线 $L$ 为曲线 $C$ 的渐近线, 等价于 $\lim_{x \to +\infty} (f(x) - (ax+b)) = 0$, 由此易得所说结果.)

    求下列曲线的渐近方程.
    \begin{tasks}[label=(\arabic*)](2)
        \task $y=x\ln\left(\mathrm{e} + \frac{1}{x}\right)$;
        \task $y=\frac{3x^2-2x+3}{x-1}$.
    \end{tasks}
\end{exercise}

\begin{solution}
    先证明,仅证明$+\infty$,另一种同理.正如提示所说,由于距离$d=\left| \frac{f(x)-(ax+b)}{\sqrt{a^2+1}} \right|$,因此$l$是渐近线,等价于$x\to+\infty$时$d$趋于$0$,等价于$f(x)-(ax+b)$趋于$0$.
    
    然后问题转化为了证明$$\lim_{x \to +\infty} (f(x) - (ax+b)) = 0 \Leftrightarrow a = \lim_{x \to +\infty} \frac{f(x)}{x}, \quad b = \lim_{x \to +\infty} (f(x)-ax).$$

    充分性:由$b = \lim_{x \to +\infty} (f(x)-ax)$可知, $$\lim_{x \to +\infty} (f(x) - (ax+b)) = 0.$$

    必要性:由$\lim_{x \to +\infty} (f(x) - (ax+b)) = 0$可知, $$\lim_{x \to +\infty} \frac{f(x)}{x} - a - \frac{b}{x} = 0, \quad \lim_{x \to +\infty} (f(x)-ax) = b$$因此$$\lim_{x \to +\infty} \frac{f(x)}{x} = a, \quad \lim_{x \to +\infty} (f(x)-ax) = b.$$



    \begin{enumerate}[(1)]
        \item
              \begin{enumerate}
                  \item 垂直渐近线, $x=- \frac{1}{\e}$: $\lim_{x \to (-\frac{1}{\e})^-} = -\frac{1}{\e} \lim_{y \to 0^+} \ln y = +\infty$;
                  \item 斜渐近线, $y=x + \frac{1}{\e}$: $\lim_{x \to \infty} \frac{y(x)}{x} = \lim_{x \to \infty} \ln(\e + \frac{1}{x}) = 1$, $\lim_{x \to  \infty} (y(x) - x) = \lim_{x \to \infty} x(\ln(\e + \frac{1}{x}) - 1) = \lim_{x \to \infty} x\ln(1 + \frac{1}{\e x}) = \lim_{x \to \infty} \frac{\ln(1 + \frac{1}{\e x})}{\frac{1}{x}} = \lim_{x \to  \infty} \frac{1 / \e x}{1 / x} = \frac{1}{\e}$ \ ($\pm\infty$两侧是同一条渐近线);
              \end{enumerate}
        \item
              \begin{enumerate}
                  \item 垂直渐近线, $x=1$: $\lim_{x \to 1} y(x) = \lim_{x \to 1} \frac{3x^2 - 2x + 3}{x-1} = \infty$;
                  \item 斜渐近线, $y=3x+1$:$\lim_{x \to \infty} \frac{y(x)}{x} = \lim_{x \to  \infty} \frac{3 - \frac{2}{x} + \frac{3}{x^2}}{1 - \frac{1}{x}} = 3$, $\lim_{x \to + \infty} (y(x) - 3x) = \lim_{x \to \infty} \frac{x+3}{x-1} = 1$;
              \end{enumerate}
    \end{enumerate}
\end{solution}

\begin{exercise}[1.3.15]
    证明: 在同一极限过程中等价的无穷小量有下列性质:
    \begin{enumerate}
        \item $\alpha(x) \sim \alpha(x)$ (自反性);
        \item 若 $\alpha(x) \sim \beta(x)$, 则 $\beta(x) \sim \alpha(x)$ (对称性);
        \item 若 $\alpha(x) \sim \beta(x)$, $\beta(x) \sim \gamma(x)$, 则 $\alpha(x) \sim \gamma(x)$ (传递性).
    \end{enumerate}
    (注意, (1) 中自然需假定 $\alpha(x)$ 不取零值; 而在 (2)、(3) 中, 条件蕴含着, 所说的无穷小量在极限过程中均不取零值.)
\end{exercise}

\begin{solution}
    解释一下,这里说的是(1)需要没有$\alpha(x)\equiv 0$这种情况.(2)(3)因为有“若xxx”的假设自然排除了这种情况.
    \begin{enumerate}[(1)]
        \item 显然, $\lim \frac{\alpha(x)}{\alpha(x)} = 1$,因此$\alpha(x) \sim \alpha(x)$.
        \item 由$\alpha(x) \sim \beta(x)$可知, $\lim \frac{\alpha(x)}{\beta(x)} = 1$,因此$\lim \frac{\beta(x)}{\alpha(x)} = 1$,即$\beta(x) \sim \alpha(x)$.
        \item 由$\alpha(x) \sim \beta(x)$, $\beta(x) \sim \gamma(x)$可知, $\lim \frac{\alpha(x)}{\beta(x)} = 1$, $\lim \frac{\beta(x)}{\gamma(x)} = 1$,因此$\lim \frac{\alpha(x)}{\gamma(x)} = \lim \frac{\alpha(x)}{\beta(x)} \cdot \lim \frac{\beta(x)}{\gamma(x)} = 1$,即$\alpha(x) \sim \gamma(x)$.
    \end{enumerate}
\end{solution}

\begin{exercise}[1.3.16]
    当 $x \to 0$ 时, 比较下列无穷小的阶:
    \begin{tasks}[label=(\arabic*)](2)
        \task $\tan x - \sin x$ 与 $x^3$;
        \task $x^3+x^2$ 与 $\sin^2 x$;
        \task $1-\cos x$ 与 $x^2$.
    \end{tasks}
\end{exercise}

\begin{solution}
    \begin{enumerate}[(1)]
        \item $$\tan x - \sin x = \frac{\sin x}{\cos x} - \sin x = \sin x \left( \frac{1}{\cos x} - 1 \right) = \sin x \cdot \frac{1 - \cos x}{\cos x}.$$
              由$\sin x \sim x, 1 - \cos x \sim \frac{x^2}{2}, \cos x \sim 1$,可知$$\tan x - \sin x \sim x \cdot \frac{x^2 / 2}{1} = \frac{x^3}{2}.$$
              因此, $$\tan x - \sin x \sim \frac{1}{2} x^3.$$
        \item $$\sin^2 x = (\sin x)^2 \sim x^2  (x \to 0).$$
              同时, $$x^3 + x^2=(x+1)x^2\sim x^2   (x \to 0).$$
              可得$$x^3 + x^2\sim \sin^2 x$$
        \item $$1 - \cos x = 2 \sin^2 \frac{x}{2} \sim 2 \cdot \left( \frac{x}{2} \right)^2 = \frac{x^2}{2} (x \to 0).$$
    \end{enumerate}
\end{solution}

\begin{exercise}[1.3.17]
    当 $x \to +\infty$ 时, 试比较下列无穷大量的阶:
    \begin{tasks}[label=(\arabic*)](1)
        \task $n$ 次多项式 $P_n(x)$ 与 $m$ 次多项式 $P_m(x)$ ($m,n$ 均为正整数);
        \task $x^\alpha$ 与 $x^\beta$ ($\alpha, \beta > 0$);
        \task $a^x$ 与 $b^x$ ($a, b > 1$).
    \end{tasks}
\end{exercise}
\begin{solution}
    \begin{enumerate}[(1)]
        \item 
        $\lim_{x \to +\infty} \frac{P_n(x)}{P_m(x)} = \lim_{x \to +\infty} \frac{a_n x^n +a_{n-1} x^{n-1} \cdots}{b_m x^m +b_{m-1} x^{m-1} +\cdots} 
        = \lim_{x \to +\infty} \frac{x^n}{x^m}\frac{a_n  +a_{n-1} \frac{1}{x} +\cdots}{b_m + b_{m-1}\frac{1}{x} +\cdots}$ 
        
        $=\frac{a_n  }{b_m}\lim_{x \to +\infty}x^{n-m}=\begin{cases}
                    \frac{a_n}{b_m}, & n = m; \\
                    0, & n < m; \\
                    +\infty, & n > m.
                \end{cases} $,即得到$\begin{cases}
                    P_n(x) \sim P_m(x), & n = m; \\
                    P_m(x) \text{更高阶}, & n < m; \\
                    P_n(x) \text{更高阶}, & n > m.
                \end{cases} $.
        \item     利用$\lim_{x \to +\infty} \frac{x^\alpha}{x^\beta} = \lim_{x \to +\infty} x^{\alpha - \beta} = \begin{cases}
                        1, & \alpha = \beta; \\
                        0, & \alpha < \beta; \\
                        +\infty, & \alpha > \beta.
                    \end{cases} $可得$\begin{cases}
                    x^\alpha \sim x^\beta, & \alpha = \beta; \\
                    x^\beta \text{更高阶}, & \alpha < \beta; \\
                    x^\alpha \text{更高阶}, & \alpha > \beta.
                \end{cases}$
        \item 利用$\lim_{x \to +\infty} \frac{a^x}{b^x} = \lim_{x \to +\infty} \left( \frac{a}{b} \right)^x = \begin{cases}
                        1, & a = b; \\
                        0, & a < b; \\
                        +\infty, & a > b.
                    \end{cases} $可得$\begin{cases}
                    a^x \sim b^x, & a = b; \\
                    b^x \text{更高阶}, & a < b; \\
                    a^x \text{更高阶}, & a > b.
                \end{cases}$

    \end{enumerate}
\end{solution}

\begin{exercise}[1.3.18]
    试用等价无穷小量代换的方法计算下列极限.
    \begin{tasks}[label=(\arabic*)](2)
        \task $\lim_{x \to 0} \frac{\sin mx}{\sin nx}$ ($m, n$ 均为正整数);
        \task $\lim_{x \to 0} \frac{\tan ax}{x}$;
        \task $\lim_{x \to 0} \frac{\sqrt[n]{1+\sin x}-1}{\arctan x}$;
        \task $\lim_{x \to 0} \frac{\sqrt{2}-\sqrt{1+\cos x}}{\sin^2 x}$;
        \task $\lim_{x \to 0} \frac{\sqrt{1+x+x^2}-1}{\sin 2x}$;
        \task $\lim_{x \to 0} \frac{\sqrt{1+x^2}-1}{1-\cos x}$.
    \end{tasks}
\end{exercise}

\begin{solution}
    \begin{enumerate}[(1)]
        \item 由$\sin x \sim x$,可知$$\lim_{x \to 0} \frac{\sin mx}{\sin nx} = \lim_{x \to 0} \frac{mx}{nx} = \frac{m}{n}.$$
        \item 由$\tan x \sim x$,可知$a\neq0$时, $$\lim_{x \to 0} \frac{\tan ax}{x} = \lim_{x \to 0} \frac{ax}{x} = a.$$
        很显然该结果对$a=0$也成立.
        \item 由$(1+x)^\alpha - 1 \sim \alpha x$, $\arctan x \sim x$,可知$$\lim_{x \to 0} \frac{\sqrt[n]{1+\sin x}-1}{\arctan x} = \lim_{x \to 0} \frac{\frac{1}{n} \sin x}{x} = \frac{1}{n}.$$
        \item $$\frac{\sqrt{2}-\sqrt{1+\cos x}}{\sin^2 x} = \frac{(\sqrt{2}-\sqrt{1+\cos x})(\sqrt{2}+\sqrt{1+\cos x})}{\sin^2 x (\sqrt{2}+\sqrt{1+\cos x})} = \frac{1 - \cos x}{\sin^2 x (\sqrt{2}+\sqrt{1+\cos x})}.$$
              由$1 - \cos x \sim \frac{x^2}{2}, \sin x \sim x$,可知$$\lim_{x \to 0} \frac{\sqrt{2}-\sqrt{1+\cos x}}{\sin^2 x} = \lim_{x \to 0} \frac{\frac{x^2}{2}}{x^2 (\sqrt{2} + \sqrt{1 + \cos x})} = \frac{1}{4\sqrt{2}}.$$
        \item 由$(1+x)^\alpha - 1 \sim \alpha x$, $\sin x \sim x$,可知$$\lim_{x \to 0} \frac{\sqrt{1+x+x^2}-1}{\sin 2x} = \lim_{x \to 0} \frac{\frac{1}{2}(x + x^2)}{2x} = \frac{1}{4}.$$
        \item 由$(1+x)^\alpha - 1 \sim \alpha x$, $1 - \cos x \sim \frac{x^2}{2}$,可知$$\lim_{x \to 0} \frac{\sqrt{1+x^2}-1}{1-\cos x} = \lim_{x \to 0} \frac{\frac{1}{2} x^2}{\frac{x^2}{2}} = 1.$$
    \end{enumerate}
\end{solution}