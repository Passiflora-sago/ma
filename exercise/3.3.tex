\section{习题3.3}
%3.13-3.15被删掉了
\begin{exercise}[3.3.1]
    设 $f(x)=(x-1)(x-2)(x-3)(x-4)$, 确定方程 $f'(x)=0$ 的实根的个数, 并指出根所在的区间.
\end{exercise}

\begin{solution}
    $f'(x)$为三次多项式, 故$f'(x)=0$最多有三个实根. 又$f(1) = f(2) = f(3) = f(4) = 0$,由Rolle定理,在$(1,2),(2,3),(3,4)$内分别各至少有一点$\xi_1,\xi_2,\xi_3$,使得$f'(\xi_1) = f'(\xi_2) = f'(\xi_3) = 0$,显然该三点互不相同.因此$f'(x)=0$有且仅有三个实根,分别在$(1,2),(2,3),(3,4)$内.
\end{solution}

\begin{exercise}[3.3.2]
    设函数 $f(x)$ 在区间 $[1,2]$ 上有二阶微商, 且 $f(1)=f(2)=0$. 记 $F(x)=(x-1)^2 f(x)$, 则在区间 $(1,2)$ 内至少有一点 $\xi$, 使得 $F''(\xi)=0$.
\end{exercise}

\begin{solution}
    由$f(1) = f(2) = 0 \Rightarrow F(1) = F(2) = 0$.由Rolle定理,在$(1,2)$内至少有一点$\xi_1$,使得$F'(\xi_1) = 0$.又$F'(1) = 2(1-1)f(1) + (1-1)^2 f'(1) = 0$.由Rolle定理,在$(1,\xi_1)$内至少有一点$\xi$,使得$F''(\xi) = 0$.
\end{solution}

\begin{exercise}[3.3.3]
    举例说明, 中值定理的下述意义的逆不成立: 设 $\xi \in (a,b)$ 是指定的一点, 则存在 $c,d \in [a,b]$, 使得 $\frac{f(c)-f(d)}{c-d}=f'(\xi)$. (提示: 考虑函数 $f(x)=x^3, \xi=0$.)
\end{exercise}

\begin{solution}
    设$f(x) = x^3, \xi = 0$,则$f'(\xi) = f'(0) = 0$.若存在$c,d \in [-1,1]$,使得$\frac{f(c)-f(d)}{c-d} = f'(\xi)$,则有
    $$
        \frac{c^3 - d^3}{c-d} = 0 \Rightarrow c^2 + cd + d^2 = 0 \Rightarrow c = d = 0.
    $$
    但$c,d$不能相等,故不存在$c,d \in [-1,1]$,使得$\frac{f(c)-f(d)}{c-d} = f'(\xi)$.
\end{solution}

\begin{exercise}[3.3.4]
    证明下列不等式:
    \begin{enumerate}
        \item 当 $a>b>0, n>1$ 时, 有 $nb^{n-1}(a-b) < a^n-b^n < na^{n-1}(a-b)$;
        \item 当 $x>0$ 时, 有 $\frac{x}{1+x} < \ln(1+x) < x$;
        \item 当 $0<a<b$ 时, 有 $(a+b)\ln\frac{a+b}{2} < a\ln a + b\ln b$.
        \item 当 $0<\alpha<\beta<\frac{\pi}{2}$ 时, 有 $\frac{\beta-\alpha}{\cos^2\alpha} < \tan\beta - \tan\alpha < \frac{\beta-\alpha}{\cos^2\beta}$.
    \end{enumerate}
\end{exercise}

\begin{solution}
    \begin{enumerate}
        \item 设$f(x) = x^n$,则$f'(x) = nx^{n-1}$.由Lagrange中值定理,在$(b,a)$内至少有一点$\xi$,使得
              $$
                  f'(\xi) = \frac{f(a)-f(b)}{a-b}
              $$
              又$a > \xi > b > 0$,$f'(x)$在$(0,+\infty)$内单调递增,故$nb^{n-1} < f'(\xi) < na^{n-1}$,于是$$nb^{n-1}(a-b) < a^n-b^n < na^{n-1}(a-b).$$
        \item 设$f(x) = \ln(1+x)$,则$f'(x) = \frac{1}{1+x}$.由Lagrange中值定理,在$(0,x)$内至少有一点$\xi$,使得
              $$
                  f'(\xi) = \frac{f(x)-f(0)}{x-0} = \frac{\ln(1+x)}{x}
              $$
              又$x > \xi > 0$,$f'(x)$在$(0,+\infty)$内单调递减,故$\frac{1}{1+x} < f'(\xi) < 1$,于是$$\frac{x}{1+x} < \ln(1+x) < x.$$
        \item 设$f(x) = x\ln x$,则$f'(x) = \ln x + 1$.由Lagrange中值定理,$\exists \xi \in \left(a, \frac{a+b}{2}\right), \eta \in \left( \frac{a+b}{2}, b \right)$,使得
              $$
                  f'(\xi) = \frac{f\left(\frac{a+b}{2}\right)-f(a)}{\frac{a+b}{2}-a} = \frac{(a+b)\ln\frac{a+b}{2} - 2a\ln a}{b-a}
              $$
              $$
                  f'(\eta) = \frac{f(b)-f\left(\frac{a+b}{2}\right)}{b-\frac{a+b}{2}} = \frac{2b\ln b - (a+b)\ln\frac{a+b}{2}}{b-a}
              $$
              又$a < \xi < \eta < b$, $f'(x)$在$(0,+\infty)$内单调递增,故$f'(\xi) < f'(\eta)$,于是$$(a+b)\ln\frac{a+b}{2} < a\ln a + b\ln b.$$
        \item 设$f(x) = \tan x$,则$f'(x) = \frac{1}{\cos^2 x}$.由Lagrange中值定理,在$(\alpha,\beta)$内至少有一点$\xi$,使得
              $$
                  f'(\xi) = \frac{f(\beta)-f(\alpha)}{\beta-\alpha} = \frac{\tan\beta - \tan\alpha}{\beta-\alpha}
              $$
              又$\beta > \xi > \alpha > 0$,$f'(x)$在$(0,\frac{\pi}{2})$内单调递增,故$\frac{1}{\cos^2\alpha} < f'(\xi) < \frac{1}{\cos^2\beta}$,于是$$\frac{\beta-\alpha}{\cos^2\alpha} < \tan\beta - \tan\alpha < \frac{\beta-\alpha}{\cos^2\beta}.$$
    \end{enumerate}
\end{solution}

\begin{exercise}[3.3.5]
    证明下列恒等式:
    \begin{enumerate}
        \item $\arctan x = \arcsin\frac{x}{\sqrt{1+x^2}}$;
        \item $\arctan x + \arctan\frac{1-x}{1+x} = \begin{cases} \frac{\pi}{4}, & x > -1, \\ -\frac{3\pi}{4}, & x < -1. \end{cases}$
    \end{enumerate}
\end{exercise}

\begin{solution}
    \begin{enumerate}
        \item 设$f(x) = \arctan x - \arcsin\frac{x}{\sqrt{1+x^2}}$,则
              $$
                  f'(x) = \frac{1}{1+x^2} - \frac{1}{\sqrt{1+x^2}} \cdot \frac{1}{\sqrt{1-\frac{x^2}{1+x^2}}} \cdot \frac{(1+x^2) - x \cdot 2x}{(1+x^2)^2} = 0.
              $$
              又$f(0) = 0$,故$f(x) = 0$,即$\arctan x = \arcsin\frac{x}{\sqrt{1+x^2}}$.
        \item 设$g(x) = \arctan x + \arctan\frac{1-x}{1+x}$,则
              $$
                  g'(x) = \frac{1}{1+x^2} + \frac{1}{1+\left(\frac{1-x}{1+x}\right)^2} \cdot \frac{-2}{(1+x)^2} = 0. \quad (x \ne -1)
              $$
              又$g(0) = \frac{\pi}{4},g(-2) = -\frac{3\pi}{4}$,故$g(x) = \begin{cases} \frac{\pi}{4}, & x > -1, \\ -\frac{3\pi}{4}, & x < -1. \end{cases}$
    \end{enumerate}
\end{solution}

这道题使用了引理:
\begin{lemma*}
    设函数 $f(x)$ 在区间 $I$ 上连续, 且在 $I$ 内部可导. 若 $f'(x)=0$ 恒成立, 则 $f(x)$ 在 $I$ 上恒为常数.
\end{lemma*}

\begin{proof}
    设$x_1,x_2 \in I$,不妨设$x_1 < x_2$,则由Lagrange中值定理,$\exists \xi \in (x_1,x_2)$,使得
    $$
        f'(\xi) = \frac{f(x_2)-f(x_1)}{x_2-x_1} = 0 \Rightarrow f(x_2) = f(x_1).
    $$
    因此$f(x)$在$I$上恒为常数.
\end{proof}

\begin{exercise}[3.3.6]
    设 $f(x)$ 是闭区间 $[0,1]$ 上的可导函数, 对任意 $x \in [0,1]$ 有 $f(x) \in (0,1)$; 并且对每个 $x, f'(x) \ne 1$. 证明在 $(0,1)$ 内有且仅有一个 $x$, 使 $f(x)=x$.
\end{exercise}

\begin{solution}
    先证明存在性:设$g(x) = f(x) - x$,则$g(0) = f(0) > 0, g(1) = f(1) - 1 < 0$.由介值定理,在$(0,1)$内至少有一点$\xi$,使得$g(\xi) = 0$,即$f(\xi) = \xi$.

    反证法证明唯一性:假设在$(0,1)$内有两点$\xi_1,\xi_2$,使得$f(\xi_1) = \xi_1,f(\xi_2) = \xi_2$,则$g(\xi_1) = g(\xi_2) = 0$.由Rolle定理,在$(\xi_1,\xi_2)$内至少有一点$\eta$,使得$g'(\eta) = 0$,即$f'(\eta) - 1 = 0$,与$f'(x) \ne 1$矛盾.因此在$(0,1)$内有且仅有一个$x$,使$f(x) = x$.
\end{solution}

\begin{exercise}[3.3.7]
    设函数 $f(x)$ 在 $[0,1]$ 上连续, 在 $(0,1)$ 内可导, 且 $|f'(x)| < 1$, 又 $f(0)=f(1)$. 证明: 对于 $[0,1]$ 上的任意两点 $x_1, x_2$, 有 $|f(x_1)-f(x_2)| < \frac{1}{2}$.
\end{exercise}

\begin{solution}
    对于$x_1,x_2 \in [0,1]$,不妨设$x_1 < x_2$,则由Lagrange中值定理,$\exists \xi_1,\xi_2,\xi_3 \st 0 < \xi_1 < x_1 < \xi_2 < x_2 < \xi_3 < 1$,使得
    $$
        f'(\xi_1) = \frac{f(x_1)-f(0)}{x_1-0}, \quad f'(\xi_2) = \frac{f(x_2)-f(x_1)}{x_2-x_1}, \quad f'(\xi_3) = \frac{f(1)-f(x_2)}{1-x_2}.
    $$
    因此$$|f(x_1)-f(0)| = |f'(\xi_1)|x_1 < x_1,$$
    $$|f(x_2)-f(x_1)| = |f'(\xi_2)|(x_2-x_1) < x_2 - x_1,$$
    $$|f(1)-f(x_2)| = |f'(\xi_3)|(1-x_2) < 1 - x_2.$$
    于是\begin{align*}
        2|f(x_1)-f(x_2)| & = |f(x_2)-f(x_1)| + |f(1)-f(x_2) + f(x_1)-f(0)|      \\
                         & \les |f(x_2)-f(x_1)| + |f(1)-f(x_2)| + |f(x_1)-f(0)| \\
                         & < (x_2 - x_1) + (1 - x_2) + x_1 = 1.
    \end{align*}
    故$|f(x_1)-f(x_2)| < \frac{1}{2}$.
\end{solution}

\begin{exercise}[3.3.8]
    若 $f(x)$ 处处可导, 且 $f'(x)=f(x)$. 证明 $f(x) = C\mathrm{e}^x, C$ 为任意常数.
\end{exercise}

\begin{solution}
    设$g(x) = \frac{f(x)}{\mathrm{e}^x}$,则
    $$
        g'(x) = \frac{f'(x)\mathrm{e}^x - f(x)\mathrm{e}^x}{(\mathrm{e}^x)^2} = \frac{\mathrm{e}^x f(x) - \mathrm{e}^x f(x)}{(\mathrm{e}^x)^2} = 0.
    $$
    因此$g(x) \equiv g(0) \Rightarrow f(x) = f(0)\mathrm{e}^x$.设$C = f(0)$.
\end{solution}

\begin{exercise}[3.3.9]
    设不恒为常数的函数 $f(x)$ 在闭区间 $[a,b]$ 上连续, 在开区间 $(a,b)$ 内可导, 且 $f(a)=f(b)$. 证明在 $(a,b)$ 内存在一点 $\xi$, 使得 $f'(\xi)>0$.
\end{exercise}

\begin{solution}
    由于$f(x)$不为常函数,故存在$x_0 \in (a,b)$,使得$f(x_0) \ne f(a)$.
    \begin{enumerate}
        \item 若$f(x_0) > f(a)$,则$\exists \xi \in (a,x_0)$,使得$f(\xi) = \frac{f(x_0)-f(a)}{x_0-a} > 0$.
        \item 若$f(x_0) < f(a)=f(b)$,则$\exists \eta \in (x_0,b)$,使得$f(\eta) = \frac{f(b)-f(x_0)}{b-x_0} > 0$.
    \end{enumerate}
\end{solution}

\begin{exercise}[3.3.10]
    设函数 $f(x)$ 在 $[a, +\infty)$ 上可微, 且 $\lim_{x \to +\infty} f'(x) = 0$. 证明:
    \begin{enumerate}
        \item $\lim_{x \to +\infty} [f(x+1)-f(x)] = 0$;
        \item $\lim_{x \to +\infty} \frac{f(x)}{x} = 0$.
    \end{enumerate}
\end{exercise}

\begin{solution}
    \begin{enumerate}
        \item 由Lagrange中值定理,$\exists \xi_x \in (x,x+1)$,使得
              $$
                  f'(\xi_x) = \frac{f(x+1)-f(x)}{(x+1)-x} = f(x+1)-f(x).
              $$
              又$\lim_{x \to +\infty} f'(x) = 0$,故$\lim_{x \to +\infty} [f(x+1)-f(x)] = 0$.
        \item $\lim_{x \to +\infty} f'(x) = 0 \Rightarrow \forall \ve, \exists X > 0 , \st \forall x > X, |f'(x)| < \ve.$
              设$x > X$,则由Lagrange中值定理,$\exists \eta \in (X,x)$,使得
              $$
                  f'(\eta) = \frac{f(x)-f(X)}{x-X} \Rightarrow f(x) = f'(\eta)(x-X) + f(X).
              $$
              因此
              $$
                  \left|\frac{f(x)}{x}\right| = \left|f'(\eta)\left(1-\frac{X}{x}\right) + \frac{f(X)}{x}\right| \les |f'(\eta)|\left|1-\frac{X}{x}\right| + \left|\frac{f(X)}{x}\right| < \ve + \left| \frac{f(X)}{x}\right|.
              $$
              也就是说,当$x > \max\left\{X, \frac{|f(X)|}{\ve}\right\}$时,$\left|\frac{f(x)}{x}\right| < 2\ve$.因此$\lim_{x \to +\infty} \frac{f(x)}{x} = 0$.
    \end{enumerate}
\end{solution}

\begin{exercise}[3.3.11]
    证明: 若函数 $f(x)$ 在(有限)开区间 $(a,b)$ 内有有界的导函数, 则 $f(x)$ 在 $(a,b)$ 内也有界. 如果有限区间 $(a,b)$ 改为无穷区间, 结论还成立吗? 命题的逆命题是否成立?
\end{exercise}

\begin{solution}
    $f(x)$ 在 $(a, b)$ 内可导, 故在 $(a, b)$ 连续.
    不妨设 $\forall x \in (a, b), f'(x) \les M$,同时取$c=\frac{a+b}{2}$, 则
    $$ \forall x \in (a, b)\setminus \{c\}, \left|\frac{f(x) - f(c)}{x-c}\right| = |f'(\xi)| \les M \Rightarrow |f(x)| \les |f(c)| + M|b-a|$$
    显然对于$c$,$|f(c)| \les |f(c)| + M|b-a|$也成立,即得$ f(x)$ 在 $(a, b)$ 内有界.

    将 $(a, b)$ 改为无穷区间, 结论不成立. 考虑 $f(x) = x, f'(x) = 1$ 在 $(0, +\infty)$ 上有界, 但 $f(x) = x$ 在 $(0, +\infty)$ 上无界.

    逆命题不成立. 考虑 $f(x) = \sqrt{x}, f(x) = \sqrt{x}$ 在 $(0, 1)$ 上有界, 但 $f'(x) = \frac{1}{2\sqrt{x}}$ 在 $(0, 1)$ 上无界.
\end{solution}

\begin{exercise}[3.3.12]
    设对所有的实数 $x,y$, 不等式 $|f(x)-f(y)| \les M|x-y|^2$ ($M$ 为常数) 都成立. 证明: $f(x)$ 恒为常数.
\end{exercise}

\begin{solution}
    $\forall x \ne y, 0 \les \left|\frac{f(x)-f(y)}{x-y}\right| \les M|x-y|$. 令 $y \to x$,
    $$ \lim_{y\to x} 0 = \lim_{y\to x} M|x-y| = 0 \Rightarrow |f'(x)| = \lim_{y\to x} \left|\frac{f(x) - f(y)}{x-y}\right| = 0 \Rightarrow f'(x) = 0. $$
    故 $f(x)$ 恒为常数.
\end{solution}

% \begin{exercise}[3.3.13]
%     设函数 $f(x)$ 在区间 $[x_0, x_0+\delta]$ 上连续 (这里 $\delta>0$), 在 $(x_0, x_0+\delta)$ 内可导. 若 $\lim_{x \to x_0^+} f'(x) = l$ (这里的 $l$ 可以是无穷大), 则 $f(x)$ 在 $x_0$ 处的右导数也为 $l$, 即
%     $$ f'_+(x_0) = \lim_{x \to x_0^+} f'(x). $$
%     (将区间 $[x_0, x_0+\delta]$ 换为 $[x_0-\delta, x_0]$, 有类似的结论.)
% \end{exercise}

% % \begin{solution}
% %     设$x \in (x_0, x_0+\delta)$,则由Lagrange中值定理,$\exists \xi \in (x_0,x)$,使得
% %     $f'(\xi) = \frac{f(x)-f(x_0)}{x-x_0}.$
% %     令$x \to x_0^+$,则$\xi \to x_0^+$,因此$$f'_+(x_0) = \lim_{x \to x_0^+} \frac{f(x)-f(x_0)}{x-x_0} = \lim_{\xi \to x_0^+} f'(\xi) = \lim_{x \to x_0^+} f'(x) = l.$$
% % \end{solution}

% \begin{solution}
%     仅证$l=+\infty$的情形.由$\lim_{x \to x_0^+} f'(x) = +\infty$,知$\forall N > 0, \exists \delta_N > 0, \st \forall x \in (x_0, x_0 + \delta_N), f'(x) > N.$

%     由Lagrange中值定理,有$\forall x \in (x_0,x_0 + \delta_N)$, $\exists \xi \in (x_0,x)$,使得$$f'(\xi) = \frac{f(x)-f(x_0)}{x-x_0}$$

%     因此$\forall N, \exists \delta_N > 0 , \forall x \in (x_0,x_0+\delta),$
%     $$\frac{f(x)-f(x_0)}{x-x_0} = f'(\xi) > N.$$
%     即证$f_+'(x_0) = \lim_{x \to x_0^+} \frac{f(x)-f(x_0)}{x-x_0} = + \infty.$
% \end{solution}

% 但是若$f_+'(x_0)=l$已知,不一定有$\lim_{x \to x_0^+} f'(x) = l$.考虑$f(x) = \begin{cases} x^2\sin\frac{1}{x}, & x \ne 0, \\ 0, & x = 0. \end{cases}$.$f_+'(0) = 0$,但$\lim_{x \to 0^+} f'(x) = \lim_{x \to 0^+} 2x\sin\frac{1}{x} - \cos\frac{1}{x}$不存在.

% \begin{exercise}[3.3.14]
%     应用上一题的结论证明:
%     \begin{enumerate}
%         \item 函数 $x^{1/3}$ 在 $x=0$ 处不可导;
%         \item 函数 $\arcsin x, \arccos x$ 在 $x=1$ 处没有左导数, 在 $x=-1$ 处没有右导数.
%     \end{enumerate}
% \end{exercise}

% \begin{solution}
%     \begin{enumerate}
%         \item 设$f(x) = x^{1/3}$,则$f'(x) = \frac{1}{3x^{2/3}}$.$\lim_{x \to 0} f'(x) = \infty$,故$f(x)$在$x = 0$处不可导.
%         \item 仅证明$\arcsin x$在$x = 1$处没有左导数, $\arccos x$的证明类似.设$f(x) = \arcsin x$,则$f'(x) = \frac{1}{\sqrt{1-x^2}}$.$\lim_{x \to 1^-} f'(x) = +\infty$,故$f(x)$在$x = 1$处没有左导数.
%     \end{enumerate}
% \end{solution}

% \begin{exercise}[3.3.15]
%     证明: 若函数 $f(x)$ 在一个区间内处处可导, 则导函数 $f'(x)$ 不能有第一类间断点, 即在(区间内)每一点处, $f'(x)$ 或者连续, 或者有第二类间断. (由本题推出, 具有第一类间断点的函数, 如 $\operatorname{sgn} x$, 不能成为某个函数的导函数.)
% \end{exercise}

% \begin{solution}
%     由Darboux定理,导函数$f'(x)$具有介值性质,故$f'(x)$不能有第一类间断点.
% \end{solution}

\begin{exercise}[3.3.16]\label{ex:3.3.16}
    设 $f(x)$ 在一个区间 $I$ 上连续, 且(至多)除了有限个点外, $f(x)$ 在 $I$ 内部的导数为正(负), 则 $f(x)$ 在 $I$ 上严格单调递增(减). (注意, 在例外的点处, $f(x)$ 可能不可导.)
\end{exercise}

\begin{solution}
    设$x_1,x_2 \in I$,不妨设$x_1 < x_2$.若$f(x)$在$(x_1,x_2)$内处处可导,则由Lagrange中值定理,$\exists \xi \in (x_1,x_2)$,使得
    $$
        f'(\xi) = \frac{f(x_2)-f(x_1)}{x_2-x_1}.
    $$
    又$f'(\xi) > 0 (f'(\xi) < 0)$,故$f(x_2) > f(x_1) (f(x_2) < f(x_1))$.

    若$f(x)$在$(x_1,x_2)$内有有限个点不可导,设这些点为$y_1,y_2,\cdots,y_n$,则将$(x_1,x_2)$分成$n+1$个子区间$(x_1,y_1),(y_1,y_2),\cdots,(y_n,x_2)$,在每个子区间内$f(x)$处处可导,由上面的结论可知,$f(x_1)<f(y_1)<\cdots<f(y_n)<f(x_2)\left(f(x_1)>f(y_1)>\cdots>f(y_n)>f(x_2)\right)$.
    
    综上所述,$f(x)$在$I$上严格单调递增(递减).
\end{solution}

\begin{exercise}[3.3.17]\label{ex:3.3.17}
    设函数 $f(x)$ 和 $g(x)$ 均在区间 $I$ 上连续, 且(至多)除了有限个点外, $f(x)$ 和 $g(x)$ 在 $I$ 内部满足 $f'(x) > g'(x)$; 设存在 $a \in I$, 使得 $f(a)=g(a)$ ($a$ 不是区间端点), 则当 $x \in I$ 且 $x>a$ 时, 有 $f(x)>g(x)$; 当 $x \in I$ 且 $x<a$ 时, 有 $f(x)<g(x)$.
\end{exercise}

\begin{solution}
    记$h(x) = f(x) - g(x)$,则$h(x)$在$I$上连续,且(至多)除了有限个点外,$h(x)$在$I$内部满足$h'(x) = f'(x) - g'(x) > 0$.又$h(a) = f(a) - g(a) = 0$.由上一题的结论可知,当$x \in I$且$x > a$时,$h(x) > 0 \Rightarrow f(x) > g(x)$;当$x \in I$且$x < a$时,$h(x) < 0 \Rightarrow f(x) < g(x)$.
\end{solution}

\begin{exercise}[3.3.18]
    若 $f(x)$ 在 $[0, +\infty)$ 可导, $f(0)=0$, $f'(x)$ 严格递增, 证明 $\frac{f(x)}{x}$ 在 $(0, +\infty)$ 严格递增.
\end{exercise}

\begin{solution}
    设$x_1,x_2 \in (0,+\infty)$,不妨设$x_1 < x_2$,则由Lagrange中值定理,$\exists \xi \in (x_1,x_2)$,使得
    $$
        f'(\xi) = \frac{f(x_2)-f(x_1)}{x_2 - x_1}.
    $$
    又由Lagrange中值定理,$\exists \eta \in (0,x_1)$,使得
    $$
        f'(\eta) = \frac{f(x_1)-f(0)}{x_1 - 0} = \frac{f(x_1)}{x_1}.
    $$
    因为$f'(x)$严格递增,故$f'(\xi) > f'(\eta)$,即
    $$
        \frac{f(x_2)-f(x_1)}{x_2 - x_1} > \frac{f(x_1)}{x_1} \Rightarrow \frac{f(x_2)}{x_2} > \frac{f(x_1)}{x_1}. \quad (\hyperref[3.5.3]{\text{习题3.5.3}})
    $$
    因此$\frac{f(x)}{x}$在$(0,+\infty)$严格递增.
\end{solution}

\begin{exercise}[3.3.19]
    设 $x_0$ 是函数 $f(x)$ 的一个驻点( $f'(x_0)=0$), 且 $f(x)$ 在 $x_0$ 处二阶可微, $f''(x_0) \ne 0$. 证明: 若 $f''(x_0)<0$, 则 $x_0$ 是 $f(x)$ 的一个极大值点; 若 $f''(x_0)>0$, 则 $x_0$ 是 $f(x)$ 的一个极小值点. 

    举例说明: 若 $f''(x_0)=0$, 则 $x_0$ 可以是 $f(x)$ 的极大值点或极小值点, 也可以不是极值点.
\end{exercise}

\begin{solution}
    若$f''(x_0) < 0$,则存在$\delta > 0$,使得$\forall x \in (x_0 - \delta, x_0 + \delta), f''(x) < 0$.由Lagrange中值定理,$\forall x\in (x_0 - \delta, x_0 + \delta),\exists \xi=\xi(x) \in (x_0 - \delta, x_0 + \delta)$,使得
    $$
        f'(\xi) = \frac{f(x_0)-f(x)}{x_0 - x}.
    $$
    又$f''(x) < 0$,故$f'(\xi) > f'(x_0) = 0 > f'(\xi_2)$,由\hyperref[ex:3.3.17]{习题3.3.17} 可知,即当$x \in (x_0 - \delta, x_0)$时,$f(x) < f(x_0)$;当$x \in (x_0, x_0 + \delta)$时,$f(x) < f(x_0)$.因此$x_0$是$f(x)$的一个极小值点.
    即当$x \in (x_0 - \delta, x_0)$时,$f(x) < f(x_0)$;当$x \in (x_0, x_0 + \delta)$时,$f(x) < f(x_0)$.因此$x_0$是$f(x)$的一个极大值点.

    例:设$f(x) = x^4$,则$f''(0) = 0$,但$x = 0$是$f(x)$的一个极小值点.设$g(x) = -x^4$,则$g''(0) = 0$,但$x = 0$是$g(x)$的一个极大值点.设$h(x) = x^3$,则$h''(0) = 0$,但$x = 0$不是$h(x)$的极值点.
\end{solution}

\begin{exercise}[3.3.20]
    设 $f(x)$ 在 $[0,1]$ 上有二阶导数, 且 $f(0)=f'(0), f(1)=f'(1)$. 求证: 存在 $\xi \in (0,1)$, 满足 $f(\xi)=f''(\xi)$.
\end{exercise}

\begin{solution}
    设$g(x) = \e^x \left( f(x) - f'(x) \right)$,则
    $$
        g'(x) = \e^x (f(x) - f'(x)) + \e^x (f'(x) - f''(x)) = \e^x (f(x) - f''(x)).
    $$
    又$g(0) = \e^0 (f(0) - f'(0)) = 0, g(1) = \e^1 (f(1) - f'(1)) = 0$.由Rolle定理,在$(0,1)$内至少有一点$\xi$,使得$g'(\xi) = 0$,即$f(\xi) = f''(\xi)$.
\end{solution}

\begin{exercise}[3.3.21]
    求下列函数的单调区间与极值.
    \begin{tasks}[label=(\arabic*)](2)
        \task $y = 2x^3-3x^2$;
        \task $y = x^{2/3}$;
        \task $y = x^2\mathrm{e}^{-x^2}$;
        \task $y = x^{1/x}$;
        \task $y = \frac{(\ln x)^2}{x}$;
        \task $y = \arctan x - \frac{1}{2}\ln(1+x^2)$.
    \end{tasks}
\end{exercise}

\begin{solution}
    高中的时候大家就已经很熟悉怎么用导数来求函数的单调区间与极值了,这里仅给出答案.
    \begin{enumerate}
        \item $$y' = 6x^2 - 6x = 6x(x-1).$$
              \begin{enumerate}
                  \item[] 单调区间:在$(-\infty,0)$和$(1,+\infty)$上单调递增,在$(0,1)$上单调递减.
                  \item[] 极值:在$x = 0$处取得极大值$0$,在$x = 1$处取得极小值$-1$.
              \end{enumerate}
        \item $$y' = \frac{2}{3}x^{-1/3}.$$
              \begin{enumerate}
                  \item[] 单调区间:在$(0,+\infty)$上单调递增,在$(-\infty,0)$上单调递减.
                  \item[] 极值:在$x = 0$处取得极小值$0$.
              \end{enumerate}
        \item $$y' = 2x\mathrm{e}^{-x^2}(1-x^2).$$
              \begin{enumerate}
                  \item[] 单调区间:在$(-1,0)$和$(1,+\infty)$上单调递减,在$(-\infty,-1)$和$(0,1)$上单调递增.
                  \item[] 极值:在$x = \pm1$处取得极大值$\frac{1}{\mathrm{e}}$,在$x = 0$处取得极小值$0$.
              \end{enumerate}
        \item 此函数只在$(0,+\infty)$有定义.
              $$y' = x^{1/x} \frac{1 - \ln x}{x^2}.$$
              \begin{enumerate}
                  \item[] 单调区间:在$(0,\mathrm{e})$上单调递增,在$(\mathrm{e},+\infty)$上单调递减.
                  \item[] 极值:在$x = \mathrm{e}$处取得极大值$\mathrm{e}^{1/\mathrm{e}}$.
              \end{enumerate}
        \item $$y' = \frac{\ln x (2 - \ln x)}{x^2}.$$
              \begin{enumerate}
                  \item[] 单调区间:在$(0,1)$和$(\mathrm{e}^2,+\infty)$上单调递减,在$(1,\mathrm{e}^2)$上单调递增.
                  \item[] 极值:在$x = 1$处取得极小值$0$,在$x = \mathrm{e}^2$处取得极大值$\frac{4}{\mathrm{e}^2}$.
              \end{enumerate}
        \item $$y' = \frac{1}{1+x^2} - \frac{x}{1+x^2} = \frac{1 - x}{1 + x^2}.$$
              \begin{enumerate}
                  \item[] 单调区间:在$(-\infty,1)$上单调递增,在$(1,+\infty)$上单调递减.
                  \item[] 极值:在$x = 1$处取得极大值$\frac{\pi}{4} - \frac{1}{2}\ln 2$.
              \end{enumerate}
    \end{enumerate}
\end{solution}

\begin{proposition*}[极值点的判别法] \quad
    \begin{enumerate}
        \item 设$f(x)$在$x_0$处连续,若$f'(x_0)$在$x_0$两侧存在,异号,则$f(x_0)$为$f(x)$的极值点.$f'(x_0)$可以不存在.这即极值存在的一阶导判别法.
        \item 若$f'(x_0) = 0, f''(x_0) > 0(<0)$,则$f(x_0)$为$f(x)$的极小值(极大值)点.此即极值存在的二阶导判别法.
        \item 若$f'(x_0) = f''(x_0) = f^{(2n-1)}(x_0) = 0, f^{(2n)} >0 (<0)$,则$f(x_0)$必是$f(x)$的极小值(极大值)点.此即极值存在的高阶导判别法.
    \end{enumerate}
\end{proposition*}

\begin{exercise}[3.3.22]
    求下列函数在所给区间上的最大值和最小值.
    \begin{tasks}[label=(\arabic*)](2)
        \task $y = x^4 - 2x^2+5$, $[ -2, 2]$;
        \task $y = \sin 2x - x$, $[-\frac{\pi}{2}, \frac{\pi}{2}]$;
        \task $y = \arctan\frac{1-x}{1+x}$, $[0,1]$;
        \task $y = x\ln x$, $(0, +\infty)$.
    \end{tasks}
\end{exercise}

\begin{solution}
    \begin{enumerate}
        \item $$y' = 4x^3 - 4x = 4x(x^2 - 1) = 0 \Rightarrow x = -1, 0, 1.$$
              计算所有极值点疑点和边界点:
              $$
                  y(-2) = 13, \quad y(-1) = 4, \quad y(0) = 5, \quad y(1) = 4, \quad y(2) = 13.
              $$
              故最大值为$13$, 最小值为$4$.
        \item $$y' = 2\cos 2x - 1 = 0 \Rightarrow x = \pm \frac{\pi}{6}.$$
              计算所有极值点疑点和边界点:
              $$
                  y\left(-\frac{\pi}{2}\right) = \frac{\pi}{2}, \quad y\left(-\frac{\pi}{6}\right) = -\frac{\sqrt{3}}{2} + \frac{\pi}{6}, \quad y\left(\frac{\pi}{6}\right) = \frac{\sqrt{3}}{2} - \frac{\pi}{6}, \quad y\left(\frac{\pi}{2}\right) = -\frac{\pi}{2}.
              $$
              故最大值为$\frac{\pi}{2}$, 最小值为$-\frac{\pi}{2}$.
        \item $$y' = -\frac{2}{(1+x)^2 + (1-x)^2} = -\frac{1}{1+x^2} < 0 \quad (x \in [0,1]).$$
              故在$[0,1]$上单调递减,最大值为$y(0) = \frac{\pi}{4}$, 最小值为$y(1) = 0$.
        \item $$y' = \ln x + 1 = 0 \Rightarrow x = \frac{1}{\mathrm{e}}.$$
              函数在$\left(0, \frac{1}{\mathrm{e}}\right]$上单调递减,在$\left[\frac{1}{\mathrm{e}}, +\infty\right)$上单调递增且趋于$+\infty$.故最小值为$y\left(\frac{1}{\mathrm{e}}\right) = -\frac{1}{\mathrm{e}}$, 无最大值.
    \end{enumerate}
\end{solution}

\begin{exercise}[3.3.23]
    证明下列不等式:
    \begin{enumerate}
        \item $\frac{1}{2^{p-1}} \les x^p+(1-x)^p \les 1, \quad x \in [0,1], p>1$;
        \item $\tan x > x - \frac{x^3}{3}, \quad x \in (0, \frac{\pi}{2})$;
        \item $\frac{\tan x_2}{\tan x_1} > \frac{x_2}{x_1}, \quad 0<x_1<x_2<\frac{\pi}{2}$;
        \item $\ln(1+x) > \frac{\arctan x}{1+x}, \quad x>0$;
        \item $1+x\ln(x+\sqrt{1+x^2}) \ges \sqrt{1+x^2}, \quad x$ 为任意实数;
        \item $\frac{x}{\sin x} > \frac{4}{3} - \frac{1}{3}\cos x, \quad x \in (0, \frac{\pi}{2})$, 且右端的常数 $\frac{4}{3}$ 不能换为更大的数;
        \item $(1-\frac{1}{x})^{x-1} (1+\frac{1}{x})^x < 4, \quad x \in (1, +\infty)$;
        \item $x^{a-1}+x^{a+1} \ges \left(\frac{1-a}{1+a}\right)^{\frac{a-1}{2}} + \left(\frac{1-a}{1+a}\right)^{\frac{a+1}{2}}, \quad x \in (0,1), a \in (0,1)$.
    \end{enumerate}
\end{exercise}

\begin{solution}
    \begin{enumerate}
        \item 设$f(x) = x^p + (1-x)^p$,则$f'(x) = p[x^{p-1} - (1-x)^{p-1}]$.因此$f(x)$在$\left[ 0 , \frac{1}{2} \right]$上单调递增,在$\left[ \frac{1}{2}, 1 \right]$上单调递减.故$f(x)$的最小值为$f\left(\frac{1}{2}\right) = \frac{1}{2^{p-1}}$,最大值为$f(0) = f(1) = 1$,即
              $$
                  \frac{1}{2^{p-1}} \les x^p + (1-x)^p \les 1.
              $$
        \item 设$g(x) = \tan x - x + \frac{x^3}{3}$,则$g'(x) = \sec^2 x - 1 + x^2 = \tan^2 x + x^2 > 0$.因此$g(x)$在$(0, \frac{\pi}{2})$上单调递增,又$g(0) = 0$,故$g(x) > 0$,即
              $$
                  \tan x > x - \frac{x^3}{3}.
              $$
        \item 设$h(x) = \frac{\tan x}{x}$,
              则$h'(x) = \frac{x\sec^2 x - \tan x}{x^2} = \frac{x\tan^2 x}{x^2} > 0$.因此$h(x)$在$(0, \frac{\pi}{2})$上单调递增,故$h(x_2) > h(x_1)$,即
              $$
                  \frac{\tan x_2}{\tan x_1} > \frac{x_2}{x_1}.
              $$
        \item 设$f(x) = \ln(1+x) - \frac{\arctan x}{1+x}$,则
              $$f'(x) = \frac{\arctan x + x^2 \arctan x + x^3 + x^2}{x^4 + 2x^3 + 2x^2 + 2x +1} > 0 , \quad x > 0$$
              因此$f(x)$在$(0, +\infty)$上单调递增,又$f(0) = 0$,故$f(x) > 0$,即
              $$
                  \ln(1+x) > \frac{\arctan x}{1+x}.
              $$
        \item[(4)] \textbf{另解} 令 $f(x) = (1+x)\ln(1+x) - \arctan x, g(x) = \arctan x, \forall x > 0$,
            $$ \exists \xi_1 \in [0, x], \frac{f(x) - f(0)}{x-0} = \frac{f(x)}{x} = f'(\xi_1) = 1 + \ln(1+x) > 1. $$
            $$ \exists \xi_2 \in [0, x], \frac{g(x) - g(0)}{g-0} = \frac{g(x)}{x} = g'(\xi_2) = \frac{1}{1+\xi_2^2} < 1. $$
            故
            $$ g(x) < x < f(x). $$
            即
            $$ \ln(1+x) > \frac{\arctan x}{1+x}. $$
        \item 令 $f(x) = \ln(x+\sqrt{1+x^2}), \forall x \in \mathbb{R}$,
              $$ \exists \xi \in [0, x], \frac{f(x) - f(0)}{x-0} = \frac{f(x)}{x} = f'(\xi) = \frac{1}{\sqrt{1+\xi^2}} \ges \frac{1}{\sqrt{1+x^2}}. $$
              $$ \frac{\ln(x+\sqrt{1+x^2})}{x} \ges \frac{1}{\sqrt{1+x^2}} \Rightarrow x \ln(x+\sqrt{1+x^2}) \ges \frac{x^2}{\sqrt{1+x^2}} = \sqrt{1+x^2} - \frac{1}{\sqrt{1+x^2}} \ges \sqrt{1+x^2} - 1 $$
              即
              $$ 1+x \ln(x+\sqrt{1+x^2}) \ges \sqrt{1+x^2}. $$
        \item 令 $f(x) = x - \frac{4}{3} \sin x + \frac{1}{3} \sin x \cos x$, 则
              $$ f'(x) = 1 - \frac{4}{3} \cos x + \frac{1}{3} \cos 2x = \frac{1}{3} (2 \cos x - 1)^2 \ges 0 \Rightarrow g(x) > g(0) = 0. $$
              又 $\forall x \in (0, \frac{\pi}{2}), \sin x > 0,$
              $$ \frac{x}{\sin x} > \frac{4}{3} - \frac{1}{3} \cos x. $$
              另一方面, 由于
              $$ \lim_{x \to 0^+} \frac{x}{\sin x} - \frac{4}{3} + \frac{1}{3} \cos x = 0, $$
              $\frac{4}{3}$ 不能被替换为更大的数.
        \item $$ \left(1-\frac{1}{x}\right)^{x-1} \left(1+\frac{1}{x}\right)^{x+1} = \left(1-\frac{1}{x^2}\right)^{x-1} \left(1+\frac{1}{x}\right)^2 < \left(1+\frac{1}{x}\right)^2 < 4 $$
        \item 令 $f(x) = x^{a-1} + x^{a+1}$, 则 $\forall x \in (0, 1), a \in (0, 1)$,
              $$ f'(x) = (a-1)x^{a-2} + (a+1)x^a = (a+1) \left(x + \sqrt{\frac{1-a}{1+a}}\right) \left(x - \sqrt{\frac{1-a}{1+a}}\right) x^{a-2}. $$
              $$\forall x \in \left(0, \sqrt{\frac{1-a}{1+a}}\right), f'(x) < 0, f(x) > f\left(\sqrt{\frac{1-a}{1+a}}\right) = \left(\frac{1-a}{1+a}\right)^{\frac{a-1}{2}} + \left(\frac{1-a}{1+a}\right)^{\frac{a+1}{2}}.$$
              $$\forall x \in \left(\sqrt{\frac{1-a}{1+a}}, 1\right), f'(x) > 0, f(x) > f\left(\sqrt{\frac{1-a}{1+a}}\right) = \left(\frac{1-a}{1+a}\right)^{\frac{a-1}{2}} + \left(\frac{1-a}{1+a}\right)^{\frac{a+1}{2}}.$$

              综上,
              $\forall x \in (0, 1), a \in (0, 1), x^{a-1} + x^{a+1} \ges \left(\frac{1-a}{1+a}\right)^{\frac{a-1}{2}} + \left(\frac{1-a}{1+a}\right)^{\frac{a+1}{2}}.$
    \end{enumerate}
\end{solution}

\begin{exercise}[3.3.24]
    试确定下列函数零点的个数及所在范围:
    \begin{tasks}[label=(\arabic*)](2)
        \task $x^3-6x^2+9x-10$;
        \task $ax-\ln x$ (其中 $a>0$).
    \end{tasks}
\end{exercise}

\begin{solution}
    \begin{enumerate}
        \item
              令 $f(x) = x^3-6x^2+9x-10$.
              我们首先分析 $f(x)$ 的单调性. 求导得:
              $$ f'(x) = 3x^2 - 12x + 9 = 3(x^2 - 4x + 3) = 3(x-1)(x-3). $$
              令 $f'(x) = 0$, 解得 $x_1 = 1, x_2 = 3$.

              \begin{itemize}
                  \item 当 $x \in (-\infty, 1)$ 时, $f'(x) > 0$, $f(x)$ 单调递增.
                  \item 当 $x \in (1, 3)$ 时, $f'(x) < 0$, $f(x)$ 单调递减.
                  \item 当 $x \in (3, +\infty)$ 时, $f'(x) > 0$, $f(x)$ 单调递增.
              \end{itemize}

              $f(x)$ 在 $x=1$ 处取得极大值, $$f(1) = 1^3 - 6(1)^2 + 9(1) - 10 = 1 - 6 + 9 - 10 = -6.$$
              $f(x)$ 在 $x=3$ 处取得极小值, $$f(3) = 3^3 - 6(3)^2 + 9(3) - 10 = 27 - 54 + 27 - 10 = -10.$$

              同时, 考察 $x$ 趋于无穷时的极限:
              $$ \lim_{x \to -\infty} f(x) = -\infty $$
              $$ \lim_{x \to +\infty} f(x) = +\infty $$

              由于极大值 $f(1) = -6 < 0$, 且 $\lim_{x \to -\infty} f(x) = -\infty$, 所以在 $(-\infty, 1]$ 上 $f(x)$ 恒为负, 没有零点.
              由于极小值 $f(3) = -10 < 0$, 且极大值 $f(1) = -6 < 0$, 所以在 $(1, 3]$ 上 $f(x)$ 恒为负, 没有零点.
              由于 $f(3) = -10 < 0$ 且 $\lim_{x \to +\infty} f(x) = +\infty$, 且 $f(x)$ 在 $(3, +\infty)$ 上连续且单调递增, 根据零点存在定理, $f(x)$ 在 $(3, +\infty)$ 上有且仅有一个零点.

              又因为
              $$f(4) =  -6 < 0 ,\quad f(5) =  10 > 0 .$$
              所以零点在 $(4, 5)$ 范围内.

              该函数只有一个零点, 位于区间 $(4, 5)$ 内.

        \item
              令 $g(x) = ax - \ln x$. 函数的定义域为 $(0, +\infty)$. (题目已给出 $a>0$)
              我们分析 $g(x)$ 的单调性. 求导得:
              $$ g'(x) = a - \frac{1}{x} = \frac{ax - 1}{x}. $$
              令 $g'(x) = 0$, 解得 $x = \frac{1}{a}$.

              \begin{itemize}
                  \item 当 $x \in (0, 1/a)$ 时, $g'(x) < 0$, $g(x)$ 单调递减.
                  \item 当 $x \in (1/a, +\infty)$ 时, $g'(x) > 0$, $g(x)$ 单调递增.
              \end{itemize}

              因此, $g(x)$ 在 $x = 1/a$ 处取得全局最小值, 最小值为:
              $$ g(1/a) = a\left(\frac{1}{a}\right) - \ln\left(\frac{1}{a}\right) = 1 - \ln(a^{-1}) = 1 + \ln a. $$

              我们考察 $g(x)$ 在定义域边界的行为:
              $$ \lim_{x \to 0^+} g(x) = \lim_{x \to 0^+} (ax - \ln x) = 0 - (-\infty) = +\infty. $$
              $$ \lim_{x \to +\infty} g(x) = \lim_{x \to +\infty} (ax - \ln x) = \lim_{x \to +\infty} x \left(a - \frac{\ln x}{x}\right) = +\infty \cdot (a - 0) = +\infty. $$

              函数 $g(x)$ 从 $+\infty$ 递减到最小值 $1+\ln a$, 然后再递增到 $+\infty$. 零点的个数取决于最小值 $1+\ln a$ 的符号:

              \begin{enumerate}
                  \item[\textbf{情况 1}] 若最小值 $1 + \ln a > 0$, 即 $\ln a > -1$, $a > \e^{-1}$ ($a > 1/\e$).
                      此时 $g(x)$ 的最小值大于 0, $g(x)$ 恒大于 0, 故函数没有零点.

                  \item[\textbf{情况 2}] 若最小值 $1 + \ln a = 0$, 即 $\ln a = -1$, $a = \e^{-1}$ ($a = 1/\e$).
                      此时 $g(x)$ 的最小值等于 0, $g(x)$ 仅在 $x=1/a = \e$ 处与 $x$ 轴相切, 故函数有且仅有一个零点, 零点为 $x=\e$.

                  \item[\textbf{情况 3}] 若最小值 $1 + \ln a < 0$, 即 $\ln a < -1$, $0 < a < \e^{-1}$ ($0 < a < 1/\e$).
                      此时 $g(x)$ 的最小值小于 0.
                      由于 $\lim_{x \to 0^+} g(x) = +\infty$ 且 $g(1/a) < 0$, $g(x)$ 在 $(0, 1/a)$ 上连续且单调递减, 故在 $(0, 1/a)$ 内必有一个零点.
                      由于 $g(1/a) < 0$ 且 $\lim_{x \to +\infty} g(x) = +\infty$, $g(x)$ 在 $(1/a, +\infty)$ 上连续且单调递增, 故在 $(1/a, +\infty)$ 内必有一个零点.
                      故函数有两个零点.
              \end{enumerate}

              结论:
              \begin{itemize}
                  \item 若 $a > \e^{-1}$, 函数有 0 个零点.
                  \item 若 $a = \e^{-1}$, 函数有 1 个零点, 位于 $x = \e$.
                  \item 若 $0 < a < \e^{-1}$, 函数有 2 个零点, 一个位于 $(0, 1/a)$, 另一个位于 $(1/a, +\infty)$.
              \end{itemize}

    \end{enumerate}
\end{solution}

\begin{exercise}[3.3.25]
    设 $a \in (0,1), b_1=1-a$,
    $$ b_{n+1} = \frac{b_n}{1-\mathrm{e}^{-b_n}} - a, \quad n=1,2,\dots. $$
    问 $\{b_n\}$ 是否收敛? 若不收敛, 则给予证明, 若收敛, 则求极限.
\end{exercise}
\begin{solution}
    我们记$f(x)=\frac{x}{1-\mathrm{e}^{-x}}-a,x>0$,
    则有$b_{n+1}=f(b_n),n=1,2,\dots,$且$$f'(x)=\frac{1-\mathrm{e}^{-x}-x\mathrm{e}^{-x}}{(1-\mathrm{e}^{-x})^2} \ges 0 $$

    % 因为
    % \begin{align*}
    %     \e^{x}\ges 1+x\quad\Rightarrow\quad
    %     1\ges (1+x)\e^{-x}\quad\Rightarrow\quad
    %     1-\mathrm{e}^{-x}-x\mathrm{e}^{-x}&\ges 0
    % \end{align*}
    % 故$f'(x)\ges 0$.

    由Lagrange中值定理可知,存在$\xi_n\in(b_n,b_{n+1})$使得
    $$b_{n+2}-b_{n+1}=f(b_{n+1})-f(b_n)=f'(\xi_n)\left(b_{n+2}-b_{n+1}\right).$$

    据此可知,$\{b_n\}$是一个单调数列,增减性由$b_2-b_1$确定.

    再设$g(x)=\begin{cases}
        f(x)-x,&x>0\\
        1-a,&x=0
    \end{cases}$
    化简得到$x>0$时,$$g(x)=\frac{x}{\e^x-1}-a,\quad g'(x)=\frac{\e^x-1-x\e^x}{(\e^x-1)^2} \les 0,$$

    % 将$-x$替换$1-\mathrm{e}^{-x}-x\mathrm{e}^{-x}\ges 0$里的$x$可得到$1-\mathrm{e}^{x}+x\mathrm{e}^{x}\ges 0$,即$g'(x)\les 0$.
    
    可知$g(x)$在$x\ges0$处连续,递减.且由于$g(0)=1-a>0$,$g(1)=\frac{1}{1-\e^{-1}}-1-a<0$,故存在唯一的$c\in(0,1)$使得$g(c)=0$,即$f(c)=c$.

    又由于
    \begin{align*}
        b_2-b_1 = f(b_1)-b_1
        = g(b_1)
        = \frac{b_1}{\e^{b_1}-1}-a
        = \frac{1-a}{\e^{1-a}-1}-a
        = \frac{1-a\e^{1-a}}{\e^{1-a}-1} \ges 0.
    \end{align*}
    % 将$a-1$替换$1-\mathrm{e}^{-x}-x\mathrm{e}^{-x}\ges 0$里的$x$可得到$1-\mathrm{e}^{1-a}-(a-1)\mathrm{e}^{1-a}=1-a\e^{1-a}\ges 0$,即$g(b_1)=b_2-b_1\ges 0$.

    由上述分析可知$b_1<c$,$\{b_n\}$单调递增,并且$b_{n+1}-c=f(b_n)-f(c)=f'(\eta_n)\left(b_n-c\right)$,于是递推得到,$b_n<c$.

    综上,$\{b_n\}$单调有界,故收敛.设$\lim\limits_{n\to\infty}b_n=b$,则
    $$b=\lim\limits_{n\to\infty}b_{n+1}=\lim\limits_{n\to\infty}f(b_n)=f(b).$$
    故$b=c$.
\end{solution}

\newpage
