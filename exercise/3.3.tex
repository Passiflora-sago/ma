\section{习题3.3}

\begin{exercise}[3.3.1]
    设 $f(x)=(x-1)(x-2)(x-3)(x-4)$, 确定方程 $f'(x)=0$ 的实根的个数, 并指出根所在的区间.
\end{exercise}

\begin{exercise}[3.3.2]
    设函数 $f(x)$ 在区间 $[1,2]$ 上有二阶微商, 且 $f(1)=f(2)=0$. 记 $F(x)=(x-1)^2 f(x)$, 则在区间 $(1,2)$ 内至少有一点 $\xi$, 使得 $F''(\xi)=0$.
\end{exercise}

\begin{exercise}[3.3.3]
    举例说明, 中值定理的下述意义的逆不成立: 设 $\xi \in (a,b)$ 是指定的一点, 则存在 $c,d \in [a,b]$, 使得 $\frac{f(c)-f(d)}{c-d}=f'(\xi)$. (提示: 考虑函数 $f(x)=x^3, \xi=0$.)
\end{exercise}

\begin{exercise}[3.3.4]
    证明下列不等式:
    \begin{enumerate}
        \item 当 $a>b>0, n>1$ 时, 有 $nb^{n-1}(a-b) < a^n-b^n < na^{n-1}(a-b)$;
        \item 当 $x>0$ 时, 有 $\frac{x}{1+x} < \ln(1+x) < x$;
        \item 当 $0<a<b$ 时, 有 $(a+b)\ln\frac{a+b}{2} < a\ln a + b\ln b$.
        \item 当 $0<\alpha<\beta<\frac{\pi}{2}$ 时, 有 $\frac{\beta-\alpha}{\cos^2\alpha} < \tan\beta - \tan\alpha < \frac{\beta-\alpha}{\cos^2\beta}$.
    \end{enumerate}
\end{exercise}

\begin{exercise}[3.3.5]
    证明下列恒等式:
    \begin{enumerate}
        \item $\arctan x = \arcsin\frac{x}{\sqrt{1+x^2}}$;
        \item $\arctan x + \arctan\frac{1-x}{1+x} = \begin{cases} \frac{\pi}{4}, & x > -1, \\ -\frac{3\pi}{4}, & x < -1. \end{cases}$
    \end{enumerate}
\end{exercise}

\begin{exercise}[3.3.6]
    设 $f(x)$ 是闭区间 $[0,1]$ 上的可导函数, 对任意 $x \in [0,1]$ 有 $f(x) \in (0,1)$; 并且对每个 $x, f'(x) \ne 1$. 证明在 $(0,1)$ 内有且仅有一个 $x$, 使 $f(x)=x$.
\end{exercise}

\begin{exercise}[3.3.7]
    设函数 $f(x)$ 在 $[0,1]$ 上连续, 在 $(0,1)$ 内可导, 且 $|f'(x)| < 1$, 又 $f(0)=f(1)$. 证明: 对于 $[0,1]$ 上的任意两点 $x_1, x_2$, 有 $|f(x_1)-f(x_2)| < \frac{1}{2}$.
\end{exercise}

\begin{exercise}[3.3.8]
    若 $f(x)$ 处处可导, 且 $f'(x)=f(x)$. 证明 $f(x) = C\mathrm{e}^x, C$ 为任意常数.
\end{exercise}

\begin{exercise}[3.3.9]
    设不恒为常数的函数 $f(x)$ 在闭区间 $[a,b]$ 上连续, 在开区间 $(a,b)$ 内可导, 且 $f(a)=f(b)$. 证明在 $(a,b)$ 内存在一点 $\xi$, 使得 $f'(\xi)>0$.
\end{exercise}

\begin{exercise}[3.3.10]
    设函数 $f(x)$ 在 $[a, +\infty)$ 上可微, 且 $\lim_{x \to +\infty} f'(x) = 0$. 证明:
    \begin{enumerate}
        \item $\lim_{x \to +\infty} [f(x+1)-f(x)] = 0$;
        \item $\lim_{x \to +\infty} \frac{f(x)}{x} = 0$.
    \end{enumerate}
\end{exercise}

\begin{exercise}[3.3.11]
    证明: 若函数 $f(x)$ 在(有限)开区间 $(a,b)$ 内有有界的导函数, 则 $f(x)$ 在 $(a,b)$ 内也有界. 如果有限区间 $(a,b)$ 改为无穷区间, 结论还成立吗? 命题的逆命题是否成立?
\end{exercise}

\begin{exercise}[3.3.12]
    设对所有的实数 $x,y$, 不等式 $|f(x)-f(y)| \les M|x-y|^2$ ($M$ 为常数) 都成立. 证明: $f(x)$ 恒为常数.
\end{exercise}

\begin{exercise}[3.3.13]
    设函数 $f(x)$ 在区间 $[x_0, x_0+\delta]$ 上连续 (这里 $\delta>0$), 在 $(x_0, x_0+\delta)$ 内可导. 若 $\lim_{x \to x_0^+} f'(x) = l$ (这里的 $l$ 可以是无穷大), 则 $f(x)$ 在 $x_0$ 处的右导数也为 $l$, 即
    $$ f'_+(x_0) = \lim_{x \to x_0^+} f'(x). $$
    (将区间 $[x_0, x_0+\delta]$ 换为 $[x_0-\delta, x_0]$, 有类似的结论.)
\end{exercise}

\begin{exercise}[3.3.14]
    应用上一题的结论证明:
    \begin{enumerate}
        \item 函数 $x^{1/3}$ 在 $x=0$ 处不可导;
        \item 函数 $\arcsin x, \arccos x$ 在 $x=1$ 处没有左导数, 在 $x=-1$ 处没有右导数.
    \end{enumerate}
\end{exercise}

\begin{exercise}[3.3.15]
    证明: 若函数 $f(x)$ 在一个区间内处处可导, 则导函数 $f'(x)$ 不能有第一类间断点, 即在(区间内)每一点处, $f'(x)$ 或者连续, 或者有第二类间断. (由本题推出, 具有第一类间断点的函数, 如 $\operatorname{sgn} x$, 不能成为某个函数的导函数.)
\end{exercise}

\begin{exercise}[3.3.16]
    设 $f(x)$ 在一个区间 $I$ 上连续, 且(至多)除了有限个点外, $f(x)$ 在 $I$ 内部的导数为正(负), 则 $f(x)$ 在 $I$ 上严格单调递增(减). (注意, 在例外的点处, $f(x)$ 可能不可导.)
\end{exercise}

\begin{exercise}[3.3.17]
    设函数 $f(x)$ 和 $g(x)$ 均在区间 $I$ 上连续, 且(至多)除了有限个点外, $f(x)$ 和 $g(x)$ 在 $I$ 内部满足 $f'(x) > g'(x)$; 设存在 $a \in I$, 使得 $f(a)=g(a)$ ($a$ 不是区间端点), 则当 $x \in I$ 且 $x>a$ 时, 有 $f(x)>g(x)$; 当 $x \in I$ 且 $x<a$ 时, 有 $f(x)<g(x)$.
\end{exercise}

\begin{exercise}[3.3.18]
    若 $f(x)$ 在 $[0, +\infty)$ 可导, $f(0)=0$, $f'(x)$ 严格递增, 证明 $\frac{f(x)}{x}$ 在 $(0, +\infty)$ 严格递增.
\end{exercise}

\begin{exercise}[3.3.19]
    设 $x_0$ 是函数 $f(x)$ 的一个可疑极值点, 且 $f(x)$ 在 $x_0$ 处二阶可微, $f''(x_0) \ne 0$. 证明: 若 $f''(x_0)<0$, 则 $x_0$ 是 $f(x)$ 的一个极大值点; 若 $f''(x_0)>0$, 则 $x_0$ 是 $f(x)$ 的一个极小值点. (提示: 现在必有 $f'(x_0)=0$.)

    举例说明: 若 $f''(x_0)=0$, 则 $x_0$ 可以是 $f(x)$ 的极大值点或极小值点, 也可以不是极值点.
\end{exercise}

\begin{exercise}[3.3.20]
    设 $f(x)$ 在 $[0,1]$ 上有二阶导数, 且 $f(0)=f'(0), f(1)=f'(1)$. 求证: 存在 $\xi \in (0,1)$, 满足 $f(\xi)=f''(\xi)$.
\end{exercise}

\begin{exercise}[3.3.21]
    求下列函数的单调区间与极值.
    \begin{tasks}[label=(\arabic*)](2)
        \task $y = 2x^3-3x^2$;
        \task $y = x^{2/3}$;
        \task $y = x^2\mathrm{e}^{-x^2}$;
        \task $y = x^{1/x}$;
        \task $y = \frac{(\ln x)^2}{x}$;
        \task $y = \arctan x - \frac{1}{2}\ln(1+x^2)$.
    \end{tasks}
\end{exercise}

\begin{exercise}[3.3.22]
    求下列函数在所给区间上的最大值和最小值.
    \begin{tasks}[label=(\arabic*)](2)
        \task $y = x^4 - 2x^2+5$, $[ -2, 2]$;
        \task $y = \sin 2x - x$, $[-\frac{\pi}{2}, \frac{\pi}{2}]$;
        \task $y = \arctan\frac{1-x}{1+x}$, $[0,1]$;
        \task $y = x\ln x$, $(0, +\infty)$.
    \end{tasks}
\end{exercise}

\begin{exercise}[3.3.23]
    证明下列不等式:
    \begin{enumerate}
        \item $\frac{1}{2^{p-1}} \les x^p+(1-x)^p \les 1, \quad x \in [0,1], p>1$;
        \item $\tan x > x - \frac{x^3}{3}, \quad x \in (0, \frac{\pi}{2})$;
        \item $\frac{\tan x_2}{\tan x_1} > \frac{x_2}{x_1}, \quad 0<x_1<x_2<\frac{\pi}{2}$;
        \item $\ln(1+x) > \frac{\arctan x}{1+x}, \quad x>0$;
        \item $1+x\ln(x+\sqrt{1+x^2}) \ges \sqrt{1+x^2}, \quad x$ 为任意实数;
        \item $\frac{x}{\sin x} > \frac{4}{3} - \frac{1}{3}\cos x, \quad x \in (0, \frac{\pi}{2})$, 且右端的常数 $\frac{4}{3}$ 不能换为更大的数;
        \item $(1-\frac{1}{x})^{x-1} (1+\frac{1}{x})^x < 4, \quad x \in (1, +\infty)$;
        \item $x^{a-1}+x^{a+1} \ges \left(\frac{1-a}{1+a}\right)^{\frac{a-1}{2}} + \left(\frac{1-a}{1+a}\right)^{\frac{a+1}{2}}, \quad x \in (0,1), a \in (0,1)$.
    \end{enumerate}
\end{exercise}

\begin{exercise}[3.3.24]
    试确定下列函数零点的个数及所在范围:
    \begin{tasks}[label=(\arabic*)](2)
        \task $x^3-6x^2+9x-10$;
        \task $ax-\ln x$ (其中 $a>0$).
    \end{tasks}
\end{exercise}

\begin{exercise}[3.3.25]
    设 $a \in (0,1), b_1=1-a$,
    $$ b_{n+1} = \frac{b_n}{1-\mathrm{e}^{-b_n}} - a, \quad n=1,2,\dots. $$
    问 $\{b_n\}$ 是否收敛? 若不收敛, 则给予证明, 若收敛, 则求极限.
\end{exercise}


\newpage
