\section{第 5 章综合习题}

\begin{exercise}[5.C.1]
    设 $m, n$ 为正整数, 证明:
    \begin{enumerate}[(1)]
        \item $\displaystyle \int_0^{2\pi} \sin mx \cos nx \mathrm{d}x = 0;$
        \item $\displaystyle \int_0^{2\pi} \sin mx \sin nx \mathrm{d}x = \int_0^{2\pi} \cos mx \cos nx \mathrm{d}x = \begin{cases} \pi, & \text{如 } m = n, \\ 0, & \text{如 } m \neq n. \end{cases}$
    \end{enumerate}
    (本题的结论, 在以后要讲的 Fourier (傅里叶) 级数理论中具有基本的重耍性.)
\end{exercise}

\begin{solution}
    \begin{enumerate}
        \item \begin{align*}
                  \int_0^{2\pi} \sin mx \cos nx \mathrm{d}x & = \frac{1}{2} \int_0^{2\pi} [\sin (m+n)x + \sin (m-n)x] \mathrm{d}x \\
                                                            & = 0
              \end{align*}
        \item \begin{align*}
                  \int_0^{2\pi} \sin mx \sin nx \mathrm{d}x & = \frac{1}{2} \int_0^{2\pi} [\cos (m-n)x - \cos (m+n)x] \mathrm{d}x               \\
                                                            & = \begin{cases} \pi, & \text{如 } m = n, \\ 0, & \text{如 } m \neq n. \end{cases}
              \end{align*}
    \end{enumerate}
\end{solution}

\begin{exercise}[5.C.2]
    设 $m, n$ 为正整数, 记
    $$ \mathrm{B}(m, n) = \int_0^1 x^{m-1} (1 - x)^{n-1} \mathrm{d}x. $$
    证明: (1) $\mathrm{B}(m, n) = \mathrm{B}(n, m)$; (2) $\mathrm{B}(m, n) = \displaystyle \frac{m! n!}{(m + n + 1)!}$.
\end{exercise}

\begin{solution}
    \begin{enumerate}
        \item 由变量替换 $x = 1 - t$ 可得
              \begin{align*}
                  \mathrm{B}(m, n) & = \int_0^1 (1 - t)^{m-1} t^{n-1} \mathrm{d}t \\
                                   & = \mathrm{B}(n, m)
              \end{align*}
        \item 由分部积分可得
              \begin{align*}
                  \mathrm{B}(m, n) & = \int_0^1 x^{m-1} (1 - x)^{n-1} \mathrm{d}x                                                                 \\
                                   & = \left. -\frac{1}{n} x^{m-1} (1 - x)^n \right|_0^1 + \frac{m - 1}{n} \int_0^1 x^{m-2} (1 - x)^n \mathrm{d}x \\
                                   & = \frac{m - 1}{n} \mathrm{B}(m - 1, n + 1)
              \end{align*}
              重复使用上式, 可得
              \begin{align*}
                  \mathrm{B}(m, n) & = \frac{(m - 1)(m - 2) \cdots 1}{n (n + 1) \cdots (n + m - 1)} \mathrm{B}(1, n + m - 1) \\
                                   & = \frac{(m - 1)!}{(n + m - 1)! / (n - 1)!} \cdot \frac{1}{n + m - 1}                    \\
                                   & = \frac{m! n!}{(m + n - 1)! (m + n)}                                                    \\
                                   & = \frac{m! n!}{(m + n + 1)!}
              \end{align*}
    \end{enumerate}
\end{solution}

\begin{exercise}[5.C.3]
    计算下列积分.
    \begin{tasks}(1)
        \task $\displaystyle \int_{\frac12}^2 \left( 1 + x - \frac{1}{x} \right) \e^{x+ \frac{1}{x}} \mathrm{d}x;$
        \task $\displaystyle \int_{0}^{n\pi} x| \sin x| \mathrm{d}x$ ($n$ 为自然数);
        \task 设 $f(x) = \displaystyle \int_x^{x+2\pi} (1 + \mathrm{e}^{\sin t} - \mathrm{e}^{-\sin t}) \mathrm{d}t + \frac{1}{1 + x} \int_0^x f(t) \mathrm{d}t$, 求 $\displaystyle \int_0^1 f(x) \mathrm{d}x.$
    \end{tasks}
\end{exercise}

\begin{solution}
    \begin{enumerate}
        \item \begin{align*}
                  \int_{\frac{1}{2}}^2 \left( 1 + x - \frac{1}{x} \right) \mathrm{e}^{x + \frac{1}{x}} \mathrm{d}x & = \int_{\frac{1}{2}}^2 \mathrm{d} \left( x \mathrm{e}^{x + \frac{1}{x}} \right) \\
                                                                                                                   & = x \mathrm{e}^{x + \frac{1}{x}} \Big|_{\frac{1}{2}}^2                          \\
                                                                                                                   & = \frac{3}{2} \mathrm{e}^{\frac{5}{2}}.
              \end{align*}
        \item \begin{align*}
                  \int_0^{n\pi} |x \sin x| \mathrm{d}x & = \sum_{k=0}^{n-1} \int_{k\pi}^{(k+1)\pi} x |\sin x| \mathrm{d}x                                             \\
                                                       & \stackrel{x=u+k\pi}{=} \sum_{k=0}^{n-1} \int_0^\pi (u + k\pi) |\sin (u + k\pi)| \mathrm{d}u                  \\
                                                       & = \sum_{k=0}^{n-1} \int_0^\pi (u + k\pi) \sin u \, \mathrm{d}u                                               \\
                                                       & = \sum_{k=0}^{n-1} \left( \int_0^\pi u \sin u \, \mathrm{d}u + k\pi \int_0^\pi \sin u \, \mathrm{d}u \right) \\
                                                       & = \sum_{k=0}^{n-1} (\pi + 2k\pi)                                                                             \\
                                                       & = n\pi + 2\pi \frac{(n-1)n}{2} = n^2 \pi.
              \end{align*}
        \item $\left( 1 + \e^{\sin t} - \e^{-\sin t} \right)$ 为以 $2\pi$ 为周期的函数, 由\hyperref[ex:5.1.21]{习题 5.1.21} 可知
              \begin{align*}
                  \int_x^{x+2\pi} \left( 1 + \mathrm{e}^{\sin t} - \mathrm{e}^{-\sin t} \right) \mathrm{d}t & = \int_0^{2\pi} \left( 1 + \mathrm{e}^{\sin t} - \mathrm{e}^{-\sin t} \right) \mathrm{d}t                              \\
                                                                                                            & = 2\pi + \int_0^{2\pi} \mathrm{e}^{\sin t} \mathrm{d}t - \int_0^{2\pi} \mathrm{e}^{-\sin t} \mathrm{d}t                \\
                                                                                                            & = 2\pi + \int_0^{2\pi} \mathrm{e}^{\sin t} \mathrm{d}t + \int_0^{-2\pi} \mathrm{e}^{\sin u} \mathrm{d}u \quad (u = -t) \\
                                                                                                            & = 2\pi + \int_0^{2\pi} \mathrm{e}^{\sin t} \mathrm{d}t + \int_{2\pi}^{0} \mathrm{e}^{\sin u} \mathrm{d}u               \\
                                                                                                            & = 2\pi.
              \end{align*}
              因此$f(x) = 2\pi + \frac{1}{x+1} \int_0^1 f(t) \dif t$,$f(x)$在$[0,1]$上可积,两边对$x$从$0$到$1$积分得
              \begin{align*}
                  I & = \int_0^1 f(x) \dif x = \int_0^1 \left( 2\pi + \frac{1}{x+1} I \right) \dif x \\
                    & = 2\pi + I \int_0^1 \frac{1}{x+1} \dif x = 2\pi + I \ln 2.
              \end{align*}
              解得 $I = \displaystyle \frac{2\pi}{1 - \ln 2}$.
    \end{enumerate}
\end{solution}

\begin{exercise}[5.C.4]
    证明: $\displaystyle \frac{1}{2n + 2} < \int_0^{\frac{\pi}{4}} \tan^n x \mathrm{d}x < \frac{1}{2n}$ ($n = 1, 2, \dots$).
\end{exercise}

\begin{solution}
    \begin{align*}
        \int_0^{\frac{\pi}{4}} \tan^n x \mathrm{d}x & = \int_0^{\frac{\pi}{4}} \tan^{n-1} x \cdot \frac{2\sin x \cos x}{2\cos^2 x} \dif x \les \int_0^{\frac{\pi}{4}} \tan^{n-1} x \cdot \frac{1}{2\cos^2 x} \dif x = \frac{1}{2n} \tan^n x \Big|_0^{\frac{\pi}{4}} =  \frac{1}{2n}; \\
        \int_0^{\frac{\pi}{4}} \tan^n x \mathrm{d}x & = \int_0^{\frac{\pi}{4}} \tan^n x \cdot \frac{2\cos^2 x}{2\cos^2 x} \dif x \ges \int_0^{\frac{\pi}{4}} \tan^n x \cdot \frac{1}{2\cos^2 x} \dif x = \frac{1}{2n + 2} \tan^{n+1} x \Big|_0^{\frac{\pi}{4}} =  \frac{1}{2n + 2}.
    \end{align*}
\end{solution}

\begin{exercise}[5.C.5]
    设函数 $f$ 在 $[a, b]$ 上可积, 且 $\displaystyle \int_a^b f(x) \mathrm{d}x > 0$, 证明: 必有一个区间 $[\alpha, \beta] \subset [a, b]$, 使得对任意 $x \in [\alpha, \beta]$, 有 $f(x) > 0$. (比较习题 5.1 中第 8 题.)

    (提示: 假设结论不成立, 则对 $[a, b]$ 的任一分割 $a = x_0 < x_1 < \dots < x_{n-1} < x_n = b$, 在区间 $[x_{i-1}, x_i]$ 上都存在 $\xi_i$, 使 $f(\xi_i) \les 0$. 由此产生一个非正的积分和, 过渡到极限, 产生矛盾.)
\end{exercise}

\begin{solution}
    反证法. 假设结论不成立, 则对 $[a, b]$ 的任一分割 $a = x_0 < x_1 < \dots < x_{n-1} < x_n = b$, 在区间 $[x_{i-1}, x_i]$ 上都存在 $\xi_i$, 使 $f(\xi_i) \les 0$. 由此产生一个非正的积分和
    $$ S = \sum_{i=1}^n f(\xi_i) (x_i - x_{i-1}) \les 0. $$
    过渡到极限, 可得
    $$ \int_a^b f(x) \mathrm{d}x = \lim_{\max (x_i - x_{i-1}) \to 0} S \les 0, $$
    这与题设 $\displaystyle \int_a^b f(x) \mathrm{d}x > 0$ 矛盾. 因此假设不成立, 结论成立.
\end{solution}

\begin{exercise}[5.C.6]
    \begin{enumerate}[(1)]
        \item 设 $f$ 是处处连续的偶函数, 证明: $f$ 必有一个原函数为奇函数;
        \item 设 $f$ 是处处连续的奇函数, 证明: $f$ 的任一原函数都是偶函数. (试比较习题 3.1 第 15 题.)
    \end{enumerate}
\end{exercise}

\begin{solution}
    \begin{enumerate}[(1)]
        \item 设 $F(x)$ 是 $f$ 的一个原函数, 则
              \begin{align*}
                  G(x) & = F(x) - F(0)                                     \\
                       & = \int_0^x f(t) \mathrm{d}t                       \\
                       & = \int_0^{-x} f(-u) (-\mathrm{d}u) \quad (u = -t) \\
                       & = -\int_0^{-x} f(u) \mathrm{d}u                   \\
                       & = -F(-x) + F(0)
              \end{align*}
              因此 $G(-x) = -G(x)$, 即 $G(x)$ 是奇函数,且$G'(x) = F'(x) = f(x)$.
        \item 设 $F(x)$ 是 $f$ 的一个原函数, 则
              \begin{align*}
                  F(-x) & = \int_0^{-x} f(t) \mathrm{d}t                 \\
                        & = \int_0^x f(-u) (-\mathrm{d}u) \quad (u = -t) \\
                        & = -\int_0^x (-f(u)) \mathrm{d}u                \\
                        & = \int_0^x f(u) \mathrm{d}u                    \\
                        & = F(x)
              \end{align*}
              因此 $F(-x) = F(x)$, 即 $F(x)$ 是偶函数.
    \end{enumerate}
\end{solution}

\begin{exercise}[5.C.7]
    举例说明, 存在一个连续的周期函数 $f$, 使得 $f$ 的原函数都不是周期函数.

    (试比较习题 3.1 第 16 题.) (提示: 选一个连续的周期函数 $f$, 使它能保证 $F(x) = \displaystyle \int_0^x f(t) \mathrm{d}t$ 不是周期函数. 注意, 不必考虑 $F(x)$ 的显式表示.)
\end{exercise}

\begin{solution}
    设$f(x) \equiv 1$, 则$f$为周期函数. $f(x)$的所有原函数为$F(x) = x + C$, 显然$F(x)$不是周期函数.
\end{solution}

\begin{exercise}[5.C.8]
    设 $\displaystyle \frac{a_0}{n+1} + \frac{a_1}{n} + \dots + a_n = 0$. 证明: 多项式 $a_0 x^n + a_1 x^{n-1} + \dots + a_n$ 在 $(0, 1)$ 内至少有一个零点. (本题是第 3 章综合习题的第 3 题, 这里要求用积分的解法来论证. 可利用习题 5.1 中第 8 题的结果.)
\end{exercise}

\begin{solution}
    设 $f(x) = a_0 x^n + a_1 x^{n-1} + \dots + a_n$, 则
    \begin{align*}
        \int_0^1 f(x) \mathrm{d}x & = \int_0^1 (a_0 x^n + a_1 x^{n-1} + \dots + a_n) \mathrm{d}x \\
                                  & = \frac{a_0}{n+1} + \frac{a_1}{n} + \dots + a_n              \\
                                  & = 0
    \end{align*}
    由\hyperref[ex:5.1.8]{习题 5.1.8}可知, $f(x)$ 在 $(0, 1)$ 内至少有一个零点.
\end{solution}

\begin{exercise}[5.C.9]
    设函数 $f(x)$ 在 $[0, \pi]$ 上连续, 且有 $\displaystyle \int_0^\pi f(x) \sin x \mathrm{d}x = \int_0^\pi f(x) \cos x \mathrm{d}x = 0$. 试证: 在 $(0, \pi)$ 内存在两点 $x_1$ 和 $x_2$, 使得 $f(x_1) = 0$ 与 $f(x_2) = 0$.

    (提示: 易知 $f$ 在 $(0, \pi)$ 内至少有一个零点 $x_1$. 若这是唯一的零点, 则 $f$ 在 $(0, x_1)$ 与 $(x_1, \pi)$ 内异号. 于是 $\displaystyle \int_0^\pi f(x) \sin(x - x_1) \mathrm{d}x \neq 0$, 这将产生矛盾.)
\end{exercise}

\begin{solution}
    由$\int_0^\pi f(x) \sin x \mathrm{d}x = 0$与\hyperref[ex:5.1.8]{习题 5.1.8}可知, $f$在$(0,\pi)$内至少有一个零点$x_1$.若这是唯一的零点,则
    \begin{enumerate}
        \item 若$f(x)$在$(0,x_1)$与$(x_1,\pi)$内同号且非零,不妨设$f(x) > 0,\forall x \in (0,x_1) \cup (x_1,\pi)$,则
              \begin{align*}
                  \int_0^\pi f(x) \sin x \mathrm{d}x & = \int_0^{x_1} f(x) \sin x \mathrm{d}x + \int_{x_1}^\pi f(x) \sin x \mathrm{d}x > 0
              \end{align*}
              与题设矛盾.
        \item 若$f(x)$在$(0,x_1)$与$(x_1,\pi)$内异号且非零,构造函数$g(x) = \sin(x_1 - x) = - \sin x \cos x_1 +  \cos x \sin x_1$,不妨设$f(x) > 0,\forall x \in (0,x_1)$;$f(x) < 0,\forall x \in (x_1,\pi)$,因此
              \begin{align*}
                  \int_0^\pi f(x) g(x) \mathrm{d}x & = \int_0^{x_1} f(x) g(x) \mathrm{d}x + \int_{x_1}^\pi f(x) g(x) \mathrm{d}x > 0
              \end{align*}
              这与\begin{align*}
                  \int_0^\pi f(x) g(x) \mathrm{d}x & =  - \cos x_1 \int_0^\pi f(x) \sin x \mathrm{d}x + \sin x_1 \int_0^\pi f(x) \cos x \mathrm{d}x = 0
              \end{align*}矛盾.
    \end{enumerate}
\end{solution}

\begin{exercise}[5.C.10]
    设 $f(x)$ 处处连续, $f(0)=0, f'(0)$ 存在. 记 $F(x) = \displaystyle \int_0^1 f(xy) \mathrm{d}y$, 证明 $F(x)$ 处处可导, 并求出 $F'(x)$.
\end{exercise}

\begin{solution}
    $F(0) = \int_0^1 f(0) \mathrm{d}y = 0$,因此$F(x)$在$x=0$处连续,且
    $$\lim_{x \to 0} \frac{F(x) - F(0)}{x - 0} = \lim_{x \to 0} \frac{1}{x} \int_0^1 f(xy) \mathrm{d}y = \lim_{x \to 0} \frac{1}{x^2} \int_0^x f(t) \mathrm{d}t = \lim_{x \to 0} \frac{f(x)}{2x} = \frac{f'(0)}{2}.$$
    当 $x \neq 0$ 时,$F(x) = \int_0^1 f(xy) \mathrm{d}y \overset{t = xy}{=}\frac{1}{x} \int_0^x f(t)  \mathrm{d}t$,则
    $$F'(x) = -\frac{1}{x^2} \int_0^x f(t) \mathrm{d}t + \frac{1}{x} f(x)$$.
    综上所述,$F(x)$处处可导,且
    $$F'(x) = \begin{cases} \frac{f'(0)}{2}, & x = 0, \\ -\frac{1}{x^2} \int_0^x f(t) \mathrm{d}t + \frac{1}{x} f(x), & x \neq 0. \end{cases}$$
\end{solution}

\begin{exercise}[5.C.11]
    \begin{enumerate}[(1)]
        \item 设 $\displaystyle f(x) = \begin{cases} \mathrm{e}^{-x^2}, & |x| \les 1, \\ 1, & |x| > 1, \end{cases}$ 记 $F(x) = \displaystyle \int_0^x f(t) \mathrm{d}t$, 试研究 $F(x)$ 在哪些点可导;

              (提示: 与习题 5.1 中第 16 题不同, 本题无法求出 $F(x)$ 的显式表示.)

        \item 设 $f(x) = \displaystyle \int_0^x \cos \frac{1}{t} \mathrm{d}t$, 求证: $f'_+(0) = 0.$
    \end{enumerate}
\end{exercise}

\begin{solution}
    \begin{enumerate}
        \item 当 $x \neq \pm 1$ 时, $f(x)$ 在 $x$ 处连续,$F(x)$ 在 $x$ 处可导,且 $F'(x) = f(x)$. 当 $x = 1$ 时,
              \begin{align*}
                  F'_+(1) & = \lim_{h \to 0^+} \frac{F(1 + h) - F(1)}{h} = \lim_{h \to 0^+} \frac{1}{h} \int_1^{1+h} f(t) \mathrm{d}t = \lim_{h \to 0^+} \frac{1}{h} \int_1^{1+h} 1 \mathrm{d}t = 1; \\
                  F'_-(1) & = \lim_{h \to 0^-} \frac{F(1 + h) - F(1)}{h} = \lim_{h \to 0^-} \frac{1}{h} \int_1^{1+h} f(t) \mathrm{d}t = \lim_{h \to 0^-} \frac{1}{h} \int_1^{1+h} \mathrm{e}^{-t^2} \mathrm{d}t = \mathrm{e}^{-1}.
              \end{align*}
              因此 $F(x)$ 在 $x = 1$ 处不可导, 同理可证 $F(x)$ 在 $x = -1$ 处不可导.
        \item 由于$x=0$是被积函数的间断点,因此不能使用变上限积分求导的方法来求$F'(0)$,只能根据定义计算,先由分部积分得
        $$f(x) = \int_0^x - t^2 \diff \left( \sin \frac{1}{t} \right) = - x^2 \sin \frac{1}{x} + \int_0^x 2t \sin \frac{1}{t} \mathrm{d}t.$$
        因此
        \begin{align*}
            f'_+(0) & = \lim_{x \to 0^+} \frac{f(x) - f(0)}{x - 0} = \lim_{x \to 0^+} \frac{f(x)}{x} = \lim_{x \to 0^+} \left( - x \sin \frac{1}{x} + \frac{1}{x} \int_0^x 2t \sin \frac{1}{t} \mathrm{d}t \right) \\
                    & = \lim_{x \to 0^+} - x \sin \frac{1}{x} + \lim_{x \to 0^+} \frac{1}{x} \int_0^x 2t \sin \frac{1}{t} \mathrm{d}t.
        \end{align*}
        其中
        \begin{align*}
            \left| \lim_{x \to 0^+} - x \sin \frac{1}{x} \right| & \les \lim_{x \to 0^+} |x| = 0; \\
            \left| \lim_{x \to 0^+} \frac{1}{x} \int_0^x 2t \sin \frac{1}{t} \mathrm{d}t \right| & \les \lim_{x \to 0^+} \frac{1}{x} \int_0^x 2|t| \mathrm{d}t = \lim_{x \to 0^+} x = 0.
        \end{align*}
        因此 $f'_+(0) = 0$.
    \end{enumerate}
\end{solution}

\begin{exercise}[5.C.12]
    设函数 $f$ 处处连续. 证明:
    $$ \lim_{h \to 0} \frac{1}{h} \int_a^b (f(x + h) - f(x)) \mathrm{d}x = f(b) - f(a). $$
\end{exercise}

\begin{exercise}[5.C.13]
    设函数 $f(x)$ 在 $[a, b]$ 上连续可微. 证明:
    $$ \lim_{\lambda \to \infty} \int_a^b f(x) \sin \lambda x \mathrm{d}x = 0. $$
    (提示: 分部积分.)
\end{exercise}

我们加强为证明
\begin{exercise}
    设函数 $f(x)$ 在 $[a, b]$ 上连续可微,$g(x)$为以$T$为周期的连续函数,则
    $$ \lim_{\lambda \to \infty} \int_a^b f(x) g(\lambda x) \mathrm{d}x = \frac{1}{T} \int_0^T g(t) \mathrm{d}t \int_a^b f(x) \mathrm{d}x. $$
\end{exercise}

\begin{solution}
    等价于证明,$$\lim_{\lambda \to \infty} \int_a^b f(x) \left( g(\lambda x) - \frac{1}{T} \int_0^T g(t) \mathrm{d}t \right) \mathrm{d}x = 0.$$
    记$h(t) = g(t) - \frac{1}{T} \int_0^T g(t) \mathrm{d}t$,则$h(t)$为以$T$为周期的连续函数,且$\int_0^T h(t) \mathrm{d}t = 0$,
    $$h(\lambda t) = g(\lambda t) - \frac{1}{T} \int_0^T g(\lambda t) \mathrm{d}t= g(\lambda t) - \frac{1}{T} \int_0^T g(t) \mathrm{d}t.$$
    因此只需证明$$ \lim_{\lambda \to \infty} \int_a^b f(x) h(\lambda x) \mathrm{d}x = 0.$$
    由分部积分可得
    \begin{align*}
        \int_a^b f(x) h(\lambda x) \mathrm{d}x & = \left. f(x) \frac{1}{\lambda} H(\lambda x) \right|_a^b - \frac{1}{\lambda} \int_a^b f'(x) H(\lambda x) \mathrm{d}x
    \end{align*}
    其中$H(t)$为$h(t)$的一个原函数,由$h(t)$的连续性可知$H(t)$在$[0,T]$上有界,设$M = \max_{t \in [0,T]} |H(t)|$,则
    \begin{align*}
        \left| \int_a^b f(x) h(\lambda x) \mathrm{d}x \right| & \les \frac{1}{\lambda} \left( |f(b)| + |f(a)| \right) M + \frac{1}{\lambda} \int_a^b |f'(x)| M \mathrm{d}x \\
                                                             & = \frac{M}{\lambda} \left( |f(b)| + |f(a)| + \int_a^b |f'(x)| \mathrm{d}x \right)
    \end{align*}
    当$\lambda \to \infty$时,上式右端趋于0,故
    $$ \lim_{\lambda \to \infty} \int_a^b f(x) h(\lambda x) \mathrm{d}x = 0.$$
\end{solution}

\begin{exercise}[5.C.14]
    证明: $\displaystyle \lim_{x \to +\infty} \frac{1}{x} \int_0^x |\sin t| \mathrm{d}t = \frac{2}{\pi}.$
\end{exercise}

\begin{solution}
    设 $x = 2k\pi + r$, 其中 $k$ 为非负整数, $0 \les r < 2\pi$, 则
    \begin{align*}
        \frac{1}{x} \int_0^x |\sin t| \mathrm{d}t & = \frac{1}{2k\pi + r} \left( k \int_0^{2\pi} |\sin t| \mathrm{d}t + \int_0^r |\sin t| \mathrm{d}t \right) \\
                                                  & = \frac{1}{2k\pi + r} \left( 4k + \int_0^r |\sin t| \mathrm{d}t \right)                                   \\
                                                  & = \frac{2}{\pi} \cdot \frac{2k}{2k + r / \pi} + \frac{1}{2k\pi + r} \int_0^r |\sin t| \mathrm{d}t
    \end{align*}
    当 $x \to +\infty$ 时, 即 $k \to +\infty$,
    $$ \lim_{x \to +\infty}\frac{1}{2k\pi + r} \int_0^r |\sin t| \mathrm{d}t \les \lim_{k \to +\infty} \frac{1}{2k\pi} \int_0^{2\pi} |\sin t| \mathrm{d}t = 0,$$
    故
    $$ \lim_{x \to +\infty} \frac{1}{x} \int_0^x |\sin t| \mathrm{d}t = \frac{2}{\pi} + 0 = \frac{2}{\pi}. $$
\end{solution}

\begin{exercise}[5.C.15]
    证明: $\displaystyle \lim_{n \to \infty} \int_0^{\frac{\pi}{2}} \sin^n x \mathrm{d}x = 0.$

    (提示: 直接用积分中值定理, 得出左端的积分为 $\displaystyle \frac{\pi}{2} \sin^n \xi_n$ (其中 $0 < \xi_n < \frac{\pi}{2}$), 但这不能导出结果, 因不能排除 $\{\xi_n\}$ 中有一个子列趋于 $\frac{\pi}{2}$. 克服这一困难可采用如下的方法. 对任意正数 $\varepsilon < 1$, 取一个参数 $\delta$ (与 $\varepsilon$ 有关), 将问题中的积分拆成两部分: 一部分用区间长度控制, 另一部分由 $n \to \infty$ 来控制, 以使得两者的和小于 $\varepsilon$. 我们有
    $$ \int_0^{\frac{\pi}{2}} \sin^n x \mathrm{d}x = \int_0^{\frac{\pi}{2} - \delta} \sin^n x \mathrm{d}x + \int_{\frac{\pi}{2} - \delta}^{\frac{\pi}{2}} \sin^n x \mathrm{d}x $$
    $$ < \left( \frac{\pi}{2} - \delta \right) \left(\sin \left( \frac{\pi}{2} - \delta \right) \right)^n + \delta $$
    $$ < \frac{\pi}{2} \left( \cos \delta \right)^n + \delta. $$
    现在取 $\delta = \frac{\varepsilon}{2}$, 因 $0 < \cos \delta < 1$, 且 $\cos \delta$ 与 $n$ 无关, 故 $n$ 充分大时, 可使上式右端第一项小于 $\frac{\varepsilon}{2}$.

    解答本题的另一方法是应用 \textsection 5.1 中例 5.1.10, 并参考第 1 章综合习题中第 1 题 (1).)
\end{exercise}

\begin{solution}
    对任意正数 $\varepsilon < 1$, 取一个参数 $\delta = \frac{\varepsilon}{2}$, 将问题中的积分拆成两部分:
    $$ \int_0^{\frac{\pi}{2}} \sin^n x \mathrm{d}x = \int_0^{\frac{\pi}{2} - \delta} \sin^n x \mathrm{d}x + \int_{\frac{\pi}{2} - \delta}^{\frac{\pi}{2}} \sin^n x \mathrm{d}x $$
    $$ < \left( \frac{\pi}{2} - \delta \right) \left(\sin \left( \frac{\pi}{2} - \delta \right) \right)^n + \delta $$
    $$ < \frac{\pi}{2} \left( \cos \delta \right)^n + \delta. $$
    因 $0 < \cos \delta < 1$, 且 $\cos \delta$ 与 $n$ 无关, 故 $n$ 充分大时, 可使上式右端第一项小于 $\frac{\varepsilon}{2}$.
    因此当 $n$ 充分大时, 有
    $$ \int_0^{\frac{\pi}{2}} \sin^n x \mathrm{d}x < \frac{\varepsilon}{2} + \frac{\varepsilon}{2} = \varepsilon. $$
    由 $\varepsilon$ 的任意性可知
    $$ \lim_{n \to \infty} \int_0^{\frac{\pi}{2}} \sin^n x \mathrm{d}x = 0. $$
\end{solution}

\begin{exercise}[5.C.16]
    设 $f(x)$ 是 $[a, b]$ 上的连续函数, 且 $f(x) \ges 0$ (对 $x \in [a, b]$). 记 $f(x)$ 在该区间上的最大值为 $M$, 证明:
    $$ \lim_{n \to \infty} \left( \int_a^b f^n(x) \mathrm{d}x \right)^{\frac{1}{n}} = M. $$
\end{exercise}

\begin{solution}
    由$f(x)$在$[a,b]$上连续可知,$\forall \varepsilon > 0$,存在$\delta > 0$,使得当$|x - x_0| < \delta$时,$|f(x) - f(x_0)| < \varepsilon$,其中$x_0 \in [a,b]$且$f(x_0) = M$.因此,当$x \in (x_0 - \delta, x_0 + \delta) \cap [a,b]$时,$f(x) > M - \varepsilon$.设$I = (x_0 - \delta, x_0 + \delta) \cap [a,b]$,记$|I| = \sup I - \inf I > 0$,则有
    $$\int_a^b f^n(x) \mathrm{d}x \ges \int_I f^n(x) \mathrm{d}x \ges \int_I (M - \varepsilon)^n \mathrm{d}x = |I| (M - \varepsilon)^n.$$
    因此
    $$\left( \int_a^b f^n(x) \mathrm{d}x \right)^{\frac{1}{n}} \ges |I|^{\frac{1}{n}} (M - \varepsilon).$$
    又因为$f(x) \les M$,所以
    $$\left( \int_a^b f^n(x) \mathrm{d}x \right)^{\frac{1}{n}} \les (b - a)^{\frac{1}{n}} M.$$
    综上所述,有
    $$|I|^{\frac{1}{n}} (M - \varepsilon) \les \left( \int_a^b f^n(x) \mathrm{d}x \right)^{\frac{1}{n}} \les (b - a)^{\frac{1}{n}} M.$$
    当$n \to \infty$时,上式两端均趋于$M$,由夹逼定理可知
    $$\lim_{n \to \infty} \left( \int_a^b f^n(x) \mathrm{d}x \right)^{\frac{1}{n}} = M.$$
\end{solution}

\begin{exercise}[5.C.17]
    (1) 设函数 $f(x)$ 是 $[1, +\infty)$ 上的递增、非负函数, 证明: 对任意正整数 $n$, 有
    $$ 0 \les \sum_{k=1}^n f(k) - \int_1^n f(x) \mathrm{d}x \les f(n); $$
    (2) 设函数 $f(x)$ 是 $[1, +\infty)$ 上的递减、非负函数, 证明: 对任意正整数 $n$, 有
    $$ 0 \les \sum_{k=1}^n f(k) - \int_1^n f(x) \mathrm{d}x \les f(1). $$
    此外, 极限
    $$ \lim_{n \to \infty} \left( \sum_{k=1}^n f(k) - \int_1^n f(x) \mathrm{d}x \right) = \alpha $$
    存在, 且 $0 \les \alpha \les f(1)$.

    (提示: 对于 (1), 应用习题 5.1 中第 5 题可知,
    $$ f(k) \les \int_k^{k+1} f(x) \mathrm{d}x \les f(k+1), $$
    对 $k = 1, 2, \dots, n-1$ 求和, 可得出结果. 类似地可证明 (2) 中的不等式.
    为证明 (2) 中说的极限存在, 可证明
    $$ g(n) = \sum_{k=1}^n f(k) - \int_1^n f(x) \mathrm{d}x, \quad n = 1, 2, \dots $$
    是单调递减的函数; 而上面已指出 $g(n)$ 以 $0$ 为下界.)

    某些 (不易直接处理的) 离散量的和—数列的和, 可以通过 (易于处理的) 连续的量—积分部分作出估计. 本题给出了最简单的这样的结果 (这在后面的无穷级数理论中还将提及). 如今, 我们现在易于给出 $\displaystyle \sqrt[n]{k}$ 及 $n!$ 的相当精确的上、下界. 我们特别提及, 对 $n \ge 1$, 有
    $$ \ln n \les \sum_{k=1}^n \frac{1}{k} \les \ln n + 1, $$
    从而 $\displaystyle \sum_{k=1}^\infty \frac{1}{k}$ 趋于无穷大, 并与 $\ln n$ 同阶. 此外,
    $$ \gamma = \lim_{n \to \infty} \left( \sum_{k=1}^n \frac{1}{k} - \ln n \right) $$
    存在, 这称为 Euler (\textsf{欧拉}) 常数.
\end{exercise}

\begin{solution}
    \begin{enumerate}
        \item 由$f(x)$在$[1,+\infty)$上递增可知,$\forall k \in \mathbb{N}^+$,有$f(k) \les f(x) \les f(k+1),\forall x \in [k,k+1]$,因此
              $$ f(k) \les \int_k^{k+1} f(x) \mathrm{d}x \les f(k+1). $$
              对$k = 1, 2, \dots, n-1$求和,得
              $$ \sum_{k=1}^n f(k) = f(1) + \sum_{k=1}^{n-1} f(k+1) \ges \int_1^n f(x) \mathrm{d}x + f(1), $$
              $$ \sum_{k=1}^n f(k) = \sum_{k=1}^{n-1} f(k) + f(n) \les \int_1^n f(x) \mathrm{d}x + f(n). $$
        \item 由$f(x)$在$[1,+\infty)$上递减可知,$\forall k \in \mathbb{N}^+$,有$f(k+1) \les f(x) \les f(k),\forall x \in [k,k+1]$,因此
              $$ f(k+1) \les \int_k^{k+1} f(x) \mathrm{d}x \les f(k). $$
              对$k = 1, 2, \dots, n-1$求和,得
              $$ \sum_{k=1}^n f(k) = f(1) + \sum_{k=2}^n f(k) \ges \int_1^n f(x) \mathrm{d}x + f(n), $$
              $$ \sum_{k=1}^n f(k) = \sum_{k=1}^{n-1} f(k) + f(n) \les \int_1^n f(x) \mathrm{d}x + f(1). $$
              设$g(n) = \sum_{k=1}^n f(k) - \int_1^n f(x) \mathrm{d}x$,则由
              $$g(n+1) - g(n) = f(n+1) - \int_n^{n+1} f(x) \mathrm{d}x \les 0,$$
              可知$g(n)$是单调递减,且有下界$0$,因此极限
              $$ \lim_{n \to \infty} g(n) = \alpha $$
              存在,且$0 \les \alpha \les f(1)$.
    \end{enumerate}
\end{solution}

\begin{exercise}[5.C.18]
    (Cauchy 积分不等式) 设 $f(x)$ 与 $g(x)$ 在 $[a, b]$ 上连续, 证明:
    $$ \left( \int_a^b f(x) g(x) \mathrm{d}x \right)^2 \les \int_a^b f^2(x) \mathrm{d}x \int_a^b g^2(x) \mathrm{d}x, $$
    等号成立的充分必要条件是: $f$ 和 $g$ 中有一个恒为零, 或 $f(x) = \lambda g(x)$ (对 $x \in [a, b]$), 这里 $\lambda$ 是一个常数.

    (提示: 本题有好几种证法. 用对积分分上限求导可得出一个证明, 参见习题 5.1 中第 28 题的提示. 最标准的方法如下: 不妨设 $\displaystyle \int_a^b g^2(x) \mathrm{d}x \neq 0$, 否则易知函数 $g$ 恒为零, 结论显然成立 (习题 5.1 第 4 题 (2)). 考虑关于 $t$ 的二次项式 $\displaystyle \int_a^b (f(x) + t g(x))^2 \mathrm{d}x$, 这总是非负的.)
\end{exercise}

\begin{solution}
    $\left( f(x)-tg(x) \right)^2 \ges 0, \forall t \in \R$.即$\int_a^b \left( f(x)-tg(x) \right)^2 \dif x \ges 0$.
    展开得$$\int_a^b f^2(x) \dif x - 2t \int_a^b f(x)g(x) \dif x + t^2 \int_a^b g^2(x) \dif x \ges 0,$$
    即$$\Delta = 4 \left( \int_a^b f(x)g(x) \dif x \right)^2 - 4 \int_a^b f^2(x) \dif x \int_a^b g^2(x) \dif x \les 0,$$
    即$$\left( \int_a^b f(x)g(x) \dif x \right)^2 \les \int_a^b f^2(x) \dif x \int_a^b g^2(x) \dif x.$$
\end{solution}

\begin{exercise}[5.C.19]
    设 $f(x)$ 在 $[0, 1]$ 上有连续的导数, 证明: 对任意 $a \in [0, 1]$, 有
    $$ |f(a)| \les \int_0^1 |f(x)| \mathrm{d}x + \int_0^1 |f'(x)| \mathrm{d}x. $$
\end{exercise}

\begin{solution}
    设$|f(x_0)| = \max_{x \in [0,1]} |f(x)|$,又$f(x) = f(x_0) + \int_{x_0}^x f'(t) \dif t,$因此
    \begin{align*}
        |f(x_0)| & = \left| \int_0^1 f(x_0) \mathrm{d}t \right| = \left| \int_0^1 f(x) \mathrm{d}t - \int_0^1 \left( \int_{x_0}^x f'(t) \mathrm{d}t \right) \mathrm{d}x \right| \\
                 & \les \int_0^1 |f(x)| \mathrm{d}x + \left| \int_0^1 \left( \int_{x_0}^x f'(t) \mathrm{d}t \right) \mathrm{d}x \right|
    \end{align*}
    对于后者
    \begin{align*}
        \left| \int_0^1 \left( \int_{x_0}^x f'(t) \mathrm{d}t \right) \mathrm{d}x \right| & \les \int_0^1 \left| \int_{x_0}^x f'(t) \mathrm{d}t \right| \mathrm{d}x \\
                                                                                          & \les \int_0^1 \int_0^1 |f'(t)| \mathrm{d}t \mathrm{d}x                  \\
                                                                                          & = \int_0^1 |f'(t)| \mathrm{d}t
    \end{align*}
\end{solution}

\begin{exercise}[5.C.20]
    证明: $0.944 < \displaystyle \int_0^1 \frac{\sin x}{x} \mathrm{d}x < 0.947$. (提示: 参看习题 5.1 中第 24 题的提示.)
\end{exercise}

\begin{solution}
    由$\sin x = x - \frac{x^3}{3!} + \frac{x^5}{5!} - \frac{x^7}{7!} + \dots$,可得
    $$ \frac{\sin x}{x} = 1 - \frac{x^2}{3!} + \frac{x^4}{5!} - \frac{x^6}{7!} + \dots $$
    因此
    $$ 1 - \frac{x^2}{6} < \frac{\sin x}{x} < 1 - \frac{x^2}{6} + \frac{x^4}{120}. $$
    对上式两端在$[0,1]$上积分,得
    $$ \int_0^1 \left( 1 - \frac{x^2}{6} \right) \mathrm{d}x < \int_0^1 \frac{\sin x}{x} \mathrm{d}x < \int_0^1 \left( 1 - \frac{x^2}{6} + \frac{x^4}{120} \right) \mathrm{d}x, $$
    即
    $$ 0.944\bar{4} < \int_0^1 \frac{\sin x}{x} \mathrm{d}x < 0.946\bar{1}. $$
\end{solution}

\begin{exercise}[5.C.21]
    设 $f(x)$ 在区间 $[0, 1]$ 上连续可微, 且 $|f'(x)| \les M$. 证明: 对任意正整数 $n$,
    $$ \left| \int_0^1 f(x) \mathrm{d}x - \frac{1}{n} \sum_{k=1}^n f \left( \frac{k}{n} \right) \right| \les \frac{M}{2n}. $$
\end{exercise}

\begin{solution}
    由$|f'(x)| \les M$可知, 对任意$x,y \in [0,1]$, 由Lagrange中值定理可得$\exists \xi \in (x,y)$使得
    $$ f'(\xi) = \frac{f(y) - f(x)}{y - x} \Rightarrow |f(y) - f(x)| = |f'(\xi)||y - x| \les M|y - x|. $$
    因此
    \begin{align*}
        \left| \int_0^1 f(x) \mathrm{d}x - \frac{1}{n} \sum_{k=1}^n f \left( \frac{k}{n} \right) \right| & = \left| \sum_{k=1}^n \int_{\frac{k-1}{n}}^{\frac{k}{n}} f(x) \mathrm{d}x - \sum_{k=1}^n \int_{\frac{k-1}{n}}^{\frac{k}{n}} f \left( \frac{k}{n} \right) \mathrm{d}x \right| \\
                                                                                                         & = \left| \sum_{k=1}^n \int_{\frac{k-1}{n}}^{\frac{k}{n}} (f(x) - f(\tfrac{k}{n})) \mathrm{d}x \right|                                                                        \\
                                                                                                         & \les \sum_{k=1}^n \int_{\frac{k-1}{n}}^{\frac{k}{n}} |f(x) - f(\tfrac{k}{n})| \mathrm{d}x                                                                                    \\
                                                                                                         & \les \sum_{k=1}^n \int_{\frac{k-1}{n}}^{\frac{k}{n}} M \left| x - \frac{k}{n} \right| \mathrm{d}x                                                                            \\
                                                                                                         & = M \sum_{k=1}^n \int_{\frac{k-1}{n}}^{\frac{k}{n}} \left( \frac{k}{n} - x \right) \mathrm{d}x                                                                               \\
                                                                                                         & = M \sum_{k=1}^n \frac{1}{2n^2} = \frac{M}{2n}.
    \end{align*}
\end{solution}

\begin{exercise}[5.C.22]
    设 $f: \mathbb{R} \to (0, +\infty)$ 是一个可微函数, 且对任意实数 $x, y$ 满足
    $$ |f'(x) - f'(y)| \les |x - y|. $$
    求证: 对任意实数 $x$, 有
    $$ (f'(x))^2 < 2f(x). $$
\end{exercise}

\begin{solution}
    对于任意$x,y$,有$f(x+y) = f(x) + \int_x^{x+y} f'(t) \dif t$,因此
    \begin{align*}
        f(x+y) & = f(x) + yf'(x) + \int_x^{x+y} (f'(t) - f'(x)) \dif t    \\
               & \les f(x) + yf'(x) + \int_x^{x+y} |f'(t) - f'(x)| \dif t \\
               & \les f(x) + yf'(x) + \int_x^{x+y} |t - x| \dif t         \\
               & = f(x) + yf'(x) + \frac{y^2}{2}.
    \end{align*}
    令$y = -f'(x)$,则
    $$ f(x - f'(x)) \les f(x) - (f'(x))^2 + \frac{(f'(x))^2}{2} = f(x) - \frac{(f'(x))^2}{2}. $$
    因为$f(x - f'(x)) > 0$,所以$(f'(x))^2 < 2f(x)$.
\end{solution}

\newpage