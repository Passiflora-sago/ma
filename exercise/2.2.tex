\section{习题 2.2}

\begin{exercise}[2.2.1]
    证明函数 $x \cdot 2^x - 1$ 在 $[0,1]$ 内有零点.
\end{exercise}

\begin{solution}
    设 $f(x) = x \cdot 2^x - 1$, 则 $f(0) = -1 < 0, f(1) = 1 > 0$. 又因 $f(x)$ 在 $[0,1]$ 上连续, 故由介值定理知, 存在 $x_0 \in (0,1)$, 使得 $f(x_0) = 0$.
\end{solution}

\begin{exercise}[2.2.2]
    证明函数 $x - a\sin x - b$ (其中 $a,b$ 为正数) 在 $(0, +\infty)$ 上有零点, 且零点不超过 $a+b$.
\end{exercise}

\begin{solution}
    设 $f(x) = x - a\sin x - b$, 则 $f(0) = -b < 0, f(a+b) = a + b - a\sin(a+b) - b = a(1-\sin(a+b)) \ges 0$. 又因 $f(x)$ 在 $[0, a+b]$ 上连续, 故由介值定理知, 存在 $x_0 \in (0, a+b)$, 使得 $f(x_0) = 0$. 又因对任意 $x > a+b$ 有 $f(x) = x - a\sin x - b > a + b - a \sin x - b \ges 0$, 故 $f(x)$ 在 $(0, +\infty)$ 上的零点不超过 $a+b$.
\end{solution}

\begin{exercise}[2.2.3]
    证明函数 $x - \sin(x+1)$ 有实零点.
\end{exercise}

\begin{solution}
    设 $f(x) = x - \sin(x+1)$, 由$ -1 \les \sin(x+1) \les 1$知,
    则 $f(-2) \les -2 + 1 = -1 < 0, f(2) \ges 2 - 1 = 1 > 0$. 又因 $f(x)$ 在 $[-2,2]$ 上连续, 故由介值定理知, 存在 $x_0 \in (-2,2)$, 使得 $f(x_0) = 0$.
\end{solution}

\begin{exercise}[2.2.4]
    设函数 $f(x)$ 在 $[a,b]$ 上连续, 且值域就是 $[a,b]$. 证明 $f(x)$ 在 $[a,b]$ 上必有不动点, 即有 $x_0 \in [a,b]$, 使得 $f(x_0)=x_0$.
\end{exercise}

\begin{solution}
    函数$f(x)$的值域为$[a,b]$, 故存在 $x_1, x_2 \in [a,b]$, 使得 $f(x_1) = a, f(x_2) = b$. 设 $g(x) = f(x) - x$, 则 $g(x)$ 在 $[a,b]$ 上连续, 且 $g(x_1) = a - x_1 \les 0, g(x_2) = b - x_2 \ges 0$. 故由介值定理知, 存在 $x_0 \in [a,b]$, 使得 $g(x_0) = 0$, 即 $f(x_0) = x_0$.
\end{solution}

\begin{exercise}[2.2.5]
    设函数 $f(x), g(x)$ 在区间 $[a,b]$ 上连续, 且 $f(a) > g(a), f(b) < g(b)$. 试证: 存在 $x_0 \in (a,b)$, 使得 $f(x_0)=g(x_0)$.
\end{exercise}

\begin{solution}
    设 $h(x) = f(x) - g(x)$, 则 $h(x)$ 在 $[a,b]$ 上连续, 且 $h(a) = f(a) - g(a) > 0, h(b) = f(b) - g(b) < 0$. 故由介值定理知, 存在 $x_0 \in (a,b)$, 使得 $h(x_0) = 0$, 即 $f(x_0) = g(x_0)$.
\end{solution}

\begin{exercise}[2.2.6]
    设函数 $f(x)$ 在 $[0, 2a]$ 上连续, 且 $f(0)=f(2a)$. 证明: 在区间 $[0,a]$ 上存在某个 $x_0$, 使得 $f(x_0) = f(x_0+a)$.
\end{exercise}

\begin{solution}
    设 $g(x) = f(x) - f(x+a)$, 则 $g(x)$ 在 $[0,a]$ 上连续, 且 $g(0) = f(0) - f(a), g(a) = f(a) - f(2a) = f(a) - f(0) = -g(0)$. 因此$g(0) g(a) = -(g(0))^2 \les 0$. 由介值定理知, 存在 $x_0 \in [0,a]$, 使得 $g(x_0) = 0$, 即 $f(x_0) = f(x_0+a)$.
\end{solution}

\begin{exercise}[2.2.7]
    试证: 若函数 $f(x)$ 在 $[a,b]$ 上连续, $x_1, x_2, \dots, x_n$ 为此区间中的任意点, 则在 $[a,b]$ 中有一点 $\xi$, 使得
    $$ f(\xi) = \frac{1}{n}(f(x_1)+f(x_2)+\dots+f(x_n)). $$
    更一般地, 若 $q_1 > 0, q_2 > 0, \dots, q_n > 0$, 且 $q_1+q_2+\dots+q_n=1$, 则在 $[a,b]$ 中有一点 $\xi$, 使得
    $$ f(\xi) = q_1f(x_1)+q_2f(x_2)+\dots+q_nf(x_n). $$
\end{exercise}

\begin{solution}
    \begin{enumerate}
        \item 设 $A = \frac{1}{n}(f(x_1)+f(x_2)+\dots+f(x_n))$. 若 $f(x_i) = A$ 对某个 $i$ 成立, 则取 $\xi = x_i$. 否则, 设 $f(x_k) = \min\{f(x_1), f(x_2), \dots, f(x_n)\}, f(x_m) = \max\{f(x_1), f(x_2), \dots, f(x_n)\}$. 则 $f(x_k) < A < f(x_m)$. 由介值定理知, 存在 $\xi \in (x_k, x_m)$, 使得 $f(\xi) = A$.
        \item 设 $A = q_1f(x_1)+q_2f(x_2)+\dots+q_nf(x_n)$. 若 $f(x_i) = A$ 对某个 $i$ 成立, 则取 $\xi = x_i$. 否则, 设 $f(x_k) = \min\{f(x_1), f(x_2), \dots, f(x_n)\}, f(x_m) = \max\{f(x_1), f(x_2), \dots, f(x_n)\}$. 则 $f(x_k) < A < f(x_m)$. 由介值定理知, 存在 $\xi \in (x_k, x_m)$, 使得 $f(\xi) = A$.
    \end{enumerate}
\end{solution}

\begin{exercise}[2.2.8]\label{exe:2.2.8}
    设函数 $f(x)$ 在区间 $[a, +\infty)$ 上连续, 且 $\lim_{x \to +\infty} f(x)$ 存在. 证明 $f(x)$ 在 $[a, +\infty)$ 上有界.
\end{exercise}

\begin{solution}
    设 $\lim_{x \to +\infty} f(x) = A$, 则存在 $M > 0$, 使得当 $x > M$ 时, 有 $|f(x) - A| < 1$, 即 $|f(x)| < |A| + 1$. 又因函数 $f(x)$ 在区间 $[a,M]$ 上连续, 故在该闭区间上有界, 即存在 $K > 0$, 使得对任意 $x \in [a,M]$ 有 $|f(x)| \les K$. 取 $N = \max\{K, |A| + 1\}$, 则对任意 $x \in [a, +\infty)$ 有 $|f(x)| \les N$. 因此, 函数 $f(x)$ 在 $[a, +\infty)$ 上有界.
\end{solution}

\begin{exercise}[2.2.9]
    证明函数 $f(x) = \frac{1+x^2}{1-x^2+x^4}$ 在 $(-\infty, +\infty)$ 上有界.
\end{exercise}

\begin{solution}
    由\hyperref[{exe:1.3.17}]{习题1.3.17} 知,$\lim_{x \to +\infty} \frac{1+x^2}{1-x^2+x^4} = 0$, 利用\hyperref[exe:2.2.8]{习题2.2.8} 知, 函数 $f(x)$ 在 $[0, +\infty)$ 上有界. 又因 $f(x)$ 是偶函数, 故 $f(x)$ 在 $(-\infty, 0]$ 上也有界. 因此, 函数 $f(x)$ 在 $(-\infty, +\infty)$ 上有界.
\end{solution}

\begin{exercise}[2.2.10]
    是否有满足下面条件的连续函数? 说明理由.
    \begin{enumerate}
        \item 定义域为 $[0,1]$, 值域为 $(0, +\infty)$;
        \item 定义域为 $[0,1]$, 值域为 $(0,1)$;
        \item 定义域为 $[0,1]$, 值域为 $[0,1]\cup[2,4]$;
        \item 定义域为 $(0,1)$, 值域为 $(2, +\infty)$.
    \end{enumerate}
\end{exercise}

\begin{solution}
    \begin{enumerate}
        \item 不存在. 这与最值定理矛盾.
        \item 不存在. 这与最值定理矛盾.
        \item 不存在. 这与介值定理矛盾.
        \item 存在. 例如, $f(x) = \frac{1}{x} + 2$.
    \end{enumerate}
\end{solution}

\begin{exercise}[2.2.11]
    举例说明, 对任意正数 $\varepsilon < \frac{b-a}{2}$, 函数 $f(x)$ 在闭区间 $[a+\varepsilon, b-\varepsilon]$ 上有界, 不能保证 $f(x)$ 在开区间 $(a,b)$ 上有界. (比较习题 2.1 第 2 题.)
\end{exercise}

\begin{solution}
    例如, 设 $f(x) = \frac{1}{x-a} + \frac{1}{b-x}$, 则对任意正数 $\varepsilon < \frac{b-a}{2}$, 函数 $f(x)$ 在闭区间 $[a+\varepsilon, b-\varepsilon]$ 上有界, 但在开区间 $(a,b)$ 上无界.
\end{solution}

\begin{exercise}[2.2.12]
    设 $y=f(x)$ 在开区间 $I=(a,b)$ 上连续并严格单调. 证明 $y=f(x)$ 的值域 $f(I)$ 也是一个开区间.
\end{exercise}

\begin{solution}
    \begin{remark}
        不能假设$\lim_{x \to a^+} f(x)$和$\lim_{x \to b^-} f(x)$存在.$f(I)$可能有无穷端点,例如$f(x)=\tan x$在$(-\frac{\pi}{2}, \frac{\pi}{2})$上连续且严格单调,但值域是$(-\infty,+\infty)$.
    \end{remark}

    先证明$f(I)$存在两个不同的点:取$x_1, x_2 \in I$, 且 $x_1 < x_2$, 由严格单调性知, $f(x_1) < f(x_2)$, 故$f(I)$中至少有两个不同的点.\begin{remark}
        如果去除单调的严格性,则$f(I)$不一定是开区间,例如$f(x)=1$在$(0,1)$上连续且单调,但值域不是开区间.
    \end{remark}

    $\forall y_1, y_2 \in f(I)$, 且 $y_1 < y_2$. 则存在 $x_1, x_2 \in I$, 使得 $f(x_1) = y_1, f(x_2) = y_2$. 因为 $f(x)$ 在 $I$ 上严格单调, 故 $x_1 < x_2$. 对任意 $y \in (y_1, y_2)$, 由介值定理知, 存在 $x \in (x_1, x_2) \subset I$, 使得 $f(x) = y$. 因此, $f(I)$ 是区间.

    下面证明$f(I)$是开区间. $\forall y \in f(I)$, 则存在 $x_0 \in I$, 使得 $f(x_0) = y$. 由$(a,b)$是开区间知, 存在 $\delta > 0$, 使得 $(x_0 - \delta, x_0 + \delta) \subset (a,b)$. 设 $\eta = \min\{f(x_0) - f(x_0 - \delta), f(x_0 + \delta) - f(x_0)\} > 0$, 则对任意 $y' \in (y - \eta, y + \eta)\subset (f(x_0 - \delta), f(x_0 + \delta))$, 由介值定理知, 存在 $x' \in (x_0 - \delta, x_0 + \delta) \subset I$, 使得 $f(x') = y'$. 因此, $f(I)$ 是开区间.
\end{solution}

\begin{exercise}[2.2.13]
    设函数 $f(x)$ 在有限区间 $(a,b)$ 上一致连续. 求证 $f(x)$ 在 $a$ 点的右极限和在 $b$ 点的左极限都存在.
\end{exercise}

\begin{solution}
    $f(x)$ 在 $(a,b)$ 上一致连续, 则对任意 $\varepsilon > 0$, 存在 $\delta > 0$, 使得当 $x_1, x_2 \in (a,b)$ 且 $|x_1 - x_2| < \delta$ 时, 有 $|f(x_1) - f(x_2)| < \varepsilon$. 现取 $x_1, x_2 \in (a, a+\delta)$, 则 $|x_1 - x_2| < \delta$, 故 $|f(x_1) - f(x_2)| < \varepsilon$. 因此, 函数 $f(x)$ 在 $(a, a+\delta)$ 上满足柯西收敛准则, 故 $\lim_{x \to a^+} f(x)$ 存在. 同理可证 $\lim_{x \to b^-} f(x)$ 也存在.
\end{solution}

\begin{exercise}[2.2.14]
    设函数 $f(x)$ 在 $(0, +\infty)$ 上一致连续, $\{a_n\}$ 是正收敛数列. 求证 $\{f(a_n)\}$ 也收敛. 又问仅假设 $f(x)$ 连续时, 结论是否还成立, 为什么?
\end{exercise}

\begin{solution}
    由$f(x)$在$(0,+\infty)$上一致连续, 则对任意$\varepsilon>0$, 存在$\delta>0$, 使得当$x_1,x_2\in(0,+\infty)$且$|x_1-x_2|<\delta$时, 有$|f(x_1)-f(x_2)|<\varepsilon$. 又因$\{a_n\}$是正收敛数列, 故存在$N\in\mathbf{N}^+$, 使得当$n,m>N$时, 有$|a_n-a_m|<\delta$. 因此, 对任意$n,m>N$有$|f(a_n)-f(a_m)|<\varepsilon$. 由柯西收敛准则知, 数列$\{f(a_n)\}$收敛.

    仅假设$f(x)$连续时, 结论不成立. 例如, 设$f(x)=\sin\frac{1}{x}$在$(0,+\infty)$上连续, 且数列$a_n=\frac{1}{(n+\frac{1}{2})\pi}$收敛于$0$, 但数列$f(a_n)=\sin\left( \left(n+\frac12\right) \pi\right)= (-1)^n$不收敛.
\end{solution}

\begin{exercise}[2.2.15]
    设函数 $f(x)$ 在 $(-\infty, +\infty)$ 上连续, $\{a_n\}$ 是收敛数列. 求证 $\{f(a_n)\}$ 也收敛.
\end{exercise}

\begin{solution}
    设 $\lim_{n \to \infty} a_n = a$, 则 $a \in (-\infty, +\infty)$. 由$f(x)$在$(-\infty,+\infty)$上连续知,$f(x)$在$a$点连续, 即$\lim_{x \to a} f(x) = f(a)$. 因此, 对任意 $\varepsilon > 0$, 存在 $\delta > 0$, 使得当 $|x - a| < \delta$ 时, 有 $|f(x) - f(a)| < \varepsilon$. 又因数列$\{a_n\}$收敛于$a$, 故存在$N\in\mathbf{N}^+$, 使得当$n>N$时, 有$|a_n-a|<\delta$. 因此, 对任意$n>N$有$|f(a_n)-f(a)|<\varepsilon$. 由数列的收敛定义知, 数列$\{f(a_n)\}$收敛于$f(a)$.
\end{solution}

\begin{exercise}[2.2.16]
    给出一个在 $(-\infty, +\infty)$ 上连续且有界但不一致连续的函数.
\end{exercise}

\begin{solution}
    例如, $f(x) = \sin x^2$ 在 $(-\infty, +\infty)$ 上连续且有界, 但不一致连续. 反证法: 若$f(x)$在$(-\infty,+\infty)$上一致连续, 则对任意$\varepsilon=1$, 存在$\delta>0$, 使得当$x_1,x_2\in(-\infty,+\infty)$且$|x_1-x_2|<\delta$时, 有$|f(x_1)-f(x_2)|<1$. 现取$x_1=\sqrt{2n\pi}, x_2=\sqrt{2n\pi+\frac{\pi}{2}}$, 则当$n>\left(\frac{2}{\delta}\right)^2$时, 有$|x_1-x_2|<\delta$, 但$|f(x_1)-f(x_2)|=\left|\sin(2n\pi)-\sin\left(2n\pi+\frac{\pi}{2}\right)\right|=1$, 矛盾.
\end{solution}

\newpage