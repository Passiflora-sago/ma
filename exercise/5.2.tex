\section{习题 5.2}

\begin{exercise}[5.2.1]
求区间 $[0, 1]$ 上 Dirichlet 函数的上积分和下积分.
\end{exercise}

\begin{solution}
    设 $f(x)$ 是区间 $[0, 1]$ 上的 Dirichlet 函数. 对任意分划 $P = \{ x_0, x_1, \dots, x_n \}$, 因为每个子区间 $[x_{i-1}, x_i]$ 上既有有理数也有无理数, 所以
    $$ M_i = \sup_{x \in [x_{i-1}, x_i]} f(x) = 1, \quad m_i = \inf_{x \in [x_{i-1}, x_i]} f(x) = 0. $$
    因此
    $$ U(f, P) = \sum_{i=1}^n M_i \Delta x_i = \sum_{i=1}^n 1 \cdot (x_i - x_{i-1}) = 1, $$
    $$ L(f, P) = \sum_{i=1}^n m_i \Delta x_i = \sum_{i=1}^n 0 \cdot (x_i - x_{i-1}) = 0. $$
    所以
    $$ \overline{\int_0^1} f(x) \mathrm{d}x = \inf_P U(f, P) = 1, $$
    $$ \underline{\int_0^1} f(x) \mathrm{d}x = \sup_P L(f, P) = 0. $$
\end{solution}

\begin{exercise}[5.2.2]
试给出 Darboux 上和与下的几何解释.
\end{exercise}

\begin{solution}
    设函数 $f(x)$ 在区间 $[a, b]$ 上有界. 对任意分划 $P = \{ x_0, x_1, \dots, x_n \}$, 记
    $$ M_i = \sup_{x \in [x_{i-1}, x_i]} f(x), \quad m_i = \inf_{x \in [x_{i-1}, x_i]} f(x). $$
    则 Darboux 上和与下可以分别表示为
    $$ U(f, P) = \sum_{i=1}^n M_i \Delta x_i, \quad L(f, P) = \sum_{i=1}^n m_i \Delta x_i. $$
    几何上, $U(f, P)$ 表示由各子区间 $[x_{i-1}, x_i]$ 上的最大值 $M_i$ 所构成的矩形条形图的面积之和; 而 $L(f, P)$ 则表示由各子区间 $[x_{i-1}, x_i]$ 上的最小值 $m_i$ 所构成的矩形条形图的面积之和.
\end{solution}

\begin{exercise}[5.2.3]
证明: 如果函数 $f(x)$ 在区间 $[a, b]$ 上可积, 则 $|f(x)|$ 在区间 $[a, b]$ 上也可积, 而且有
$$ \left| \int_a^b f(x) \mathrm{d}x \right| \les \int_a^b |f(x)| \mathrm{d}x. $$
\end{exercise}

\begin{solution}
    设 $f(x)$ 在区间 $[a, b]$ 上可积. 对任意分划 $T = \{ x_0, x_1, \dots, x_n \}$, 记
    $$M_i = \sup_{x \in [x_{i-1}, x_i]} f(x), \quad m_i = \inf_{x \in [x_{i-1}, x_i]} f(x), \quad \omega_i = M_i - m_i,$$
    $$M_i' = \sup_{x \in [x_{i-1}, x_i]} |f(x)|, \quad m_i' = \inf_{x \in [x_{i-1}, x_i]} |f(x)|,  \quad \omega_i' = M_i' - m_i'.$$
    因为$f(x)$在区间$[a,b]$上可积,所以$\lim_{\lambda(T) \to 0} \sum_{i=1}^n \omega_i \Delta x_i = 0$.又因为
    $$\omega_i' = M_i' - m_i' \les |M_i| + |m_i| \les |M_i - m_i| = \omega_i,$$
    所以$\lim_{\lambda(T) \to 0} \sum_{i=1}^n \omega_i' \Delta x_i = 0$.因此$|f(x)|$在区间$[a,b]$上可积.

    对$-|f(x)| \les f(x) \les |f(x)|$两边同时积分,得
    $$ -\int_a^b |f(x)| \mathrm{d}x \les \int_a^b f(x) \mathrm{d}x \les \int_a^b |f(x)| \mathrm{d}x, $$
    即
    $$ \left| \int_a^b f(x) \mathrm{d}x \right| \les \int_a^b |f(x)| \mathrm{d}x. $$
\end{solution}

\begin{exercise}[5.2.4]
设 $f(x)$ 在区间 $[a, b]$ 上可积, 且 $f(x) \ges c > 0$. 求证: $\displaystyle \frac{1}{f(x)}$ 在区间 $[a, b]$ 上也可积.
\end{exercise}

\begin{solution}
    设 $f(x)$ 在区间 $[a, b]$ 上可积, 且 $f(x) \ges c > 0$. 对任意分划 $T = \{ x_0, x_1, \dots, x_n \}$, 记
    $$M_i = \sup_{x \in [x_{i-1}, x_i]} f(x), \quad m_i = \inf_{x \in [x_{i-1}, x_i]} f(x), \quad \omega_i = M_i - m_i,$$
    $$M_i' = \sup_{x \in [x_{i-1}, x_i]} \frac{1}{f(x)}, \quad m_i' = \inf_{x \in [x_{i-1}, x_i]} \frac{1}{f(x)},  \quad \omega_i' = M_i' - m_i'.$$
    因为$f(x)$在区间$[a,b]$上可积,所以$\lim_{\lambda(T) \to 0} \sum_{i=1}^n \omega_i \Delta x_i = 0$.又因为
    $$\omega_i' = M_i' - m_i' = \frac{1}{m_i} - \frac{1}{M_i} = \frac{M_i - m_i}{M_i m_i} = \frac{\omega_i}{M_i m_i} \les \frac{\omega_i}{c^2},$$
    所以$\lim_{\lambda(T) \to 0} \sum_{i=1}^n \omega_i' \Delta x_i = 0$.因此$\frac{1}{f(x)}$在区间$[a,b]$上可积.
\end{solution}

\newpage