\section{习题 7.2}

\begin{exercise}[7.2.1]
    证明: 两个在共同区间 $I$ 上一致收敛的级数的和, 也在 $I$ 上一致收敛.
\end{exercise}

\begin{solution}
    设$\sum_{n=1}^\infty u_n(x)$和$\sum_{n=1}^\infty v_n(x)$在区间$I$上一致收敛,则对任意$\varepsilon > 0$,存在$N_1, N_2$,当$m > n \ges N_1$时,有
    $$\left| \sum_{k=n}^m u_k(x) \right| < \frac{\varepsilon}{2},$$
    当$m > n \ges N_2$时,有
    $$\left| \sum_{k=n}^m v_k(x) \right| < \frac{\varepsilon}{2}.$$
    取$N = \max\{N_1, N_2\}$,则当$m > n \ges N$时,有
    \begin{align*}
        \left| \sum_{k=n}^m [u_k(x) + v_k(x)] \right| & = \left| \sum_{k=n}^m u_k(x) + \sum_{k=n}^m v_k(x) \right|                   \\
                                                      & \les \left| \sum_{k=n}^m u_k(x) \right| + \left| \sum_{k=n}^m v_k(x) \right| \\
                                                      & < \frac{\varepsilon}{2} + \frac{\varepsilon}{2} = \varepsilon.
    \end{align*}
    故级数$\sum_{n=1}^\infty [u_n(x) + v_n(x)]$在区间$I$上一致收敛.
\end{solution}

\begin{exercise}[7.2.2]
    确定下列函数项级数的收敛域.
    \begin{tasks}(2)
        \task $\displaystyle \sum_{n=1}^\infty n \mathrm{e}^{-nx};$
        \task $\displaystyle \sum_{n=2}^\infty \frac{x^{n^2}}{n};$
        \task $\displaystyle \sum_{n=1}^\infty \frac{(-1)^n}{2n - 1} \left( \frac{1 - x}{1 + x} \right)^n;$
        \task $\displaystyle \sum_{n=1}^\infty \frac{1}{x^{2n}} \sin \frac{\pi}{2^n};$
        \task $\displaystyle \sum_{n=1}^\infty \frac{(x - 3)^n}{n - 3^n};$
        \task $\displaystyle \sum_{n=1}^\infty n! \left( \frac{x}{n} \right)^n;$
        \task $\displaystyle \sum_{n=1}^\infty \frac{\cos nx}{\mathrm{e}^{nx}};$
        \task $\displaystyle \sum_{n=1}^\infty \frac{x^n}{1 - x^n}.$
    \end{tasks}
\end{exercise}

\begin{solution}
    \begin{enumerate}
        \item $$\frac{u_{n+1}(x)}{u_n(x)} = \frac{(n+1) \mathrm{e}^{-(n+1)x}}{n \mathrm{e}^{-nx}} = \left( 1 + \frac{1}{n} \right) \mathrm{e}^{-x} \to \mathrm{e}^{-x}.$$
              故当$x > 0$时,级数收敛;当$x \les 0$时,级数发散.收敛域为$(0, +\infty)$.
        \item $$\sqrt[n]{|u_n(x)|} = \sqrt[n]{\frac{|x|^{n^2}}{n}} = \frac{|x|^n}{\sqrt[n]{n}} \to |x|^n.$$
              故当$|x| < 1$时,级数收敛;当$|x| > 1$时,级数发散;当$x = 1$时,级数$\sum_{n=2}^\infty \frac{1}{n}$发散;当$x = -1$时,级数$\sum_{n=2}^\infty \frac{(-1)^{n^2}}{n} = \sum_{n=2}^\infty \frac{(-1)^n}{n}$收敛.收敛域为$[-1, 1)$.
        \item $$\sqrt[n]{|u_n(x)|} = \sqrt[n]{\frac{1}{2n - 1} \left| \frac{1 - x}{1 + x} \right|^n} = \left| \frac{1 - x}{1 + x} \right| \cdot \sqrt[n]{\frac{1}{2n - 1}} \to \left| \frac{1 - x}{1 + x} \right|.$$
              故当$x \in (- \infty, 0)$时,级数发散;当$x \in (0, +\infty)$时,级数收敛;当$x = 0$时,级数$\sum_{n=1}^\infty \frac{(-1)^n}{2n - 1}$由Liebnitz判别法收敛.收敛域为$[0, +\infty)$.
        \item $$\frac{1}{x^{2n}} \sin \frac{\pi}{2^n} \sim \frac{\pi}{(2x)^{n}}.$$
              故当$|x| > \frac{1}{2}$时,级数收敛;当$|x| < \frac{1}{2}$时,级数发散;当$x = \pm \frac{1}{2}$时,级数$\sum_{n=1}^\infty \sin \frac{\pi}{2^n}$发散.收敛域为$(-\infty, -\frac{1}{2}) \cup (\frac{1}{2}, +\infty)$.
        \item $$\sqrt[n]{|u_n(x)|} = \sqrt[n]{\frac{|x - 3|^n}{|n - 3^n|}} = \frac{|x - 3|}{\sqrt[n]{|n - 3^n|}} \to \frac{|x - 3|}{3}.$$
              故当$|x - 3| < 3$时,级数收敛;当$|x - 3| > 3$时,级数发散;当$x = 0$时,级数$\sum_{n=1}^\infty \frac{(-3)^n}{n - 3^n}$发散;当$x = 6$时,级数$\sum_{n=1}^\infty \frac{3^n}{n + 3^n}$发散,.收敛域为$(0, 6)$.
        \item $$\frac{u_{n+1}(x)}{u_n(x)} = \frac{(n+1)! \left( \frac{x}{n+1} \right)^{n+1}}{n! \left( \frac{x}{n} \right)^n} = \frac{x}{n} \to 0.$$
              故级数在$(-\infty, +\infty)$上收敛.收敛域为$(-\infty, +\infty)$.
        \item $x \ne 0$时,$\cos nx$的部分和
              $$\left| \sum_{k=1}^n \cos kx \right| = \left| \frac{\sin \frac{nx}{2} \cdot \cos \frac{(n+1)x}{2}}{\sin \frac{x}{2}} \right| \les \frac{1}{|\sin \frac{x}{2}|}.$$
              故当$x \in (-\infty, 0) \cup (0, +\infty)$时,级数收敛;当$x = 0$时,级数$\sum_{n=1}^\infty 1$发散.收敛域为$(-\infty, 0) \cup (0, +\infty)$.
        \item $$\frac{u_{n+1}(x)}{u_n(x)} = \frac{\frac{x^{n+1}}{1 - x^{n+1}}}{\frac{x^n}{1 - x^n}} = \frac{x (1 - x^n)}{1 - x^{n+1}} \to x.$$
              故当$|x| < 1$时,级数收敛;当$|x| > 1$时,级数发散;当$x = 1$时,级数$\sum_{n=1}^\infty \frac{1}{1 - 1^n}$无定义;当$x = -1$时,级数$\sum_{n=1}^\infty \frac{(-1)^n}{1 - (-1)^n}$发散.收敛域为$(-1, 1)$.
    \end{enumerate}
\end{solution}

\begin{exercise}[7.2.3]
    在区间 $[0, 1]$ 上, 定义
    $$ u_n(x) = \begin{cases} \frac{1}{n}, & x = \frac{1}{n}, \\ 0, & x \neq \frac{1}{n}. \end{cases} $$
    证明: 级数 $\displaystyle \sum_{n=1}^\infty u_n(x)$ 在 $[0, 1]$ 上一致收敛, 但是它没有 Weierstrass 判别法中的控制级数.
\end{exercise}

\begin{solution}
    对任意$\varepsilon > 0$,取$N > \frac{1}{\varepsilon}$,则当$m > n \ges N$时,对任意$x \in [0, 1]$,有
    $$\left| \sum_{k=n}^m u_k(x) \right| \les \sum_{k=n}^m |u_k(x)| \les \sum_{k=n}^m \frac{1}{k} < \varepsilon.$$
    故级数$\sum_{n=1}^\infty u_n(x)$在$[0, 1]$上一致收敛.

    若存在控制级数$\sum_{n=1}^\infty M_n$使得对任意$x \in [0, 1]$,均有$|u_n(x)| \les M_n$,则取$x = \frac{1}{n}$时,有
    $$\frac{1}{n} = \left|u_n \left(\frac{1}{n} \right)\right| \les M_n,$$
    故级数$\sum_{n=1}^\infty M_n$发散,与控制级数矛盾.故级数$\sum_{n=1}^\infty u_n(x)$没有控制级数.
\end{solution}

\begin{exercise}[7.2.4]
    研究下列级数在给定区间上的一致收敛性:
    \begin{tasks}(2)
        \task $\displaystyle \sum_{n=1}^\infty \frac{\sin nx}{n^2}, -\infty < x < +\infty;$
        \task $\displaystyle \sum_{n=1}^\infty \frac{1}{2^n (1 + (nx)^2)}, -\infty < x < +\infty;$
        \task $\displaystyle \sum_{n=1}^\infty (-1)^{n-1} x^n, -1 < x < 1;$
        \task $\displaystyle \sum_{n=1}^\infty x^2 \mathrm{e}^{-nx}, 0 \les x < +\infty;$
        \task $\displaystyle \sum_{n=1}^\infty \frac{(-1)^n}{x + n}, 0 \les x < +\infty;$
        \task $\displaystyle \sum_{n=1}^\infty \frac{1}{n^x}, 1 < x < +\infty;$
        \task $\displaystyle \sum_{n=1}^\infty \frac{\cos nx}{n}, 0 < \delta \les x \les 2\pi - \delta;$
        \task $\displaystyle \sum_{n=1}^\infty \frac{x^2}{(n \mathrm{e}^n)^x}, 0 \les x < +\infty.$
    \end{tasks}
\end{exercise}

\begin{solution}
    \begin{enumerate}
        \item $$|u_n(x)| = \left| \frac{\sin nx}{n^2} \right| \les \frac{1}{n^2}.$$
              由Weierstrass判别法知,级数$\sum_{n=1}^\infty \frac{\sin nx}{n^2}$在$(-\infty, +\infty)$上一致收敛.
        \item $$|u_n(x)| = \left| \frac{1}{2^n (1 + (nx)^2)} \right| \les \frac{1}{2^n}.$$
              由Weierstrass判别法知,级数$\sum_{n=1}^\infty \frac{1}{2^n (1 + (nx)^2)}$在$(-\infty, +\infty)$上一致收敛.
        \item $$\sup_{x \in (-1, 1)} |u_n(x)| = \sup_{x \in (-1, 1)} |(-1)^{n-1} x^n| = 1.$$
              因此$\lim_{n \to \infty} \sup_{x \in (-1, 1)} |u_n(x)| \neq 0$.故级数$\sum_{n=1}^\infty (-1)^{n-1} x^n$在$(-1, 1)$上不一致收敛.
        \item 考虑函数$f(x) = x^2 \mathrm{e}^{-nx}$在$[0, +\infty)$上的最大值,由$f'(x) = 2x \mathrm{e}^{-nx} - n x^2 \mathrm{e}^{-nx} = x \mathrm{e}^{-nx} (2 - nx)$可知,当$x = \frac{2}{n}$时,$f(x)$取得最大值,即
              $$\sup_{x \in [0, +\infty)} |u_n(x)| = f\left( \frac{2}{n} \right) = \frac{4}{n^2} \mathrm{e}^{-2}.$$
              故级数$\sum_{n=1}^\infty x^2 \mathrm{e}^{-nx}$在$[0, +\infty)$上一致收敛.
        \item $$\lim_{n \to \infty} \sup_{x \in [0, +\infty)} \left| \frac{(-1)^n}{x + n} \right| = \lim_{n \to \infty} \frac{1}{n} = 0.$$
              故级数$\sum_{n=1}^\infty \frac{(-1)^n}{x + n}$在$[0, +\infty)$上一致收敛.
        \item $\forall N ,\forall n > N, \exists p = n, x = 1 + \frac{1}{\ln 2N} \in (1, +\infty)$,使得
              \begin{align*}
                  |u_{n+1}(x) + \dots + u_{n+p}(x)| & = \sum_{k=n+1}^{2n} \frac{1}{k^x} = \sum_{k=n+1}^{2n} \frac{1}{k^x} \cdot \frac{1}{k^{\ln 2N}}                      \\
                                                    & > \frac{1}{(2N)^{\frac{1}{\ln 2N}}} \sum_{k=n+1}^{2n} \frac{1}{k}                                                   \\
                                                    & = \frac{1}{\mathrm{e}} \sum_{k=n+1}^{2n} \frac{1}{k} > \frac{1}{\mathrm{e}} \int_{n+1}^{2n+1} \frac{\mathrm{d}x}{x} \\
                                                    & = \frac{1}{\mathrm{e}} \ln \frac{2n + 1}{n + 1} > \frac{1}{\mathrm{e}} \ln \frac{3}{2}.
              \end{align*}
              由Cauchy准则知,级数$\sum_{n=1}^\infty \frac{1}{n^x}$在$(1, +\infty)$上不一致收敛.
        \item $\cos nx$的部分和
              $$\left| \sum_{k=1}^n \cos kx \right| = \left| \frac{\sin \frac{nx}{2} \cdot \cos \frac{(n+1)x}{2}}{\sin \frac{x}{2}} \right| \les \frac{1}{|\sin \frac{x}{2}|} \les \frac{1}{\sin \frac{\delta}{2}}.$$
              有界,且$\frac{1}{n}$单调递减趋于$0$,由Dirichlet判别法知,级数$\sum_{n=1}^\infty \frac{\cos nx}{n}$在$[\delta, 2\pi - \delta]$上一致收敛.
        \item 由(4)知,$\sum_{n=1}^\infty x^2 \mathrm{e}^{-nx}$在$[0, +\infty)$上一致收敛,且
              $\frac{1}{n^x} \les 1$对$x$单调,对$n,x$一致有界,由Abell判别法知,级数$\sum_{n=1}^\infty \frac{x^2}{(n \mathrm{e}^n)^x}$在$[0, +\infty)$上一致收敛.
    \end{enumerate}
\end{solution}

\begin{exercise}[7.2.5]
    证明: 若级数 $\displaystyle \sum_{n=1}^\infty a_n$ 收敛, 则级数 $\displaystyle \sum_{n=1}^\infty \frac{a_n}{\mathrm{e}^{nx}}$ 在 $0 \les x < +\infty$ 中一致收敛.
\end{exercise}

\begin{solution}
    $\sum_{n=1}^\infty a_n$作为函数列时,对$x$而言是常数,故对$x$一致收敛.同时$\frac{1}{\mathrm{e}^{nx}} \les 1$对$x$单调,对$n,x$一致有界,由Abell判别法知,级数$\sum_{n=1}^\infty \frac{a_n}{\mathrm{e}^{nx}}$在$[0, +\infty)$上一致收敛.
\end{solution}

\begin{exercise}[7.2.6]
    证明: 函数 $\displaystyle \zeta(x) = \sum_{n=1}^\infty \frac{1}{n^x}$ 在 $(1, +\infty)$ 内连续, 且具有的各阶导数.
\end{exercise}

\begin{solution}
    $$\zeta(x) = \sum_{n=1}^\infty \frac{1}{n^x} := \sum_{n=1}^\infty u_n(x), \quad \sum_{n=1}^\infty u_n^{(k)}(x) = \sum_{n=1}^\infty \frac{(-\ln n)^k}{n^x}.$$
    $\sum_{n=1}^\infty u_n^{(k)}(x)$在$(1,+\infty)$中并不一致收敛,但对每一点$x \in (1, +\infty)$,总是存在$1< \alpha < x < \beta$,在区间$[\alpha, \beta]$上,有
    $$\left| \frac{(-\ln n)^k}{n^x} \right| \les \frac{(\ln n)^k}{n^\alpha}.$$
    由Cauchy根值判别法知,级数$\sum_{n=1}^\infty \frac{(\ln n)^k}{n^\alpha}$收敛,故由Weierstrass判别法知,级数$\sum_{n=1}^\infty u_n^{(k)}(x)$在$[\alpha, \beta]$上一致收敛.由教材定理7.36 知,$\zeta(x)$在$[\alpha, \beta]$上处处可微.

    递推的可以得到,$\zeta(x)$在$[\alpha, \beta]$上具有任意阶导数,由$x$的任意性知,$\zeta(x)$在$(1, +\infty)$内连续,且具有各阶导数.
\end{solution}

\begin{exercise}[7.2.7]
    证明: $\displaystyle f(x) = \sum_{n=1}^\infty \frac{\sin nx}{n^4}$ 当 $|x| < +\infty$ 时, 具有连续的二阶微商.
\end{exercise}

\begin{solution}
    $$|u'(x) = \frac{\cos nx}{n^3}| \les \frac{1}{n^3}$$
    由Weierstrass判别法知,级数$\sum_{n=1}^\infty \frac{\cos nx}{n^3}$在$(-\infty, +\infty)$上一致收敛.
    $$|u''(x) = -\frac{\sin nx}{n^2}| \les \frac{1}{n^2}$$
    由Weierstrass判别法知,级数$\sum_{n=1}^\infty -\frac{\sin nx}{n^2}$在$(-\infty, +\infty)$上一致收敛.
    因此$f''(x) = \left( \sum_{n=1}^\infty \frac{\sin nx}{n^4} \right)'' = \sum_{n=1}^\infty -\frac{\sin nx}{n^2}$在$(-\infty, +\infty)$.

    因为$\frac{\sin nx}{n^2}$在$(-\infty, +\infty)$上连续,由教材定理7.34知,$f''(x)$在$(-\infty, +\infty)$上连续.
\end{solution}

\begin{exercise}[7.2.8]
    设 $\displaystyle f(x) = \sum_{n=1}^\infty \frac{x^n \cos \frac{n\pi}{x}}{(1 + 2x)^n}$, 求 $\lim_{x \to +\infty} f(x)$ 及 $\lim_{x \to 1} f(x)$.
\end{exercise}

\begin{solution}
    $$\left| \frac{x^n}{(1 + 2x)^n} \cos \frac{n\pi}{x} \right| \les \frac{1}{2^n}.$$
    由Weierstrass判别法知,级数$\sum_{n=1}^\infty \frac{x^n \cos (n\pi)}{(1 + 2x)^n}$在$[1, +\infty)$上一致收敛.

    又$u_n(x) = \frac{x^n \cos \frac{n\pi}{x}}{(1 + 2x)^n}$在$[1, +\infty)$上连续,由教材定理7.34知,$f(x)$在$[1, +\infty)$上连续,故
    $$\lim_{x \to +\infty} f(x) = \sum_{n=1}^\infty \lim_{x \to +\infty} \frac{x^n \cos \frac{n\pi}{x}}{(1 + 2x)^n} = \sum_{n=1}^\infty \frac{1}{2^n} = 1.$$
    $$\lim_{x \to 1} f(x) = f(1) = \sum_{n=1}^\infty \frac{1^n \cos n\pi}{(1 + 2 \cdot 1)^n} = \sum_{n=1}^\infty \frac{(-1)^n}{3^n} = -\frac{1}{4}.$$
\end{solution}

\begin{exercise}[7.2.9]
    设 $f(x) = \displaystyle \sum_{n=1}^\infty n \mathrm{e}^{-nx}$, 求 $\displaystyle \int_{\ln 2}^{\ln 3} f(x) \mathrm{d}x$.
\end{exercise}

\begin{solution}
    $$\int_{\ln 2}^{\ln 3} n \e^{-nx} \mathrm{d}x = \left. -\mathrm{e}^{-nx} \right|_{x = \ln 2}^{x = \ln 3} = \frac{1}{2^n} - \frac{1}{3^n}.$$
    且
    $$\sum_{n=1}^\infty \left( \frac{1}{2^n} - \frac{1}{3^n} \right) = \sum_{n=1}^\infty \frac{1}{2^n} - \sum_{n=1}^\infty \frac{1}{3^n} = 1 - \frac{1}{2} = \frac{1}{2}$$收敛,且$\sum_{n=1}^\infty n \mathrm{e}^{-nx}$在$[\ln 2, \ln 3]$上一致收敛,
    故可以交换积分和求和顺序,有
    $$\int_{\ln 2}^{\ln 3} f(x) \mathrm{d}x = \sum_{n=1}^\infty \int_{\ln 2}^{\ln 3} n \mathrm{e}^{-nx} \mathrm{d}x = \sum_{n=1}^\infty \left( \frac{1}{2^n} - \frac{1}{3^n} \right) = \frac{1}{2}.$$
\end{solution}

\begin{exercise}[7.2.10]
    递归定义 $[0, 1]$ 上的连续可微函数列 $\{f_n\}$ 如下: $f_1 = 1$, 在 $(0, 1)$ 上有
    $$ f_{n+1}'(x) = f_n(x) f_{n+1}(x), \quad f_{n+1}(0) = 1. $$
    求证: 对每个 $x \in [0, 1]$, $\displaystyle \lim_{n \to \infty} f_n(x)$ 存在, 并求出其极限函数.
\end{exercise}

\begin{solution}
    设$f_k(x) = \sum_{n=0}^\infty a_n^{(k)} x^n$,则由$f_{k+1}'(x) = f_k(x) f_{k+1}(x)$,有
    $$\sum_{n=1}^\infty n a_n^{(k+1)} x^{n-1} = \left( \sum_{n=0}^\infty a_n^{(k)} x^n \right) \left( \sum_{n=0}^\infty a_n^{(k+1)} x^n \right).$$
    故
    $$\sum_{n=0}^\infty \left( (n+1) a_{n+1}^{(k+1)} - \sum_{m=0}^n a_m^{(k)} a_{n-m}^{(k+1)} \right) x^n = 0.$$
    因此$(n+1) a_{n+1}^{(k+1)} - \sum_{m=0}^n a_m^{(k)} a_{n-m}^{(k+1)} = 0$,且$a_0^{(k+1)} = f_{k+1}(0) = 1$,由此可得
    $$a_{n+1}^{(k+1)} = \frac{1}{n+1} \sum_{m=0}^n a_m^{(k)} a_{n-m}^{(k+1)}.$$
    于是
    $$a_0^{(2)} = 1,$$
    $$a_1^{(2)} = a_0^{(1)} a_0^{(2)} = 1,$$
    $$a_2^{(2)} = \frac{1}{2} (a_0^{(1)} a_1^{(2)} + a_1^{(1)} a_0^{(2)}) = \frac{1}{2} (1 + 0) = \frac{1}{2},$$
    $$a_0^{(3)} = 1,$$
    $$a_1^{(3)} = a_0^{(2)} a_0^{(3)} = 1,$$
    $$a_2^{(3)} = \frac{1}{2} (a_0^{(2)} a_1^{(3)} + a_1^{(2)} a_0^{(3)}) = \frac{1}{2} (1 + 1) = 1,$$
    $$a_3^{(3)} = \frac{1}{3} (a_0^{(2)} a_2^{(3)} + a_1^{(2)} a_1^{(3)} + a_2^{(2)} a_0^{(3)}) = \frac{1}{3} (1 + 1 + \frac12) = \frac{5}{6}.$$
    依此类推可得$$a_0^{(k)} = a_1^{(k)} = \cdots = a_{k-1}^{(k)} = 1,$$或者说$a_n^{(k)} = 1, \forall n < k$.

    由此我们能写出部分和函数$S_n^{(k)}(x) = \sum_{m=0}^{n} a_m^{(k)} x^m$,当$n < k$时,$$S_n^{(k)}(x) = \sum_{m=0}^{n} x^m = \frac{1 - x^{n+1}}{1 - x},$$

    因为$\sum_{m=0}^{n} x^m$在$(0, 1)$上内闭一致收敛到$\frac{1}{1 - x}$,故对每个给定的$x \in (0, 1)$,当$n \to \infty$时,$S_n^{(k)}(x) \to \frac{1}{1 - x}$,即$\lim_{k \to \infty} f_k(x) = \frac{1}{1 - x}$.

\end{solution}

\begin{exercise}[7.2.11]
    证明 Dini (迪尼) 定理:
    \begin{enumerate}[(1)]
        \item 设 $\{u_n(x)\}$ 是定义在闭区间 $[a, b]$ 上的非负连续函数列, 且在此区间上逐点收敛到零. 若对任意固定的 $x \in [a, b]$, 数列 $\{u_n(x)\}$ 是单调递减的, 则 $\{u_n(x)\}$ 在 $[a, b]$ 上一致收敛到零.
        \item 如果函数项级数 $\displaystyle \sum_{n=1}^\infty u_n(x)$ 在区间 $[a, b]$ 上逐点收敛到 $S(x)$, 且通项 $u_n(x)$ 在区间 $[a, b]$ 上是连续且非负的, 那么函数 $S(x)$ 在 $[a, b]$ 上连续的充分必要条件是此级数在 $[a, b]$ 上一致收敛.
    \end{enumerate}
\end{exercise}

\begin{solution}
    \begin{enumerate}[(1)]
        \item 对于任给的$\varepsilon > 0$,$\forall x \in [a,b]$,存在$N_x, U(x)$,使得当$ n \ges N_x$且$x \in U(x)$时,$u_n(x) < \varepsilon$.

              由$[a,b]$的紧性,或者说闭区间的有限覆盖定理,存在$x_1, x_2, \cdots, x_m$,使得$[a,b] \subset \bigcup_{i=1}^m U(x_i)$.

              取$N = \max\{N_{x_1}, N_{x_2}, \cdots, N_{x_m}\}$,则当$n \ges N$时,对任意$x \in [a,b]$,存在$i \in \{1,2,\cdots,m\}$使得$x \in U(x_i)$,因此有
              $$u_n(x) \les u_n(x_i) < \varepsilon.$$
              故$\{u_n(x)\}$在$[a,b]$上一致收敛到零.
        \item 对部分和函数$S_n(x) = \sum_{k=1}^n u_k(x)$,由(1)知,$\{S(x) - S_n(x)\}$在$[a,b]$上一致收敛到零,即级数$\sum_{n=1}^\infty u_n(x)$在$[a,b]$上一致收敛.

              反之,若级数$\sum_{n=1}^\infty u_n(x)$在$[a,b]$上一致收敛,则部分和函数$S_n(x) = \sum_{k=1}^n u_k(x)$在$[a,b]$上一致收敛到$S(x)$,且每个$S_n(x)$在$[a,b]$上连续,由教材定理7.34知,$S(x)$在$[a,b]$上连续.
    \end{enumerate}
\end{solution}

\newpage