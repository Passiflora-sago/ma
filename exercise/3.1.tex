\chapter{单变量函数的微分学}

\section{习题3.1}

\begin{exercise}[3.1.1]
    讨论下列函数在点 $x=0$ 处是否可导:
    \begin{tasks}[label=(\arabic*)](2)
        \task $f(x) = |\sin x|$;
        \task $f(x) = \begin{cases} x+1, & x \ges 0, \\ 1, & x < 0; \end{cases}$
        \task $f(x) = \begin{cases} x^2\sin\frac{1}{x}, & x \ne 0, \\ 0, & x=0; \end{cases}$
        \task $f(x) = \begin{cases} \ln(1+x), & x \ges 0, \\ x+1, & x < 0; \end{cases}$
        \task $f(x) = |x|\mathrm{e}^x$;
        \task $f(x) = |x^3|$.
    \end{tasks}
\end{exercise}

\begin{exercise}[3.1.2]
    求 $a,b$ 的值, 使下列函数处处可导:
    \begin{tasks}[label=(\arabic*)](2)
        \task $f(x) = \begin{cases} x^2, & x \les 1, \\ ax+b, & x > 1; \end{cases}$
        \task $f(x) = \begin{cases} \ln(1+x), & x < 0, \\ ax+b, & x \ges 0. \end{cases}$
    \end{tasks}
\end{exercise}

\begin{exercise}[3.1.3]
    设函数 $g(x)$ 在 $x=a$ 处连续, 记 $f(x) = (x-a)g(x)$. 证明 $f'(a)=g(a)$.
\end{exercise}

\begin{exercise}[3.1.4]
    若函数 $f(x)$ 在 $x_0$ 处可导, 证明
    $$ \lim_{h \to 0} \frac{f(x_0+\alpha h) - f(x_0-\beta h)}{h} = (\alpha+\beta)f'(x_0) \quad (\alpha, \beta \text{为常数}). $$
\end{exercise}

\begin{exercise}[3.1.5]
    设函数 $f(x)$ 在 $x=a$ 处可导, 且 $f(a) \ne 0$, 证明函数 $|f(x)|$ 在 $x=a$ 也可导. 若 $f(a)=0$, 结论是否仍成立?
\end{exercise}

\begin{exercise}[3.1.6]
    求下列函数的导数.
    \begin{tasks}[label=(\arabic*)](2)
        \task $y = \frac{x}{3x^2+5x-2}$;
        \task $y = \sin x \tan x + \cot x$;
        \task $y = x^2 \log_3 x$;
        \task $y = \frac{x}{1-\cos x}$;
        \task $y = \frac{1+\ln x}{x^2+1}$;
        \task $y = \frac{(1+x^2)\ln x}{\sin x + \cos x}$;
        \task $y = (x^2+1)(3x-1)(1-x^3)$;
        \task $y = x^3 \cdot \tan x \cdot \ln x$.
    \end{tasks}
\end{exercise}

\begin{exercise}[3.1.7]
    求下列函数的导数:
    \begin{tasks}[label=(\arabic*)](2)
        \task $y = x\sqrt{1-x^2}$;
        \task $y = \sqrt{1+\ln^2 x}$;
        \task $y = \arccos\frac{2x-1}{\sqrt{3}}$;
        \task $y = (\sin x + \cos x)^3$;
        \task $y = (\sin x^3)^3$;
        \task $y = \sqrt{x+\sqrt{x+\sqrt{x}}}$;
        \task $y = \sin[\sin(\sin x)]$;
        \task $y = \sin[\cos^5(\arctan x^3)]$;
        \task $y = \left(\frac{x^2-1}{x^2+1}\right)^3$;
        \task $y = x\sqrt{1+x^2}\sin x$;
        \task $y = \mathrm{e}^{\sqrt{x^2+1}}$;
        \task $y = \ln[\ln^2(\ln^3 x)]$;
        \task $y = x^{x^x}+x^{x}+x^{2^x}$;
        \task $y = (\ln x)^x$;
        \task $y = (\tan x)^{\cot x}$;
        \task $y = 10^x, \quad (\sin x)^{\cos x}$;
        \task $y = \frac{(x+5)^2(x-4)^{1/3}}{(x+2)^3(x+4)^{1/2}}$;
        \task $y = \frac{1-\sqrt{x}}{1+\sqrt{x}}\sqrt{\frac{1+x}{1+x^2}}$.
    \end{tasks}
\end{exercise}

\begin{exercise}[3.1.8]
    设 $f(x)=x^3$. 求 $f'(x^2)$ 与 $[f(x^2)]'$.
\end{exercise}

\begin{exercise}[3.1.9]
    设 $f(x)=\ln(x+\sqrt{1+x^2}), g(x)=\mathrm{e}^{\sqrt{x^2+1}}$. 求 $f'[g(x)], [f(g(x))]'$.
\end{exercise}

\begin{exercise}[3.1.10]
    设 $f(x)$ 处处可导. 求 $\frac{\mathrm{d}y}{\mathrm{d}x}$:
    \begin{tasks}[label=(\arabic*)](2)
        \task $y = f(x^3)$;
        \task $y = f(\sin^2 x) + f(\cos^2 x)$;
        \task $y = f(\mathrm{e}^x+x^{\mathrm{e}})$;
        \task $y = \sin[f(\sin f(x))]$;
        \task $y = f[f(f(x+\cos x))]$;
        \task $y = f(\mathrm{e}^x) \mathrm{e}^{f(x)}$.
    \end{tasks}
\end{exercise}

\begin{exercise}[3.1.11]
    求下列函数的导数:
    \begin{tasks}[label=(\arabic*)](2)
        \task $y = \begin{cases} x\mathrm{e}^{1/x}, & x \ne 0, \\ 0, & x=0; \end{cases}$
        \task $y = |1-2x|\sin x$.
    \end{tasks}
\end{exercise}

\begin{exercise}[3.1.12]
    设 $n$ 为正整数, 考虑函数 $f(x) = \begin{cases} x^n \sin\frac{1}{x}, & x \ne 0, \\ 0, & x=0. \end{cases}$ 证明:
    \begin{enumerate}
        \item 当 $n=1$ 时, $f(x)$ 在点 $x=0$ 处不可导;
        \item 当 $n=2$ 时, $f(x)$ 在点 $x=0$ 处可导, 但导函数在 $x=0$ 处不连续 (事实上, 在这一点有第二类间断);
        \item 当 $n \ges 3$ 时, $f(x)$ 在点 $x=0$ 处可导, 且导函数在 $x=0$ 处连续.
    \end{enumerate}
\end{exercise}

\begin{exercise}[3.1.13]
    证明: 函数 $f(x) = \begin{cases} x^2\sin\frac{1}{x^2}, & x \ne 0, \\ 0, & x=0 \end{cases}$ 在区间 $[-1,1]$ 上处处可导, 但导函数在这个区间上无界.
\end{exercise}

\begin{exercise}[3.1.14]
    求下列函数的反函数的微商.
    \begin{tasks}[label=(\arabic*)](2)
        \task $y = x\mathrm{e}^x$;
        \task $y = \arctan\frac{1}{x}$;
        \task $y = 2x^3 - \mathrm{e}^{-2x}$;
        \task $y = \ln(\mathrm{e}^x+\sqrt{1+\mathrm{e}^{2x}})$.
    \end{tasks}
\end{exercise}

\begin{exercise}[3.1.15]
    证明: 可导的偶函数的导数为奇函数; 而可导的奇函数的导数为偶函数.
\end{exercise}

\begin{exercise}[3.1.16]
    证明: 可导的周期函数的导数仍是周期函数.
\end{exercise}

\begin{exercise}[3.1.17]
    求下列各式之和:
    \begin{enumerate}
        \item $P_n=1+2x+3x^2+\dots+nx^{n-1}$;
        \item $Q_n=1^2+2^2x+3^2x^2+\dots+n^2x^{n-1}$;
        \item $R_n = \cos x + 2\cos 2x + \dots + n\cos nx$.
    \end{enumerate}
\end{exercise}

\begin{exercise}[3.1.18]
    求下列函数的二阶导数:
    \begin{tasks}[label=(\arabic*)](2)
        \task $y = \mathrm{e}^{-x^2}$;
        \task $y = x^2 2^{2x}$;
        \task $y = (1+x^2)\arctan x$;
        \task $y = \begin{cases} x^2, & x \ges 0, \\ -x^2, & x < 0. \end{cases}$
    \end{tasks}
\end{exercise}

\begin{exercise}[3.1.19]
    设函数 $f(x)$ 处处有三阶导数, 求 $y'', y'''$.
    \begin{tasks}[label=(\arabic*)](2)
        \task $y=f(x^2)$;
        \task $y=f(\mathrm{e}^x+x)$.
    \end{tasks}
\end{exercise}

\begin{exercise}[3.1.20]
    设 $f(x)=x^n|x|$ ($n$ 为正整数), 证明 $f^{(n)}(0)$ 存在, 但 $f^{(n+1)}(0)$ 不存在.
\end{exercise}

\begin{exercise}[3.1.21]
    证明: 如果 $x_0$ 是多项式 $P_n(x)$ 的 $r$ 重根, 即 $P_n(x)$ 可以分解成
    $$ P_n(x) = (x-x_0)^r Q_{n-r}(x), $$
    其中 $Q_{n-r}(x)$ 是一个 $n-r$ 次多项式, 且 $Q_{n-r}(x_0) \ne 0$. 则 $P_n(x)$ 满足条件
    $$ P_n(x_0)=0, \quad P_n'(x_0)=0, \quad \dots, \quad P_n^{(r-1)}(x_0)=0, \quad P_n^{(r)}(x_0) \ne 0. $$
\end{exercise}

\begin{exercise}[3.1.22]
    求下列函数的高阶导数:
    \begin{tasks}[label=(\arabic*)](2)
        \task $(x^2\mathrm{e}^x)^{(n)}$;
        \task $[(x^2+1)\sin x]^{(n)}$;
        \task $\left(\frac{1}{x^2-3x+2}\right)^{(n)}$;
        \task $(\sin x \cdot \cos x)^{(n)}$.
    \end{tasks}
\end{exercise}

\begin{exercise}[3.1.23]
    求曲线 $y=\cos x$ 在 $x=\frac{\pi}{4}$ 处的切线方程.
\end{exercise}

\begin{exercise}[3.1.24]
    证明: 双曲线 $xy=1$ 上任一点处的切线, 与两坐标轴构成的三角形的面积为定值.
\end{exercise}

\begin{exercise}[3.1.25]
    有一底半径为 $r$ cm, 高为 $h$ cm 的正圆锥形容器, 现以 $a \text{ cm}^3/\text{s}$ 的速度自顶部向其内注水, 求水面上升的速度.
\end{exercise}

\begin{exercise}[3.1.26]
    水自高为 18 cm, 底半径为 6 cm 的圆锥形漏斗流入直径为 10 cm 的圆柱形筒中. 已知水在漏斗中深度为 12 cm 时水平面下降的速率为 1 cm/min. 试求圆柱形筒中水面上升的速度.
\end{exercise}

\newpage