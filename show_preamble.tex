\documentclass[lang=cn,12pt,scheme=chinese,mode=simple,black]{elegantbook}
\usepackage{amsmath}
\usepackage{booktabs}

% ====== 自定义数学运算符 ======

% ----- command -----

%# -*- coding: utf-8 -*-

% preamble_command_mathematics_v2_0.tex

% ---------- macro definition ----------
% ----- equation setting -----
\everymath{\displaystyle}

% ----- command -----
\newcommand{\les}{\leqslant}
\newcommand{\ges}{\geqslant}
\newcommand{\degree}{^{\circ}}

% \renewcommand\pi{\piup}
\newcommand{\PI}{\uppi}
% \renewcommand\pi{\uppi}
% \renewcommand\pi{\text{\textpi}}
% \renewcommand\pi{\textrm{\greektext p}}

\newcommand{\nimplies}{\centernot\implies}
\newcommand{\nimpliedby}{\centernot\impliedby}

\DeclareMathOperator{\sgn}{sgn}
\DeclareMathOperator{\dif}{d\!}
\DeclareMathOperator{\diff}{d}
\DeclareMathOperator{\trans}{T}
\DeclareMathOperator{\tr}{tr}
\DeclareMathOperator{\grad}{\mathbf{grad}}
\DeclareMathOperator{\rot}{\mathbf{rot}}
\DeclareMathOperator{\divg}{\mathbf{div}}
\DeclareMathOperator{\rank}{rank}
\DeclareMathOperator{\diag}{diag}
\DeclareMathOperator{\sinc}{sinc}
\DeclareMathOperator{\ad}{ad}
\DeclareMathOperator{\ex}{E}
\DeclareMathOperator{\var}{Var}
\DeclareMathOperator{\argmin}{arg\,min}
\DeclareMathOperator{\argmax}{arg\,max}
\DeclareMathOperator{\softmax}{softmax}
\DeclareMathOperator{\arccot}{arccot}
\DeclareMathOperator{\st}{s.t.}
\DeclareMathOperator{\iid}{i.i.d.}
\DeclareMathOperator{\Exp}{Exp}
\DeclareMathOperator{\Poi}{Poi}
\DeclareMathOperator{\const}{Const.}
\DeclareMathOperator{\sym}{sym}
\DeclareMathOperator{\cyc}{cyc}

\renewcommand{\Re}{\mathrm{Re\,}}
\renewcommand{\Im}{\mathrm{Im\,}}
\newcommand{\ve}{\varepsilon}
\newcommand{\laplace}{\Delta}
\newcommand{\Ln}{\mathop{\mathrm{Ln}}\nolimits}
\newcommand{\Arg}{\mathop{\mathrm{Arg}}\nolimits}
\newcommand{\supp}{\mathop{\mathrm{supp}}\nolimits}
\newcommand{\Aut}{\mathop{\mathrm{Aut}}\nolimits}

\let\oldexists\exists
\renewcommand{\exists}{\oldexists\,}
\newcommand{\llfloor}{\left\lfloor}
\newcommand{\rrfloor}{\right\rfloor}
\newcommand{\llceil}{\left\lceil}
\newcommand{\rrceil}{\right\rceil}

\newcommand{\rA}{\mathrm{A}}
\newcommand{\rB}{\mathrm{B}}
\newcommand{\rC}{\mathrm{C}}
\newcommand{\rD}{\mathrm{D}}
\newcommand{\rE}{\mathrm{E}}
\newcommand{\rF}{\mathrm{F}}
\newcommand{\rG}{\mathrm{G}}
\newcommand{\rH}{\mathrm{H}}
\newcommand{\rI}{\mathrm{I}}
\newcommand{\rJ}{\mathrm{J}}
\newcommand{\rK}{\mathrm{K}}
\newcommand{\rL}{\mathrm{L}}
\newcommand{\rM}{\mathrm{M}}
\newcommand{\rN}{\mathrm{N}}
\newcommand{\rO}{\mathrm{O}}
\newcommand{\rP}{\mathrm{P}}
\newcommand{\rQ}{\mathrm{Q}}
\newcommand{\rR}{\mathrm{R}}
\newcommand{\rS}{\mathrm{S}}
\newcommand{\rT}{\mathrm{T}}
\newcommand{\rU}{\mathrm{U}}
\newcommand{\rV}{\mathrm{V}}
\newcommand{\rW}{\mathrm{W}}
\newcommand{\rX}{\mathrm{X}}
\newcommand{\rY}{\mathrm{Y}}
\newcommand{\rZ}{\mathrm{Z}}
\newcommand{\ra}{\mathrm{a}}
\newcommand{\rb}{\mathrm{b}}
\newcommand{\rc}{\mathrm{c}}
\newcommand{\rd}{\mathrm{d}}
\newcommand{\re}{\mathrm{e}}
\newcommand{\rf}{\mathrm{f}}
\newcommand{\rg}{\mathrm{g}}
\newcommand{\rh}{\mathrm{h}}
\newcommand{\ri}{\mathrm{i}}
\newcommand{\rj}{\mathrm{j}}
\newcommand{\rk}{\mathrm{k}}
\newcommand{\rl}{\mathrm{l}}
\renewcommand{\rm}{\mathrm{m}}
\newcommand{\rn}{\mathrm{n}}
\newcommand{\ro}{\mathrm{o}}
\newcommand{\rp}{\mathrm{p}}
\renewcommand{\rq}{\mathrm{q}}
\newcommand{\rr}{\mathrm{r}}
\newcommand{\rs}{\mathrm{s}}
\newcommand{\rt}{\mathrm{t}}
\newcommand{\ru}{\mathrm{u}}
\newcommand{\rv}{\mathrm{v}}
\newcommand{\rw}{\mathrm{w}}
\newcommand{\rx}{\mathrm{x}}
\newcommand{\ry}{\mathrm{y}}
\newcommand{\rz}{\mathrm{z}}

\newcommand{\0}{\bm{0}}
\newcommand{\1}{\bm{1}}
\newcommand{\A}{\bm{A}}
\newcommand{\B}{\bm{B}}
\newcommand{\C}{\mathbb{C}}
\newcommand{\D}{\bm{D}}
\newcommand{\E}{\bm{E}}
\newcommand{\F}{\bm{F}}
\newcommand{\G}{\bm{G}}
\renewcommand{\H}{\bm{H}}
\newcommand{\I}{\bm{I}}
\newcommand{\J}{\bm{J}}
\newcommand{\K}{\mathbb{K}}
\renewcommand{\L}{\bm{L}}
\newcommand{\M}{\bm{M}}
\newcommand{\N}{\mathbb{N}}
\renewcommand{\O}{\bm{O}}
\renewcommand{\P}{\bm{P}}
\newcommand{\Q}{\mathbb{Q}}
\newcommand{\R}{\mathbb{R}}
\renewcommand{\S}{\bm{S}}
\newcommand{\T}{\bm{T}}
\newcommand{\U}{\bm{U}}
\newcommand{\V}{\bm{V}}
\newcommand{\W}{\bm{W}}
\newcommand{\X}{\bm{X}}
\newcommand{\Y}{\bm{Y}}
\newcommand{\Z}{\mathbb{Z}}
\renewcommand{\a}{\bm{a}}
\renewcommand{\b}{\bm{b}}
\renewcommand{\c}{\bm{c}}
\renewcommand{\d}{\bm{d}}
\newcommand{\e}{\mathrm{e}}
\newcommand{\f}{\bm{f}}
\newcommand{\g}{\bm{g}}
\newcommand{\h}{\bm{h}}
\renewcommand{\i}{\mathrm{i}}
\renewcommand{\j}{\bm{j}}
\renewcommand{\k}{\bm{k}}
\renewcommand{\l}{\bm{l}}
\newcommand{\m}{\bm{m}}
\newcommand{\n}{\bm{n}}
\renewcommand{\o}{\bm{o}}
\newcommand{\p}{\bm{p}}
\newcommand{\q}{\bm{q}}
\renewcommand{\r}{\bm{r}}
\newcommand{\br}{\bm{r}}
\newcommand{\s}{\bm{s}}
\renewcommand{\t}{\bm{t}}
\renewcommand{\u}{\bm{u}}
\renewcommand{\v}{\bm{v}}
\newcommand{\w}{\bm{w}}
\newcommand{\x}{\bm{x}}
\newcommand{\y}{\bm{y}}
\newcommand{\z}{\bm{z}}

\newcommand{\balpha}{\bm{\alpha}}
\newcommand{\bbeta}{\bm{\beta}}
\newcommand{\bchi}{\bm{\chi}}
\newcommand{\bdelta}{\bm{\delta}}
\newcommand{\bvarepsilon}{\bm{\varepsilon}}
\newcommand{\bphi}{\bm{\phi}}
\newcommand{\bvarphi}{\bm{\varphi}}
\newcommand{\bgamma}{\bm{\gamma}}
\newcommand{\bmeta}{\bm{\eta}}
\newcommand{\biota}{\bm{\iota}}
\newcommand{\bkappa}{\bm{\kappa}}
\newcommand{\blambda}{\bm{\lambda}}
\newcommand{\bvarLambda}{\bm{\varLambda}}
\newcommand{\bmu}{\bm{\mu}}
\newcommand{\bnu}{\bm{\nu}}
\newcommand{\bpi}{\bm{\pi}}
\newcommand{\bvarpi}{\bm{\varpi}}
\newcommand{\btheta}{\bm{\theta}}
\newcommand{\bvartheta}{\bm{\vartheta}}
\newcommand{\brho}{\bm{\rho}}
\newcommand{\bsigma}{\bm{\sigma}}
\newcommand{\bvarsigma}{\bm{\varsigma}}
\newcommand{\btau}{\bm{\tau}}
\newcommand{\bupsilon}{\bm{\upsilon}}
\newcommand{\bomega}{\bm{\omega}}
\newcommand{\bxi}{\bm{\xi}}
\newcommand{\bpsi}{\bm{\psi}}
\newcommand{\bzeta}{\bm{\zeta}}
\newcommand{\bAlpha}{\bm{\Alpha}}
\newcommand{\bBeta}{\bm{\Beta}}
\newcommand{\bChi}{\bm{\Chi}}
\newcommand{\bDelta}{\bm{\Delta}}
\newcommand{\bEpsilon}{\bm{\Epsilon}}
\newcommand{\bPhi}{\bm{\Phi}}
\newcommand{\bGamma}{\bm{\Gamma}}
\newcommand{\bEta}{\bm{\Eta}}
\newcommand{\bIota}{\bm{\Iota}}
\newcommand{\bKappa}{\bm{\Kappa}}
\newcommand{\bLambda}{\bm{\Lambda}}
\newcommand{\bMu}{\bm{\Mu}}
\newcommand{\bNu}{\bm{\Nu}}
\newcommand{\bPi}{\bm{\Pi}}
\newcommand{\bTheta}{\bm{\Theta}}
\newcommand{\bRho}{\bm{\Rho}}
\newcommand{\bSigma}{\bm{\Sigma}}
\newcommand{\bvarSigma}{\bm{\varSigma}}
\newcommand{\bTau}{\bm{\Tau}}
\newcommand{\bUpsilon}{\bm{\Upsilon}}
\newcommand{\bOmega}{\bm{\Omega}}
\newcommand{\bXi}{\bm{\Xi}}
\newcommand{\bPsi}{\bm{\Psi}}
\newcommand{\bZeta}{\bm{\Zeta}}

% 偏导数,源于physics package,但physics包会冲突
\DeclareDocumentCommand\pdv{ s o m g g d() }
{
  \IfBooleanTF{#1}
  {\let\fractype\flatfrac}
  {\let\fractype\frac}
  \IfNoValueTF{#4}
  {
    \IfNoValueTF{#6}
    {\fractype{\partial \IfNoValueTF{#2}{}{^{#2}}}{\partial #3\IfNoValueTF{#2}{}{^{#2}}}} {\fractype{\partial \IfNoValueTF{#2}{}{^{#2}}}{\partial #3\IfNoValueTF{#2}{}{^{#2}}} \argopen(#6\argclose)} } { \IfNoValueTF{#5}
    {\fractype{\partial \IfNoValueTF{#2}{}{^{#2}} #3}{\partial #4\IfNoValueTF{#2}{}{^{#2}}}} {\fractype{\partial^2 #3}{\partial #4 \partial #5}} } 
}

% 对\oiint等命令的支持
\usepackage{esint}

% 重定义\thmref{}
\renewcommand{\theoremautorefname}{定理}
\renewcommand{\figureautorefname}{图}

% 欧几里得几何作图
\usepackage{tkz-euclide}

% 无引用脚注
\newcommand\blfootnote[1]{%
    \begingroup
    \renewcommand\thefootnote{}%
    \footnote{#1}%
    \addtocounter{footnote}{-1}%
    \endgroup
}

\begin{document}
\section*{规范化符号}
比较重要的是规范表达自然底数$\e$,虚数单位$\i$,微分算子$\dif$,大于等于$\ges$,小于等于$\les$时
前三种尽量规范表达成正体的形式,后面两个这种表达成这种等于号部分是斜的,如下表所示;

另外区别小写字母有些带有\verb|var|字样,另外我们还有对\verb|\varepsilon|的一个简写成\verb|\ve|.
以及我们加入了esint包用以支持闭合曲面曲线的积分符号.

\begin{center}
\setlength{\tabcolsep}{10pt}
\begin{tabular}{|c|c|c|}
\hline
\textbf{命令} & \textbf{结果} & \textbf{名称}\\
\hline
\verb|\e| & $\e$ & 自然底数 \\
\verb|\i| & $\i$ & 虚数单位 \\
\verb|\dif| & $\dif$ & 微分符号 \\
\verb|\les| & $\les$ & 小于等于 \\
\verb|\ges| & $\ges$ & 大于等于 \\
\verb|\ve| & $\ve$ & $\varepsilon$ 小量 \\
\verb|\oint| & $\oint$ & 闭合曲线积分 \\
\verb|\oiint| & $\oiint$ & 闭合曲面积分 \\
\hline
\end{tabular}
\end{center}

对于偏导数,提供命令\verb|\pdv|来简化表示,常用表达放在了下面.
\begin{center}
\renewcommand{\arraystretch}{1.4}
\setlength{\tabcolsep}{10pt}
\begin{tabular}{|l|l|l|}
\hline
\textbf{命令} & \textbf{结果} & \textbf{说明} \\
\hline
\verb|\pdv{f}{x}| & $\pdv{f}{x}$ & 一阶偏导 \\
\verb|\pdv[3]{f}{x}| & $\pdv[3]{f}{x}$ & 对角矩阵单一变量高阶偏导 \\
\verb|\pdv{f}{x}{y}| & $\pdv{f}{x}{y}$ & 二阶混合偏导 \\
\verb|\pdv{x}| & $\pdv{x}$ & 偏导算子 \\
\verb|\pdv{f^4}{x\partial y^2\partial z}| & $\pdv{f^4}{x\partial y^2\partial z}$ & 表示复杂偏导的一种技巧 \\

\hline
\end{tabular}
\end{center}

\newpage
\section*{其他符号}
大多是正体的某些常见函数,用以规范表达.仅做查询,大概按常用的领域分了一下类.

\begin{center}
\setlength{\tabcolsep}{10pt}
\begin{tabular}{|c|c|c|c|}
\hline
\textbf{命令} & \textbf{结果} & \textbf{名称} & \textbf{示例/备注}\\
\hline
\verb|\exists| & $\exists$ & 存在 & \\
\verb|\diff| & $\diff$ & 微分符号 (同上) & 实际上同上文\\
\verb|\sgn| & $\sgn$ & 符号函数 & \\
\verb|\llfloor x \rrfloor| & $\llfloor x \rrfloor$ & 向下取整 & 已经\verb|\right\left |调整大小\\
\verb|\llceil x \rrceil| & $\llceil x \rrceil$ & 向上取整 & 同上\\
\verb|\arccot| & $\arccot$ & 反余切 & \\
\verb|\sinc| & $\sinc$ & sinc 函数 & $\sinc x = \frac{\sin x}{x}$\\
\verb|\supp| & $\supp$ & 支撑集 & \\
\verb|\argmin| & $\argmin$ & 取最小值的自变量 & $\argmin_{i \le n} a_i$\\
\verb|\argmax| & $\argmax$ & 取最大值的自变量 & \\
\verb|\grad| & $\grad$ & 梯度 & \\
\verb|\rot| & $\rot$ & 旋度 & \\
\verb|\divg| & $\divg$ & 散度 & \verb|\div|是除号$\div$\\
\verb|\laplace| & $\laplace$ & 拉普拉斯算子 & \\
\verb|\trans| & $\trans$ & 转置 & $A^{\trans}$\\
\verb|\tr| & $\tr$ & 迹 & \\
\verb|\rank| & $\rank$ & 秩 & \\
\verb|\diag| & $\diag$ & 对角矩阵 & $\diag\{a_1,\dots,a_n\}$\\
\verb|\Ln| & $\Ln$ & 多值自然对数 & \\
\verb|\Arg| & $\Arg$ & 多值辐角 & \\
\verb|\Aut| & $\Aut$ & 自同构 & \\
\verb|\Re| & $\Re$ & 实部 & \\
\verb|\Im| & $\Im$ & 虚部 & \\
\verb|\ex| & $\ex$ & 期望 & \\
\verb|\var| & $\var$ & 方差 & \\
\verb|\Exp| & $\Exp$ & 指数分布 & \\
\verb|\Poi| & $\Poi$ & 泊松分布 & \\
\verb|\st| & $\st$ & such that (使得) & \\
\verb|\iid| & $\iid$ & 独立同分布 & \\
\verb|\const| & $\const$ & 常数 & \\
\hline
\end{tabular}
\end{center}

\begin{center}
\setlength{\tabcolsep}{10pt}
\begin{tabular}{|c|c|c|c|}
\hline
\textbf{命令} & \textbf{结果} & \textbf{名称} & \textbf{示例/备注}\\
\verb|\softmax| & $\softmax$ & softmax 函数 & $\softmax(x_k)=\frac{e^{x_k}}{\sum_i e^{x_i}}$,机器学习常用\\
\verb|\ad| & $\ad$ & 对合算子 (Lie 代数) & \\
\verb|\sym| & $\sym$ & 对称部分 & \\
\verb|\cyc| & $\cyc$ & 循环和 & \\
\verb|\degree| & $\degree$ & 角度制的度 & \\
\hline
\end{tabular}
\end{center}


\newpage
\section*{字母表}
字母表分成三个系列,第一个系列是\verb|\*|这种模式的字母,大部分是黑体,少数是为了方便的表示替换成其他字体了,请注意需要黑体的时候额外修改.

大写字母里的表示常见数域$\N,\Z,\Q,\R,\C$,以及不特指的数域$\K$;小写字母的$\e$,$\i$是例外.
%==========================
% 黑体字母
%==========================
\begin{center}
\renewcommand{\arraystretch}{1.4}
\begin{tabular}{|c|c|c|c|c|c|c|c|c|c|c|c|c|}
\hline\hline
\verb|\A| & \verb|\B| & \verb|\C| & \verb|\D| & \verb|\E| & \verb|\F| & \verb|\G| & \verb|\H| & \verb|\I| & \verb|\J| & \verb|\K| & \verb|\L| & \verb|\M|\\
\hline
$\A$ & $\B$ & $\C$ & $\D$ & $\E$ & $\F$ & $\G$ & $\H$ & $\I$ & $\J$ & $\K$ & $\L$ & $\M$\\
\hline\hline
\verb|\N| & \verb|\O| & \verb|\P| & \verb|\Q| & \verb|\R| & \verb|\S| & \verb|\T| & \verb|\U| & \verb|\V| & \verb|\W| & \verb|\X| & \verb|\Y| & \verb|\Z|\\
\hline
$\N$ & $\O$ & $\P$ & $\Q$ & $\R$ & $\S$ & $\T$ & $\U$ & $\V$ & $\W$ & $\X$ & $\Y$ & $\Z$\\
\hline\hline
\verb|\a| & \verb|\b| & \verb|\c| & \verb|\d| & \verb|\e| & \verb|\f| & \verb|\g| & \verb|\h| & \verb|\i| & \verb|\j| & \verb|\k| & \verb|\l| & \verb|\m|\\
\hline
$\a$ & $\b$ & $\c$ & $\d$ & $\e$ & $\f$ & $\g$ & $\h$ & $\i$ & $\j$ & $\k$ & $\l$ & $\m$\\
\hline\hline
\verb|\n| & \verb|\o| & \verb|\p| & \verb|\q| & \verb|\r| & \verb|\s| & \verb|\t| & \verb|\u| & \verb|\v| & \verb|\w| & \verb|\x| & \verb|\y| & \verb|\z|\\
\hline
$\n$ & $\o$ & $\p$ & $\q$ & $\r$ & $\s$ & $\t$ & $\u$ & $\v$ & $\w$ & $\x$ & $\y$ & $\z$\\
\hline\hline
\end{tabular}
\end{center}

\vspace{1cm}

%==========================
% 直立体(\rA 等)
%==========================
第二个系列是\verb|\r*|这种模式的字母,表示的均是正体.

\begin{center}
\renewcommand{\arraystretch}{1.4}
\setlength{\tabcolsep}{6pt}
\begin{tabular}{|c|c|c|c|c|c|c|c|c|c|c|c|c|}
\hline\hline
\verb|\rA| & \verb|\rB| & \verb|\rC| & \verb|\rD| & \verb|\rE| & \verb|\rF| & \verb|\rG| & \verb|\rH| & \verb|\rI| & \verb|\rJ| & \verb|\rK| & \verb|\rL| & \verb|\rM|\\
\hline
$\rA$ & $\rB$ & $\rC$ & $\rD$ & $\rE$ & $\rF$ & $\rG$ & $\rH$ & $\rI$ & $\rJ$ & $\rK$ & $\rL$ & $\rM$\\
\hline\hline
\verb|\rN| & \verb|\rO| & \verb|\rP| & \verb|\rQ| & \verb|\rR| & \verb|\rS| & \verb|\rT| & \verb|\rU| & \verb|\rV| & \verb|\rW| & \verb|\rX| & \verb|\rY| & \verb|\rZ|\\
\hline
$\rN$ & $\rO$ & $\rP$ & $\rQ$ & $\rR$ & $\rS$ & $\rT$ & $\rU$ & $\rV$ & $\rW$ & $\rX$ & $\rY$ & $\rZ$\\
\hline\hline
\verb|\ra| & \verb|\rb| & \verb|\rc| & \verb|\rd| & \verb|\re| & \verb|\rf| & \verb|\rg| & \verb|\rh| & \verb|\ri| & \verb|\rj| & \verb|\rk| & \verb|\rl| & \verb|\rm|\\
\hline
$\ra$ & $\rb$ & $\rc$ & $\rd$ & $\re$ & $\rf$ & $\rg$ & $\rh$ & $\ri$ & $\rj$ & $\rk$ & $\rl$ & $\rm$\\
\hline\hline
\verb|\rn| & \verb|\ro| & \verb|\rp| & \verb|\rq| & \verb|\rr| & \verb|\rs| & \verb|\rt| & \verb|\ru| & \verb|\rv| & \verb|\rw| & \verb|\rx| & \verb|\ry| & \verb|\rz|\\
\hline
$\rn$ & $\ro$ & $\rp$ & $\rq$ & $\rr$ & $\rs$ & $\rt$ & $\ru$ & $\rv$ & $\rw$ & $\rx$ & $\ry$ & $\rz$\\
\hline\hline
\end{tabular}
\end{center}

\vspace{1cm}
\newpage
第三个系列是\verb|\b*|的粗体希腊字母表示向量等.

这里有一个比较特别的是\verb|\bmeta|多了个\verb|m|,是$\eta$的粗体$\bmeta$,主要是为了避免\verb|\beta| 即$\beta$重复.
%==========================
% 粗体希腊字母
%==========================
\begin{center}
\renewcommand{\arraystretch}{1.4}
\begin{tabular}{|c|c|c|c|c|c|c|c|}
\hline\hline
\verb|\balpha| & \verb|\bbeta| & \verb|\bgamma| & \verb|\bdelta| & \verb|\bvarepsilon| & \verb|\bzeta| & \verb|\bmeta| & \verb|\btheta| \\
\hline
$\balpha$ & $\bbeta$ & $\bgamma$ & $\bdelta$ & $\bvarepsilon$ & $\bzeta$ & $\bmeta$ & $\btheta$ \\
\hline\hline
\verb|\biota| & \verb|\bkappa| & \verb|\blambda| & \verb|\bmu| & \verb|\bnu| & \verb|\bxi| & \verb|\bpi| & \verb|\brho|\\
\hline
$\biota$ & $\bkappa$ & $\blambda$ & $\bmu$ & $\bnu$ & $\bxi$ & $\bpi$ & $\brho$ \\
\hline\hline
\verb|\bsigma| & \verb|\btau| & \verb|\bupsilon| & \verb|\bphi| & \verb|\bchi| & \verb|\bpsi| & \verb|\bomega| &\\ 
\hline
$\bsigma$ & $\btau$ & $\bupsilon$ & $\bphi$ & $\bchi$ & $\bpsi$ & $\bomega$ &  \\
\hline\hline
\verb|\bPhi| & \verb|\bGamma| & \verb|\bDelta|&\verb|\bTheta| & \verb|\bLambda| & \verb|\bXi| & \verb|\bPi| & \verb|\bSigma| \\
\hline
$\bPhi$ & $\bGamma$ & $\bDelta$ &$\bTheta$ & $\bLambda$ & $\bXi$ & $\bPi$ & $\bSigma$  \\
\hline\hline
 \verb|\bUpsilon|&\verb|\bOmega| & \verb|\bPsi| & & & & &\\
\hline
 $\bUpsilon$&$\bOmega$ & $\bPsi$ & & & & &\\
\hline\hline
\end{tabular}
\end{center}


\end{document}
